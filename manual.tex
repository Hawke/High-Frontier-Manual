\documentclass[a4paper]{book}

\usepackage{hyperref}

\setcounter{tocdepth}{2}

\title{High Frontier \\ Alive \& Complete, 3rd Edition}
\author{Phil Eklund}

\begin{document}
\maketitle

\newpage
A game of exoglobalization for 1 to 5 players. This edition includes the Basic, Colonization, and Interstellar games, as well as other variants.

\begin{description}
\item[RULES:] Phil Eklund, P-M Agapow, Kyrill Melai, Rus Belikov, Andy Graham, Mike Dommett
\item[NEW CARD \& MAP RESEARCH:] Dr. Noah Vale, Dave Bostwick, Pawel Garycki
\item[RULES EDITORS:] Brandon Waite, John Menichelli, Alex Mauer
\item[PLAYTESTERS]: Derek Drake, Jeff \& Eric Chamberlain, Andrew Doull, Francisco Colmenares, Xavier del Valle Muñoz

\end{description}

Last updated July 23, 2015

Copyright © 2015, Sierra Madre Games.

http://games.groups.yahoo.com/group/HighFrontier/


\large{\textbf{VASSAL version}} 

courtesy of Sam Williams

\url{http://www.vassalengine.org/wiki/Module:High_Frontier}

\textbf{Requirements}: Vassal version 3.2.6 or later. 

\textbf{Compatibility}: Saved games from 40.3.x to 40.5.0 will only be compatible if the VP2 file is active instead of VP4. (Only load one or the other file.)

\large{\textbf{Cyberboard version}}

courtesy of Pawel Garycki: 

\url{http://www.garycki.com/highfrontier.html}

\textbf{Requirements}: Cyberboard.

\tableofcontents

\part{BASIC AND COLONIZATION RULES}

\chapter{BASIC GAME OVERVIEW}
Players, representing the space-faring factions of Earth, bid for patents (space technology cards) and boost them into Low Earth Orbit (LEO) to be assembled into rockets and their cargos. Once fueled with water tanks (WTs), these rockets fly to promising industrial and science sites (planets, asteroids, etc.). If prospecting succeeds, a claim is made. A claim can be upgraded to a factory to produce useful new technologies. By extracting water from a site, a rocket can refuel (WTs are used both as rocket propellant and as currency). The winner is determined by the number of his off-world claims and factories, the resource value of his industrialized sites, and his exploration achievements (ventures and glories).
·   Terms being defined are listed in bold. Italicized words are terms that are defined elsewhere in the rules. Terms defined in the glossary are underlined. Easily missed rules are listed white on black.
·   Changes from Second Edition. 34 new cards. This includes new robonaut, thruster, refinery, reactor, generator, radiator, freighter, and gigawatt thruster cards, as well as cards used to indicate ventures and glory (I1 \& I2), and alternate crew cards provided for all factions (C0.1). The Rocket Diagram is replaced by a Fuel Strip (E4) allowing fungible fuel tanks (K6) and the carrying of different fuel grades. The Lander Fuel Penalty is replaced by lander burns (G4). Triangle burns are eliminated. Water Theft is replaced as a felony by Hijacking. Powersats and Push Factories only push thrusters with the push icon. The cost of multiple operations has changed (M1). You can use Colonists and move cards the turn they come into play. Ending the game has changed (U5).
 
A1. THIRD EDITION COMPONENTS
·   This rulebook includes Interstellar solitaire (Section Y).
·   Operations List on back page.
·   2 maps (inner \& outer solar system)
·   1 Solar System Chart
·   5 playmats with Fuel Strip
·   80 clear blue disks (water tanks, busted mines and futures, Rocket dry mass disks and  acceleration indicators, 1 politics, 1 solar cycle, 5 exploitation track)
·   15 clear red disks for Bernal dry mass disks and acceleration indicators. Also for value 5 water tanks
·   45 opaque disks (in 5 colors) for claims
·   20 yellow disks for outposts and isotope FFT. (alternate) Small colored rings
·   10 black disks for dirt FFT
·   30 cubes (in 5 colors) for factories \& mobile factories
·   5 big cubes (in 5 colors) for freighters
·   35 cylinders (in 5 colors) for Space Colonies (alternate) hemispheres
·   5 rockets (in 5 colors), plus 15 rockets (in 3 colors) for rocket location \& fuel
·  10 hemispheres (in 5 colors) plus 15 hemispheres (in 3 colors) for Bernal Sphere location \& fuel. (alternate) cylinders
·   120 cards for patents, colonists, crew, and achievements
·   1 six-sided die (1d6) for prospecting and hazards
·   2 counter-sheets with punched components for a 3D card-holder for the 9 patent decks. This is a stand that holds each of the nine decks at about a 30° angle.
B. COMPONENT DESCRIPTIONS
B1. PATENT CARDS
In the basic game, there are three types of patent cards: Thruster (used to move Rocket Stacks), Robonaut (used to prospect, refuel, and build factories), and Refinery (used to build factories).
The white side of a patent card shows a product built on Earth; the black side shows an improved product built in space. Until you build your first space factory, you will only be using the white side of the patent cards.
Mass. The mass of a card for the basic game is shown in the white box located in the upper-left corner. All information in the red box (upper right corner of the card) is only used in the Colonization advanced game (O).
ISRU. Some cards have an In Situ Resource Utilization rating, which is used to ISRU refuel (H5) and Prospect (H6).
Thrust Triangle. Cards capable of moving stacks have a thrust triangle (F1).
Product Letter. The letter shown on the black side of a patent (C, D, M, S, V) indicates what type of factory can build them (H8).
TIP: patent cards represent unbuilt ideas (if held in your hand) or hardware in space (in locations marked by figures on the Map).
B2. STARTING CARD AND PIECES
You start with the following:
A starting card identifying your color and faction privilege.
TIP: The crew cards have a thruster triangle (F1), so they can be used as the thruster for your Rocket Stack. For instance, the PRC Taikonauts with 7 tanks of fuel can fly alone from LEO to an aerobrake landing on Mars!
Nine disks of your color to mark claims and futures, plus two yellow disks to mark your outposts
Six cubes of your color to mark factories
One big (10mm) cube of your color to mark the location of your freighter
One rocket figure in your color to mark the location of your rocket.
One blue disk to mark your rocket’s dry mass and one to mark your rocket’s acceleration.
IMPORTANT: You are limited to this number of cubes, disks, and rockets. See logistics.
B3. PLAYMAT AND SOLAR SYSTEM CHARTS
Each player starts with a Playmat, used to store his cards and his WTs (water tanks stored in a WT depot in your home orbit, used for money and fueling). It contains a Fuel Strip and Acceleration Track (used to keep track of the fuel and net thrust of his Rocket Stack).
The Solar System Charts contain the five Exploitation Tracks (C, S, M, V, and D). The other tables (Sunspot Cycle, Event Table, Futures Stars, and Space Government) are only used in the Colonization game (K).
B4. SOLAR SYSTEM MAP
The map shows the spaces (i.e. the orbits in the Solar System where a rocket can “halt”), and the routes between the spaces. Some of the most useful routes are colored and labeled with a signpost (e.g. “Earth-Mars”). The various kinds of spaces are described below:
Sites. BThe black hexes are sites, i.e. planets, moons and asteroids that can be landed on. Each site is characterized by a size (a number indicating the surface gravity--the larger the number the more likely prospecting will be successful and the more powerful your rocket has to be to take off and land there), spectral type (a letter - C, D, M, S, V – showing what can be mined there and thus the product letter of cards able to be factory-produced there), and hydration (a number of waterdrops indicating how easy it is to obtain water). A microscope icon indicates a science site or a TNO science site. A plant icon indicates an Astrobiology Site, and a blue icon indicates a Subsurface Ocean (both used for certain Glory Cards). See Map Anatomy.
 Intersections. The intersection between two routes is called a Hohmann if the two lines touch, and a Lagrange Point if is marked by a hollow or filled circle. (If the lines don’t touch, they just pass by each other without intersecting). Turning requires a pivot in a Hohmann, but is free in a Lagrange Point. Each sharp turn (for instance, the zigzag routes in the outer solar system) also require a pivot.
Burns. The filled magenta spaces, called burns, cost fuel for a rocket to enter, and the number of burns a rocket can enter in a turn is limited by its net thrust. There are two types of burns: Lagrange Point (circular) and Lander (lander-shaped).
TIP: The majority of burns are either around planets or at the outer edge of each heliocentric zone. Closer to the sun, there are paths through the edges of heliocentric zones which don’t have burns, representing the Interplanetary Travel Network. Use these to trade fuel consumption for time.
Spaces can be marked with the hazard (F5), lander (G4), flyby (K1), or radiation (K2) icons.
Aerobrake Routes. Routes along a dashed brown line have special coasting rules (F6), and you cannot move along the route against the arrowhead. 
Buggy Roads. The dashed yellow lines are buggy roads, used only for buggy prospecting, not movement.
Heliocentric Zones. These divide the solar system into concentric zones to mark the decreasing solar energy as you move further away from the sun. Each zone is named after a planet (from Mercury to Neptune). These zones modify the thrust of solar-powered rockets and sails, see F2. Solar-powered cards or functions do not work in the outermost zone.
NOTE: The Small white triangles found in some burns on older edition maps are not used in this edition.
B5. WATER TANKS \& INDICATOR DISKS
Clear blue disks simulate 40-ton water tanks (WTs) stored in your home orbit and are used both as “money” and as rocket fuel. Each clear blue disk represents 1 WT. Your WTs are stored on your playmat, and unowned WTs are stored in a central pool.
Clear blue disks are also used to mark busted sites where prospecting has failed and to track game information. They are also used to track rocket dry mass and net thrust.
B6. COMPONENT LIMITATIONS
Because of mission control limitations, you are limited to the claim disks, factory cubes, and colony cylinders you start with. If you need another disk, you may withdraw a claim disk (the vacated site must be re-prospected to be claimed again). If you need another cube, you may withdraw your freighter’s Big Cube (E5) or a mobile factory cube (P6).

C. BASIC GAME SETUP
1.  Choose factions. Each Crew Card has a “legacy” faction on one side, and a “radical” faction on the other. Decide jointly The players jointly decide if the legacy or radical side will be used for by all this game. If playing with fewer than 5 players, legacy and radical factions may can be mixed as long as no two players have the same faction privilege. Then each player chooses or is given a Crew Card and a playmat.
2.  ReservesFaction Cubes, Disks, and Rockets. Place Put the 1 big cube, 1 rocket, 6 small cubes, dry mass indicator disk, and 9 disks of your faction color anywhere on your playmat.
3.  Starting Water Water Tanks. Give each player 4 clear blue WT disks.Each player places 4 clear blue WT disks on his playmat. These disks are WTs (water tanks), used as both fuel and currency.
4.  Patent cards are separated into types (Thruster, Robonaut, and Refinery), shuffled, and placed white-side up alongside each other. Patent decks. Separate the patent cards into decks by type (Thruster, Robonaut, and Refinery). Shuffle and place white-side up near the game board.
Resource exploitation. Place Ffive blue disks are placed on the starting positions of the resource exploitation tracks.
5.  Determine the starting player by any means you choose.
6.  Place the 5 Glory Cards and the 4 Venture Cards yellow-side up near the game board.in a public area.
C1. OPTIONAL QUICK START VARIANT (Basic \& Colonization Games)
This variant starts the players with a hand of patent cards, selected during special turns called Idea Turns.
Players start with 3 WT instead of 4.
Idea Turns. The first few turns are used to build up each player’s hand. Starting with the first player, each player in turn may pay 1 WT and take the top card off of one of the patent decks (including the colonist deck). Exception: The first player pays 2 WT for his first card, to offset his advantage in drawing first.
Ending Idea Turns. If for your turn you decide to perform a standard operation instead of an Idea Turn, you become the new first player for the rest of the game. No further Idea Turns can be played.
COLONIZATION: Start with 1 WT per patent deck available instead of 3 WT. During Idea Turns, there is no hand limit and the Sunspot Cycle is not active. The card drawn comes alone, without supports.
D. BASIC GAME PLAY SEQUENCE
On your turn, you may move each of your spacecraft up to its full movement (F), perform operations (H), and perform free actions (D1). In the Basic Game, you are allowed two spacecraft (your rocket and your freighter), one operation, and any number of free actions. The advanced game will make additional spacecraft and operations available (see P, Q, and R).
·       Moves and operations can be taken in any order. For instance, you may move your rocket, take an operation, and then move your freighter. 
·       When your turn is complete, play passes to the next player in clockwise order.
D1. FREE ACTIONS
You may perform the following free actions without expending an operation. Free actions may be taken in any order.The first three can be done anytime in your turn except during movement. The last six can be done at any time in your turn.
D1.1. Cargo Transfer. Form a stack per E4, E5, or E6 by transferring cargo from colocated stacks into the appropriate slot on your playmat. For instance, a Rocket Stack can be formed at LEO per E4 by transferring cards boosted to the LEO stack. Except for Freighter Stacks, a stack can have any number of cards. If you transfer cargo to your rocket or Bernal, perform a dry mass adjustment (D2). You can also transfer cargo between existing stacks this way, even between stacks of cooperating players. If you transfer cargo to or from a rocket or Bernal, perform a dry mass adjustment (D2) on each affectedsuch stack.
COLONIZATION: You cannot transfer or jettison (D1.5) dedicated cards.
D1.2. Home Orbit Refueling. In the Basic Game, the Home Orbit for all players is LEO, where you may transfer WTs to your rocket as fuel tanks. Each WT added to your rocket moves the fuel figure on your Fuel Strip one step to the right, following the dotted red line.
D1.3. Liquidate Fuel. If in your Home Orbit, you may convert one or more tanks of water or isotope fuel in your rocket or Bernal into the same number of WT disks. You can also do the reverse operation, converting WTs into fuel. Both free actions gain or lose one WT for every step the fuel figure moves on the Fuel Strip (E4), following the dashed red line. {Advanced Game} Fuel liquidation can occur away from your Home Orbit by turning fuel steps into FFTs (K6) instead of WTs.
D1.4. Jettison any amount of fuel from your rocket by simply moving the fuel figure to the left a desired number of steps along either the red or black line (E4). If this is done before you move your rocket, this may have the advantage of reducing your wet mass and improving your net thrust, see F2.
D1.5. Jettison any amount of cargo from your rocket or Bernal, which causes a dry mass adjustment (D2). This can be done at any point in the move, but it only impacts your movenet thrust if done before you start moving and the dry mass adjustment is delayed until the end of the move. The cargo can be left as an outpost (this is an exceptional case of stack formation during movement), or decommissioned.
NOTE: You normally need an operational thruster to move (for exceptions see F6).
D1.6. Decommission cards or your freighter, which returns them to your hand for reuse. Cards you wish to decommission from your rocket or Bernal must be jettisoned first. If you decommission your freighter, form an outpost with the cargo it carries. You cannot voluntarily decommission your crew except by building a space colony (D1.8) or claiming a glory (D1.9) unless you are allowed to perform felonies (e.g. the PRC or Anonymous faction privilege). You cannot decommission your Bernal card. A crew card can only be decommissioned at LEO or one of your factory sites. If the latter, it converts into a cylinder (representing a space colony).
NOTE: Decommissioning also occurs during free market, industrialization, hazards, flares, radiation belts, and combat.
D1.7. Discard cards from your hand to the bottom of the patent deck.
D1.8. Build or disband Space Colonies. To build a space colony, decommission your crew or a human colonist (R2) at your factory and place replace it with a cylinder at the location factory representing a space colony. To disband a space colony, you must have your crew card in your hand. Replace the space colony cylinder with an outpost containing your crew card.
D1.9. Claim Glory (I1) or Ventures (I2).
D1.10. Withdraw a stack counter, claim disk or mobile factory cube from the map. You can only withdraw a stack counter if it contains no cards and the stack’s fuel has been jettisoned (D1.4) or liquidated (K6). You can only withdraw a claim disk if you need to use the claim disk elsewhere. If you have a promoted freighter you may withdraw a factory cube if you industrialize elsewhere and have no factory cubes left to represent the new factory.
D2. DRY MASS ADJUSTMENT
If the rocket or Bernal takes on or discharges cargo, you will need to perform a dry mass adjustment after you complete your rocket’s movement. Simply move the dry mass disk left or right along the dotted red line to the new dry mass. Then move the fuel figure along its dotted red line the same number of steps and the same direction as the dry mass movement.

Example: Your rocket has a dry mass disk on the “4” spot and a fuel figure on 4-1/3. You drop off cargo with a mass of 2. Move the dry mass disk two steps left to the number “2” spot. Move the fuel figure two steps left as well, to the 2-1/6 spot. <illustration xxx>[BW4] 
NOTE: You may transfer cards to create or add to the stack of any cooperating player in the same space at the beginning or end of your move. FFTs can be transferred as well.
D3. DEAL-MAKING
At any time, you may exchange WTs, FFTs, claims, factories, stack cards, or services or actions as terms of a deal. These services can include faction privileges such as use of the ESA powersat or the UN cycler. If the promised service is to be provided on a subsequent turn, it is non-binding. Hand Cards can be swapped as part of a deal, as long as the number of hand cards for each trader remains unchanged (this rule is necessary to avoid abusing the hand card limits during research auctions). Crew or Bernal (Colonization) Cards can’t be sold or traded.
FFTs or cards colocated in space can be exchanged among players either to create new stacks or between colocated stacks. Transferred cards can’t be further transferred until next turn. WT transfers can occur between home orbits.
If colonists are traded, they may only perform their operation once per year.
E. HAND \& STACK CARDS
Cards enter your hand through trade (D3), research (H2), or decommission. They represent patents and potential products. Once played into a stack, they become tangible objects, either boosted from Earth (if played on their white side), or built in space (if played on their black side).
Stacks can be created by cargo transfer (D1.1), jettisons (D1.5), boosting (H4), production (H8), or (Colonization) digital swap (N1).
Your stacks can have any number of cards, except your Freighter Stack is limited to one card when not using the Freighter Module (P).
If a stack contains no cards or FFTs (K6), and the spacecraft it represents has jettisoned all its fuel (D1.4), you may remove the spacecraft figure from the map to create a stack elsewhere. IMPORTANT: Your WTs, hand cards, and stacks are free for anyone to examine.
E1. HAND CARDS
Hand cards are stored face-up to the right of your playmat.
·       Your hand is not a stack. Operations move cards from hand to a stack, decommission moves cards from a stack to hand.
·       Limits. There is no Hand Limit, however there are drawbacks to having too many Hand Cards (H2). You can discard Hand Cards per D1.7.
E2. STACK LIMITS
To represent mission control and spacesuited human limitations, the number of stacks you are allowed at the end of your turn is strictly limited to the following five stacks:
1.  Rocket Stack - this stack is represented on the map by your rocket figure.
2.  Freighter Stack - this stack is represented on the map by your big cube.
3.  LEO Stack - this stack is always at the space on the map marked LEO, and is not represented by a figure.
4 \& 5.  Two Outpost Stacks - these stacks are represented by your two yellow outpost disks.
If the Bernal Module (Q) is being used, you may have two additional stacks (for a total of seven stacks):
6.  Bernal Stack - this stack is represented on the map by a hemisphere figure of your faction color.
7.  Ersatz-Bernal Stack - this stack is always located at your home orbit and is represented by a hemisphere of any available color.
·   	Limits. You are allowed two outpost stacks, and one of each every other type of stack. During your turn you may boost or build an outpost stack over your limit, but by the end of your turn you must decommission one of them.
E3. LEO STACK CREATION AND DEFINITION
Your LEO stack represents the game components located in Low Earth Orbit (marked LEO on the map). Unlike other stacks, this stack does not have its own figure. Any cargo in the stack is simply assumed to be in LEO. A LEO stack can be created through either of the following mechanisms.
1.  When using the boost operation (H4), if cards are boosted to LEO and your LEO stack does not yet exist, you can create your LEO stack and place the boosted cards into it.
2.  As a free action by transferring converting a stack (D1.1) located in LEO to your LEO stack.
NOTE: Once your rocket is underway, you can no longer use WTs at LEO for refueling. You also cannot use fuel carried by the rocket for currency (e.g. spending in auctions).
E4. ROCKET STACK CREATION AND DEFINITION
Designate a stack as a Rocket Stack by placing cards to the left of your playmat. It will usually include a thruster (F1) and perhaps crew, but neither are required. Indicate its location on the map using a rocket figure, and indicate its mass and fuel by placing a disk and rocket figure on your playmat’s Fuel Strip per the following:
·  Compute Dry Mass. Add up the mass of every card in the Rocket Stack [Remember: use the top left red mass number for Basic Game and the top right white mass for the Advanced Game]. This is your dry mass. Place a translucent blue disk, called your dry mass disk, in the spot on your fuel strip with this number.
·  Fueling Your Rocket. Place a rocket figure, called the fuel figure, on top of the dry mass disk. This indicates your tanks are empty. The color of the fuel figure depends on the fuel grade, see below.
·  You may add fuel to your rocket by home orbit refueling (D1.2), site refuel (H5), or liquefying on-board FFTs (K6).
·  Fuel Grades. The fuel figure must be black, blue, or (Colonization) yellow, representing loading dirt, water propellant, or (Colonization) isotope fuel, respectively. The fuel grade is important since only dirt rockets can use dirt propellant, and GW/TW rockets (S) can only use isotope fuel. When mixing fuels, the fuel grade becomes the same as the lowest grade fuel in the mix (where isotope is the highest and dirt the lowest). Indicate this grade by the color of the fuel figure.
<<FUEL STRIP ILLUSTRATION>>[BW5] 

Example: A rocket has a Hall Effect thruster (mass 2), a crew (mass 1), plus a Kuck mosquito robonaut (mass 0) in cargo. Its dry mass is 3. Loaded up with one fuel tank, it has 4 steps of fuel, as shown. <<illustration xxx>>[BW6] 
WARNING: If your fuel figure ever reaches the fuel step where the dry mass disk is located, your rocket is out of fuel. The fuel figure can never be on the left side of the dry mass disk.
TIP: A rocket will often have a crew, but does not need one. It may also have more than one thruster, although it can only use one in any turn.
E5. FREIGHTER STACK CREATION, MOVEMENT AND DEFINITION
A Freighter Stack can only be created by an ET production operation (H8), which creates both the freighter cube and the black card as its cargo at the same time. When the freighter stack is created, place your big cube in the space where it was produced (either next to the factory, or next to the dirtside factory’s Bernal), and move the freshly built black card into the “Freighter Stack” slot.
·       Committed Cargo. Your freighter can only carry the one card that was produced at the same time that it was. Once this card is jettisoned (D1.5) or decommissioned (D1.6), return the cube to your reserves.
·       Movement. A freighter can move the turn it is produced. It moves one burn at a time (plus coasting per F6 and, in the advanced game, slingshots per K1 and Push Factories). Fuel expenditure is not tracked. It can land or lift-off per G1. Freighter movement is unchanged if using the Freighter Module (P).
E6. OUTPOST STACK CREATION
At the beginning or end of your movement, you can freely convert a stack into outpost \#1 or outpost \#2 (e.g. for a Rocket Stack that has broken down or is out of fuel). You can also drop off outposts by jettisoning them during movement per D1.5. In either case, move the cards into the Outpost Stack location on the playmat, and mark their map position with a yellow disk so they can be retrieved later. 
LIMIT: You can only have two outposts at the end of any turn. You must keep track of which Outpost Card Stack belongs to which yellow disk on the map.  You may however, have more than two outposts at any given time provided you start and end your turn with no more than two.
NOTE: Crew stranded in space can either be rescued (by yourself or another player if you agree, which allows the card in his stack without passing ownership), or left as an outpost. If you choose the latter, you won’t have crew available (since new volunteers are scarce if you are in the habit of abandoning them in space!). Crew normally can’t be voluntarily decommissioned.

Example 1: A big Rocket Stack arrives at Mercury L1, too heavy to land for a powered landing, even by decommissioning its primary thruster and switching to a 8•8 afterburning crew. The refinery becomes outpost \#1, allowing the crew \& robonaut to land for free. On its next turn, the crew lifts off to pick up the refinery, and the robonaut remains as surface outpost \#2. On their next turn, the crew \& refinery may land and build a factory.
 
Example 2 (advanced game): A solar flare knocks out the thruster card of your Rocket Stack. At the start of your turn, you choose to convert it into Outpost Stack \#1 and move all the surviving cards to this slot on your playmat - this allows you to form a Rocket Stack elsewhere, potentially as a rescue mission. You have the option to convert tanks of fuel into FFT, which are put on top of the stack.
F. ROCKET MOVEMENT PROCEDURE
Overview: To move a rocket figure on the map, choose a thruster, calculate net thrust, and move up to this many burns on the map. For each burn entered, expend fuel steps on the Fuel Strip equal to the thruster’s fuel consumption, following the solid black line.
No Stacking Limits. You may freely pass and share a space with all other units, both yours and your opponents’.
F1. ACTIVATE THRUSTER
Choose one card with a thruster triangle to be the activated thruster, and place this card on top of your Rocket Stack. An example of a The Card Anatomy for a Thrust Triangle is shown to the right:
Base Thrust: the first number shows how many Thrust Movement Points (unmodified TMPs) the rocket has. It is modified per F2.
Fuel Consumption: how many steps of fuel are consumed per burn entered.

Example: The NASA crew has a (terrible) fuel consumption of 8, representing its SSME chemical engines. It must move its fuel figure 8 steps to enter each burn!
NOTE: thrusters cannot be combined. Only one thruster can be used for a stack in a turn.
TIP: thrusters can be divided into two broad types – high thrust (chemical, nuclear) or high fuel efficiency (sails, electric rockets). Often, it makes sense to equip a rocket with both types and use the best each turn.
Afterburn: the flame icon shows steps of fuel expended to increase the net thrust (F2) by one.
Solar-Powered: the solar icon indicates the thruster’s net thrust is modified by the heliocentric zone (B4) that the rocket starts in.
Fuel Grade: The color of the thrust triangle indicates its minimum grade—dirt (black), water (blue), or (Colonization) isotope (yellow). Dirt is the lowest grade, and isotope the highest. A thruster with a blue triangle can accept either water or isotope fuel, and a black triangle dirt rocket can accept all three grades.
Pushable: The “▶)” icon indicates that the spacecraft may receive a Beamed Power Modifier bonus to its thrust (F2).
You cannot activate a thruster if it requires a fuel grade higher than your fuel grade (indicated by the color of the fuel figure, see E4). REMEMBER you may use free actions to change the fuel type in the rocket (K6).
COLONIZATION: isotope fuel (S2) is used only for GW thrusters.
F2. CALCULATE NET THRUST
Before the rocket figure is moved, but after expending fuel for both afterburning and lift-off (G), compute the net thrust to see how many thrust movement points (TMPs) your rocket has, and how big a world it can land on. To compute net thrust, start with the thruster’s Base Thrust (F1), then add or subtract the following modifiers:
Wet Mass Modifier. This modifier is determined by which column of the fuel strip the fuel figure is on. There are five columns, corresponding to the five wet mass classes: Wisp (+2), Probe (+1), Scout (+0), Transport (-1), and Tug (-2).
Solar Power Modifier. This modifier is listed on the map according to which heliocentric zone (B4) the rocket map figure starts in. It is only used if your thruster (or one or more of its supports if using Module O) has the solar icon on its triangle (F1).
 Beamed Power Modifier. If your spacecraft is pushable (F1), it may receive has the “▶)” u) icon it gets either a +1 modifier with power from a Powersat faction privilege, or (Colonization) a +2 modifier from a Push Factory on Mercury, Venus, or Io. (This can be from your own factory or part of a deal, per D3).
Afterburning. If the thruster has the flame icon and you decide to afterburn, reduce fuel the number of steps indicated for a +1 modifier.
COLONIZATION: Afterburning increases cooling as well as boosting thrust, see open cycle cooling.
NOTE: If using the Support Module (O2), see thrust-modifying supports.

Place a blue disk in the Acceleration Track on your playmat to mark your net thrust. Once set, this value remains constant for your entire move.
 Movement Requirements. To move, a rocket needs a thruster that is operational (O1) with a net thrust of at least 1. See F6 for exceptions.

Example: A rocket with a dry mass of 7 and a wet mass of 8½ is transport class (wet mass thrust modifier of -1). If its thruster has a base thrust of 1, its net thrust is 1 - 1 = 0. It can’t move with zero thrust, so it jettisons one step of fuel to bring it to scout class (wet mass 7½). This class has a wet mass modifier of 0, allowing it to move.
NOTE: Even if a rocket sheds mass (cards) during a turn, its net thrust depends on its state before its starts moving.
F3. THRUST MOVEMENT POINTS
Each rocket (including sails) has a number of thrust movement points (TMPs) equal to its net thrust (F2). During a rocket’s move, each burn and each pivot expends both TMPs and fuel (see next section).

Freighters move with a thrust of 1 and do not expend fuel. [If using the Freighter Module, in order to move, a freighter’s cargo is mass-limited by its load-limit (P4)].
Outposts cannot move.
F4. MOVE FIGURE AND EXPEND FUEL
A rocket figure moves from space to space along the routes (B4B2). It may choose to stop in any space it enters. It expends neither TMPs nor fuel to enter an intersection or space which is not a burn.
·       Burn (magenta circle): For each burn entered, move the fuel figure on your Fuel Strip to the left, following the solid black line, a number of fuel steps equal to the fuel consumption of the thruster. Round up any fractions at the end of the move. The fuel figure may never move to the left of the dry mass indicator.
·       Pivots. If in mid-move you turn on a Hohmann intersection, so as to move in a new direction, this is called a pivot. This maneuver costs 2 TMPs plus fuel equal to entering two burns. If you begin your move on a Hohmann, you may freely move in any direction, regardless of the direction moved last turn. Turning is free in Lagrange Points. <<<EXAMPLE XXX>>>>[BW7] 
·       No U-Turn. You can’t reverse direction during your move. However, if you halt on a space, on your next turn, you may move in any direction.
·       Jettisons. If you jettison cargo (D1.5) during your move, make a dry mass adjustment (D2) upon completing the move.
·       Landing. Moving into a site triggers a landing (G) and ends movement.

Example: An ion drive thruster (output 2 • ½) expends half a step for each burn entered. If it enters one burn in a turn, half a step is expended, which is rounded up to one step to the left.
NOTE: You cannot pick up items in mid-move.
TIP: In the early game, water is scarce and largely spent on boosting tech into orbit or fuelling ships. Later in the game, water becomes less valuable as boosting is replaced by ET production, and income is replaced by water obtained at sites.
F5. HAZARDS
As you move a stack, roll one die for each hazard space entered, both crash (the skull icon) and aerobrake (the parachute icon). If you roll a 1, the entire spacecraft stack is decommissioned!
·   	Failure is Not An Option. You may avoid this risk by paying 4 WT before rolling. (This simulates paying a tiger team of elite programmers back on Earth for a software upload.)
F6. COAST
After you have moved over the desired number of burns, you may coast (not entering any burns or making any pivots) until you either land or wish to stop.
You must stop to land if you enter a Site.
You need an operational thruster to coast. Exceptions are factory-assisted landings or lift-offs (G3), moving along an aerobrake path (B4), and the next bullet.
Dry Slingshot (Colonization). You may gain slingshot bonuses while coasting, and you may use slingshot bonuses (K1) even if your thruster stops working (see “Mission to Callisto” Example Q8).
G. LANDING \& LIFT-OFF MOVEMENT
Entering a site hex requires landing on a world, and leaving a site hex requires a lift-off. For a rocket, there are three types of landings and lift-offs: powered, aerobrake, and factory-assist.
·       Cards and figures that have already moved in a turn cannot move again (e.g. as cargo).
·       You may not use an aerobrake path when lifting off.
·       You may continue to move after lifting-off.
G1. POWERED LANDING OR LIFT-OFF
A powered descent or ascent allows you to land or lift-off of any site if your net thrust (F2) is greater than the site size (B4). (This means your rocket’s thrust must be greater than its weight on the surface). For the worlds without lander burns, this means you land on a site without expending any fuel (or expending an amount of fuel small enough to be neglected).
·       Freighters. The freighter base thrust of 1 is too low to make a powered landing or lift-off on even a size 1 site, unless its thrust is boosted (e.g. with a push factory). A freighter can land using aerobrakes (G2), or land or lift-off using factory assist (G3) or being towed (P4).
·       Aerodynamic Airship Exception. You may lift-off sites with aerostats (Venus, Saturn, Uranus, or Neptune) regardless of your net thrust (because with aerostat technology your rocket uses the dynamic and static lift of large low thrust Zeppelins to achieve orbital velocity).

Example: A crewed rocket with a net thrust of 3 moves next to Nysa (size 3). Since its thrust is too low to land, it halts and on its next turn switches its thrusters to its crew card, obtaining a net thrust greater than three for a powered landing. On its next turn, it reduces its dry mass (by dropping off a refinery and the crew on Nysa), which increases its net thrust to 4. It can blast-off without burning fuel using a powered lift-off.
 
G2. AEROBRAKE LANDING
Some sites have a path to the surface marked by a parachute. Following this path to land avoids needing a net thrust greater than the site size (G1), but you still expend fuel for any burns entered. As soon as a spacecraft enters an aerobrake hazard space parachute spot, it rolls 1d6 for an aerobrake hazard (F5).
G3. FACTORY-ASSIST LANDING AND LIFT-OFF
If the site is industrialized (i.e. contains a factory), your spacecraft may land or lift-off regardless of net thrust. However, this entails a risk equal to a crash hazard![1] (F5).
·       You may use a foreign factory to land or lift-off, with permission.
·       You are not allowed to make a factory-assist landing from a lander burn or a factory-assist lift-off to enter a lander burn (G4). 
FREIGHTER MODULE: A mobile factory cube can use itself as the factory for factory-assisted lift-off or landing (P6).
G4. LANDER BURNS
Lander burns are burns shaped like a lander vehicle, and are found next to planets and large moons to represent the delta-v required to land or lift-off them. You are not allowed to halt on a lander burn, and you can’t enter it if this causes you to spend more fuel or TMPs than you have, or causes you to make a powered landing with insufficient thrust.
·   Half Burns. The lander burns shaped like half a lander vehicle are treated exactly like lander burns except it costs only half your fuel consumption to enter, rounding up.

Example 1: You are moving to Ganymede (uruk sulcus) with a photon sail. Your sail is not allowed to enter the “geyser” lander burn because it has insufficient thrust for a soft landing. If you stop in the HEO immediately before this burn, and on your next turn activate an afterburning crew card with net thrust = 10, you may make a powered landing.

Example 2. Your mission is to take a mass 3 refinery from LEO to Mercury. You use the Space X 10•8 thruster crew, a 1•0 sail card, and 5 tanks of water as lander fuel to make a rocket stack with a dry mass of 5 and a wet mass of 10. But this is transport class, so your sailing net thrust is 0 and you can't move. So you add 6 more tanks of water, bringing the wet mass to 16.  During movement, first activate the 10•8 thruster to get through the first burn in the orange path. This lowers the fuel figure 8 steps to wet mass 9 ½, and the rocket is now in the Venus heliocentric zone, which gives a +1 thrust to sails. Now you can switch your thruster to the sail, sail over the next 4 years to just outside the lander burn of Mercury. Finally switch to your crew thruster again, and expend your last 10 steps of fuel for an afterburning landing.  The afterburn is necessary to bring the net thrust to 11, greater than Mercury’s size of 10.
 
H. BASIC GAME OPERATIONS
In the basic game, you may choose and perform one operation on your turn.
H1. INCOME OPERATION (engineering, science, or finance)
Take 2 WTs (water tanks) from the pool. (Represents boosting water into your home orbit from the Earth’s surface.)
COLONIZATION: The income operation can be performed by a crew or colonist of any specialization.
H2. RESEARCH OPERATION (science)
Select one patent card from the top of one of the available decks to auction. All players can examine both sides of this card (and see the next card in the deck). Starting with the phasing player (i.e. the player whose turn it is), players bid WTs for this card. Bids can start at 0 WT and must raise or see any previous bid. Bids can be placed in any order.
·    Hand Limit. A player with more than 3 hand cards (not counting starting cards) cannot bid in or initiate an auction. However, a player with the Skunkworks faction privilege can perform this operation regardless of his hand size.
·   Ties. The phasing player wins ties if his bid is tied with another. If two or more non-phasing players are tied, the phasing player decides between them.
·       The winner pays the phasing player the price, and takes gets this card into his hand. If the phasing player is the winner, he pays the pool and takesgets the card.
SUPPORT MODULE: The card chosen in an auction comes with supports (O3).
COLONIST MODULE: This is a science op, so it can be performed by a colonist with the science specialization.
H3. FREE MARKET OPERATION (finance)
Sell a white card in your hand to the bottom of its patent deck for 5 WTs. Alternatively, you can decommission (D1.6) a black card in LEO for WTs equal to the current VPs of the card’s product letter on the exploitation track.
NOTE: Your starting cards cannot be sold.
COLONIZATION: You can also sell black cards to a Bernal (Q9). Note that if you sell to your Bernal, the card is returned to the patent deck (i.e. you sold the idea), but if you sell to a foreign Bernal, the card is decommissioned (i.e. you sold a product).
H4. BOOST OPERATION (engineering)
Move one or more white cards (including crew) from your hand to your LEO Stack (or a stack in your home orbit) by paying water equal to its combined mass.
·       You may move a cardstack that was boosted earlier in the turn.
REMEMBER: The Black side of cards can only be produced in space factories and can’t be boosted from earth.
H5. SITE REFUEL OPERATIONS (engineering)
There are four site refuel ops: Factory, ISRU, Scooping, or Dirtside. All four extract fuel from a site and add it to your rocket or Bernal, moving the fuel figure on the Fuel Strip one step to the right (along the red dashed line) for each tank added. The maximum is a wet mass of 32.
H5.1. Factory Refuel Op. If the site is industrialized (and the factory owner is cooperating), each factory refuel op adds 20 dirt tanks, 8 water tanks, or {GW Thruster Module} one isotope tank. The isotope fuel obtained is the same as the spectral type of the factory.

Example: Your blue fuel figure is on the 7-1/2 position, and you perform a factory refuel. This moves the fuel figure 8 steps to the right, ending on the “15” spot.

H5.2. ISRU Refuel Op. If not using factory refuel, you can still refuel using a unit with an ISRU rating. The number of water or isotope tanks obtained by this operation equals one plus the site’s hydration (i.e. number of drops) minus the ISRU rating of the refueling unit. If refueling an operational rocket or Bernal with a black fuel figure (e.g. a dirt rocket), add up to 10 tanks of fuel, even if no ISRU unit is present. Alternatively, you may use an ISRU unit to load 10 black FFTs (dirt) into any spacecraft. 

Example: An unfueled Rocket Stack with an ISRU of 3 and a dry mass of 4 sits on Mercury (hydration = 3). By performing an ISRU refuel operation, it gains 1 + 3 – 3 = 1 water fuel tank, which lowers its net thrust by one. If a factory is present, it gains up to 8 tanks, lowering its net thrust by two.

H5.3. Atmospheric Scooping Op. Certain colonists and operational refineries, carried as cargo, may scoop and liquefy an atmosphere to use as a propellant. This op yields 8 WT of water. Unlike other site refueling ops, scooping must be performed in the same turn after the rocket or Bernal has spent its entire move on any aerobrake hazard (other than those around Earth, due to environmental concerns). 
H5.4. Dirtside Refuel Op {Bernal Module}. Refueling at a Bernal with a dirtside is exactly like the Factory Refuel Op (above), except that if refueling with isotope fuel, the refueling rate is one tank per dirtside of the correct spectral type.
Mixing Fuel Grades. Indicate the fuel grade loaded by the color of the fuel figure (E4). If you mix fuels in a rocket or Bernal, replace the fuel figure to a color equal to the lowest fuel grade used.
You can also transfer fuel to your rocket as a Free Action, see D1 (first two bullets) and K6.
NOTE: Sites never run out of fuel or water.
IMPORTANT: Unless you are allowed hijacking, you cannot factory refuel without the owner’s permission. You must instead ISRU refuel.
WARNING: Refueling often increases your wet mass class, resulting in a lower net thrust (F1).
H6. PROSPECT OPERATION (science)
Attempt to make a claim at an unprospected site. To prospect, choose a card at the site (or adjacent to the site, in the case of raygun prospecting) that has an ISRU less than or equal to the site hydration (i.e. number of drops). Then, roll a die. If the result is less than or equal to the site size, then the prospect is successful; place a claim disk. If unsuccessful, place a blue “busted” disk showing no one may prospect there again. (But rockets can still perform the site refuel operation (H5) there.) ISRU units with the “raygun” or “buggy” icons can perform special types of prospecting:
Raygun Prospecting. Able to prospect at range, prospecting all adjacent non-atmospheric sites.
Buggy Prospecting. If the unit fails a prospecting roll, youit may re-roll the die one time. Alternately, buggy units are able to prospect multiple sites along a road (i.e. dashed yellow line).
NOTE: The prospect procedure can be modified by raygun prospecting, buggy prospecting, or assaying smelters.
NOTE: If you run out of blue disks, use substitutes. If you run out of claim disks, see logistics.
NOTE: If your card has multiple platforms (e.g. raygun and missile), choose the platform used before making the roll.

Example: You land on the lunar polar rim with an ISRU = 1 buggy and prospect. Because the site size is 6 or higher, prospecting is automatically successful. The site is connected to another lunar site by a buggy road. Because this site is bone dry (hydration = 0), your buggy has insufficient ISRU to prospect or claim it.
H7. INDUSTRIALIZE OPERATION (engineering)
Place a factory cube on top of your claim disk by decommissioning a robonaut and a refinery card at thea prospected site. This lowers the exploitation track by one step in the column corresponding to the site’s spectral type. This factory can produce the black side of patent cards with this spectral type.
IMPORTANT: Any robonaut or refinery card, regardless of its product letter, can be decommissioned to Industrialize the site. Its product letter is only used when it is produced at a factory.
NOTE: If you are out of cubes, see logistics.

Example: The UN player decommissions a refinery and robonaut to build a factory on Luna (Crater Shackleton). He adds a purple cube and lowers the S Resource Exploitation disk.
SUPPORT MODULE: You must additionally decommission all supports other than radiators for both the robonaut and the refinery. If the supports are solar-powered, they cannot be used for industrializing in the Neptune heliocentric zone (or beyond).
H8. ET PRODUCTION OPERATION (engineering)
Produce extraterrestrial technology by choosing a black card from your hand with a product letter (B12) matching the spectral type at your factory, and playing it into an existing or new stack at the factory site.
The produced card goes into a rocket stack (E4), Freighter Stack (E5), Outpost Stack (E6), or Bernal Stack (Q2). This stack must be colocated with the factory which is doing the ET production. If you are at your stack limit and thus cannot create a stack at your factory, you cannot produce anything.
·       Freighter Production, see (E5). 
·       A card or freighter produced at a factory can move the turn it is produced.
BERNAL MODULE: If you perform an ET Production using a dirtside factory, the product appears either at the factory or in the Bernal Stack, as desired.

Example: The UN player from the previous example decides to build and ship his first lunar product. He plays his S black card into his Freighter Stack slot, and places a purple freighter cube next to his lunar claim disk. He can produce as many S cards as he wishes. Note that if playing with the Freighter Module, the card produced goes into an Outpost Stack and no freighter cube is produced.
H9. EXAMPLE MISSION TO LUNA AND BACK
The PRC (i.e. Player Red) has two mission objectives: to land a robonaut on Luna as a pedestal for a future factory, and to return the Taikonauts quickly.

Year 1: Launch of Lunar Mission. After years of research, the PRC pays 4 WT to boost his crew, a mirror steamer (output 3•4), and a cat fusion robonaut (ISRU = 2, mass = 3, assuming basic game masses). He puts a rocket figure in LEO, and a blue dry mass disk with a fuel figure on the number “4” spot. He spends 6 WT to add 6 tanks of fuel, moving the fuel figure to “10”. It is now “Transport Class”, with a net thrust of 2.

Year 1 (continued). Cis-Lunar Move. The Rocket Stack burns 4 steps of fuel to enter Earth HEO and coast to the Van Allen Belt (VAB radiation space). It is now wet mass 8.

Year 2 Lunar Landing. The PRC switches the active thruster to his crew (output 10•8) for the landing. He spends 8 fuel steps to enter the lunar lander burn, leaving him with 2/3 of a tank of fuel. His net thrust of 10 is greater than the size of Luna, so he makes a powered landing. During prospecting, the PRC robonaut claims Luna (prospecting is automatically successful since the most you can roll on 1d6 is “6”).

Year 3: ISRU Refueling. The rocket does not have enough fuel for a lift-off. So the robonaut spends a year digging up regolith and squeezing out water drop by drop, moving the fuel figure one step to the right (to 5-2/3). The stack is divided into an Outpost stack with the robonaut and steamer, and the Rocket Stack with just the crew. The dry mass disk and the fuel figure are moved three steps to the left, to “1” and “2-1/2” respectively.

Year 4: Ticker Tape Return. The crew card (dry mass = 1, wet mass = 2-1/2) is a “Probe Class” rocket, with a plus one thrust bonus. This is more than enough for a powered take-off. Eight steps of fuel are expended to enter the lunar burn, leaving 2 steps of fuel remaining. This is insufficient to enter the HEO burn, so the crew takes the aerobraking “shortcut” to LEO. Assuming their parachute opens properly, the crew is decommissioned in LEO.

Year 5: The Next Mission. The PRC plans its next mission, flying a refinery to Luna. Together with the robonaut already there, an Industrialize Operation creates an S Factory in Crater Shackleton.
COLONIZATION: A Radiation Risk (K2) must be rolled in years 1 (using a net thrust of 2) and in year 4 (using a net thrust of 11, so no danger). It can avoid the first risk by not coasting in year 1, then using the crew thruster to move through the radiation hazard and land. In Colonization, prospecting Luna is not allowed (L).
I. VENTURES \& GLORY
These achievements, if achieved before your opponents, give you victory points (VPs) at the end of the game. Use the orange side of the card if playing with the Endgame Module (U), and the yellow (basic) side otherwise. If you are the first to fulfill the stated conditions, claim the relevant card as a free action (D1.9D1) by placing it face-up near to you.
I1. GLORY
There are five glory cards, each two-sided.
You can claim two cards with one action, if both are unclaimed. For instance, bringing a human back from a moon of Neptune satisfies the conditions for both King of the Gods and Father Sky.
Glory is awarded for a site even if it is busted (B5).
If the card specifies returning humans to the Earth’s surface, they must travel the entire distance by spacecraft (e.g. no decommissions), and be decommissioned in LEO.
I2. VENTURES
There are four venture cards, each two-sided. Each confers 3 VP. The orange side of the card costs 5 WT to obtain, and confers a special ability as listed below:
·       Space Pharmacy Venture – You are first to claim three S sites. If you perform a research operation in a turn, you gain 1 WT profit, taken at the end of your turn.
·       Space 3D Printing Venture – You are the first to claim three M claims. Your ET Production and Digital Swap Ops each gain 1 WT profit, taken at the end of your turn.
·       Space Elevator Venture – You are first to claim four C sites: you are able to consider as colocated the spaces connected by the “space elevator” between LEO and L2 or between Phobos and the Caves of Mars. Your stacks adjacent to one end of the elevator are considered adjacent to the other end.

Example: Your Bernal on the burn between Phobos and Deimos has a combined hydration of 8 assuming you have the space elevator and dirtside factories on Phobos, Deimos, and the Arsia Mons Caves.

·       Space Tourism Venture – You are first to claim three V sites. Gain 1 WT profit, taken at the end of each of your non-war or non-anarchy turns.
J. BASIC GAME END \& VICTORY
J1. THE GAME ENDS WHEN…
After the specified number of Factories are built, the game ends at the end of the following year. So if the third player ends the game, the remaining players for the year get their turn, and then everyone gets one final turn the following year.
·       The number of factories needed to end the game are four (if 2 players), six (if 3 players), or seven (if 4+ players).
J2. VICTORY POINTS
The player with the most victory points (VPs) at the end of the game is the winner. In the case of a tie, the tied player with the most WTs is the winner. VPs are awarded for:
Each claim, factory, Bernal, or freighter (including mobile factories) (i.e. each disk or cube of your color on the map): 1VP. Each cylinder on the map of your color representing a space colony: 2 VP. If using the Bernal module, hemispheres also count for 1 VP if they are either an  Ersatz-Bernal or promoted with a dirtside.
Each factory: Extra VP as listed on the Exploitation Track (the exploitation disk should be adjusted to reflect the number of factories built for each spectral type - C, S, M, V, or D).
Each factory at a science site (i.e. with a microscope icon): an extra 2VP, or 4VP if a TNO science site (i.e. if the site contains the yellow star-microscope icon).
The VP listed on any Glory or Venture cards.
Colonization: If in power at the end of the game: 5 VP.
 Colonization: Each claim disk on a futures star (U1): VP as indicated.
NOTE: VP conditions stack. a factory scores for the cube and for the factory value, a science site scores for the claim and the science.

Endgame example: The red player ends with claims on the V worlds of Mercury, Vesta, and one of its moonlets. He also has a factory on Mercury, which is worth 8 VP on the Exploitation Track. He has 3 disks and 1 cube on the map, plus the Space Tourism venture (3 VP). His total is 8 + 3 + 1 + 3 = 15 VP.
K. COLONIZATION (Advanced Game)
The Advanced Game, called Colonization, introduces optional modules and an endgame for an expanded experience.
Space colonies orbiting Earth now employ hundreds and provide services such as antimatter manufacture, beamed solar energy, cycler satellites, space pharming, maglev bus service on launch loops, and of course, the local tax office. New Bernal, Colonist, Freighter, and GW Thruster modules allow you to move your base of operations further afield in the High Frontier. The promoted side of many of these cards includes new paths to victory called Futures.
K1. SLINGSHOTS \& MOON BOOSTS
A spacecraft that enters a flyby space can perform a slingshot maneuver, giving it a number of free non-lander burns up to the planet’s slingshot rating. These are used during the remainder of its move. You can use this bonus to continue to move even if the thruster is shut down, for instance by radiation.
As listed on the map, the Venus flyby bonus may only be used during the blue sector of the Sunspot Cycle.
Moon Boost. A space marked with the crescent moon icon is a special type of flyby called a moon boost. Entering it gives you one free burn for no fuel or TMP cost.

Example: You have a solar-powered Bernal which ceases to operate in the Ceres zone, yet wish to get to the Jupiter flyby. Your Bernal enters the Venus flyby, getting a bonus of two burns. On your way to Jupiter, you pass four burns on the green route, and spend fuel for the first two. You save your two free burns for the last two, since the final one is in the Ceres zone where your thruster is shut down. Upon entering the Jupiter flyby, you may now enter four more burns (or a combination of burns and pivots), again without expending TMPs or fuel.
K2. RADIATION HAZARD
Radiation Hazards surround five worlds (Sol, Earth, Jupiter, Saturn, and Uranus). When entering one of these spaces, find its radiation level by rolling 1d6 and subtracting the spacecraft net thrust (F2). All cards in the stack with a rad-hardness lower than this modified roll are decommissioned. If the thruster is thereby shut down, the spacecraft halts on the spot; see example xxx.[BW8] 

Example: The green route to Enceladus passes through 7 radiation belts.

·       Solar Active Year. If the sunspot disk (L) is in the red sector, add 2 to the radiation levels of all radiation hazards.
·       UN Cycler. If he has a Bernal in his home orbit, Player Purple may designate a spacecraft as immune to the radiation hazards surrounding Earth (the Van Allen Belts).

Example: A rocket with a net thrust of 2 moves from LEO to GEO, crossing the Van Allen belt. A 4 is rolled, so the radiation level is 4 – 2 = 2. The rocket’s solar panels (rad-hard = 1) are decommissioned, and without power, its electric thrusters stop working. Having failed to reach GEO, the stack may be left as an outpost, or else entirely decommissioned.
K3. COLONIZATION SPACE GOVERNMENT
A disk on the Sol Space Government diagram, called the Politics, indicates who is in power, and which policies (K4) are in effect.  
·       Polity. The seven spots on the Space Government diagram with a faction color are called Polities. If the Politics is on a Polity, that faction is in power.
K4. COLONIZATION POLICIES
Policies are special rules enforced depending on the position of the Politics. The policies are:
Anarchy. All players are allowed to commit felonies, and all players lose their faction privileges. The War Declaration faction privilege of Player Red allows him to move the Politics from anarchy into an adjacent war spot as a free action at the beginning of his turn. Per Q3, he must have a Bernal figure in his home orbit to perform a War Declaration.
War. All players are allowed felonies and combat. All players lose their faction privileges. During elections, colonists vote for their employer (N6).
Press Gangs. Player Red may hostile recruit (N5) any colonist, regardless of loyalty.
Egalitarianism. During an income operation (H1), you may take 2 WTs from a player who has more WTs than you, instead of from the pool. Blue and yellow FFTs (K6) count as WTs if in your home orbit.
Anti-nuke. You may not Boost or Free Market reactor cards or cards with on-board nuclear power (P5). You may discard them per D1.7 D1.
Nationalism. Only Player White may perform the income operation (H1).
Paleoconservatism. Except Player White, Research Ops (H2) only permitted at a lab.
Capitalism. During an income operation (H1), a player receives as many WTs as the number of factories he owns.
K5. COLONIZATION FREE MARKET
The Politics indicates how many WTs each player receives when selling a hand card in a free market operation. Superseding the basic game rule H3, whereby hand cards are always sold for 5 WTs, cards can be sold for a free market value ranging from 2 WTs to 6 WTs. A boosted white or purple card cannot be sold (unless, of course, it is first decommissioned).
Free Market Sales of Black Cards. A black card in LEO, including freighters and GW thrusters, can be sold for the price as specified on the exploitation track. It goes to the bottom of its patent deck (since you are selling the license to manufacture the product). A black card can also be sold to a promoted Bernal per Q9, in which case the sold card goes into your hand.
EXCEPTION: Cards with the legend “Free Market sale as hand card only” cannot be sold in LEO or at Bernals.

Example: A player has built a black card. He can move it to LEO or a Bernal and sell it for 6 to 10 WTs, or he can decommission it and sell it from his hand for 2 to 6 WTs.
 
K6. COLONIZATION FUNGIBLE FUEL TANK
WTs stored in a WT depot in your home orbit are treated as “cash on hand.” But once moved into a stack, they are called Fungible Fuel Tanks (FFTs). These are carried or stored as clear blue disks as part of a stack. This represents tanks of water or blocks of ice carried as cargo. Once it leaves your home orbit, an FFT cannot be used as cash (e.g. spent during auctions).
·       Each FFT is treated as a card with a mass of one and a rad-hardness of 8. It can be carried as cargo in a freighter, jettisoned (D1.5D1) to an outpost stack (E5), decommissioned, or used to fuel a rocket (E4). If carried by a rocket or Bernal, each one adds 1 to the dry mass.
·   	Fuel Grade FFT. Use different color disks to represent different fuel grades: black = dirt, blue = water, and yellow = isotope. Isotope fuel is of a particular spectral type (S2).
·       Each blue or yellow FFT in your home orbit is additionally treated as a WT for all purposes.
·       FFT Liquidation. An FFT on a rocket or Bernal can be converted to fuel steps by discarding its disk and moving the dry mass disk to the left one step on the Fuel Strip, following the dashed red line. If the fuel added is lower grade, swap the fuel figure with one of the new fuel grade.
·       Reverse-Liquidation. Fuel steps can be converted into FFT. This moves the dry mass disk to the left, following the dotted line on the Fuel Strip for a desired number of steps. Add one FFT to the rocket or Bernal stack for each step moved. It should match the color of the fuel figure.
·       Transfer of FFTs between colocated stacks, and liquidation of FFTs into fuel steps or vice versa, are Free Actions (D1.1D1).
K7. SUMMARY OF COLONIZATION MODULES
Colonization provides seven modules, i.e. sets of rules and cards that can be added or omitted to the advanced game.
·       Support Module. This starts three new patent decks: generators, reactors, and radiators. This module is required for any of the other modules.
·       Freighter Module. This starts a new freighter patent deck. You are now required to have a freighter card to produce a freighter. If promoted, it allows all your factories to move independently.
·       Bernal Module. This gives each player a starting card and a hemisphere figure representing an orbiting Bernal Sphere space colony. This gives you a higher home orbit that upgrades your boost capacity. If moved to another world, it allows additional colonists to be boosted.
·       Colonist Module. This patent deck allows you to recruit and boost hardy pioneers giving you additional abilities, votes, glitch protection, and specialized operations.
·       Gigawatt Thruster Module. This patent deck allows you to research GW Thrusters allowing rapid transport of colonists or Bernals to sites as distant as the TNO's (Trans-Neptunian Objects).
·       Combat Module. During War, players may attack each other using humans, robots, robonauts, and factories.
·       Endgame Module. The game ends when a specified number of Futures (U1) are accomplished by any player. Futures are listed on the promoted side of Freighters, TW Thrusters, and some Colonists. Each grants VP depending on who is the first, second, third, or fourth to accomplish one of them.
L. COLONIZATION SETUP
Colonization sets up per (C), plus these extra rules, depending on which modules are in play:
Place the Politics and Sunspot Disks. On the Solar System Charts, start one blue disk on the “Start” dot (center) in the Space Government, and another on the “Start” dot (uppermost) in the Sunspot Cycle.
Moon Treaty. Place a blue “busted” disk on both Luna sites (indicating they may not be prospected). The Mine Revival refinery ability is not permitted on these sites.[11]
(Support Module): Start with three extra patent decks: Generators, Reactors, and Radiators.
(Bernal-Colonist Module): Your hand starts empty, but your crew and Bernal cards start in a Bernal Stack (Q2). You also start with the hemispherical figure of your color in its home orbit (Q1). 
(Support, GW-Thruster, Freighter, and Colonist Modules): Decks for these card types brings the number of patent decks up to as much as nine. Stack the three support decks black side down, and the others purple-side down.
(Endgame Module): Place the 5 Glory Cards and the 4 Venture Cards  orange-side up instead of yellow-side up in a public area.
M. COLONIZATION PLAY SEQUENCE
The turns begin with the first player and proceeds clockwise. When the last player has taken his turn, he advances the disk on the Sunspot Cycle clockwise one spot. This round of turns, called a year, has ended.
·       As in the basic game, during your turn you take your operations, and move any or all of your spacecraft on the map in any order.
M1. MULTIPLE OPERATIONS
Your crew card can perform any single operation, regardless of whether it is in your hand or in space. You cannot lose or trade your crew, so you are guaranteed one operation in your turn. In addition to your crew’s operation, each of your boosted colonist cards enables you to perform one additional operation on your turn.
·       Extra Operations Using a Colonist. As soon as he is boosted, each colonist can perform one operation matching the specialization icon on the colonist card. If the card has more than one such icon, choose one for the turn. If it has none, it allows no extra operations.
·       Extra Operations Costs. On your turn, your first operation for each specialization (not counting your free crew operation) costs 1 WT, and each additional operation of that specialization costs an additional 1 WT (e.g. the second op of that specialization cost 2WTs, the third 3 WTs, and so on).
·       Location. To perform an operation, a colonist can be anywhere in space. Unless its ISRU ability is used, it does not have to be at the location where the operation is performed (remote-control operations are assumed).

Example: Player Purple holds two engineers and a scientist in his Rocket Stack. He may perform his crew operation (free), one engineering operation (1 WT), a science operation (1 WT), and a second engineering operation (2 WT).

·       Order of Operations. You can perform operations in any order before, after, or between your moves. You may perform the same operation multiple times using different crew or colonists. This includes operations using robonauts.
EXCEPTION: You are only allowed one card auction (Research or Recruit) per turn.
·       A robonaut can ISRU refuel (H5.2) twice in the same turn if tele-operated by both your crew and an engineering colonist. A robonaut can also prospect (H6) twice in the same turn if tele-operated by both your crew and a science colonist (for example, if it has moved after the first prospect and before the second prospect).
M2. MULTIPLE SPACECRAFT MOVES
Depending on the modules used, your spacecraft can include your rocket figure, your big cube, your promoted Bernal (Q8), and (if you have promoted your Freighter Card, see P6) all of your small cubes (corresponding to your rocket, freighter, Bernal stacks and mobile factories respectively). Each of these figures may take their movement once per turn only.
You may move a card the same turn it is boosted, digitally swapped, or ET Produced.
No card can be moved or digitally swapped twice in a turn.

Example: Player Red can perform 3 operations in his turn. He moves his freighter to land at a factory. Its cargo, a robonaut, performs an ISRU refuel for a GW Thruster that is also there, which can then lift-off. The factory produces a generator, which the refueled GW rocket can bring along.
M3. COLONIZATION EVENT ROLL
Roll one die on the Sol Event Table (M4) whenever the sunspot disk bypasses an event threshold (it’s labeled “event” on the Sunspot Cycle).
Thus, an event is rolled every other year.
M4. SOL EVENT TABLE
When called for by a threshold (M3) on the Sunspot Cycle, roll one die on the table on the last page of the Rules. Each event potentially impacts all players.
1–2	 Inspiration. Put the topmost card of each patent deck (including the colonist deck) on the bottom.
3	 Glitch. Each player decommissions his heaviest card NOT colocated with one of his humans (including Bernals) or cubes (big or small). If multiple cards are equally heavy, choose one.
4	Pad Explosion/Space Debris. Each player decommissions his heaviest non-Bernal card in either LEO or in stacks colocated with his unpromoted Bernal. If multiple cards are equally heavy, choose one.  
5–6	Election Year, Budget Cuts, or Solar Flare. The event depends on what color sector the sunspot disk is in; see the Sunspot Cycle.
Blue Sector: Election YearAuction. This event generates an election auction (N6).
Yellow Sector: Budget Cuts. Each player discards a hand card (if he has any) to the bottom of its corresponding deck. You can’t discard starting cards.
Red Sector: Solar Flare/Coronal Mass Ejection. Roll 1d6 for radiation level, which impacts all stacks on the map. The radiation level is locally modified by adding the heliocentric zone modifier (B4), and the result is compared to the rad-hardness of all cards in each stack. Any cards that have a rad-hardness less than the modified roll are decommissioned.
EXCEPTION: Stacks in LEO, HEO (near Earth), or any Radiation Hazard are immune to solar flares (they are shielded by the magnetic field). Stacks on sites are also immune (they can quickly bury themselves).
N. COLONIZATION OPERATIONS
These operations are additional to those in (H).
N1. DIGITAL SWAP OPERATION (engineering)
This operation allows you to decommission one or more white or black cards from a stack and then replace the card(s) with one black-side card from your hand (which can also be any one of the cards just decommissioned). This operation may only be used with the Freighter Module (P).
3D Printers: Any card being swapped must be colocated with your big cube or mobile factory (P6).
Product Letter. The cards being swapped must have the same product letter—C, M, S, V, or D—as the spectral type of your Freighter Card.
Limits. You cannot digital swap using the HIIPER Beam-Rider Freighter (P3). You cannot digital swap your Freighter Card itself.
Conservation of Mass. The stack must not end up with more mass than before (i.e. the combined mass of the decommissioned stack card(s) must be greater than or equal to the mass of the black-side card replacing it).
Dry Mass. If your Rocket Stack ends up with less mass (e.g. if it carries a mobile factory and swaps with a lighter card), make a dry mass adjustment (D2).
Like all production, a card produced by a digital swap can be moved and used the turn it is produced.
Because consciousness is not fungible, neither humans nor emancipated robots can be digitally swapped.

Example: Player Red has a Fission GCR Freighter ("M" type freighter) carrying the Flux-Pinned Superthermal Radiator ("M" type, mass 1). He performs a digital swap operation, decommissioning the radiator, and replacing it with the Electrophoretic Sandworm ("M" type, Mass 1) from his hand.

Technical Note: During a digital swap, the instructions for printing the part are sent to a 3D printer, which prints the part up layer by layer, using its supply of “ink.” The ink is manufactured by melting down and separating the elements of the card being replaced. This is much faster than ET Production, which requires you to dig, extract, and transport ores that are then purified by beneficiation and other complicated processes.
N2. PROMOTE OPERATION (science)
Besides a white side (i.e. boosted from Earth), or a black side (i.e. built in space), some cards have a purple side. A card is turned to its advanced purple side only by promotion. Promotion has no effect on a card without a purple side.
·       In order to promote a Bernal Card to its purple side, see Q6.
·       In order to promote a Freighter, Colonist, or GW Thruster Card to its purple side, you must bring it to a Lab and perform a Promote Operation. This flips the card over. The spectral type of the Lab does not matter.
NOTE: You can promote at an opponent’s Lab with his permission, or as a felony if it is undefended by humans.

Example: Player White sends Refugee Colonists to his factory on Huya (a TNO science site). He then uses his crew operation to promote them.
N3. SUFFRAGE OPERATION (science)
To perform this operation, you must be in power, and at least one robot colonist must be in play (by any player). If so, roll 1d6. If the result is less than the number of cards in play on their purple side (including Freighters, Bernals, Colonists, and GW Thrusters), you become the Robot Emancipator. No other player can become the Emancipator for the rest of the game.
·       Emancipation Effects. If emancipated, robots are treated exactly as humans, except that they still do not count against your Maximum Human Colonists In Space (R4). They prevent glitches and, in peacetime, have loyalty to the Emancipator. Also see Peace Loyalty (N5).
N4. RECRUIT OPERATION (finance)
If your hand size is not more than 3 cards (not counting starting cards), this operation allows you to remove the top three cards from the Colonist Deck (revealing the fourth), and select one of the three. The other two are returned to the bottom of the deck. The chosen card is then auctioned (H2).
·       A player with the Skunkworks faction privilege can perform this operation regardless of his hand size.
N5. HOSTILE RECRUIT OPERATION (finance)
This operation allows you to woo colonists owned by other players during peacetime.[2] The targeted colonist is moved to a colocated stack of yours containing at least one card.
Max Colonists. You must have enough dirtside hydration (Q5) to support the targeted colonist (R4).
Colocation. The target is an opponent’s boosted colonist that is colocated with one of your stacks.
No hostile recruiting is allowed during War.
Peace Loyalty. The targeted colonist must have loyalty to you. However, any colonist held by the Robot Emancipator (N3) can be hostile recruited, regardless of loyalty, as long as it is not loyal to the Robot Emancipator. For instance, if Player Green is the Emancipator, all of his colonists are vulnerable to hostile recruiting, except robots and those with green election ballot icons.

Example: Player Purple moves his freighter to Player White’s Bernal containing the Vatican Observers (owned by White but loyal to Purple). Purple pays 1 WT to perform a hostile recruit with one of his financier colonists (who can be anywhere in space), stealing the Observers, who are transferred to his freighter.
N6. ACTIVISM \& ELECTIONS (finance)
If the Sunspot Cycle is in the blue sector, use this operation to initiate an Election Auction. This is held per H2, except the auction is blind, all bids are paid to the pool regardless of outcome, and ties are decided by the faction in power.
·       Blind Auction. To perform a blind auction, players make an outstretched fist containing their secret bid of WTs, and simultaneously reveal their bid. Each WT = 1 vote.
·       Maximum Ballot Stuffing. Your secret bid cannot be greater than 10 WTs.
·       Votes (War). During war, each of your colonists you own in space gives you one vote per ballot box icon (R3). Robots cannot vote unless emancipated.
·       Votes (Peace). Except during war, colonists in space with an election button of your faction color add to your election bid, regardless of ownership of the Colonist Card. Robots vote for the Emancipator (N3), if any.
·       Ties. If two or more winning bids are tied, the player in power decides. If nobody is in power during an election tie, then the Politics remains unchanged.
·       The player with the most votes may move the Politics by one spot.

Example: An election occurs in peacetime, and both players reveal their bids. Player Orange spends 1 WT for one vote, and Player Red spends nothing but receives one vote from the Juiced Cosmonauts owned by Player Orange (who vote for Red because they have a red election button with one ballot box). The tied vote is decided by the player in power, and if no one is in power, Orange's bid would be lost.
N7. ANTI-TRUST OPERATION (finance)
To perform this operation, you must be in power, and must have a victim with two or more hand cards of the same category (thruster, GW thruster, robonaut, refinery, generator, radiator, reactor, freighter, colonist). If so, your victim must choose one of his hand cards in any of the exceeded categories to surrender to your hand. It does not matter how many cards are in your hand.

Example: Player White is in power and performs an Anti-Trust Operation on Player Red, who has two radiator cards in his hand. Player Red chooses a radiator to surrender to Player White's hand.
O. SUPPORT MODULE
Summary. This module is required for all of the other modules. It adds three new patent decks: generators, reactors, and radiators. These three new card types are collectively called support cards.
·       Advanced Game Mass. When using the support module, use the mass and other data in the top upper right corner of all patent cards (in the red field) rather than the Basic Game Mass in the top left corner (B1).
O1. SUPPORTS
If a card lists a support card in its red data field (upper right corner), that card is non-operational unless the listed support is included in the stack. The support cards themselves also will often need supports. If you have multiple cards that can act as a support, only one can be chosen.
·       Non-Operational. A card without its operational supports cannot be used for movement, prospecting, refueling, etc.
·       Industrialization. To build a factory, you need to decommission a robonaut and refinery, along with their supports (and supports of those supports). You need only decommission one of each kind of support (e.g. if both the robonaut and the refinery need a generator, decommissioning just one will power the resulting factory). Exception: You do not need any radiators during industrialization (because the nightside of the world provides all the cooling you need).
·   	Sharing Supports. Reactors and generators can service any number of systems in a stack. Radiators can similarly be shared, up to the number of therms it provides (O5). However, radiator therms used during movement (e.g. cooling thrusters) can only be shared with those used for operations (e.g. robonaut prospecting).

Example: A Rocket Stack contains a refinery, the generator needed by the refinery, and a robonaut that requires a radiator. The site is industrialized by decommissioning the refinery, robonaut, and generator, even though no radiators are present.
O2. SUPPORT TYPES
In order to be used as a support, the support card must be the type specified:
·   	Generator Types. Electric ⓔ or pulsed .
·   	Reactor Types. Neutron ⓝ, burst 💣, or exotic ⓧ.
·   	Radiators. 1–4 therms , see O5.
·   	Thrust modifying, and fuel economy supports. These are cumulative and modify a rocket’s TMPs or fuel consumption, if they are needed to support its thruster. However, GW thrusters (S) are not affected by these supports (see next bullet).
·   	Solar-powered modifiers. These modify a rocket’s TMPs per (F2), but unlike other thrust-modifying cards, they do not stack and do modify the thrust of GW thrusters that use them.
O3. SUPPORTS DURING RESEARCH
During Research (H2) with the support module, if an auctioned card lists supports, the winner is also awarded the top card from each category deck (generator, reactor, or radiator) listed as a support. These extra cards are free.

Example: The bid for Player Purple wins the cermet NERVA thruster in an auction. This card lists a support: a reactor ⓝ. Player Purple takes the top reactor card. Unfortunately, this is a reactor ⓧ, which doesn’t support the cermet NERVA. So he sells the reactor on the free market.
O4. SPECIAL SUPPORTS
·       Heavy or Light Radiators. Each radiator card lists one mass on one end of the card, and a heavier mass on the opposite end. During boosting (H4) or ET production (H8), choose one of these two orientations to play the card to your playmat. Once boosted, a heavy radiator can be freely reoriented into its light version, but not vice versa. Reorientation causes a dry mass adjustment (D2). If a heavy radiator is involuntarily decommissioned, re-orient it into its light version instead.
·       Marx Capacitor Bank Generator. This “pulse” generator requires an “electric” generator as a support. If researched, remember to draw the generator card beneath it as well (O3).
·   	Magnetocaloric Refrigerator Radiator. This radiator also needs an “electric” generator, and as noted, it can cool its own supports. If these supports are used by the Magnetocaloric Refrigerator only, and not by the thruster, they cannot be used as a thrust-modifying support.
O5. OVERHEATING
Some cards indicate a number of therms (the “thermometer” icon) of radiator cooling required to keep from overheating. For instance, if the thruster card and its supports in a Rocket Stack together need 3 therms of cooling, you will need one or more radiators that add up to at least 3 therms.
·       Open-Cycle Cooling. You can use afterburning to satisfy 1 therm of cooling for any component in your thruster chain during a move or operation.
IMPORTANT: Heat rejection is necessary only if the card is in use. If a thruster system does not move, or a robonaut system does no prospecting or refueling, then it does not need radiators that turn.

Example: The Free Electron Laser robonaut needs the two supports shown. For its generator, it carries the In-Core Thermionic, which itself needs a reactor (either 💣 or ⓝ), plus another 3 therms of radiators. The Pebble Bed Fission reactor is added to the stack, plus a heavy Ti/K heat pipe, and a heavy bubble membrane (each able to reject two therms of heat). The complete robonaut stack has these five cards: 1 robonaut, 1 generator, 1 reactor, and 2 radiators. Its dry mass is 8.

Engineering Note: Open-Cycle Cooling, i.e. dumping water coolant into your nozzle, has three advantages (1) It increases thermal efficiency, which increases the percentage of power available for thrust. This is because the coolant captures power that would have been lost as waste heat. (2) Not only is thrust power increased, but the additional mass flow also increases the thrust. Doubling the kg/sec out the nozzle doubles the thrust (+1 thrust in High Frontier scale). (3) Less waste heat means fewer radiators are needed to cool the rocket.  The downside is that your specific impulse goes way down, a fancy way of saying that instead of expelling a white-hot trickle, you are shoveling out a lukewarm waterfall . Since mass is far more precious to a rocket than energy, you don’t want to do this often. Furthermore, water coolant just can’t intercept the types of energy emitted by some reactions. For a matter-antimatter reaction for instance, you need tungsten coolant instead of water.
P. FREIGHTER MODULE
Summary. When using this module, freighters are ET produced separately from their cargo no longer produced with the same op that produces factory products. Instead, research freighters form their own patent deck, and must be researched before being built with an ET production operation according to their product letter.
·       Dedicated Card. Producing a Freighter Card generates a Freighter Stack represented on the map by your big (10mm) cube. Just as in the basic game, freighters do not track fuel and move 1 burn per turn. The freighter can carry card(s) of any type up to a mass equal to its listed load limit.
·       Promotion. Promoting your Freighter Card allows all of your cubes to also be freighters, henceforth called mobile factories, and gives you access to Futures VP. A promoted Freighter Card cannot be decommissioned while you have any factories remaining.
P1. FREIGHTER CARD SETUP
Freighter Cards are black on one side, and purple on the other. They form their own patent deck--the Freighter Deck—and are researched and ET produced (on their black side) like any black support card.
P2. FREIGHTER RULES COMPARED TO BASIC
If not using this module, then freighters are exactly as in the basic game. If using the Freighter Module, these rules completely replace the Basic Game rules:
·       Freighter Deck. Freighters are researched from the Freighter Deck.
·       ET Production. If using this module, you must make a separate ET production operation (H8) to build the Freighter Card (on its black side, see P3.2). (If not using this module, freighters are produced per E5)
·       Production Limit. You may only have one Freighter Card built at a time, and only one Freighter Stack containing this card.
·       Additional Glitch Protection. A stack colocated with the big cube freighter or a mobile factory (P6) is immune to the glitch event (M4.3). (Thematically, a Smelter-3D printer recycles old parts into new ones.)
·       Cargo Transfer \& Jettison. Freighters can exist without their cargo and can jettison and transfer cargo without requiring a dry mass adjustment. Cargo can be transferred to the freighter as long as its Load Limit is respected and the Factory Loading Only rules are followed (P4). The freighter card is a dedicated card and cannot be transferred or jettisoned from the freighter stack.
·      Decommission. See P7.
P3. BUILDING A FREIGHTER
Freighter Big Cube. Use the ET production operation (H8) to build a Freighter Card and place the big cube of your color on the factory site. This is a freighter of the mass, rad-hardness, and load limit as specified on the Freighter Card.
·       Product Letter. The product letter of the Freighter Card must match the spectral type of the factory site where it is built.
·       The product letter of your Freighter Card determines the type of digital swap (N1) it can facilitate.
·       The HIIPER Beam-Rider Freighter (and its promoted side) has no product letter and can only be built at a Push Factory. As noted, the HIIPER can be sold on the Free Market only as a hand card, not after it is built.
·       As the final step, move the Freighter Card from your hand to the slot for your Freighter Stack (E5).
P4. FREIGHTER MOVEMENT AND LOAD LIMIT
Fuel is not burned or tracked on freighters, but otherwise it moves as a rocket with a base thrust of one.
·       Pushed Freighters. If the freighter has the push symbol, add 2 to your freighter’s net thrust for a Push-Factory, or one to net thrust for the Powersat faction privilege. These are not cumulative.
·       Load Limit. The Freighter Card specifies a load limit, indicating how much mass the freighter can carry, not counting the mass of the freighter card itself.
·       Factory Loading Only. A freighter with the cargo ship icon (labeled “Factory Loading Only”) is not allowed to add any cargo anywhere except at a factory or a promoted Bernal (but can drop off cargo anywhere). Furthermore, it is never allowed to add cargo in LEO.
·       Cruiser Bonus. Some freighters (i.e. Z-pinch, fission fragment and HIIPER beam-rider) receive a number of Pivots as a free movement bonus, as specified in the cruiser icon on the card.
·   Towing Freighters. If a Rocket or Bernal Stack carries a cube (either the big cube or a mobile factory cube), its mass (including cargo if any) adds to the dry mass of the rocket or Bernal. The Freighter Stack cannot itself move while being towed, instead moving with the stack towing it. For purposes of supports and combat only, both it and the stack towing it are treated as one stack.

Example: The magnetic mirror freighter gets a bonus of +3 Free Pivots.  Its Freighter Stack is at the termination shock point en route to Sedna. It moves through the next 3 Hohmanns for free, then coasts past the heliopause (and the Voyager Easter Egg) to stop at the fourth pivot just beyond.
P5. ON-BOARD NUCLEAR REACTORS \& GENERATORS
Certain Freighter and Colonist Cards contain the “On-board Nuclear Reactor” ability and/or the “On-board Nuclear Generator.” The former signifies the card can be used as a reactor card, either neutronic ⓝ, burst plasma 💣, or exotic ⓧ, as specified. The latter signifies that the card can be used as a generator, either pulsed  or electric ⓔ, as specified.
·   	A card with both an on-board generator and reactor (e.g. the Rotary Dirt Launcher) can be used as both types and can simultaneously support two pieces of hardware.
P6. FREIGHTER PROMOTE \& MOBILE FACTORIES
To flip a built Freighter Card to its purple side, bring it to a Lab and perform a Promote Operation. Promotion is normally permanent (see P7 for exceptions). Once promoted, your Freighter Card converts all of your cubes (big and small) into mobile factories. Beginning with the turn they are promoted, each may move as a freighter using the specifications on your Freighter Card.
·   	Abilities. Your small cubes now have all the characteristics listed in the thruster triangle and the red field in the upper right corner of the freighter card, including mass (relevant if the cube is being carried as cargo), rad-hard, and free pivots. Your small cubes are not allowed to carry cargo or use any other characteristics of the big cube (on-board generators, futures, starship weapons, etc.).
·       Mobile Factory Landing \& Lift-Off. To land or lift off a claim without a factory, a mobile factory cube may use factory-assist (G3), using itself as the factory. When landing a mobile factory on a site with an unindustrialized claim, place its cube on top of the claim disk. When landing your mobile factory at an industrialized site or a foreign claim, place it next to the claim disk to indicate that it is not part of the current claim.
·   	Factory Establish or Abandon. Your mobile factory is a factory only if it sits on top of your claim disk. Moving a cube on or off of your own unindustrialized claim immediately establishes or abandons a factory, adjusting the relevant Exploitation Track a step (to a maximum of 10 VP or a minimum of 6 VP). You may land your mobile factory next to but not on top of an opponent’s claim.
·   	Space Colonies. Mobile factories are not allowed to lift off or abandon a claim or be decommissioned or be used for a Big Cube Swap (P8) if they are on a claim disk that has any space colony cylinders.
·       Mass. All of your cubes have a mass equal to your Freighter Card Mass. This is relevant if the cube is taken as cargo in a rocket or freighter.
·   	Small Cube Creation. You may create a mobile factory cube by performing an industrialize operation and decommissioning an operational robonaut and refinery. Mobile factory creation is the same as factory industrialization, e.g. it must be created on one of your claims, does not need radiators, etc. If it is created on an industrialized site, place it next to the claim disk to show it is not part of the claim. Unlike industrialization of static factories, you may industrialize mobile factories in your promoted Bernal at a dirtside or in its home orbit.
P7. FREIGHTER DECOMMISSION
If a small cube is decommissioned (e.g. by combat, radiation, flares, landings, or lift-offs), return it to your reserves (you can build a new one per P6). If your big cube or freighter card is decommissioned, convert any surviving cargo excluding the freighter card into an Outpost Stack. 
Promoted. If your Freighter Card is promoted, and you have any mobile factories (small cubes) remaining, replace one of them with the big cube. Keep the freighter card in your freighter stack.
Unpromoted. If the card is not yet promoted, or if it is promoted but you have no remaining mobile factories in play, return the big cube to your reserves and return the Freighter Card to your Hand (it can be rebuilt per P3).
NOTE: A promoted Freighter Card is never returned to your Hand, either voluntarily or involuntarily, unless your big cube is decommissioned and at a time when you have no remaining mobile factories.
P8. FREIGHTER BIG CUBE SWAP
As a free action (i.e. one not requiring an operation), when your freighter is not carrying any cards you can swap its Big Cube with any mobile factory that has not moved this turn. Your Big Cube can then take its movement as part of the Big Cube Swap free action.

Technical Note: During a Big Cube Swap, your mobile factory uses its 3D printer to add a cargo hold to its chassis.
Q. BERNAL MODULE
“Oh give me a locus where the gravitons focus, where the 3-body problem is solved, where microwaves play down at three degrees K, and the cold virus never evolved.” Home on Lagrange (The L5 Song), Higgins and Gehm.

Summary: Your faction’s Crew Card and Bernal Card start boosted in your Bernal Stack, and your Bernal figure starts in a home orbit at a faction-specific Lagrange point. Your boost operations are upgraded so you can boost directly to your Bernal for the cost of boosting to LEO. Your WTs are stored at your Bernal if it’s still in its home orbit. If it leaves, your WTs are stored either in LEO or an Ersatz-Bernal. Promote your Bernal using a working generator. This protects the cards from space debris, gives you an extra privilege, and allows the Bernal to self-propel as a dirt rocket. Moving it to a space adjacent to one or more factories allows boosting additional human colonists at a card limit equal to half the combined hydration of those factories.
NOTE: Using this module adds two more stacks to the basic five allowed in the base game (E)0: your Bernal Stack (Q2) and your Ersatz-Bernal Stack (Q4).
Q1. SETUP OF BERNAL STACK AND FIGURE
Bernal Cards are white on one side and purple on the other. They are marked in the five faction colors, and one is given to each player along with his starting crew card.
·       Bernal Stack. Your Bernal Card and crew card start boosted in a special stack called the Bernal Stack (Q2). Note that both these starting cards are human.
·       Bernal Figure. Place a Bernal figure of your color into your home orbit on the map. This half-sphere represents an O’Neill-style orbiting colony used to support your colonists in space.
Q2. BERNAL STACK
The stack of cards with the Bernal Card (supports, visiting colonists, etc.) is your Bernal Stack. Your Bernal figure indicates its map location.
·       Event Vulnerability. A Bernal Card is indestructible, has no rad-hardness, and no involuntary decommission instances can be applied to them. Furthermore, if your Bernal is promoted (Q6) and at a dirtside or home orbit, all colocated cards except radiators in active use are immune to radiation hazards, events, and solar flares.
·       Hazards and Combat. A Bernal Card cannot be decommissioned or captured. If it would be destroyedcrashes (by a crash hazard, epic hazard, factory-assisted lift-off, etc.), the other cards in the Bernal Stack are decommissioned, but the Bernal Card remains in the accident spot. If it was promoted, demote it (but if promoted it unpromotes). In combat, the other cards in a Bernal Stack can attack and be attacked as if in a Rocket Stack.
·       Boost Enhancement. If you have a Bernal figure in your home orbit, your boost capacities are enhanced, so that a Boost Operation (H4) gives you the option of boosting any white card or cards to either your home orbit or LEO. You can even boost partly to LEO, and partly to your Bernal in a single operation.[3]
·       Van Allen BeltsVAB. If a boost goes through a Radiation Hazard around Earth (i.e. the Van Allen Belts), the net thrust is considered to be three. This cannot be modified.

Example: Player Orange boosts three cards to his Bernal in L2. He rolls a 6 for his radiation roll going through the VAB. The Sunspot Cycle is in the red sector, so two is added to this roll. With a net thrust of three, the radiation level is 6 + 2 - 3 = 5. One of the cards boosted has a radiation level of four, and is returned to his hand.

·       Bernal WT Storage. If you have a Bernal figure in your home orbit, your WTs are stored there. Otherwise, they are stored in LEO.
Q3. MOVING YOUR BERNAL
Your Bernal Stack can be moved per Q8. If so, it has the following consequences:
·       Moving To A Dirtside. As soon as your Bernal moves into a new orbit with at least one dirtside, this enables dirtside refueling (H5.4), increased VP (J2), increased humans in space (R4)., and Bernal Free Market (Q9).
·       You may use rayguns in a Bernal to prospect adjacent sites.
·       Privilege Loss. Moving your Bernal out of your its home orbit shuts down your faction privilege, Boost Enhancement (Q2), and Bernal WT Storage (Q2), but not your Bernal Privilege (Q7).

Example: Player White uses his GW rocket to move his white Bernal figure to Ganymede. He stops collecting launch fees and must boost to LEO instead of L5.
Q4. ERSATZ-BERNAL
If your home orbit has no Bernal figure, as a Boost Operation (H4) you can pay 10 WTs and boost a replacement Bernal there. Indicate this replacement, called an Ersatz-Bernal, using a Bernal figure of any available color in your home orbit.
·       The Ersatz-Bernal is always promoted, regardless of the status of your first Bernal. It functions as the original one did (except it can’t be moved and needs no supports). It restores your Faction Privileges, you can boost to its location, and you store your WTs there.
·       Ersatz-Bernal Stack. Assuming they survive the VAB Radiation Hazard, cards boosted at the same time or after your Ersatz-Bernal is boosted go into the Ersatz-Bernal stack on your playmat.
Q5. BERNAL DIRTSIDE HYDRATION
The sum of the waterdrops in all the dirtsides (adjacent factories) of a Bernal is called its dirtside hydration. You can have half as many Human Colonist Cards in space (R4) as your dirtside hydration (round down).

Example: Your Bernal in Mars LMO has a dirtside hydration = 7 if you have industrialized both the buried glaciers and the caves of Mars, allowing you 3 human colonists. The third factory is not adjacent.

·       Bernal Products and Water. If you perform an ET production operation (H8) using a dirtside factory, the product appears either at the factory or in the Bernal Stack, as desired. See Dirtside Refueling (H5.4) for refueling at Bernals with dirtsides. (Thematically, dirtside refineries catapult rawstuffs to be zero-gee processed in shirtsleeve conditions into final space products or isotope fuels.)
·       Hazards. During ET Production, ignore crash hazards between the Bernal and its dirtside, and radiation hazards on the Bernal location.

Science Note: There are four reasons for locating a Bernal in a Lagrange point of one or more dirtside factories: (1) Rawstuffs and water can readily be electromagnetically catapulted from the surface to the Bernal for processing and assembly by humans in shirtsleeve conditions. (2) The freighter traffic between the LaGrange and cislunar space can be conducted with low thrust rockets. (3) Humans can control dirtside robonauts without the problematic time lag that teleoperation from earth would entail. (4) Production of the final space product can be performed at zero-gee, allowing unique materials and processes not possible dirtside (or back at Earth). A Bernal at L5 has been the goal of the L5 Society, of which I have been a member since its founding.
Q6. PROMOTING A BERNAL CARD
If your Bernal Stack contains a working generator and is at its home orbit or has a dirtside, you may flip your Bernal Card to its purple side by performing a Promote Operation (N2). For the Orange Bernal, you need a reactor (any type) instead of a generator.
·       Demotion. A Bernal remains promoted even if it moves away from its home orbit or dirtside. However, if a Bernal loses its supports (e.g. its radiators are decommissioned by an event or moved to a different stack), it becomes unpromoted. Flip the card back to its white side.
·   	Solar-Powered Bernals. Solar-powered supports for a promoted Bernal do not work in the Ceres zone or beyond, forcing the Bernal to unpromote. If you (or a cooperating player) have a Push Factory, then your Bernal with solar-powered supports unpromotes in the Uranus zone or beyond.

Technical note: The unpromoted Bernal is a dumbbell-shaped colony with a capacity of 2000 souls (if supplied by a dirtside) or 50 (traveling). It has two spheres, 67m in diameter, separated by a 334m connection. The dumbbell rotates at 1.9 rpm for 0.85 gees of artificial gravity. Assuming ½ atmospheric pressure, the structural mass is 10 mass points. Upon reaching its destination, the dumbbell Bernal is replenishes its air (mass = 5) and its radiation shield (dirt and water in a honeycomb of nanofibers, mass = 3500), using local ISRU materials. Once promoted, a Bernal metamorphoses from a dumbbell into a 250m sphere with a capacity of 10,000 souls. The structure is 453 mass points, the atmosphere is 250 mass points, and the shield is 2320 mass points.  It rotates at 3-rpm for 0.7-gees.
Q7. BERNAL PRIVILEGES
For as long as it remains promoted, a Bernal gives you the following Bernal Privileges. These privileges do not stack if you have a Bernal and an Ersatz-Bernal:
·       Extra Colonists (all factions). A promoted Bernal allows you additional human colonists in space (R4).
·       Event Protection (all factions). All cards in the promoted Bernal Stack, except radiators in active use, are immune to events and solar flares.
·       NASA Science. Any of your human colonists in the promoted Bernal can perform a Research Operation. You are still limited to one Research per turn per M1.
·       Shimizu Marketplace. Any of your human colonists in the promoted Bernal can perform a Free Market Operation.
·       ESA Collimator. The promoted People’s Bernal is allowed to have functioning solar-powered supports all the way to the Neptune zone, as an exception to Q6.
·       PRC Antimatter Production. The promoted Military Bernal gives Player Red one free pivot (P4) for each of his rockets and freighters that begins its move on the Military Bernal.
·       U.N. Diplomatic Immunity. The promoted Political Bernal allows gives Player Purple the additional privilege to ignore the effects of an opponent’s policy (K4).
NOTE: NASA, Shimizu, and PRC can use both Bernals for their bonus operations, while the ESA cannot use its privilege on the Ersatz Bernal as it cannot move from its home orbit.

Example: The current polity is Egalitarianism. As long as Player Purple has a promoted Bernal, no matter how rich he is, players may not draw income from him.
Q8. ROCKET-PROPELLED BERNALS
A promoted Bernal can fuel and move as a rocket, and follows all rocket rules, including movement (F), refueling, solar-power, thrust-modifying supports, ESA beamed power, etc. When moving on the map, it uses a Bernal figure of your faction color. On the fuel strip, the Bernal uses a red dry mask disk, and a hemispherical fuel figure, color coded as follows: black = dirt, blue = water, and yellow = isotope.
Landing and Lift-off. A Bernal follows all rocket rules for landing and lift-off, except they are prohibited from entering lander burns (G4) or the Kreutz-Sungrazer.
Refueling. Bernals refuel using all of the rocket rules (H5). Note that all players except Player Orange use a Bernal that follow the dirt rocket rules. (Thematically, they sacrifice their own shielding as propellant. This shielding, made of regolith, is hundreds of times more massive than the hull and can be replaced when the Bernal arrives at a dirtside).

Example: Player White has a Bernal Stack (dry mass 15) in its home orbit and 8 WTs. As a free Home Orbit Refuel (D1), he loads all 8 WTs into the Bernal as fuel. He places a clear red disk, acting as the dry mass indicator, on 15 and a blue hemisphere, acting as a fuel figure, on the wet mass of 23. He uses a blue fuel figure instead of black even though this is a dirt Bernal. By indicating the higher grade fuel with a blue figure, he preserves the option to convert some of it back into water FFTs later.

Thruster Activation. Like a rocket, a Bernal can activate (F1) a another thruster that is part of the Bernal stack. This new thruster follows the rules for a rocket, using the Bernal dry mass disk and the Bernal fuel figure. Of course, it cannot use fuel loaded into the Bernal unless it is the correct grade (E4). Note that a Bernal Stack cannot be “towed” by another spacecraft.
 
Example Mission to Callisto: Player Red has a promoted Bernal at his L3 home orbit with an in-core thermionic generator supported by a dual-mode fission reactor and a heavy microtube radiator. Together with his taikonauts, the dry mass = 15. Without any fuel, it has a net thrust of 3 - 1 + 1 = 3. (The +1 modifier comes from its dual-mode fission). Initially traveling along route yellow, the dirt-propelled Bernal travels through the Sol-Earth L3 and stop at the Hohmann pivot near Khufu. On its next turn it lands on the small asteroid Mjolnir. Because it meets the movement requirements of F2, both moves are made without any fuel on-board. At Mjolnir, the taikonauts dirt-refuel once (8 tanks), moving the fuel figure to 23 (Tug Class) with a net thrust of 2. Blasting-off of Mjolnir, the Bernal goes through Sol-Earth L3 and Sol-Mars L4, taking route purple to land on the small asteroid Gaspra. Six tanks were emptied going through two burns, so the taikonauts break out their shovels again. From Gaspra, the Bernal takes the green route for a near-suicidal flyby of Jupiter. Trying to aerobrake to reach Callisto, the Bernal is struck by a superbolt of lightning, vaporizing everything except the Bernal shell. However, the demoted Bernal takes its four free burns from the flyby to reach its destination orbit, a highly eccentric orbit around Callisto. Mission accomplished, and Player Red will replace its lost heroes and equipment later.

Example Mission to Mars: Player Red has a nuclear-powered Bernal stack in its home orbit with a dry mass of 15 and its maximum wet mass of 32. He wishes to go to LMO (low Mars orbit) to obtain a Mars dirtside. His first move follows the yellow route and stops near L5.  On the next turn, his net thrust of 1 allows his dirt-propelled Bernal to travel from L3 to Sol-Mars Lagrange L2 (next to Deimos) in one turn, expending 3 steps for the burn. The black fuel figure drops 3 steps from 32 to 26. On his next turn, he expends 3 more steps of fuel for the burn and coasts to LMO, ending with a wet mass of 22.

Q9. FREE MARKET AT A BERNAL
If you bring a black card to a Bernal, you may sell it for the same price that it would be sold in LEO (i.e. perform a Free Market Operation, per H3, to receive WTs equal to the VP value of the card’s product letter, as shown on the Exploitation Track). The Bernal must be promoted and either with a dirtside or in its home orbit. It can be yours or an opponent’s, if he agrees.
·       Market Saturation. If you sell to your own Bernal (including at its home orbit), the card sold must be returned to the bottom of the deck (i.e. the same as selling to LEO, per K5). If you sell to an opponent’s Bernal, the card sold goes to your hand, but you must pay the owner of the Bernal the amount of WT he requests.
·       The Product Letter of the card being sold cannot be the same as the spectral types of any of the Bernal’s dirtsides.


Example: Player Green has a factory on Laocoon, the only "D" type factory in this particular game. Player Green performs an ET production operation to produce the Rotary Dirt Launcher (a "D" type freighter), launches it with factory-assist, and moves it to Player White's Bernal in Mars LMO. Because this Bernal has only C dirtsides, Player Green may sell the D Freighter Card, which is decommissioned. Player Green receives 9 WTs, and Player White receives 1 WT.
R. COLONIST MODULE
Summary. Colonists are researched or recruited from their own deck. The main advantages of a colonist are the extra operation it confers (M1) and its special ability. Some have thrusters, on-board reactors, ISRU, and Futures. There are two types: human and robot.
R1. COLONIST CARD SETUP
Colonist Cards are either white or black on one side, and purple on the other. They form their own patent deck, the Colonist Deck.
R2. HUMAN AND ROBOT COLONISTS
Human colonists are white and purple; robot colonists are black and purple. Both are taken procured into your hand by during a research (H2) or recruit (N4) operation.H However, humans are then boosted (H4), while robots are produced by ET production (H8).
·       Human colonists are limited to one in space boosted per player at the start. However, if playing with the Bernal Module, you can boost additional human colonists (R4). Human colonists vote according to their ideology, except during war when they vote for their employer (N6). Their presence provides protection against the glitch event and prevents felonies.
·       Robot colonists can only be built at factories and cannot be boosted from Earth. They don’t count against your Max Human Colonists in Space (R4), even if emancipated. Robot Colonists do not prevent glitches unless emancipated (N3).
NOTE: Your crew is human, but not a Human Colonist.
R3. COLONIST CARD ANATOMY
If in space, a Colonist Card has one or more of the following attributes. <<see xxx illustration>>[BW13] 
·       Specialization.
·       Election Button of a faction color, and containing one or more ballot box icons.
·       Platform and ISRU Rating.
·       Thrust Triangle (F1).
·       Human on Board Placard. Shows if a colonist is human.
·       Special Abilities. As listed in the white field on the card. If the ability says “In Bernal”, then the colonist must be in a Bernal Stack (including Ersatz-Bernal) to perform this ability. If in green text, the ability is used in Interstellar (Section Y) as well as in Colonization.
·       On-Board Nuclear Reactor. Colonists with this capacity can act as a reactor card (P5) and can act as the reactor card support when establishing a space colony factory (R6). They are affected by the Antinuke Policy (K4).
·       Interstellar. The icons in the green border (right side of the card) are only used with the Interstellar Solitaire Game (Y).
R4. MAXIMUM HUMAN COLONISTS IN SPACE
You are always allowed one human colonist card in space (plus your crew). If playing with the Bernal Module, additional human colonists may be boosted into space if your Bernal is promoted (Q6). If so, the maximum number of human colonists allowed in space is equal to half the dirtside hydration (Q5) of your Bernal (rounding fractions down, but always allowing at least one). For instance, if you have industrialized the entire Pluto system (4 x 4 = 16 drops), a Bernal there can support 8 human colonists in space.[4]
·       Robots (both emancipated and unemancipated) and your crew do not count against this limit.
·       Falling Limits. If at your maximum, and your dirtside hydration falls for some reason (e.g. your Bernal moves away from the dirtsides), then you do not have to decommission colonists to reach your lower limit. Instead, you simply cannot boost new human colonists until your limit rises again.

Example: Player White moves his Bernal with one colonist to LMO, where he has industrialized two Martian sites. Since his dirtside hydration is 7, he is free to boost a second and third colonist.
R5. COLONIST CARD PROMOTION
To flip a boosted Colonist Card to its purple side, bring it to a Lab and perform a Promote Operation (N2).
·       If promoted, the abilities on the facedown side are then lost.
NOTE: Your crew card cannot be promoted and goes into your hand if decommissioned.

Example: Player Orange sends a robonaut and refinery to Hi’iaka (a TNO science site), succeeds in his prospect roll, and subsequently industrializes the site, forming a TNO lab. After sending the Islamic Refugee colonists to Hi’iaka, he may perform a promote operation to flip them to their purple side.
R6. COLONIST CARD DECOMMISSION
It is a felony to decommission humans, except for obtaining glory or establishing space colonies (see decommission). Decommissioned human colonists go to the bottom of the colonist deck instead of your hand. (Colonists have no loyalty to their former employer if decommissioned).
Exception: Certain colonists have special decommissioning rules, as stated on the card. For instance, the Svalbard Caretakers allow all decommissioned humans, including the Caretakers themselves, to go to your hand.
S. GW THRUSTER MODULE
Jon's Law: "Any interesting interstellar drive is a weapon of mass destruction."
Summary. A Gigawatt (GW) Thruster cannot fuel using WTs; it may only fuel at a spectral site matching its fuel isotope or with your yellow FFTs. If used to afterburn, spend one fuel step to gain one therm of cooling plus the thrust addition listed in the afterburn icon. Otherwise, they move and burn fuel as the MW rockets in the basic game.
·       Modifiers. A rocket using a GW thruster ignores Thrust and Fuel Consumption Modifying Supports other than solar-power modifiers (O2F2) when calculating net thrust and fuel consumption.
S1. GIGAWATT THRUSTER DECK SETUP
Gigawatt thrusters are black on one side and purple on the other. They form their own deck, the GW Thruster Deck, and are researched and power Rocket Stacks like the megawatt thrusters of the basic game. They are produced on their black side by an ET production operation (H8).
You may only have one GW Thruster Card produced at a time.
S2. GIGAWATT THRUSTER FUEL
A Gigawatt Thruster can only fuel at sites of the spectral type specified on its card, or with your yellow FFTs. This includes ISRU (H5.2) or factory refueling (H5.1). It can also factory refuel at a promoted Bernal if it has a dirtside of the specified spectral type.
Isotope Fuel Site Refueling. During factory, ISRU, and Dirtside refueling, the refueling rate for isotope fuel is limited to 1 WT per factory (H5.1, H5.2, and H5.4).
TIP: for GW thrusters, ISRU refueling is sometimes faster than factory refueling! This is because isotope fuel purified from huge amounts of ore is required.

Example: A GW Dense Plasma Focus rocket sits on a factory at Hydra, a D spectra moonlet of Pluto. A “blue goo” science colonist with ISRU 1 is also there. Since Hydra is a wet world (appropriate given its name!), by performing an ISRU refuel the colonists extract 1 + 4 - 1 = 4 tanks of fuel. This is 4x faster than if a factory refuel is performed. Note that if the player has an engineer in space (and thus has an extra engineering operation), he could spend 1 WT and perform 2 ISRU refuels, gaining 8 tanks altogether.

Isotope Fuel. A GW Thruster always uses the yellow fuel figure to show that it runs on pure fuel isotopes and cannot use lower grade fuels such as water (E4). Additionally, a GW rocket can only use fuel obtained from a site matching its “fuel isotope” spectral type.

Example: Player Green builds a Spheromak GW thruster on Memphis Facula of Ganymede (type S). The dry mass of the rocket is 12 (including its supports), and during factory or ISRU refueling the rocket is loaded with 7 tanks of S isotope fuel (helium-3). The rocket is now tug class, so even with afterburning, the Spheromak net thrust (11 - 2 = 9) is insufficient to lift-off Ganymede (size 9). Therefore the rocket uses factory-assist to lift-off.

 Storing Isotope Fuel as FFTs  (F4). You can create yellow FFTs without the presence of a GW thruster.
Specifying Isotope Fuel Spectral Type. You are allowed to have only one GW rocket produced at a time, and the spectral type of this card specifies the spectral type of all the isotope fuel you carry and have stored as FFTs. If this card changes to a new spectral type, all of your isotope fuel is converted to water. You are not allowed to have or store more than one type of isotope fuel. If you do not have a GW thruster, you must specify the isotope fuel you have by putting a claim disk above the exploitation track of the spectral type matching the fuel. To change this, you must first convert all your existing isotope fuel to water. This claim disk counts towards your logistic limits.
 
S3. GIGAWATT ENHANCED AFTERBURNING
If you afterburn (F2) using a GW thruster, you expend one step of fuel and increase the net thrust by the value shown in the flame icon. This provides one therm of open-cycle cooling.

Example: A Spheromak GW thruster has a 6 • 1/10 thrust triangle with an afterburning modifier of +5. It is transport class, so the wet mass modifier is -1. By expending the minimum one step of fuel, it may move up to its net thrust (5 burns) and land on a world of size 4 or smaller. If it afterburns a second step of fuel, it may move up to 10 burns and land on a world of size 9 or smaller.
S4. TERAWATT THRUSTERS
To flip a built GW Thruster Card to its purple side, bring it to a Lab and perform a Promote Operation (N2). The promoted GW Thruster, called a TW (terawatt) Thruster, follows all the rules for GW thrusters.
A GW thruster may be promoted at any lab, regardless of the lab’s spectral type.
NOTE: The spectral type of the lab does not affect GW thruster promotions.
Starship Thrusters. Some thruster and freighter cards have a small green thrust triangle alongside the standard thrust triangle. Rockets with these cards active are called starships. The numerical values in these triangles are only used in the Interstellar Solitaire Game (Y).
Synodic Sites. A starship can enter a synodic comet site during any sector.
Colliding FRC Fusion Thruster. This TW thruster requires a generator, plus two reactors as supports (because its initiator energy requirements are so high due to low Q)[5]. 
If decommissioned, the card reverts to a GW thruster in your hand.

Example: A Rocket Stack contains an 8•0 thruster, with a “thrust +5” afterburn. If the net thrust is 8, this starship can expend zero fuel and move 8 burns or one step and move 13 burns.

Science Note: In order to achieve their remarkable Isp performance, GW thrusters burn pure nuclear fuel with no added propellant or open-cycle coolant. Thus, the burned fuel becomes the propellant (in this they are like chemical rockets). The fusion or fission fuels are rare isotopes, which must be purified by isotope separation. The game makes the following assumptions on where to find these rare isotopes: The fission fuel curium-245 on M worlds, lead antimatter thermalizer is also found on M worlds, the fusion fuel lithium-6 on V worlds, the fusion fuel boron-11 on D worlds, and the fusion fuel helium-3 on S worlds. S worlds also have tritium for the Vista drive and uranium-235 for the Zubrin drive.
T. COMBAT MODULE
When the politics marker is in war (K4), combat is possible between players with colocated stacks. You can initiate combat in two ways:
1.   Attack: After all of your spacecraft have completed movement in your turn, you may pick a space where your stacks or cubes are located and declare that you are attacking the colocated stacks or cubes of another player. You are the attacker; that player is the defender.
·   	Interception: During another player’s turn, you may declare you are intercepting the colocated spacecraft of another player as it is exiting a space where you have stacks or cubes. This interrupts the move until combat is resolved, with you as the defender and the other player as the attacker.
·   	Multiple combats can be declared during your turn, but in each space on the board, combat can only occur once during each player’s turn. Exception: if multiple players wish to intercept a third party in a space, the combats are separated and resolved in player turn order.
·   	LEO security zone. No combat is allowed in LEO.
T1. COMBAT SEQUENCE
When combat occurs in a space, every factory or space colony (T5T4) in the space, and each operational card with an ISRU rating, or operational thruster or freighter with a Jon’s Law rating (T2) in a stack in the space may make  an attack roll according to the combat sequence below. Every card, cube, or space colony in the space is a potential target of each attack roll.
The defender may attack with each of his rayguns (T2).
Attacker may attack each of his rayguns (T2).
 Defender may attack with any or all of his missiles (T3).
Attacker may attack with any or all of his missiles (T3).
The player with the higher net thrust (if any) may attack with any or all of his buggies (T4).
Cube or Cylinder Combat. Any cube on a claim disk is a factory which attacks and defends (T5T4). Cubes not on a claim disk (including legacy freighters and mobile factories) attack and defend with their cargo and are eliminated if they lose their cargo. Exception: Buggies can capture these cubes per T4. A cylinder on a claim disk is a space colony, which attacks and defends like a factory.
Entrenchment: The cards in a stack that spent the entire phasing player’s turn on a site have increased defense against rayguns and missiles. Their rad-hardness is increased by two. Entrenchment does not apply to cubes/cylinders.
Supports: During combat, supports can be shared for the purpose of determining whether a card is operational (O1), and open-cycle cooling can be used to satisfy the cooling requirements of a missile robonaut or thruster. Also a factory if present can supply some support requirements (see T5). If an attacking unit becomes non-operational due to support loss, it can’t make its attack.
The combat modifier listed on some cards is applied for that card only.
Cards with multiple ISRU platforms may make an attack with each platform.
T2. RAYGUN ATTACK
Choose a single card, cube, or cylinder as the target, and roll 1d6. If the result is greater than its rad-hardness, it is decommissioned.
·       Jon’s Law. During combat, the mass driver, MPD T-Wave, or non-freighter starship (S4) thrusters and the promoted Green Bernal may attack as robonaut rayguns rolling 2d6 instead of 1d6.
·       During combat, promoted Bernals attack as robonaut rayguns.
T3. MISSILE ATTACK
Choose a single stack as the target. Roll 2d6 and apply the sum against each card in the stack, as well as against each cube present (T5). A sum greater than its rad-hardness will decommission cards or cubes. The missile card used to perform the attack and all of its supports are decommissioned!
Kamikazes. When declaring a missile attack, a non-Bernal spacecraft can end its movement on a site without needing a net thrust higher than the site size. The entire stack is decommissioned after missile attacks are resolved.
Orion. A Rocket Stack in space carrying the operational Project Orion reactor or the n-6Li microfission thruster is immune from missiles if not on a site (it can launch fission bombs toward anything that approaches, and it has a shield designed to survive nuclear blasts.)

Example: A Rocket Stack containing just the Squid Turing robots attacks a Bernal stack. The Bernal attacks first with a raygun that is aboard and rolls a “4,” not quite enough to destroy the Turing. The missile then strikes and rolls a “7.” This is enough to decommission all the cards in the stack, except the Bernal Card itself, which is indestructible.
T4. BUGGY ATTACK
A buggy may not attack unless the net thrust of its stack is greater than the net thrust of the target stack. To determine this, both players recalculate their thrust as if they were beginning their move. They may use afterburners or jettisons to improve net thrust. An outpost stack or factory has a net thrust of 4 minus its ISRU if it contains an operational buggy on a site. Otherwise, its thrust is 0.
For each buggy capable of attacking, pick a card in an opponent’s stack as the target. If there are no cards, pick the cube as the target. Roll 2d6. The buggy attack’s success depends on what units the opposing player has present:
·   	If the opposing player has a human present (including space colonies), the attack succeeds if the roll is greater than 8.
·   	Otherwise, the attack succeeds if the roll is greater than 6.
If the attack is successful and the target was a card, decommission the card. If the target was a cube on a claimed site, see T5. If the target was a cube not on a claimed site (i.e. freighter or mobile factory), it is destroyed unless you have a promoted freighter, whereupon it is captured and replaced by a small cube of your color. If the target is a FFT (K6), it is stolen. Captured freighters or FFT require a dry mass adjustment (D2) in both stacks.
T5. FACTORY COMBAT
Each factory and space colony on a claim attacks and defends as a robonaut raygun with a rad-hardness of 8. (See Example 2 below and T4 for an exception for unmanned factories defending against buggies.)
·   	Factory Capture. If you make a successful buggy attack (T4) and chose to target your opponent’s factory cube and he has no space colonies present, capture it. Swap the cube and claim disk with ones of your color.
·   	Space Colony Capture. If the site location of the successful buggy attack contains both a factory and space colonies, you must destroy/capture the space colony before capturing the factory. Replace the opponent's space colony cylindercube with one of your own.
·   	Factory Destruction. If there are no longer any factories on a claim site, the appropriate exploitation track goes up one step (P6).

Example 1: Project Orion lands on a space colony at an M site. The cube and the colony each fire first, targeting Orion’s missiles. But, one missile survives, rolling a 9, which removes everything but the claim. The M resource Exploitation Track is increased one step toward “START.”

Example 2: You successfully attack an enemy colony with your buggy beating a roll of 8 due to the human presence. This captures the colony.  On the next turn, your buggy attacks again. If the roll is 6 or higher, you capture both the factory and the claim.
U. COLONIZATION ENDGAME MODULE
This optional module adds perhaps an hour per player to the game. If not using it, ending and winning the game follows the main base game rules (J).
U1. FUTURES
Certain promoted Freighter, Colonist, and GW Thruster Cards list a Future with conditional victory points (VP) indicated by the white star icon. This icon means “if this card is promoted and operational, and if you meet all the listed requirements, place a disk of your color on the futures star icon on the end page.”
·       You are limited to one Freighter Future, one Colonist Future, and one GW Thruster Future.
NOTE: Only one disk per future, and once placed, it can’t be removed.
U2. FREIGHTER FUTURES
·       Planet Hunt Future. Place disk in a Futures Star if you have a factory in Sedna (upper left corner of the map).[6]
·       Beehive Ark Future. Place disk in a Futures Star if you perform an epic hazard op using an emancipated robot at your D site factory.[7]
·       Spacefaring Future. Place disk in a Futures Star if you have ten or more dirtside hydration (Q5). (Tip: Orbits with a potential dirtside hydration of 10 or more include Saturn Norse Moonlets, Saturn dusty ring, Uranus Portia Group, Pluto, and Haumea).
·       Antimatter Creation Future. Place disk in a Futures Star if you perform an operation at a factory on a science site using a promoted colonist with the engineering specialization. As denoted by the double skull icons, you must make two epic hazard ops.
·       Climate Mirror Terraform Future. Place disk in a Futures Star if you perform an operation at a promoted Bernal manned with at least two humans with an atmospheric dirtside. As denoted by the skull icon, this is an epic hazard op.[8]
·       Star Wisp Future. Place disk in a Futures Star if you have a freighter cube or mobile factory cube in either the EM or the neutrino sunlens points on the map, plus a Push Factory and a TNO Lab.[9]
·       Beanstalk Future. Place disk in a Futures Star if you have successfully claimed the C or V venture and have factories on Pluto and Charon.

Example: You promote your HIIPER freighter, move mobile factory cubes to the two sunlens points, and industrialize Quaoar and Io, thus accomplishing the Star Wisp Future. As the first player to complete a Future, place your disk in the leftmost (12 VP) star on the end page (upper right corner).
U3. COLONIST CARD FUTURES
·       Revolutionary Future. Place a disk in a futures star (U1) by performing an epic hazard op while the Politics is in the “War” polity, and you have at least three human colonists in space.
·       Pan-Sapiens Future. Place a disk in a futures star if you hostile recruit (N5) two colonists using the Group Mind Immortalists. Both of these operations are epic hazard ops.
·   	New Venus Future. By decommissioning a TW thruster (S4) with its supports at a synodic comet that you have industrialized, you announce your intention to crash the comet into Venus, bringing enough water to create a new Earth. This announcement changes the Politics into the “war” position of your choice. Unless the factory is destroyed or leaves, you get a futures star 12-years later. This action destroys the comet site (place two blue disks to indicate this), and transfers its factory and its claim to the “Venus Aerostat-Xity” site. All units on Venus are decommissioned by the impact.
·   	Footfall Future. By decommissioning a TW thruster (S4) with its supports at a synodic comet that you have industrialized, you announce your intention to crash the comet into Earth, causing an apocalyptic impact winter. This announcement changes the Politics into the “war” position of your choice. Unless the factory is destroyed or leaves, the game will end 12-years later, and you claim a Futures Star. This action destroys the comet site (place two blue disks to indicate this).
·       Uplift Future. Place a disk in a Futures Star if you have emancipated the robots and your scientist at a Lab performs an epic hazard op.
·       Von Neumann Future. Place a disk in a Futures Star if you perform an epic hazard op at a factory of yours using two emancipated robots.
·       Supreme Cult Leader Future. Place disk in a Futures Star if you win an election in peacetime (N6) and there are more ballot box icons voting for you than there are for anyone else. Count robotic votes only if you own them and you are the Emancipator. If the Cult Leader Colonist is later decommissioned, you lose the disk (without sliding any other disks over to fill the gap). 
·   	SETI Future. Industrialize one asteroid in the Jovian Trojans (Trojan Camp), i.e. Patroclus, Glaukos, Menoetius, Antenor, Laocoon, Äenaes, or Tithonus; and one asteroid in the Jovian Trojans (Greek Camp), i.e. Agamemnon \& moonlet, Icarion, Philoctetes, Nestor, Telamon, Achilles, Hektor, or the Hektor moonlet.
·   	Artificial Consciousness Future. At a TNO lab, claim any busted futures star as an epic hazard op. If successful, replace the disks with one of your own. If you fail, decommission all non-Bernal humans of all players.
·   	NEO Mines Future. Place a disk in a Futures Star if you have claims on 3 sites of size 1 in the Mars, Earth, or Venus heliocentric zones.
U4. TERAWATT THRUSTER FUTURES
·       Mini-Black Hole Creation Future. Place disk in a Futures Star if you perform an epic hazard op on a factory on a size 1 S site containing a promoted scientist.
·       Protium Fusion Future. Place disk in a Futures Star if you perform an epic hazard op at a site with 4 promoted cards, including a promoted colonist with the scientist specialization.[10]
·   	Lithiated Ammonia Starship. Place disk in a Futures Star if your operational rocket contains the Solem Medusa thruster with at least 8 fuel tanks and three colonists, one of each specialization. You must then fly them to exit Sol (Jupiter-Sol-Jupiter exit, Sol Exit Neptune, or Sol Exit Oort), roll for a crash hazard not modifiable by “Failure is not an option”€, and decommission the Rocket Stack (including colonists) to your hand.
·       Enzmann Starship. Place disk in a Futures Star if your operational rocket has at least 8 fuel tanks and contains the Colliding FRC thruster and three colonists, one of each specialization. You must then fly them to exit Sol (Jupiter-Sol-Jupiter exit, Sol Exit Neptune, or Sol Exit Oort), roll for a crash hazard not modifiable by “Failure is not an option”€, and decommission the Rocket Stack (including colonists) to your hand.
·       Mass Beam Future. Place disk in a Futures Star if you have factories on Mercury, Venus, and Io (i.e. the three push-factory sites).
·       Daedalus Future. Place disk in a Futures Star if you have factories on the sites labeled “Saturn aerostat”, “Uranus aerostat”, and a TNO Lab. (Thematically, the Daedalus starship needs 1350-tanks of 3He-D, wrested by aerostat-mining from the atmospheres of gas giants.)
·       Fusion Candle Future. Place a disk in a Futures Star if decommissioning this working card in the Uranus or Neptune aerostat (epic hazard op), while having space colonies in two of its moons or aerostats.
U5. ENDING THE GAME
The game ends two event thresholds (M3) after either of the following occurs:
·       If the number of Futures occupied by disks is one less than the number of players. For instance, in a 2-player game the game ends the event after the first future disk is placed. (But either player can take additional Futures before the game ends.)
·       Twelve years (one solar cycle) after a Footfall is announced (U3).
U6. ENDGAME ELECTION
At the end of the game, perform an election auction per N6, and count VP per J2.
GLOSSARY, Basic \& Colonization
Terms underlined in the rules are defined here. These include the game’s core rules.

Adjacent – A spacecraft is adjacent to a position if it is in the next space, where each intersection, burn, and site counts as a space. However, for this purpose, crash hazards and lander burns (G4) do not count towards adjacency. Also, sites connected by buggy roads (B4) are not adjacent (because line of sight is blocked by the horizon).
·   	Any space adjacent to one end of a space elevator (I2) is also adjacent to the space on the other end.

Aerobrake Hazard – A space marked with the parachute icon (F5). When entering an aerobrake hazard, roll the die. A “1” = spacecraft decommissioned.

Assaying Smelters – Certain refineries (as listed on the card) improve the ISRU rating or prospecting roll, if at the site being prospected (Exception: Superlens ability affects colocated lasers).

Atmospheric Site – If the map image of the site is encircled by a blue halo, the site is atmospheric. This includes Venus, Mars, Saturn, Titan, Uranus, and Neptune. This is relevant in raygun prospecting (H6), atmosphere scooping (H5), climate mirrors (U2), ionosphere lasing refinery usage (F2), and some futures.

Bernal (Bernal module) – A type of space station that can move like a rocket, and follows all rocket rules. It forms its own stack, and has its own dome-shaped map and fuel figures. You can boost a second (stationary) Bernal called an Ersatz-Bernal (Q4). Your WTs are stored in the Bernal or Ersatz-Bernal parked in your home orbit.

Bernal Privilege (Bernal module) – You gain this privilege, as listed on your Bernal card, upon promoting your Bernal. The privilege is active as long as your Bernal remains promoted with a working generator.

Buggy Prospecting – Using a buggy allows for two attempts at a successful prospecting roll with one prospect operation. Alternately, if prospecting a site linked to others by a dashed yellow line (indicating a buggy road), you may prospect all the linked sites with one prospect operation (H6), assuming the buggy has the necessary ISRU for those sites.
Example: A buggy prospects Dresda. The roll is a 3, which fails because Dresda is size 2. But a second roll of 2 succeeds and places a claim disk.

Burn – A magenta-colored circular or lander-shaped space. It costs 1 TMP and (for rockets) fuel to enter a burn. You may not halt on a lander burn (G4).

Card Anatomy – The patent and crew cards have one or more of the following data:
·       Mass.
·       Product Letter. (H8).
·       Radiation-hardness (Colonization). Resistance to combat damage, solar flares, or radiation hazards. (K2).
·       Support Cards required (Support Module). The example data field shown in the rules requires an “x” reactor, plus one therm of radiator cooling (Q5).
·       Thrust Triangles (F1).
·       Support Triangles (Support Module). See Fuel Economy Reactors and Thrust Modifying Supports.

 Cargo – The cards in a stack. Fuel is not cargo, but FFTs are (F4). If using the Freighter Module, cubes may be carried as cargo per P4.

Claim Disk – A disk of a player’s color showing that he has successfully prospected and has mines and claims at a site. Each claim is worth 1 VP. Player Purple collects taxes (1 WT from the pool) every time a claim disk is placed.

Claim Jump – If you are allowed to commit felonies, you may land on the unindustrialized claim of another player, and immediately replace the claim disk with one of your color. You must have a human at the site, and the site must be undefended by humans.
·       It is not a felony to merely land or ISRU refuel (H5.2) on another player’s claim or factory. You need the owner’s permission (D3) to factory refuel.
Example: Both NASA and PRC have rockets on Encke’s comet. NASA prospects successfully, placing a white claim disk. On his turn, the manned PRC rocket feloniously decommissions its refinery and robonaut to industrialize the claim, replacing the NASA disk with a red disk and a red cube.

Colocated – Occupying the same space.

Crash Hazard – A space marked with the skull icon. When entering a crash hazard, roll the die. A 1 = spacecraft decommissioned (F5). Making a factory-assisted landing or lift-off (G3) also suffers this risk.

Decommission – Scrap a card from one of your stacks to your hand, so that it can be rebuilt anew (either by boosting or by ET production). This is a free action at any point in your turn (D1.6D1). If part of your Rocket Stack is decommissioned, see dry mass adjustment (D2).
Freighter Cube. You are allowed to decommission a freighter or mobile factory cube, which goes into your reserves.
Radiators. If a radiator is “heavy” (O4), a decommission other than a hazard (F5) or epic hazard decommission rotates it to its “light” orientation.
Decommissioned robots (unless emancipated) and crew go into your hand, but decommissioned human colonists go into the bottom of the colonist deck. (Colonists have no loyalty to their former employer if decommissioned).
Your humans can only be decommissioned if you are allowed felonies, or per the next two bullets. The Botany Bay Convicts can self-decommission, because they are allowed to commit felonies (this represents them going AWOL).
Ticker Tape Parade. Humans can be decommissioned in LEO (e.g. for obtaining a glory card per I1).
Space Colony. Humans can be decommissioned at a factory to start a Space Colony. If they have an on-board nuclear reactor (R3), they can act as a support for the factory. They can also decommission at an existing colony (which absorbs the crew without benefit).
Involuntary Decommission. If a stack is destroyed or loses its last card by a hazard roll, epic hazard op, event, radiation, combat, etc., return all cargo cards to your hand and remove its map and fuel figures. Note that the Bernal Card is invulnerable (Q2).
TIP: Should a rocket end up in a bad situation (e.g. out of fuel), either decommission its patent cards, or mount a rescue mission (e.g. convert the rocket to an outpost, assemble a new rocket, and send it to rescue the old one.

Dedicated Card (Colonization) – The Bernal and Freighter cards are dedicated. The Bernal cannot be removed from the Bernal Stack, and the Freighter Card can be removed from the Freighter Stack only by decommissioning. Although dedicated cards cannot be part of your rocket stack, you may tow freighters (P4), and move Bernals with thruster cards (Q8).

Dirt Rocket – A card with a black thrust triangle that is allowed to use regolith (space dirt) as propellant. When fueled partially or wholly with dirt, use the black fuel figure. All of the promoted Bernals, except for Player Orange, are dirt-rockets (Q8).
·   	Phileas Fogg Refueling. A rocket or Bernal with a black fuel figure can decommission its cards as fuel (the machinery is ground up and fed into the engine hopper). Each mass point so decommissioned lowers the dry mass disk one step, following the dashed red line. This is a free action before you move.

Dirtside (Bernal Module) – Each of your factories on sites adjacent to your Bernal figure is called a dirtside. See Example Q5.
·       You cannot use an opponent’s factory as a dirtside.
·       A Bernal must land on a synodic comet to use it as a dirtside.
Example: If there are red factories on the Martian moonlets Phobos and Deimos, a red Bernal on the burn between them has both as dirtsides. Factories on Mars would not be dirtsides, since the aerobrake paths to the Martian surface do not indicate adjacency.

Dry Mass – The mass of a rocket, not counting its fuel. Find the dry mass by adding the masses of all the cargo (i.e. cards, cubes, and FFT) in the Rocket or Bernal Stack. The dry mass is tracked with a disk on the Fuel Strip. If it changes, see D2.
·   	The minimum dry mass of a rocket is one. If the mass of the cards is less than one, treat it as one. The maximum dry mass allowed is 23.

Emancipated (Colonist Module) – Robots are emancipated using the suffrage operation (N3). If emancipated, robots are treated exactly as humans except they still do not count against your Maximum Human Colonists In Space (R4). They prevent glitches and in peacetime have loyalty to the Emancipator (N3). Because consciousness is not fungible, neither humans nor emancipated robots can be digitally swapped (N1).

Epic Hazard Op. (Colonization) – Certain Futures (U1) call for an operation marked with a skull icon. This operation is called an epic hazard op. It must be performed by either the colonist containing the future, or a colonist colocated with the card containing the future. This operation cannot be performed at an opponent’s factory.
·       Epic Hazard Roll. Make a 1d6 roll, which cannot be modified by Failure is Not An Option (F5). You fail if the roll is a “1” and if you have one of the following: a TNO Lab, factories on sites of at least four different Spectral Types, or claims on least two synodic comets with one having your Bernal Lab. If you don’t have one of these, you fail if the roll is anything but a 6.
·       Failure Effects. If you fail the roll, all cards and cubes colocated with the performing colonist are decommissioned (except Bernals). If on a site, the site itself is destroyed (place a black disk to indicate this). Bust the Futures Star (U1) with a black disk stacked on your disk, indicating that although the future was not met, the game is one step closer to ending.

Exoglobalization – The elimination of government-enforced restrictions on exchanges across the Earth and extended to extraterrestrial resources and facilities, creating an interglobal marketplace.

Faction – Each player is a distinct “basal societal unit” (BSU): world organization (purple), national government (white), socialist regime (red), worker union (green), or private entrepreneur (orange).

Faction Privileges - Each faction has a special privilege, depending on which side of the crew card (legacy or radical) you are playing. Except for “War Declaration,” this privilege is lost during Anarchy or War (K4). If playing with the Bernal module, it is also lost per Q3.
·   	Launch Fees (LegacyPlayer White or Radical Orange). Receive a 1 WT bonus from the pool after any player (including yourself) boosts one or more cards.
·   	Skunkworks (LegacyPlayer Orange or Radical Purple). You may participate in a Research Operation (H2) or a Recruit Operation (N4) regardless of your hand size.
·   	Powersat (LegacyPlayer Green or Radical White). You may add one thrust to any spacecraft marked by the push icon, during any player turn (unless it is already being pushed; e.g. by a push factory).  <<Illustration of the push icon, taken from the map>>
·   	Territoriality/Extralegal Activism (LegacyPlayer Red or Radical Green). As free actions during your turn, you may perform felonies: including claim jumping, human decommission, and hijacking.
·       War Declaration (LegacyPlayer Red, Colonization). As a free action during the start of your turn, you are allowed to move the Politics from anarchy to war per K4.
·   	Taxes/Protection Fees (LegacyPlayer Purple or Radical Red). Receive a 1 WT bonus from the pool for each claim disk (H6) or Factory (H7) established by any player (including yourself). No tax bonus for claim jumping.
·   	Cycler (LegacyPlayer Purple, Colonization). You may grant passage of any spacecraft through the radiation hazards surrounding Earth without risk.

Factory – A claim disk topped with a cube of the same color (large or small) is called a factory. Sites with factories are said to be industrialized. There are three special types of factories: dirtside, lab, and push. Only one cube can be on a claim (plus a cylinder, if it’s a space colony).

Felonies – Free actions that are allowed only to Player Red, Radical Green, or during Anarchy or War. You may perform them only with humans, and if no opposed humans are present. They include claim jumping, hijacking, and human decommission.

Flyby (Colonization) – A space representing a close pass to a planet, allowing slingshots (K1). It follows the rules for Lagrange Points. A moon boost (K1) is a type of flyby. Another flyby is the Solar Oberth, which gives you a slingshot rating equal to the rocket’s base thrust (F1), plus one if afterburning. You must spend fuel for this burn before using the slingshot, and each spacecraft can use the Oberth only once per move. 

Fuel Economy Reactors (Support Module) – Some space-built reactors have a triangle that halves or quarters your fuel expenditure, if used in support of a thruster (O1).

Hazard - See Aerobrake Hazard, Crash Hazard, Epic Hazard Op, or Radiation Hazard.

Hijacking – It is a felony to perform an operation using a foreign installation for refueling (H5) or promotion (N2) without the owner’s permission.

Hohmann – An intersection of two routes, unmarked by a circle. Changing direction mid-move in a Hohmann requires a pivot maneuver.
·   	If routes cross without touching, pivots are not allowed.

Home Orbit – Your home orbit is LEO (Low Earth Orbit). If using the Bernal module, your home orbit is the orbit where your Bernal starts, if you have a Bernal figure there, and LEO otherwise. The Bernal of Player White starts in Earth-Luna L5. Player Orange starts in Earth-Luna L2. Player Green starts in GEO. Player Red starts in Earth-Luna L3. Player Purple starts in HEO. These home orbits are shown on the map by a seven-pointed star. Cards boosted from Earth go to your home orbit.

Human – A crew card, Bernal card or ersatz-Bernal, Robotic Colonist Card (if emancipated), or Human Colonist Card. A space colony is also human. Humans can’t be decommissioned, and they protect against felonies and glitches. The maximum number of Human Colonists in space (but not humans, so not including crew) is limited per R4.

Hydration – The availability of water on a site. This will be between zero and four, and is denoted by the number of waterdrops on a site hex. This is used for refueling (H5) and prospecting (H6), as well as the optional colonist module (R4).

In Power (Colonization) – The seven spots on the Space Government that are colored with a faction color (see end page). You are in power if the Politics is on a polity of your faction color.

ISRU – Units able to prospect or dig for water have an “In Situ Resource Utilization” rating, ranging from zero (the best) to four (the worst). In order for a unit to be able to prospect or ISRU refuel at a site, it must have an ISRU number less than, or equal to, the site’s hydration.

Lab – There are two kinds of Labs. A factory at a TNO science site (the yellow star-microscope icon) is called a TNO Lab. A promoted Bernal with at least one dirtside on a science site is a Bernal Lab.
Either kind of Lab is required to promote a Colonist, Freighter, or GW Thruster (N2).

Lagrange Point – Any intersection marked with a circle, either filled (i.e. a burn) or unfilled, indicates a junction where you can turn for no movement or fuel cost. (Course corrections are easy in these areas because the heliocentric-dominated gravity is partially nullified by the presence of planets or large moons. In physics lingo, these are where the gravitational pulls of two massive bodies provide the centripetal force required to orbit both of them.) Note that there are many circular spaces not on an intersection that are not technically Lagrange Points. These places (which include LEO, radiation spaces, and spaces on zone boundaries) nevertheless follow the rules for a Lagrange Point.

Logistics – Because of mission control limitations, you are limited to the disks, cubes, and cylinders you start with. If you need another disk, withdraw a claim disk (the vacated site must be re-prospected to be claimed again). If you need another cube, withdraw your freighter cube or a mobile factory cube. Other cubes cannot be withdrawn.

Loyalty (Colonist Module) – When not at War (K4), the loyalty of a colonist is indicated by the faction color of its election button (R3). A gear-shaped button indicates the colonist is robotic with loyalty to the Robot Emancipator (N3), if one exists. Loyalty is used during Hostile Recruiting (N5) and peacetime elections (N6).
·       Loyalty (wartime). During War, all colonists have loyalty to the holder of their card (i.e.e.g. their employer).

Map Anatomy –[BW26] 


Maximum Mass – In order to move, your Rocket and Bernal Stacks are limited to a dry mass of 23 and a wet mass of 32.

Open-cycle Cooling (Colonization) – If you afterburn at any point in a move, in addition to adding to your thrust, you can also satisfy one therm of cooling. (This simulates dumping coolant into the exhaust to increase thrust and get rid of heat.) Afterburners can be used to cool missile robonauts during prospecting or site-refuel, if they have expended the afterburning fuel that turn.
·   	Expenditure of afterburning fuel also counts as cooling during combat, for combat-capable thrusters or missile robonauts.
·       GW Thrusters have enhanced afterburning (S3).
Example [Support Module]: A rocket uses a vortex-confined thruster (1 therm) and a D-T Tokamak as a support (2 more therms). This rocket has but a single 2-therm radiator, however, and thus must use open-cycle cooling every move the thruster is used, to keep from melting down. This increases the thrust (from 6 to 7), at a cost of an extra fuel step.

Order of Decision-Making (Colonization) – Rarely, an event (M3) or combat (T) requires an order-significant decision from multiple players simultaneously. If so, the first player makes his decisions first; then, go clockwise around the table.
Example: A Pad Explosion is rolled. The first player has two mass 5 cards in his LEO Stack and chooses one to decommission. Then, the next player in clockwise order chooses.

Pivot – A maneuver whereby a spacecraft changes direction on a Hohmann. A pivot costs 2 TMPs and fuel equivalent to making two burns. Certain freighters, Bernal Privileges, and colonists affect pivots.

Politics (Colonization) – The position of the transparent blue disk on the Space Government Diagram, indicating the faction in power, the policies (K4) enforced, and the endgame VP.

Push Factory (Colonization) – A factory on Mercury, Venus, or Io[12] is called a Push Factory. These sites are marked on the map with a push icon u). A push factory adds two thrust to any spacecraft marked by the push icon, or to legacy freighters (E5), during any player’s turn (unless it is already being pushed by a powersat or another Push Factory). It also lets a Bernal use solar-powered supports as far out as Saturn (Q6).

Radiation-Hardness (Colonization) – A number measuring how robust a card is to radiation damage, abbreviated “rad-hard” or “rad-hardness.” First determine a radiation level (during flares, combat, or radiation hazards), then decommission any card with a radiation-hardness less than this level.

Radiation Hazard (Colonization) – A space marked with the radiation icon. Upon entering, a spacecraft rolls a die and subtracts its net thrust (F2) from the roll to obtain a radiation level. All cards and cubes in the stack with a rad-hardness less than this level are decommissioned.
·   	Freighters have a net thrust equal to their TMPs for this purpose. This is usually 1 but can be higher (e.g. push factories, antiproton sails).

Raygun Prospecting – If you are prospecting with a raygun, you may prospect any number of adjacent site hexes (where each intersection, non-lander burn, and site counts as a space). However, ignore crash hazard spaces, as they are not considered spaces when determining adjacency.
·       You can’t fire into an atmospheric site.
Example 1: A raygun (ISRU = 0) on the HEO for the Koronis Family may prospect ten asteroids in one operation! This includes the asteroids in the Karin Cluster. Roll separately for each prospect.
Example 2: A raygun (ISRU = 1) on the surface of Kleopatra can prospect Kleopatra and both of its moonlets in one operation, since they are joined by a crash hazard space.

Sails – Four cards are sails (heliogyro/electric, photon kite/mag, fission fragment/antiproton, and the Calypso/wet nano sails). Physically, they are huge gossamer films usually propelled by the sun (solar photons, solar wind, or solar magnetic field). Although they are technically not rockets, they move as rockets with a fuel consumption of zero.
Sails are decommissioned immediately upon entering an aerobrake hazard.
·       The Mag Sail receives a +1 moon boost for each radiation hazard it enters in a move. However, each radiation hazard can boost a sail only once per turn, in the case where a sail circles and re-enters the same point.


Example 1: A sail with a mass of 1 takes on cargo with a mass of 6. The total dry mass is thus 1 + 6 = 7. Place the fuel figure in the empty position of the “7 dry” row (it’s flying without any fuel). If it has a thrust of 1, it would be fully loaded, since it cannot take on any more mass or fuel without going into transport class, which modifies its thrust to less than 1.

Example 2: A sail spends its 1 burn to enter the Mars HEO (highly eccentric orbit). It then coasts to the aerobrake hazard. The sail card is decommissioned, but the rest of the stack parachutes onto the Hellas Basin.

Scale –
Every round of player turns is one Earth year.
Each mass point is a quadecaton (40-tonnes, or 40,000 kg).
Every human mass point is an eight-man crew with life-support.
Fuel consumption is inversely proportional to a rocket’s specific impulse in seconds, as follows: 6 = 4.6 km/sec (460 sec Isp), 4 = 10 km/sec (1000 sec Isp), 2 = 20 km/sec (2000 sec Isp), 1 = 40 km/sec (4000 sec Isp), ½ = 80 km/sec (8000 sec Isp), ¼ = 160 km/sec (16,000 sec Isp), 0 ≥ 0.4% lightspeed (121,000 sec Isp).
A thrust of one is 0.75 kN (750 Newtons, or 169 lbs., the weight of the game designer on Earth!). Each additional point doubles this.
An acceleration (in the game this is called net thrust) of one is 0.38 milligees or 0.38 cm/sec2, and each step more doubles this.
A size one world has a surface gravity of 0.75 milligees, and each additional step doubles this. Size 1 worlds have the following diameters based upon density: comet nucleus 52 km (only Centaur comets approach this size), S-type asteroid 22 km, M-type asteroid 14 km.
A site with hydration 4 is an icy or permafrost body. Hydration 3 has small amounts of surface or subsurface ice. The Sahara Desert is hydration 3. At hydration 2, trace amounts of water can be extracted from kilotons of regolith. At hydration 1, concrete would be considered a good source of water. At hydration 0, oxides and hydrides are the only alternatives. Dr. Noah Vale.
Reactors produce from 650-2000 MWth of thermal energy, either in neutrons, pions, or plasma jets. From this, generators produce 60 MWe of electricity (Subscript e = electricity, th = thermal).
Beamed power from a GEO solar power satellite emits a 60 MW laser beam. A push factory emits a 1 GW laser or mass beam.
Each radiator therm rejects 120 MWth of heat at 1200K when used to cool MW rockets, 240 MWth at 1427K when used to cool GW rockets, and 960 MWth at 2000K for TW rockets. The increasing temperature of rejection simulates increases in technology as each class of rocket is attained, and radiator technology remains the main limitation to increasing rocket power. Note that for a TW rocket, the entire rocket will glow white hot (perhaps with the exception of the thermally-insulated payload). [13]
Each burn or lander burn requires a delta-v (velocity change) of 2.5 km/sec. Each pivot (brachistochrone maneuver) is 5.0 km/sec and a distance of about 2 AU.
A solar flare die roll of 1 is an M1 flare with an X-ray power density of 10-5 Watts/m2.  Each point more is 4 times this amount. Thus, a die roll of 6 is a X95 (Carrington-class) flare with a power density of 10-2 Watts/m2.
Equipment with a rad-hardness of 1 can withstand a total ionizing dose of 4 X 10-7 krad (Si) without failing. Each point more is 4 times this amount. For example, equipment with a rad-hardness of 5 can survive a Mrad of dosage. This is the scale used by the industry.
Solar insolation is 1.38 kW/m2 at 1 AU (1 AU = Earth-Sol average distance). Maximum sailing thrust is 12.2 N/km2 from photon pressure at 1 AU, or 0.002 N/km2 (2 nPa) from solar wind dynamic pressure. These values are in the Earth zone. Each zone closer to Sol doubles them.
Each water tank (WT) is a 40-tonne bag with a diameter of 4.25 meters. For rockets that use hydrogen as propellant, 40-tonnes of LH2 or slush hydrogen is held in a cryo-tank cylinder 7.5 meters in diameter and 16 meters long, including active refrigeration for zero boil-off (ZBO).
The boost cost to LEO is $4 million/tonne to LEO using the Falcon 9 by SpaceX, or about $160 million/mass point.

Science Site – A site marked with the microscope icon. Science sites are useful for Glory (I1), and claims on science sites are worth extra VP (J2). Science sites with a yellow star are TNO science sites. In Colonization, science sites must be industrialized to claim the VPs, and some may be used as Labs.
·       Eureka. The science site status of the Martian Trojan moon Eureka is questionable. If you are the first to land on Eureka, immediately roll 1d6. There is a 50% chance of it not being a science site.

Signposts – If taking the red, orange, yellow, green, blue, indigo, or violet routes, a sign helpfully lists the number of burns to get to the destination. This number, multiplied by your fuel consumption (F1), equals the fuel steps you will need. The lander icon in a signpost lists the net thrust needed for a powered landing (G1).
·   	Gravity Assist. The numbers of burns in the basic and Colonization games may differ, due to the presence of slingshots in the latter. The signpost lists figures for both games, separated by a slash.
·       The number of burns listed assumes a stop at every Hohmann, to avoid fuel costs for pivots.

Example: A rocket with an output of 3 • 2 takes the orange route to Mercury. It will need 7 X 2 = 14 steps of fuel for the 7 impulses. Once it gets there, it will need another thruster for the landing, with a high enough thrust to obtain a net thrust of 11.

Solar-Powered – Cards with the solar icon. If a thruster or thruster support is solar-powered, the thrust is modified per F2. Solar-powered cards do not operate in the Neptune Heliocentric zone (labeled “NO SOLAR POWER”). A colonist card with solar-powered on-board equipment or sails can operate in this zone, but its equipment is shut down.

Space Colony – An industrialized claim disk topped with a cylinder is called a space colony. The cylinder is added by decommissioning a human there, with a maximum of one cylinder per factory.
·   	Colony Disbandment. If your crew is in your hand, you can convert a cylinder that began the turn as a space colony into a crew in another colocated stack (regardless which humans were used to found the colony to begin with). A colony must be disbanded if it spends one entire turn without a factory to support it (e.g. as a result of combat).

Spectral Type – Each site has a letter indicating its resource type: C = carbon, S = stone, M = metal, V = vestoid, and D = dark.

Specialization (Colonist Module) – In order to perform an operation, a colonist must have an icon matching the operation’s specialty: wrench icon = engineering operations, microscope icon = science operations, and handshake icon = finance operations.

Starting Card – You start the game with one starting card, a crew card indicating your faction color. If playing with the Bernal Module, you start with a second starting card, your Bernal. Starting cards are human, don’t count towards your hand during research and recruiting, and cannot be sold or traded.

Sunspot Cycle (Colonization) – This diagram tracks the years, where each spot is one year, and the entire cycle is twelve years (about the solar sunspot cycle length). The Sunspot Cycle is divided into three sectors: red, blue, and yellow.

Synodic Comet (Colonization) – A site hex with a border color (red, blue, or yellow). It cannot be entered or raygun prospected at range unless the current sector of the Sunspot Cycle is in the same color. (This simulates synodic planetary alignment and launch windows.) It can be exited anytime.

Thrust Modifying Support (Support Module) – Some reactor and generator cards have a thrust modifier listed in a support triangle on the card. This modifies your net thrust (F2) only if your thruster (or one of its supports) needs the card as a support. Modifiers anywhere in the entire support chain are effective. Supports for the Refrigerator Radiator are an exception (O4). GW thrusters (S0) and interstellar starships (Y5) are affected only by solar-power modifiers (F2), not by support modifiers to thrust or fuel economy.

TMPs – Thrust Movement Points, equal to a rocket’s net thrust (F2), determine how many burns it can move in a turn. (Physically, this corresponds to a rocket’s acceleration, which is its Base Thrust divided by its mass. As it loses propellant mass throughout its journey, its acceleration increases.)

TNO Science Site – A site marked with the yellow star-microscope icon. TNO stands for Transneptunian Object, a world far enough out that it may contain the original condensate from the protostellar nebula. If so, it will have rare isotopes (such as helium-3) in relatively high “cosmic” abundances.

Wet Mass – The mass of a rocket including its fuel. The wet mass is the dry mass plus one for each fuel tank carried. The wet mass is indicated by the position of the fuel figure.
·   	The maximum wet mass allowed is 32.

WTs – Water Tanks, the transparent blue disks used as fuel and money in the game and normally stored in a WT depot in your home orbit. WT’s are immune to combat or events until stored or transferred to a stack, where they become FFTs (F4). 
High Frontier - Alive \& Complete, 3rd Edition
VOLUME II: SCENARIOS, TIPS, VARIANTS, INTERSTELLAR
VOLUME II CONTENTS
Section V. Advanced Game Scenarios \& Strategy Guides, page 63
Section W. Experimental Exomigration Module, page 80
Section X. Experimental Simulation Module, page 86
Section Y. Interstellar Solitaire, page 90
Section Z. Card Descriptions, page 112
V1-V4. SOLITAIRE VARIANTS
All the solitaire variants (V1 to V4) have these rules in common:
V0a. No hostile recruit, activism, anti-trust operations: Ignore these ops, since there are no other factions in play. Individual cards that involve such operations are discussed below.

V0b. Research auctions. Pay 2 WT plus 2 WT for each support to research a card normally (comes with supports, per O3), or 1 WT to buy one unseen off of the bottom (does not come with supports). Ignore bid limits (H2).

V0c. Ignore elections. Instead of elections, change the blue special event to “World Science Symposium.” Choose three decks and put the top card of each deck at its bottom. The new top cards represent the hottest topics at the Symposium. 

V0d. Static politics. The politics disk remains on start which fixes the free market value at 5 WT. No faction, not even UN, is awarded VPs for exoglobal politics. No faction is considered to be “in power” (individual cards that state this need are discussed below). No felonies are allowed, unless you are Red Player or have a card (e.g. botany bay convicts) stating otherwise.

V0e. Suffrage: Instead of the “in power” requirement, in order to execute this operation the player must have at least one promoted robot and pay 10 WTs to attempt the emancipation roll (1d6, needs to be lower than the total of purple cards in play). Thematically, the WTs represent the expenditure of money needed to lobby for the robotic emancipation. A failed roll may be attempted again in a new suffrage operation, but the cost of 10 WTs must be paid every time.
V1. WERNER’S STAR SOLITAIRE (Phil Eklund, all modules)
 
Little Werner dreams of becoming an astronaut (or cosmonaut, taikonaut, etc.) and traveling to another star. Can this dream come true? Werner is represented by your crew card, and he wins if he attains a TW thruster future before he dies of old age. Play with all modules and events. It should take 2 hours.
 
In this variant, each Human Colonist has a number of white disks on them showing its age. (White = 1, Red = 5, and Blue = 10 disks). Each Age Disk is 12-years. Every time the Sunspot Disk passes the Senility Threshold (12:00 position in the Sunspot Cycle), add one Age Disk to all Human Colonists in space, and then for each of them roll a 1d6 Cancer Risk.
·       Cancer Risk. If a Human Colonist needs to make a Cancer Risk check, roll 1d6. Remove the card from the gameDiscard the card out of the game if the roll is less than the colonist’s his age (i.e. number of Age Disks). Werner is cancer-free, and never needs to make a Cancer Risk check..
·       Starting Age. Werner starts in your Bernal with two Age Disks. All Human Colonists acquire two Age Disks when boosted. You lose if Werner is decommissioned.
·       Hazardous Duty Wages. Performing an operation with a human costs a number of WTs equal to the colonist’s age (instead of 1 WT). Your crew continue to enjoy a free operation, and self-promotion (next bullet) and robot ops continue to cost 1 WT.
·       Self-Promotion. A Human Colonist can spend 1 WT to perform his own Promote operation and roll 1d6 to attempt to flip his own card to its purple side. This operation is successful if the roll is less than his age. Since this represents the maturation of the original humans, self-promotion preserves their Age Disks.
·       Auctions. If you pay Hazardous Wages, you obtain cards during Research or Recruit at no extra cost.
·       Irradiation. If a Crew or Human Colonist fails a radiation roll (e.g. from flares or radiation hazards), then instead of being decommissioned it gains acquires one Age Disk and then rolls a Cancer Risk.
·       Parenthood. If you have enough dirtside hydration to support an additional human colonist, your Crew or Human Colonist in a Bernal may attempt to become a parent by performing (and paying for) its operation and rolling 1d6. If the result is greater than the parent’s age, then draw a random Human from the Colonist Deck into your Bernal Stack, making a dry mass adjustment (D2). This is the next generation. It starts with zero Age Disks, and cannot be moved from the Bernal Stack nor perform any operations until it receives its first Age Disk.
NOTE: Emancipated robots are sterile but cancer-free.
·       Werner’s Star Victory. Win if you attain a TW thruster future (U4) before Werner becomes 60 years old (i.e. before he attains his 5th Age Disk).
·       Interstellar Victory. If you have the Interstellar map (Y), send Werner off as a Passenger in a starship. You win if his son reaches a habitable or living planet.
V2. HERMES’ FALL SOLITAIRE (Phil Eklund, support module only)
Earth is threatened by the binary asteroid Hermes [expanded map], which has been calculated to impact in 23 turns. You need to achieve the “Saving The Earth” colonization glory before the sunspot disk passes the senility threshold for the second time. (The refineries represent mass drivers that gradually deflect the path of the twin asteroids away from Earth. Prospecting is unnecessary.)
NOTE: Travel is always possible from Hermes a to Hermes b, and vice versa, regardless of the synodic comet rule.
Revised Sequence of Play. Setup per Section C as any faction. Use only the Support Module (O).
1.   Move your rocket.
·       Pick one operation. For research, see V0b.
·       Make an event roll, and then advance the sunspot disk one step. See V0c.
·       New Privileges. The UN privilege is to start with 10 WT. He also has his cyclers faction privilege. The PRC starts with six extra hand cards, drawn at random from each patent deck. The other three faction privileges are unchanged.
 
V3. MARS FLOTILLA SOLITAIRE (by Andy Graham)
This is a solo game to pass the time between playing High Frontier or to practice building rockets.

Objective. Make as many valid prospecting rockets as possible using just the High Frontier thruster, robonaut and support decks. To be considered a valid prospector, a rocket must be able to travel to Deimos - a moon of Mars - and be able to prospect that site, which needs a working robonaut with an ISRU of 3. This requires two burns worth of fuel and a minimum thrust of 2 to land on the size 1 site. Assume that all rockets are successfully boosted and fueled in LEO, and that they travel through the radiation belt safely. Likewise, do not consider the success or failure of the prospecting roll.
All of the normal High Frontier rules for making a functional rocket apply. So, support cards of the correct type and quantity must be in place. In most cases, the validity of a prospector will be self-evident, but for some, you may have to check the map.

Setup. Sort the thruster, robonaut, reactor, generator, radiator into their stacks. Shuffle each stack leaving them white face up.

Procedure. Draw the cards one at a time, choosing from any of the five stacks. Place the chosen card white-side up onto any existing Rocket Stack or start a new Rocket Stack. Once placed, the card cannot be moved.

Victory. Win if you get all of your ISRU 3 or lower robonauts to Mars. For a more challenging game, add the requirement to transport a functional refinery in addition to all of the above requirements.
V4. CEO SOLITAIRE (by Victor Caminha)
A. Starting WTs: 10 (but no idea turns).
B. Busted site: Luna.
C. No epic hazard roll without TNO: All futures which require an epic hazard op may only be attempted if a factory on a TNO site has been built. The roll is successful on 1–5 in 1d6.
 
D. Directory Meeting Round: At the end of a round which is marked with the “Senility” sign on the solar cycle, there is a high-level executive meeting, concerning whether the space program of the faction should continue or not. Unless the player keeps making significant achievements, the board will eventually decide to disband the program, causing the game to end.
In order to keep playing, a player must score a minimum number of victory points at the end of a directory meeting round. Whenever the solar cycle disk passes through the “senility” sign, put a blue disk at the center of the solar activity cycle, which represents 10 victory points. At the end of the actions phase of the next directory meeting round, the accumulated disks represent the minimum VPs a player must have accrued to avoid the termination of the space program.
So, when the solar cycle disk passes through the senility sign for the first time, the first directory meeting takes place, which demands no minimum VP requirement (but see “loss of crew,” below), but sets the standard for 10 VPs for the next directory round, twelve rounds later. At the end of the second directory meeting round, the player must have accumulated at least 10 VPs or it is game over and the minimum VP requirement will be raised to 20, checked at the end of the third directory round and so on. Eventually, the space program will be cancelled, unless the player successfully creates a future, which ends the game with victory, silencing the most skeptic members of the board.
 
E. Loss of crew: Every time a human crew card is decommissioned, add a black disk to temporarily raise the minimum VP requirement by 3 for the next directory meeting, unless when building a colony, achieving glory or successfully creating a future. This means the loss of human life reflects badly on the space program, and there will be need for an extra effort to convince the board to maintain it. Note, a player might need to score at the end of the first directory board if he loses crew due to pad explosions, hazards, etc. After the directory meeting, remove all the black disks from the center of the solar cycle. This means the loss of a human crew won’t affect the next meeting twelve years later, as public attention is diverted elsewhere. However, the loss of a robot colonist won’t earn a black disk, even if emancipated.
 
F. Winning the solo game: The game ends if the player’s faction successfully creates a future. If you fail a future, put a black disk on it, like normal. If all of the future’s spots are failed, the space program is doomed, and the game ends.
 
G. The following colonists have been altered on this variant:
New Attica Secessionists (Revolutionary Future): Instead of the requirement of setting the Politics to “War,” pay 10 WTs to the supply, representing the military investment needed by the revolutionaries to fulfill their future. The other requirements are unchanged.
Group Mind Immortalists (Pan-Sapiens Future): Instead of hostile recruiting, make recruit operations instead. Fulfill this future by successfully making two epic hazard rolls when this card is colocated in a stack along with at least two other human colonists.
Wet Nano Ecologists/ Eugenic Pilgrims (New Venus Future/ Footfall Future): These futures are unchanged, but they only happen 12 years later, which means a directory meeting round will occur in the meantime, and thus, the minimum VP requirement will be higher by the time waiting period comes to an end. The victory points for  this future are not awarded to the player until the waiting period comes to an end, so it is possible that the space program might be cancelled before the directory board sees its fruits.
Vatican Observers: This card halves the cost of Suffrage to 5 WT.
Cryo Librarians: Decommissioning humans does not earn a black disk, which would normally affect the minimum VP requirement.
Rock Rats Miner’s Guild: Instead of hostile recruit, this card raises the minimum number of colonists in space to 2. However, the colonist may only execute its op if it is in the same stack of the Rock Miner’s Guild in order to execute its operation. Also, this ability is superseded by the Bernal dirtside if its total hydration is 4 or more.
H. Measuring the faction’s space program success at the end of the game: After successfully creating a future, deduct the minimum VP requirements measured by the blue disks on the solar cycle from the total victory points accumulated by the player’s faction. This difference represents the net success the faction has scored.
I. Scale of Success (Total VP – Minimum VP Requirement)
-1 or lower: Controversial
0-11: Fair                 	
12-16: Good
17-20: Very Good
21+ Excellent
 
V5-V10. ADVANCED GAME SCENARIOS
V5. SPACE RACE- 2 to 5 Players (Phil Eklund)
The winner is the first to land a crew on Titan and return them to LEO.
V6. ALIEN INVASION 3-PLAYER (Phil Eklund)
Player Red is an alien race based on Titan. The two human players must find a way to cooperate against the vastly superior aliens.
·       Titans. The Titan player uses the Radical Green Crew Card (Anonymous P2P Radicals).
·       Research. Research auctions are conducted like election auctions (N6), with the winning bid always going into the pool. The Titans must use the black side of its cards researched.
·       Titan Water Tank WT Depot. Titan boost operations and WT start in the Titan HEO (Highly Eccentric Orbit) space instead of LEO.
·       War. All players can attack and perform felonies.
·       Factories. No factories are allowed on D worlds or any of Saturn’s moons. Factories built by the Titan player do not lower the Exploitation Table. Black cards can be sold by the Titan player in LTO for 8 WTs each.
·       End of Game. The game ends if the Titan rocket enters LEO, or if an Earth rocket enters a site hex on Titan.
·       Victory Conditions. Each player gets 1 VP for each S factory (military base) and 5 VP for an outpost with at least one operational robonaut in the Rabbithole L3 point in the Mercury zone.
V7. QUICK 2-PLAYER (Scott Muldoon)
Quick Start: Each player draws one thruster card at random into his hand during setup.
Downtime: You may forgo your entire Move Phase to collect 1 WT. This means you may not move any spacecraft.
Research \& Recruit Operations: Use the solitaire bidding rules (V0b).
V8. THE RED GIANT (Phil Eklund)
After a long (!) dark age period*, Earth-based astronomers of our future species discover that Sol is about to flash into a Red Giant. This will engulf the planets out to Mars and make the inner solar system uninhabitable. To survive this crisis, the factions prepare to colonize the outer solar system, which will then become habitable.
*Our sun will expand into a red giant 5 billion years from now.
A. Bright Future Sun. Counterintuitively, as Sol burns up its nuclear fuel, it becomes brighter. Therefore add +2 to the solar thrust modifier of each heliocentric zone (including Neptune). Cards which are limited to given heliocentric zones can make it 2 zones further away.
(Environmental concern. This additional flux is more than enough to boil away the oceans and turn Earth into another Venus. The future species will have had to deal with this.)
B. Desiccation. All sites out to Saturn are one less in hydration due to solar heating.
C. Victory Conditions. VP are not awarded for any Glory, Claims, or Factories, except at sites in the Uranus zone or beyond. VP are also not awarded for Space Ventures. Space government VP is unchanged.
D. The New Earth. Each colony is worth 3 VP, but only if it is the Uranus zone or beyond.
V9. THE GRAND TOUR (Andy Graham)
This is an era of space exploration during which space budgets are driven by Discoveries and Glory. The game proceeds as normal with the following changes:
A. End Of Game. The game ends when there is at least one claim in each heliocentric zone and the requisite number of claims on TNO science sites:
·       Two in a two-player game.
·       Three claims in three- or four-player game.
·       Four claims a five-player game.
Space Ventures and Factories have their normal effects, but don't count towards End of Game conditions.
B. Victory Points: When calculating victory points, only consider Glory, Claims, Colonies, Space Ventures, Science Sites and TNO Science Sites. Each claim is worth 1 VP, plus the value of the Heliocentric thrust modifier. So, more distant claims are worth more.
·       You may receive VP for only one claim per Heliocentric Zone. Science sites, TNO science sites, and Astrobiology Sites are exempt from this limit.
·       Factories can be built as usual and have the normal in-game effects, but they don't count for Victory Points.
·       Rules for Glory and Politics are unaffected.
 
Example: A Claim at Mercury results in 1VP+2VP=3VP where as a Claim at a moon of Uranus would earn you 1VP+5VP=6VP.

V10. QUEST FOR GLORY (Andrew Doull)
This scenario has a harder set of victory conditions than the basic game, without the  complexity of the futures module.
End of Game Glories: Modify the end game conditions and glory achievements so that:
1. Each player must claim a glory by meeting the prerequisites to the basic game side of the card. If they then fulfill the requirements on the colonization side of the glory they have claimed, they win the game immediately (no VP is used for this scenario). Victory also occurs for the player holding the glory if no sites are left available to check for the Search for Life and Submersible Prospecting glories.
2. The Heroism glory can be claimed by any player as a free action at any time - there is no need to meet the requirements on the card.
3. Each player may only claim one glory card. Once all players have a glory claimed, any player can abandon an already claimed glory as a free action.
V11. ALTERNATE COLONIZATION START (Dave Hammond)
This set of variant quick start rules is meant to be played prior to beginning a game of Colonization with all modules. It is intended to simulate starting the game having completed a Basic game + support cards, and then introducing the additional Colonization components. At the resolution of this quick start, proceed with the normal rules of High Frontier as indicated by the living rules, with the next player being "Player 1".
Each player begins with 25 WT, a faction card, and their Bernal card. Set out the 6 white/black card patent decks - the three decks from the basic game (Thrusters, Robonauts and Refineries), and three additional support decks (Generators, Reactors and Radiators). 
Each player blind bids for the first turn. Each player secretly selects an amount of water, and all players reveal simultaneously. In the case of a tie, the tied players re-bid, adding zero or more WT to their original bid. In the case that two or more tied players bid equally on the re-bid, the tiebreaker is decided by die rolls until a winner is found. All WT used in the bid are sent to the pool - even for players who lose the bidding. Play begins with whoever wins the bid, and proceeds clockwise.
On their turn, a player executes one of two operations.
1. Research Legacy Operation: The phasing player examines all available patent cards, chooses one, and puts it in their hand. The player pays 1WT to the pool. A player may not choose the Research Legacy Operation if they have 4 or more patents in their hand (including Shimizu).
2. Industrial Legacy Operation: The phasing player chooses a non synodic comet site in the Ceres zone or closer, and makes a free prospect roll. If the roll fails, the site is busted and the player's turn is over. If the prospect roll succeeds an auction takes place initiated by the phasing player. Whoever wins the auction pays the pool their bid and places a claim disk and ET factory cube on the prospected site.
The pre-game immediately ends when there are a number of factories on the board equal to the victory condition in the legacy game for the number of players currently playing. All players retain all cards, WT, and claims that were acquired in the pre-game phase, and the remaining decks (Colonists, GW Thrusters and Freighters) are added to the game.
V12. NO POLITICS OR EVENTS (Andrew Doull)
The game starts with the politics token on an a space on the political diagram that all players agree to. Roll 1d6 if players cannot agree: 1-Capitalism, 2-Paleoconservatism, 3-Nationalism, 4-Press Gangs, 5-Egalitarianism, 6-Anti-Nuke. The faction matching the chosen politics is not considered in-power.

The following policies are modified: Paleoconservatism: Your research operations do not come with support cards.
Nationalism: Income ops only give 1 WT.

Note in both these instances, player White is affected by the policy.

Play otherwise proceeds normally except do not roll for events, and the Activism operation is not permitted. Hold an election at the end of the game to determine which player receives 5 VP, moving the election token directly to a policy of the winner's color.
V12-V17. STRATEGY GUIDES
V12. MISSION PLANNING (Andrew Doull)
There are 3 primary considerations whenever you plan a mission: 1) the number of burns required to reach the destination, 2) ensuring you have enough thrust to land at the destination and fuel to pay for the landinger burns using this thruster, and 3) ensuring you have low enough an ISRU to either prospect or refuel at the destination - noting you can refuel dirt powered rockets at any site. Of additional consideration is the site spectral class, the likelihood of prospecting the site and whether you need to cross any hazards to reach it - use these factors to guide selection of which acceptable destination to choose.

Fuel Requirements: To calculate the total fuel requirement for a one way trip, multiply the number of burns required to get to the destination by the number of fuel steps your thruster consumes. If you plan on switching thruster to a higher thrust rocket for landing, remember to calculate the fuel cost for landinger burns using the more expensive fuel consumption of this rocket. Until you begin producing black cards, a good rule of thumb is to plan for trips of no more than 6 burns in total distance divided by the fuel consumption of the thruster. This typically means the asteroid belt for most white card thrusters, and the moons of Saturn for the high efficiency white card thrusters (Metastable Helium and Pondermotive Vasimir).

Until you are confident of your ability to plan routes, just stop at line intersections, end your move and wait for the next turn to change direction, rather than paying the additional burn cost to change direction.

Thrust Requirements: You can only land on sites where your rocket’s thrust exceeds the site size at the time you plan to land. Remember that you must reduce the thrust of any thruster by the wet mass modifier, which may affect your ability to land or (more importantly) prevent you moving the rocket at all – especially if your thruster is solar powered.

Don’t make the mistake of thinking you can take off from any site you land on - the additional fuel needed for the return trip may increase the wet mass thrust penalty to the point where you have insufficient thrust. You can often avoid paying for a return trip by decommissioning your thruster and supports and boosting them again - noting that this may not be possible with crew.

Refueling: The ISRU of your crew and/or robonaut will normally determine whether you can refuel – it must be less than or equal to the number of hydration drops at the site you are planning to travel to. Sites are typically drier the closer they are to the sun. Dirt fuelled rockets have a significant advantage here as they can refuel at any site hex. Low mass rockets can significantly benefit from refueling en route - either at a high hydration site or by staging a fuel depot - as they get multiple fuel steps per WT of fuel if they keep their wet mass low.
If you are carrying a working robonaut and refinery you can build a factory to refuel at any site you successfully prospect.

Staged Rockets: When you begin carrying cargo such as robonauts and refineries, you will be tempted to build high mass rockets to carry everything at once. It can be better splitting your cargo into multiple loads ferrying each load separately - but only if you are able to keep the stacks mass low enough to get multiple fuel steps from each WT, or if your wet mass modifier prevents you otherwise moving or landing at a site.

This applies especially to solar sails, whose low thrust will prevent you moving much cargo at any one time. Take advantage of the fact you can decommission sails and boost them cheaply to save on return trips. The Mirror Steamer and other low mass rockets may also be used this way.
If you cannot benefit from low mass rocket stacks, and have high enough thrust, the additional cost of reboosting your rocket components may exceed any benefit gained from multiple missions.
V13. EASILY MISSED RULES (Eric Schiedler)
a) Remember the re-rolls for buggy ISRU's. Notice that these re-rolls raise chances of success on a comet from 16.7% to 30.6%.

b) A solar flare does not affect stacks on a site hex or on a radiation hazard space.

c) The MAG sail (only, not the other sail cards) receives a +1 acceleration boost if it flies through each radiation hazard! This makes for a powerful sail in the later stages of the game. It makes for a very nice Mercury trip with crew, etc.

d) Thrust modifying supports are NOT applied if the thrust card does not need them as support. This includes sails. The powered landing rules (G1) allow you to cruise to a large destination such as Ceres or Callisto with your main rocket, then switch to a high-thrust rocket (such as crew cards) to allow you to land.

e) A lot of 10•8 or 8•8 thrusters can't land on the large site hexes if they are carrying too much mass or because the fuel cost of landing burns is prohibitive. Large Rocket Stacks usually have wet mass net thrust -1. That's why a lot of factories will take 2 trips to large planets and certain asteroids are easy for factories.
This also makes Chinese claim jumping a time consuming affair—certain locations will require the Chinese to make 1 trip to jump the claim and a second trip to bring the factory. The solution is the +2, +4, and +8 net thrusters. But the ESA beamed power bonus of +1 becomes a powerful solution here. This series of effects of various rules, as listed under this paragraph (e), is not debatable; it simply must clearly be understood in order to effectively plan missions.
V14. BASIC GAME STRATEGY (Chad Marlett)
Concepts that should guide strategy that may not be clear from a quick read through the rulebook:
1. Exploitation Track: Each Factory of a given spectral type has a VP value (or WT value, for a product) that goes down in value as the number of factories increase. Depending on the board situation, you need to break the monopolies of the leader, and try to get some of your own.
2. Robonauts/ISRU: Of the 12 white robonauts, 3 are buggy, 5 are raygun, and 4 are missile. If you play the quick start and get a random robonaut, your first claim/Factory might be dictated. Otherwise, pick a robonaut aligned to your chosen target.
·       The missile robonauts are ISRU 2 or 3, and all act as inefficient thrusters. They are best at size 3+ sites with two or three water drops. The 10•8 can land on Luna, but doesn't have enough ISRU to prospect there. The two 5•4s have good targets at Hertha, Eichsfeldia, and Minerva.
·       The buggy robonauts are essential for the isolated size 1 sites to give you a decent success chance (Khufu, Deimos, Phobos, Phaethon, Olijato, Aneas, etc.) They are almost required for Mars—why go through the trouble of landing on Mars and not claim the whole enchilada?
·       The raygun robonauts have the highest mass, but they are the keys to the asteroid families and  do not have to land to prospect (unless the site has an atmosphere). A raygun robonaut has the only 1 ISRU in the white cards.
3. Refinery: There are 10 refineries. The mass 2 (basic game mass 3) ones are the easiest to carry. Try to get one that matches your first factory so that you can build black refineries there, but remember: There are no refineries of spectral type C.
4. Thrusters: The 0•0 sails are useful for early claims and glory in the inner sites, but have limited use in the late game. The 3•1 and 5•1 are quite useful. The 3•2, 5•2, and 4•3 are not very efficient and fairly heavy. The 3•4 solar thruster seems tricky to use, but it does weigh 0. The 7•4 would seem only to be useful on a few targets as a lander (size 6 Vesta and Ceres).
5. Crew: The crew with thrusters can go for glory or early claim on Mars - or Luna with an ISRU 2 robonaut. All of their ISRU ratings are poor, so only a few sites could be claimed. You can decommission crew at a Factory to form a colony for VPs—the only one-way trip you can really plan with crew, other than the PRC player. Note that the PRC crew must be present to claim jump, and other players’ crews prevent this.
6. ISRU Refueling: You don't have to have a claim to refuel at a site. This is handy at Deimos, since it is difficult to prospect but easy to land on.
7. Aerobraking: Don't forget that you don't have to meet the site minimum thrust requirement to land after aerobraking. Of course, taking off again is another story.
8. Failure Is Not an Option: Paying the 4 WTs to avoid making a crash or aerobrake roll seems like a tough call. A roll of 1 seems very unlikely, but ultimately, you have a lot more riding on the mission than 4 WTs. If you have the most VPs of any other player, you should probably play it safe.
9. Space Ventures: Aiming for a three-of-a-kind set of claims seems to be a decisive VP edge.
V15. ADVANCED GAME STRATEGY (Eric Schiedler)
These are strategy ideas for the expanded game (all expanded rules, cards, and both maps). These apply to 3–5 player games. The solitaire and 2-player games probably are a bit more tactical.
These strategies are presented in order of importance (my opinion).
1.   Reduce Mission Cycle Time.
Complete missions (successfully or unsuccessfully) as quickly as possible. It almost doesn’t matter what the missions are.
You can't get anything useful to advance you on VPs or the later stages of the game until you complete a mission. You can only really have one mission going with only one Rocket Stack. Other players are doing the same process of trying to get their next mission off the launch pad. Your decisions should focus on getting this time between missions down to a minimum. One key way of doing this is ensuring you can ET produce your robonaut and refinery in relative proximity - either because they are the same spectral class or because they can be produced by two sites on one moon or two asteroids in the same family.
I like to think of my mission cycle as the time between boost operations. I only want to boost once per mission, because I lose the opportunity of gaining 2 WTs in income (or more) for each turn that I boost. But, you likely have to boost for each mission, because you often decommission cards from the previous mission that you will use in the next mission. It's a convenient method to track your mission cycle time; you may find others.
·       Take Winning Risks.
As in many games, a strategy with a slight degree of risk is more likely to win than a conservative strategy that takes no risks. Of course, reckless risks are to be avoided. You are more likely to lose if you take more risks (because of higher mission failure), but you are more likely to win if your risks pay off, because they will lock-up valuable points.

·       Find Income Sources.
Researching then selling cards is an efficient income source. Even more efficient is researching then letting other players win the auction, because then you don’t bloat your hand and don’t have to spend operations clearing your hand of excess cards later. However, this is quite situational, as there is no cookbook way to make the most income. (Jeffrey Chamberlain)

·       Get Free Operations.
If I buy at an auction that another player has initiated, I saved having to use an operation. If a player sells me a patent card, I saved having to use an operation. If a player sells me a stack of water at a factory, I save several operations of refueling. Look for free operations, as they directly reduce the all-important mission cycle time.
·       Have a black card plan.
If you start one or two factories and you know which black card you will produce at them, and plan accordingly, you will crush your opponents. Black cards are powerful, but not if they just sit there—you need to have a plan to really make use of them. I want to point out that it's more important to have a crushing plan for your black cards, and to get that plan going, than to worry about how many victory points your factories will be worth at the end of the game. The end VP value depends too much on the actions of other players and random events. Turn your awesome black cards into solid VPs from space ventures that won't drop in value.
·       Store WTs for elections.
The final election in the game is worth up to 5 VPs. Either crush your opponents with a huge lead, or win this final election. Also, influence the election in the middle of the game so that the nasty effects of politics don't hurt you. I don't try to work too hard to try to win the 5 VPs for myself; I just like other people not to have the special advantages and 5 VPs.
·       Use the free decommission rule to make missions feasible.
All players can perform a free decommission at any time during their turn (except only the PRC player can decommission crew—but then, an election pushes this to anarchy/war and all bets are off). That means that not everything has to make the full trip (both one-way and round trips). Drop mass when you can! Plan to drop mass and fuel! Use multiple thrusters to jigger the mass/fuel effects! You can even drop everything but a thruster and crew card. As they limp back home, you can plan your next mission.

·       Have a small mission going if planning a huge mission.
A two-card Rocket Stack (or a one card robonaut/thruster Rocket Stack) can do something, anything! An efficient thruster and a crew card can try to claim glory points. This is not always an effective strategy if you are about to launch a mission, or are planning a medium-sized mission. But, if you are thinking of building a Monster Rocket Stack to take over Saturn or Jupiter, for example, you'll have the time to have a small mission going.
·       Learn to use the patents you have; don't fall in love with a card or a mission.
No card is perfect for every mission. No mission is always great; it depends on the stage of the game. No card is awesome unless combined with another complimentary card. No single purchase will get you through the whole game.
Space ventures require multiple (or huge) missions, so work these in depending on the flow of the game.
·        Don't be afraid to make deals.
How often have you desperately needed your ship to have one more thrust for the turn? Make a deal with the ESA, and you've got it. Need a robonaut that is powered by exotics and masses exactly 1? Talk to Shimizu, they've likely got something stored away in their databases. Need money now for that software upload? NASA always seems to have extra cash lying around.
For Victor Caminha’s brilliant ratings of High Frontier basic \& first edition expanded cards, see:
http://boardgamegeek.com/thread/686375/impressions-of-the-hf-cards-expansion-rules
V16. FREQUENTLY ASKED QUESTIONS (Joe Schlimgen)
Q: What makes a valid rocket? Does a rocket need a crew?
A: A valid rocket is any number of cards, regardless of type, with a total wet mass of no more than 32. In particular, a rocket does not require a crew (but a crew will protect the rocket from glitches in the expanded game). However, to be able to move, a rocket needs a working thruster and enough fuel to get it to its destination, and possibly the means to return. In the advanced game, the thruster may require additional support cards, such as reactors or generators in order to operate (and the support cards may require support cards).

Q: In order to build a factory, do you have to decommission cards with product letters that match the site's letter?
A: No. The product letters of the refinery, robonaut, and support cards decommissioned to industrialize are irrelevant. For example, in the basic game you could build a factory at a type `M' world by decommissioning a Basalt Fiber Spinning refinery (type `S') and a Kuck Mosquito robonaut (type `V'). However, the card you choose to be the factory product must have a product letter that matches the site's spectral type.

Q: The signpost from LEO to Mars lists three burns, but the colored path requires three burns and a Hohmann transfer (two more burns). Shouldn't the signpost list five burns?
A: Signposts list the minimum number of burns required to reach the destination. They do not list the Hohmann pivots because the rocket may stop at each transfer orbit and change directions at the start of the next turn without spending any burns.

Q: How do I perform an Apollo-style moon mission?
A: See H9.

Q: How realistic is it to land without using fuel?
%A: The largest asteroid you can land on without a lander burn is size 5, for instance Psyche with a surface gravity of 0.06 m/s2 and an escape velocity of 0.13 km/sec.  The wet mass needed to land a dry mass = 5 with a liquid chemical rocket with exhaust velocity of 4.5 km/sec on Psyche is straightforward using the rocket equation:  Wet Mass = Dry Mass times e^(delta-v/exhaust velocity) = 5 * e^(0.13/4.5) = 5.15.  Thus the amount of fuel needed is 5.15 – 5 = 0.15 tanks.
%
Q: What is the Solar Oberth?
A: The flyby rules approximate TWO effects: gravity slingshot and the Oberth effect. (You got to love those crazy German scientists. Thanks to the email correspondence with Professor Nathan Strange of NASA for describing both effects).

The slingshot effect describes the momentum transfer between a spacecraft and a planet (as the most useful example in game terms). For instance, by passing in front of an oncoming planet, the planet may speed up, and the spacecraft slows down, conserving momentum. Notice that this effect does not depend upon the rocket engine, so it works for unpowered or ballistic vehicles. Also, notice that this effect does not help the spacecraft with respect to the gravity field of the planet used in the slingshot. A slingshot by Earth does not help effect a capture by Earth, for instance. In general, using Sol for a slingshot is useful only for a solar system escape, e.g. using the Jupiter-Sol-Jupiter Exit (Y1.C).

But the second effect, the Oberth effect, describes a multiplier if one thrusts close to a planet, as opposed to further away. This is because if you are discarding propellant by expelling it, you gain an energy advantage if you expel it at low altitudes rather than at high altitudes. To give a terrestrial analogy, suppose you are at the base of a mountain that you have to climb. You are carrying a liter of water. You should drink it all before your trip, at the lowest altitude, and sweat it off during the ascent, rather than haul it to the top and then drink it.

The Oberth multiplier does not violate conservation of energy. Remember that whenever your rocket thrusts by expelling propellant, some of the energy goes into moving the rocket and the rest into moving the propellant. (Swimmers have the same problem, wasting as much energy moving water backwards as moving the swimmer forward. Joggers have it easy, direct momentum transfer with the planet Earth.) If you thrust your rocket at periapsis, more of the energy goes to moving your rocket and less "wasted" into the propellant.

The Oberth delta-v multiplier is equal to the square root of one plus twice the escape velocity divided by the delta-v burn. Thus it is most effective for high thrusts at fast speeds, such as a periapsis close to a high gravity world with a chemical rocket. It is not so good for an electric rocket, which can only expel a trickle of propellant at the interval when it is going the fastest. And, it is useless for sails and ballistic spacecraft.

The Solar Oberth, as marked on the map, takes you through 5 burns and allows you a slingshot equal to your unmodified thrust (plus one if you afterburn). The solar escape velocity in this region is 68 km/sec, so the multiplier comes to 3.4 for a thrust 6 spacecraft able to execute these burns in a timely fashion. This is comparable to the net game value of 6 – 2 = 4 (two burns are lost exiting the Solar Oberth region).
 
Q: Why are the planets static?
A: A partial answer is that for a map of potential energy from Sol, the alignments do not matter for nearly-circular orbits, only the distance from Sol. Only for highly-eccentric comets, like Halley's, do I bring synodic alignments (i.e. launch windows) into play.
One conceptual difficulty with having simulated how planets move, even in a computer game, is that the Hohmann transfer orbit between two planets occurs when the two worlds are the furthest apart. How to explain to a player that he must wait until he must travel the greatest distance? Not intuitive at all.
But the biggest reason for not having moving planets is, if starting at LEO, and going to every destination except Mars, your wait for the perfect astrology (i.e. planetary alignment) is less than a year. If going to an inferior destination like Mercury, and the alignment of the planets happens to be the worst possible, just wait half of a Mercurian year (44 days), and it’s perfect. If going to a slow planet—like Saturn, which is practically "fixed" in position ("mercurial" means "fast," "saturnine" means "slow")—and again your astrology is the worst possible, wait one-half of an Earth year, and it will be perfect.
A fourth reason is, for high ISP rockets (including most of the ones in the game), the difference in energy between the worst and best synodic positions is less than 15%. Only in low-specific-impulse rockets with miniscule delta-v (i.e. all NASA missions until Dawn) do such celestial billiards become significant.
It happens that the Earth-Mars trip is the only planetary trip from LEO where the wait for favorable alignments might be longer than the High Frontier turn length (1 year). None other than Bob Zubrin, famed for the Zubrin Drive and Mars Direct, reviewed High Frontier before it went to press. Bob requested omitting the asteroids to concentrate on Mars, and using moving markers to represent Earth and Mars, or perhaps switching to polar coordinates. I objected that with decent specific impulse rockets, alignment did not matter much, but did agree to rework the map to allow for Conjunction versus Opposition routes to Mars. The latter route, using a Venus slingshot, is favored for the Zubrin Athena Mars Direct plan (see endnote {11} in Volume I, where these routes and Mars Direct are illustrated).
V17. ADVANCED FLIGHT PLAN (Andy Graham)
 (also contributions from Wulf, Tarwin, Francisco, Sam, Jeff)
Rule one is to have fun. You can have a great game of Colonization without getting anywhere near winning, especially when that mission to distant Pluto looks much more exciting than just building another factory. But, whatever your goal, having a plan helps you get there! So, here is my plan to help launch your game of Colonization (all modules assumed).

A. PIONEER COLONIST. Earlier is better when boosting colonists. Financiers are best in the early game, and scientists in the late game. However, because you are going to only have one or two colonists total, it’s usually best to boost an engineer, because they are the best over the long haul.
·       Get a generator boosted and your Bernal promoted. This activates your Bernal ability and gives you a protected location for your colonists and rocket. But, generators are in short supply, so don’t delay. Remember that three of the freighters provide generators. If the generator requires radiators, you will need to replace them unless, they have a high-enough rad-hardness to survive events.
·       If you have an easy way to shift stuff out of the Bernal to an alternate safe location (with a sail, for example), there isn't necessarily any hurry to get it promoted, if you are the UN or the ESA, since their Bernal bonuses are more relevant mid/late game, rather than in the early game.

B. ROBOTIC PROSPECTING. Often, your first robonaut determines your strategy.
·       Buggies give you the best chance to prospect a site, especially missions to Hertha or the Trojans.
·       Rayguns might save you from needing to make a landing and can even prospect several sites at once. With a working ISRU 2 raygun robonaut, go blanket-prospecting or try space ventures (more so for the UN).
·       Missile robonauts prevent you from needing to take a separate thruster card. If you can get a fuel factory going with just a missile robonaut, you can bypass step C below and go straight to setting up a GW thruster.

C. THRUSTER. Try to get a thruster with supports that match your robonaut, but don’t waste time on perfection either. Get your mission going with whatever you can throw together. Remember, you can have more than one thruster on a rocket, such as high-thrust crews for that landing on Vesta. Crews are also handy against glitches.
·       Never get too attached to a specific card or a specific spectral type. Change plans if you fail a key prospect or card acquisition.
·       You may not need a MW rocket if you use your Bernal to get to the first site for your factory. If you don’t use solar power, make aggressive use of flybys, and have a really good lander (such as certain legacy Crews); your Bernal can carry everything you need (up to a mass of 32). This may not be an option for the UN player, who is generally one burn more distant from everything than the other players.

D. PLAN YOUR FIRST MISSION. You will be getting technology and throwing together a mission opportunistically. But, do not fail to check the black side of each card and think, “Where can I build this?” Early missions include:
·       Avoid the race to Mars. Although it’s only a burn away, Mars can be a trap if your long-range plan is to reach TNO and/or claim futures. It just takes too long to build up from there to attain futures or grab TNO labs (which score a minimum of 10 VP per cube). Your diversity suffers at Mars too, especially for GW rockets, because only C is available. 
·       A quick dash for Glory. With just a crew card and a thruster (especially a sail) you grab quick Glory Points.
·       The W-class asteroids. Lutetia and Hertha (both “wet metallic,” see footnote on page 3) have 3-drops and with a buggy have a 75% shot of bagging the hard-to-get M spectral class.
·       The minor planets. Ceres and Vesta are both big enough to make prospecting a sure thing, and by making a Mars flyby you usually need only one burn to reach them (plus a half burn of landing fuel). Save fuel by landing on smaller worlds in the same family, and refueling for the larger worlds if prospecting does not work out. Or, bring along a high-thrust chemical rocket for landing.
·       The Galilean Moons. The four big moons of Jupiter have science, certain prospects, and enough water to support 4 colonists should you move your Bernal there. Callisto is 5-burns away (or less if you use flybys), so you will need a fairly good rocket or lots of WTs. The moon Io, although radioactive, hazardous, and dry, is a push factory site of spectral class M.

E. SET-BUILDING. Always strive for cards that match the product letter of your claims. So, while you are making your first claims, get that hand ready and have your refinery boosted. It is often better to leave the robonaut at your claim as an outpost and head back to pick up your refinery. You may find that after building your first factory, your rocket is no longer functional due to losing supports. Sometimes you can rescue it by production, digital swap, or another mission. Other times, it’s simpler and quicker to scrap your first rocket back to your hand and to boost it again. Or, just take an additional support with you.
·       Helium-3. Early claiming of an S site is worth the trouble, since S is the most valuable, followed by D, M, V, and finally C.

F. RETURN ON INVESTMENT. Money is still important in the midgame. So, right after establishing that factory, build all your black cards of that product letter, and ship them to LEO or another player’s Bernal. If you are the first to industrialize a product letter, you get a free freighter that you can build and use for shipping. Or just move it to LEO and sell it. Be sure to check if its future is interesting.

G. GOING FOR BLACK. Start stockpiling factory products needed for the game win.
·       A GW thruster can get you to the science sites, grab a space venture, or move your Bernal. But, it takes a long time to fuel one, and don’t forget that you need enough fuel to return to your fuel factory.
·       Robot Colonists are not limited by your dirtside hydration, so they are worth having. Except for the Smart Pets, this requires a D factory.

H. HEAVY HAULAGE. If by now you have at least one claim in one of those sweet spots with 6 to 8 drops of combined hydration, or at least 4 drops and science, then it’s high time to move your Bernal!
·      You are limited to a dry mass of 23 and a total mass of 32. You will need a lightweight rocket with a thrust of three or more, and may need three trips to bring the Bernal and its supports and your colonists.
·       If using a GW rocket, you may be able to share supports.
·       If you move your Bernal on its own dirt power, a glance at the fuel strip shows that something this heavy will have a maximum range of about four burns. Take your lightweight crew or an electric robonaut for refueling, and move from one rock to another. Your first rock is often in the Mars zone: Deimos, Didymus, Oljato (wet and metallic!), or Toro (science!). But it is possible with two burns to get into the main asteroid belt as well. To go anywhere except 
Venus or Mercury, it will need nuclear power, as solar power has insufficient thrust and range.

I. HOME ON LAGRANGE. If all is per plan, you should have the capacity to recruit and boost more colonists. Remember, colonists need only to be in space to do work. They don’t have to be on the Bernal.
·       Lab Work. You will need a Lab to promote cards and have access to futures. If your Bernal is not at a science site, you will need a GW thruster to industrialize a TNO science site.

J. MEANWHILE, BACK HOME (Peter Card). Once your Bernal has moved on, when to build a replacement is a tricky question. An early Ersatz restores your Bernal Privilege, keeps the WTs rolling in, and simplifies the logistics.

K. LOOKING TO THE FUTURE. The endgame consists of promoting cards and grabbing futures.
·       When promoting colonists, remember that they change their loyalty which may either benefit or hinder you. Consider changing politics to War to keep your colonists loyal.
·       Abandon claims or factories if something better appears.
·       Futures are not the only way to earn a victory. You can win the game on factories and space ventures. Robot emancipation and politics are more often a means to an end than an end in themselves.
·       Once your freighter is promoted, swap your big cube with your small cubes, as required. For instance, manufacture a black card, then swap the big cube and use it to move the product to where it is needed. And the reverse is true when needed your big cube can be a factory.
·       If politics is important for you, watch out for the game-ending moves of your opponents and keep a lot of WTs for the endgame election.

“Having a one-page strategy guide in High Frontier is like having a page of suggestions included with a jumbo box of Lego blocks.” Phirax

W. EXOMIGRATION \& EXPLORATION MODULES (Andrew Doull \& Pawel Garycki)
Summary: The experimental exomigration modules W1 to W6 increase the scope and detail of the political side of the game by making space politics a dynamic system that can evolve over time independently of the players. This allows politics to be used in solitaire game while making political choices in the multiplayer game more interesting. The experimental exploration modules W7 to W10 are for those who wish a more exploratory approach.  All ten modules are independent of each other.
If you decommission a colonist to build a colony, you may move your claim disk from the colony to the Interplanetary council (W1), where each claim disk gives you 2 votes in every election and 2 VP. Independent colonies still act as a claim for factories, require another human present to decommission them as a felony, and your colonies become independent during War if you control the New Attica Secessionists.
All five players are included in the election auction even if they are not otherwise present in the game (W2). Elections won by absent players or by players who don’t reach the political threshold (W5) move the counter randomly towards the faction’s political position. This allows the Politics to change in ways that may not benefit players playing the game.
An election should always occur every time the sunspot disk transitions through the blue sector on the Sunspot Cycle (W3) unless the Politics is Authoritarian - this is enforced immediately before entering the yellow sector.
The face up colonist on the Colonist Deck votes in elections (W4), with a chance equal to the Suffrage op chance of the votes coming from the promoted side of the card.
The political threshold (W5) accrues one counter every time the game exits the blue sector (after the mandatory election, if any). This is the minimum number of votes an election winner must win by to be able to choose where the politics token moves - otherwise it moves according to the political struggle (W2), if that rule is used. 
Once a colonist future has been achieved, the faction the colonist is loyal to always gets their votes and the special ability text on the promoted colonist required for this future is applied to all factions (W6) and persists even if the colonist is demoted, decommissioned or removed from the game. ‘This card’ applies to all promoted colonists; ‘this stack’ applies to all stacks.
Exploration Module. Both sides of glory cards can be claimed (W7a \& b). They have additional requirements to meet and they confer rewards as well as victory points to the explorer. There are 4 new ventures (W7c).
Endgame conditions are modified to be dependent on the total amount of achieved glories, ventures, starshots and futures (W8). Therefore the game is faster than when playing just the Endgame Module (however it is still recommended).
Mercenary crews can be hired for a price. It is used for a better net thrust or a buggy/missile ISRU ratings which either makes access to larger sites easier or prospect of sites better and faster (W9).
New exploration tokens for additional VP for TNO Science, astrobiological, and subsurface ocean sites (W10).
W1. INTERPLANETARY COUNCIL
If you decommission a colonist to make a colony you may choose to make the colony independent. Remove your claim disk from the site and place it in the Interplanetary Council. Your claim disks on the Interplanetary council each give you 2 votes in every election and are worth 2 VP (instead of 1 VP as a claim). You can continue to use your small cubes as factories at your independent colonies as if you had a claim there (and your big cube if you have a promoted freighter) and the colony gives you 2 VP at the end of the game as normal.

Decommission. If your independent colony is decommissioned, you must withdraw one of your claim disks (if any) and place it at the site of the colony claim. You must have another human present if you decommission independent colonies as a felony, and the colony must be decommissioned before withdrawing or moving your factory at the same site.

Revolution: The player controlling the New Attica Secessionists makes all their colonies independent if the politics is or changes to War.
W2. POLITICAL STRUGGLE
If playing with the Colonist module (R), when conducting the election auction, all five factions are considered part of the auction, even if they are not otherwise present in the game.
Each faction accrues votes from the electoral buttons of boosted colonists if the Politics is not at War (K4), and from all exomigrant colonists (W4) regardless of whether the policy is at War - but only factions with players present in the game can add votes through the blind bid process and these votes can only be added to the players’ own faction.
If the faction who wins an election is either missing from the game or did not win by enough votes to reach the political threshold (W5), the Politics disk moves one step towards the policy or policies of their colour along the shortest path, taking either the north or west (1-3) or south or east (4-6) path if equal length choices exist.
Flipping. If the winner is missing from the game or did not reach the political threshold and is NASA or ESA and they are already in power, they switch policy half the time (1-3) to the other policy of the same color.
W3. MANDATORY ELECTIONS (DEMOCRACY)
An election must occur at least once each time the sunspot disk moves through the blue sector of the Sunspot Cycle (Democracy) - counting elections initiated from the Sol Event Table (M3) and from the Activism operation (N6) regardless of who initiated it. If an election has not happened this time through the blue sector by the time the sunspot disk exits the blue sector of the Sunspot Cycle, it occurs automatically (and at no action or operation cost) before the counter is moved into the yellow section of the Sunspot Cycle.
EXCEPTION: An election is not required when exiting the blue sector if the current political status is Authoritarian at the time and it will not be triggered by the exit when this policy applies.

Politics Note: While uncommon in presidential elections, there is a tradition in parliamentary democracies of allowing the election date to be chosen by the in power party in order to maximise their reelection chances. It is also possible for an opposition party to force an election to be held if it is able to win a vote of no confidence or block the passage of legislation critical to the ongoing governing of the country, such as a budget or supply bill.
W4. EXOMIGRATION
If playing with the Colonist module (R), the face up colonist on the Colonist deck, if human, will influence the election auction by adding their votes to the auction total. To determine how the votes are cast, before the blind auction, roll 1d6. If the number rolled is less than the number of cards in play on their purple side (including Freighters, Bernals, Colonists and GW Thrusters), the colonist will contribute the votes from their promoted side (turn the card over for the duration of the election auction), otherwise they will contribute votes from their unpromoted side. Because they have no employer, their allegiance does not change while politics is at War (K5).

W5. POLITICAL THRESHOLD
When you enter the yellow sector, place one clear blue counter (after conducting any mandatory election) next the end of the blue sector. The number of clear blue counters here represents the political threshold. This is the minimum number of votes (electoral boxes and WT) that an election winner must win by in order to be able to choose where to move the politics token. If this threshold is not met, the politics token moves according to the political struggle.
W6. COLONIST FUTURES
Once a colonist future has been achieved, the colonist’s electoral boxes always contribute votes to their faction (instead of being affected by war loyalty) and the special ability text on the promoted colonist required for this future is applied as if all factions control the card. This represents a permanent change to the human society and persists even if the colonist is demoted, decommissioned or otherwise removed from play. The following specific modifications also occur:
1. Text referring to ‘this card’ applies to all promoted colonist cards.
2. Text referring to ‘this card’s stack’ or ‘colocated stacks’ applies to all stacks.
Note: The text on the Creeper Neogen referring to the black side limits this specific ability to promoted robots only.

Example: The Supreme Cult Leader future is achieved. Now, no robonaut or robot controlled by any player can perform an op without getting permission of all other players unless a human is present, and all other players get 1 WT each any time anyone performs a FINAO. Player White gets the 2 votes from the Josephson Implants, regardless of whether it is peace or war time or the card is not in play.

W7. DOUBLE-SIDED GLORIES \& VENTURES
Players can claim both sides of Glory and Venture cards. Whoever claims one side of the card has an exclusive right to claim its reverse side. Victory points from both sides are cumulative.

Handling of multiple Glory subconditions: If the humans are killed on the way to Earth’s surface or a lab, visit the requested bodies again. The condition of being first means being first to collectively fulfill all subconditions together. Different humans can visit different bodies, provided all of them return to Earth’s surface, each fulfilling the given subcondition. The phrase “yet-unvisited” means that the player cannot use this body for the glory if another player has visited it first with humans or it has been already used for another glory, however if there are no such sites left, the phrase is ignored. All "random" site characteristic additions are performed immediately after fulfillment of the glory.
 (Red font indicates an addition to the printed cards)

A. BASIC SIDE OF GLORY CARDS
Messenger of the Gods/Pushing the Boundary – First to land humans on Mercury and another yet-unvisited Push Factory body and return safely to Earth’s surface.
Reward: You can consider Mercury North Pole to be a subsurface ocean site with a 3d6 roll instead of 2d6.
God of War/Vision for Space Exploration – First to land humans on Mars and another yet-unvisited atmospheric body and return safely to Earth’s surface.
Reward: You can consider a random Mars site to be an astrobiology site.
Doomsayer of the Gods/One Big Leap for Science – First to land humans on a science site and another such site of yet-unvisited 5+ size body and return safely to Earth’s surface.
Reward: You can consider a random Chiron/Methone/Asbolus site to be a TNO science site.
King of the Gods/HOPE: Human Outer Planet Exploration – First to land humans on a site in the Jupiter zone or beyond and another such site of yet-unvisited 5+ size body and return safely to Earth’s surface.
Reward: You can consider a random Ganymede/Callisto non subsurface ocean site to be a subsurface ocean site.
Father Sky/Following Voyager – First to land humans on a site in the Uranus zone or beyond and another such site of yet-unvisited 5+ size body.
Reward: You can consider a random Triton site to be a TNO science site.
Heroism/The New Space Race – First to land humans on sites in three different heliocentric zones one of them being a 5+ size site of a yet-unvisited body.
Reward: You can consider a random Io site to be an astrobiology site.

B. ENDGAME MODULE SIDE OF GLORY CARDS
Golden Apples of the Sun – First to decommission a refinery plus non-radiator supports on the Kreutz Sungrazer and a working raygun robonaut in the Rabbithole.
Reward: You have +1 rad hardness during Solar Flares/CME rolls and for radiation belts during the red sector.
Search for Alien Life – First to discover ET life and return the humans safely to the Earth or a lab. Additionally must land humans on Titan. To search for life, have a human spend a Science Op. in an astrobiology site. Roll 1d6 to explore; a ‘1’ or ‘2’ = success. Each site may be explored once.
Reward: Your industrialized (for Halley: claimed) sites with alien life are 'living' labs and each one lets you add 2 to your epic hazard op roll. If you already meet the criteria for the epic hazard roll to fail only at ‘1’, reroll a failed roll instead.
Submersible Prospectors – First to subsurface prospect and return the humans to the Earth or a lab. Additionally must land humans on Europa. To prospect, make a special 2d6 buggy prospecting roll at one of your claimed subsurface ocean sites. Roll for each site; only one attempt per site. Succeed if 2d6 <= size. Human presence at the site is required when special-prospecting.
Reward: Your special-prospected sites have increased hydration by 1.  and the corresponding dirtside hydration is rounded up, not down, provided you have at least Possession of at least 2 such industrialized sites (anywhere) lets your dirtside hydration be rounded up, not down.
Search for CUDOs – First to land humans on your claimed TNO science site and return safely to Earth’s surface. Additionally must land humans on Triton.
Reward: When industrializing any TNO site, you may change its spectral type to S (for Helium Scarcity: H). Only one such change is allowed per gamethat TNO site, you may one-off change the spectral type of the site to any chosen.
Saving the Earth: First to decommission a refinery plus non-radiator supports on both Hermes A and B and unless it is done in the first 24 turns, also decommission a refinery plus non-radiator supports in a Jovian aerobrake (It will watch for impacting comets).
Reward: You can access synodic comets 1 year before or after its solar period.
Outer Diaspora: first to simultaneously have a space colony in 4 heliocentric zones with at least one of them on an aerostat site.
Reward: Each of your colonies on an explored site adds two additional votes.

C. VENTURE CARD CHANGES
Ignore the former legacy side of Venture cards and use the texts from below instead:
Reverse of Space 3D Printing: Rare materials (4VP).
Each financial op. (excl. income) profits 1WT at the end of your turn.
First to have 3 D site claims and pay 5WT.
Reverse of Space Pharmacy: Helium Infrastructure (4VP).
You can factory-refuel 2 more tanks of fuel of any grade.
First to claim 2 aerostat sites and pay 5WT.
Reverse of Space Tourism Venture: Mass beaming (4VP).
Your push factory pushes +3.
First to claim 2 push factory sites and pay 5WT.
Reverse of Space Elevator Venture: Extraterrestrial Agriculture (4VP).
You can ET produce ‘C’ products at astrobiology sites. ET production at ‘C’ astrobiology sites profits 1WT at the end of your turn. 
First to claim 3 astrobiology sites and pay 5WT.

W8. ENDGAME CONDITIONS
The game ends 2 events after the required amount of achievements (glories, ventures, starshots, futures) is reached: 2 players = 9, 3 players = 15, 4 players = 20, 5 players = 22. The game ends also 12 turns after the Footfall future is announced or when all players have performed a starshot (Y1).
Futures are worth 3 achievements each and they are no longer necessary to end the game.
Starshots (Y1) are worth 2 achievements each and 3VP.
Each side of Glory and Venture is worth 1 achievement.
When the game ends, execute an endgame election.

UNRESTRICTED AMOUNT OF FUTURES
It is possible to score any amount of futures. All 5+th futures are worth 4VP.


W9. MERCENARIES
If your crew is unboosted, you can boost a mercenary crew from its reverse side. They are helpful for landing on glory-rich heavy gravity sites (high thrust) or for a better prospect operations (buggy, raygun).
Faction privileges do not change.
If mercenaries are decommissioned due to anything other than a ticker tape parade or a colony, then you are no longer allowed to boost mercenaries.

FEES:
1WT fee should be paid whenever:
using a mercenaries’ thrusters for landing or lift-off or a hazardous flyby (X5),
using a mercenaries’ ISRU ratings for prospecting or combat,
establishing a colony with a mercenary.

W10. EXPLORATION TOKENS
Place exploration (clear red) tokens at the beginning of the game on each site which can be explored to achieve the Search for Cudos (TNO science), Search for Alien Life (astrobiological) or Submersible Prospecting glories. Take the exploration token when you attempt the operation described on the card to achieve glory - this requires humans to be present (and a buggy for Submersible Prospecting). This exploration token is worth 1 VP regardless of the success or failure of the roll for glory. You may continue to acquire exploration tokens in the same manner even after the appropriate glories have been achieved.
Whenever a site becomes relevant for the token placement due to certain glory rewards (e.g. new astrobiology site on Mars), place additional tokens then (even if there are already such).

X. SIMULATION MODULES
The following experimental modules are independent of each other:
·    Bernals start ‘on the ground’ instead of boosted (X1).
·       Certain generators have open cycle cooling which gives them a +1 thrust modifier and 1 therm of cooling if expending 1 fuel step of fuel (X2).
·       For takeoff, landing and avoiding radiation hazards, GW/TW thrusters only increase their net thrust by 1 if they afterburn instead of the increased afterburn value shown on the card, and sails regard their first pivot as 1 burn instead of 2 (X3).
·       Alternate solar flux values for Ceres zone and beyond (X4) apply to solar-powered rockets to more realistically model available solar power at these ranges.
·       The ability to declare a flyby to be a hazardous burn (X5) to improve the free burns it provides if you meet a minimum thrust requirement and spend fuel to get through the burn.
·       A new spectral class H (X6) replaces the spectral class listed for the gas giant aerostats and the Lunar Astrachus Plateau site. Cards which use Helium-3 become H spectral class.
·      Factories at Astrobiological and Subsurface ocean sites add +2 WT when performing a Site Refuel of water or isotope fuel.
X1. BERNALS START ON THE GROUND (Phil Eklund)
This module starts the player factions on the ground, with no space presence. It adds about 15 minutes per player to the game.
Setup. Players start with their Bernal Card in their hand, not in space. It must be boosted like any other white card (into LEO at a cost of 10 WT).
Privilege. You start with your faction privilege deactivated, but it becomes permanently activated as soon as you boost your mass 10 Bernal into LEO.
Humans In Space. You can boost crew, but you cannot boost any human colonists until you boost your Bernal. A colonist in a Bernal at LEO is safe from the Pad Explosions/Space Debris event.
Home Orbit. As soon as you move your Bernal to its home orbit (see Q3), you can boost cards directly to your home orbit, and store your WTs there (Q). This also activates your Bernal privilege (Q7).
Hazard. It is possible to destroy the unpromoted Bernal with a hazard roll or an epic hazard op. Then it starts again on the ground. Combat cannot destroy it.
Lunar Treaty: Luna is not busted at start. Reactors and cards with onboard reactors are not allowed to be ET produced there. Each operation involving a Lunar site other than ISRU refueling costs a 5 WT fee. 
Ersatz. You cannot boost an ersatz-Bernal (Q4) until your Bernal is moved to a dirtside.
X2. OPEN CYCLE GENERATORS
The Cascade Thermoacoustic, MHD open cycle, Z-pinch Microfission, and Granular Rainbow Corral generators have open cycle cooling which gives them a +1 thrust modifier and 1 therm of cooling for expending 1 fuel step of water or isotope fuel. This is cumulative with a thruster’s afterburn, if the afterburn fuel cost is expended, but can also be used to cool generator powered robonauts and other supported cards which afterburn may not be able to cool.
Note: GW thrusters are not modified by supports other than solar power so the +1 thrust modifier does not apply to them but they can benefit from the 1 therm additional cooling.
X3. TORCH SHIPS AND SAILS
In order to model a wide range of different thrusters, the game made some approximations for continuous thrust rockets such as GW thrusters and sails. These rules are more realistic:
GW and TW thrusters typically have a low thrust but high fuel efficiency which allows them to expel propellant for the entire duration of the journey (torch ships), which is modelled using their high afterburn bonus. To more correctly simulate these thrusters, their increased afterburn bonus should only apply for increasing their thrust movement points (TMP) used to calculate how many burns and pivots they can enter in a turn. For all other uses of thrust - e.g. take off, landing and avoiding rad-hazards - consider the GW/TW thruster’s net thrust to be only increased by 1 if the thruster afterburns.
Sails are capable of a greater delta V per year than their thrust would suggest. To reflect this, the first pivot moved through each turn by a sail costs 1 burn instead of two burns. This also applies to the Calypso2EcoWarriors/Wet Nano Ecologists colonist thruster.
X4. MODIFIED SOLAR FLUX (Dr. Noah Vale)
The following solar flux values for Ceres zone and beyond apply to solar-powered thrusters (and thrusters with solar-powered supports) to more realistically model available solar power at these ranges.
Zone
New Flux
Ceres
-3 thrust
Jupiter
-5 thrust
Saturn
-7 thrust
Uranus
No solar power
Exception: The Electric Sail uses solar wind instead of solar power, so use the existing on map thrust modifiers (as do Solar Flare / CME events). Green Bernal’s Collimator makes the genuine Bernal’s thrust immune to the above solar flux modifications.
X5. HAZARDOUS FLYBYS (Dr. Noah Vale)
You can elect to make a hazardous flyby in order to increase the free burns that you receive from the flyby space, as follows:
·       The flyby or moon boost space becomes a hazardous burn when you attempt a hazardous flyby (spend a burn to enter, and roll for a hazard or FINAO). You cannot use free burns from another flyby to traverse this burn - you must spend fuel.
·       Net thrust must be ≥ 9 for a moon-boost
·       Net thrust must ≥ 8 for a slingshot of 1
·       Net thrust must ≥ 7 for a slingshot of 2
·       Net thrust must ≥ 6 for a slingshot of 4 (Jupiter)
·       The Solar Oberth is modelled using its own hazardous burn rules.       You receive additional free burns by completing a hazardous flyby are equal to the original flyby bonus plus one.
·       The same slingshot can only be used once during a move. Only one hazardous flyby (including Oberth) is allowed per move.

Example: The player performs an Oberth over Venus (+2) with his NERVA 7*4 rocket. He performs a burn over the Venus Flyby, using 4 steps. He safety rolls a 3 on his hazard roll. He now has 3 free burns (for a net of 2, since he used a burn over the Flyby L-space) in addition to the 2 from the Venus flyby for a total of 5 free burns (net of 4). He uses it to reach Psyche.
X6. HELIUM SCARCITY
Helium-3 is a rare isotope used in the clean 3He-D fusion reaction (clean aneutronic reactions are necessary to reduce the need for heat rejection). In our solar system, helium-3 is found only in gas giants or on Luna. Therefore, the three gas giant aerostats (Saturn, Uranus, and Neptune) and the Lunar Astrachus Plateau (0 hydration formerly M site) become H spectral class sites (a new class) replacing their existing spectral type, and use an H spectral class exploitation track for VP determination and black card values.

The following 6 cards become H spectral class cards, as either H Isotope refueling, digital swap H, flip side H or H products:
·   	Project Orion / Project Valkyrie reactors
·   	Penning Trap / 3He-D Fusion Mirror Cell reactors
·   	D-T Fusion Tokamak / Antimatter Bottle reactors
·   	D-T Vista / Daedalus 3He-D Inertial Fusion GW thrusters
·   	Spheromak 3He-D Magnetic Fusion / Colliding FRC 3He-D Fusion GW thrusters
·   	Z-Pinch D-T/6Li Fusion / Z-Pinch 3He-D Target Fusion freighters

When using the Exploration Module, the Helium Infrastructure venture requires 2 H claims (instead of 2 aerostat claims).

X7. ASTROBIOLOGY \& SUBSURFACE OCEAN FACTORIES
When refueling at a factory at an astrobiology or subsurface ocean site, add two WT to each Site Refuel operation (H5) when refueling with water or isotope fuel. 
Y. HIGH FRONTIER INTERSTELLAR SOLITAIRE
Copyright © 2015, Philip Eklund. VERSION April 27, 2015
Developer: Neal Sofge of Fat Messiah Games, Andrew Doull
 
Playtesting: Francisco Colmenares, Sam Williams, Andrew Graham, Wulf Corbett, Daniel Eliot Boese, Paul-Michael Agapow, Dan DiTursi, David Harris, Adam LaJuene, Pagou, Thompson Stubbs, Pawel Garycki
 
“The Drake Equation, used to determine how many interstellar civilizations exist, indicates the determining factor in whether there are large or small numbers of civilizations in the universe is the civilization lifetime, or in other words, the ability of technological civilizations to avoid self-destruction.”
 
To journey to the stars, you will need a good starship engine, the right team of humans and robot supports, and enough fuel tanks speed up to cruising speed and then to slow to a dead stop at the destination. If you do not have enough fuel to stop, the dry mass should be reduced while underway, either by dust erosion of the Bernal’s shielding, hull reconfiguration, or, as a last resort, jettisoning life support. Your pilots have the special ability to brake the starship without using fuel, by drogue braking at gas giants, heliopause bow shock surfing on the Local Interstellar Cloud (LIC), using antimatter-fueled beam-core rockets, or using interplanetary magnetic fields. Your scientists dream up new ideas, which your engineers make into reality using 3D printers and nano-reconfiguration refineries. Your engineers also repair dust and radiation damage, while your entrepreneurs keep the passengers sane with products and services. Marriages help reduce stress and provide for the next generation. Raygun crew are able to beam-push ultralight probes, either to potential destinations or back to Earth for help. Spacewalking robots and robonauts fix punctured radiators and land on planets to explore them. However, the passengers both age and accumulate stress, and can mutiny if not aligned with the Politics. Radiation can interfere with reproduction and increase aging and cancer enough to turn your starship into a ghost ship.
 
A. NEEDED TO PLAY.
You will need a copy of the Interstellar Poster-Map, which includes the Starship Government, Event Table, and Rocket Diagrams. This map is available at Zazzle. http://www.zazzle.com/sierramadregames
 
B. INTERSTELLAR GLOSSARY (Capitalized terms defined here).
Age Track. Cards are placed in this track to indicate how old they are, see Y1.E.

Beehive. A Starship on a comet using a thruster with the Beehive thrust icon (Y1.B). Beehives always moves one space each turn and do not track mass or fuel, and have a 50% chance of survival in a desperation braking maneuver through a star (Y4.D). Cards aboard a Beehive are handled differently in three respects: Colonist Cards are recycled to the bottom of the Colonist Deck when killed. and the drydock and bioengineering operations can be performed while underway to rebuild the hull and colonists respectively. Mutinies and gray goo outbreaks are more virulent in a Beehive resulting in beehive warfare (Y6.D) and Beehive gray goo war (Y6.C). Retiring a Passenger to the vats (Y4.I) is felonious in a Beehive, and moving a passenger to the Vats using any other means incites a Vats Mutiny.
 
Bernal Stack. The ship’s passenger section, containing both dust shields and EM radiation shields. The more Bernals a ship carries, the more Humans can be on alert at the same time (Y3.B). The Bernals are stored as a stack on the Age Track, and when they reach the last Age slot, the dust shields have eroded away, and all Passengers age faster (Y2.D). If a ship flies through the LIC, its Bernals are demoted, indicating that its EM shields are shut down (Y2.B), and subsequently destroyed. The Bernals in the Bernal stack demote together, but are promoted one at a time. If a Bernal is discarded, it goes to the Graveyard.

Data Disk. Black claim disks placed on breakthrough spots, representing data generated by high energy experiments or computational modelling contributing towards an eventual breakthrough. The amount of data this represents exceeds the carrying capacity of an SOS Wisp, which cannot add Data Disks or take advantage of them.

Discontent. A Passenger whose ballot box does not match the current Politics.

Factory Cube. Factory Cubes represent stocks of nanites and metamaterials used for advanced reconfiguration of the Starship hull, rebuilding Bernals during dry dock and building newly researched technologies and breakthroughs. When carried, they increase the chance that gray goo outbreaks will overwhelm the alert Passengers. Cubes on a breakthrough or hostile goo are not factory cubes and cannot be used for factory cube actions. Factory cubes have a mass of 5.

Felonies. The following actions are felonious:
1. Decommissioning Humans (including promoted Robots).
2. Retiring passengers into the Vats on Beehive starship.
3. Decommissioning a Bernal when this decreases the maximum number of alert Humans below the number of Humans currently alert, unless Beehive Braking.
To perform a felony, you must have an alert Passenger other than the passengers affected by the felony, and the Politics must be anarchy. Alternatively, you may (once per game) declare a desperation action when the politics is not Anarchy to permit a felony, which halves your final score and additionally allows a Passenger to commit a felony on themselves (“I’m going outside. I may be some time.”).
When you attempt a felony, the Passengers affected by the felony immediately mutiny even if content (Y6.D). In a Beehive Starship, resolve the mutiny as beehive warfare (Y6.D). If a desperation action, the mutineers are not assisted by any other Passengers. The felony is only permitted if the mutiny fails.
Note: Using Bioengineering in a Beehive is not a felony (or Parenthood when there is no space for the children). However it triggers a Vats Mutiny.

Graveyard. Killed crew and non-robotic colonists go to the Graveyard unless in a Beehive starship. Bernals are discarded to the Graveyard if they are damaged by travelling through the LIC or cannibalized by hostile goo during a gray goo war (Y4.C) in a Beehive. Other decommissioned cards return to your hand. The Graveyard represents genetic diversity and food which can be used for raising children as well as ship facilities available for housing new colonists.

Hand Cards. Black cards that are ideas in the ship’s library and spare mass of the correct spectral type required to build them. Includes supports, Robots, and Robonauts. They may be built with a 3D Print operation. You may not have humans, either colonists or crew, as hand cards - these are decommissioned to the bottom of the deck in a Beehive, otherwise into the Graveyard.
 
Human. A Colonist or Crew with the “humans on board” placard (R3). Humans can become Domestics and suffer a stress from each operation performed, and can go into the vats for stress relief. In Interstellar, Bernals are not Human, and only promoted Robots are emancipated and act as Humans (compare with Robots, below).
 
Midwife. Any alert Human who can act to assist Passengers leaving the vats (Y4.I). Alert promoted Robots can act as Midwives only if they are Biotechs.

Passenger. Any card on the Age Track (except for the Bernal Cards), including crew,  Colonists (both Human and Robotic), and Robonauts. All Passengers except Robonauts can have a maximum of two profession counters on them and can perform one operation per turn for each profession counter, but only promoted Passengers are permitted two profession counters of the same color. Robonauts are limited to a single profession counter. Humans suffer one stress for each operation performed or by fighting hostile goo (Y6.C); non-Humans suffer from stress only during mutinies (Y6.D), or being 3D printed with an age 9+ Bernal or without a Bernal (Y4.A), if a Robot.

Patent Decks. The Patent decks do not represent an exclusive right to use a technology as patents are unenforceable over Interstellar distances and long since expired. Instead, the topmost cards on the patent deck represent the technologies most readily adopted into the starship configuration, for which components which cannot be 3D printed (wiring, isotopes) are available or readily manufactured from existing materials.

Robot and Robonaut. Black cards that are stored on the Age Track or in the hand. Robots can get stress by attempting to mutiny (Y6.D), or being 3D printed in an Age 9+ Bernal (Y4.A) or starship missing a Bernal, or, if human by fighting hostile goo (Y6.C). Robots are susceptible to Alzheimer’s and Mutinies, but Robonauts are not. Both are vulnerable to accidents. If a Robot is promoted to its purple side for any reason, then treat it as a Human, except that it is sterile, cancer-immune, and does not count as alert against the Bernal Capacity (Y3.B). If killed, decommission it to your hand. Promoted robots can go to the vats and decant as if Human. Robonauts can instead be powered down (Y2.A).
 
Ship’s Stack. Any card carried by the Starship not on the Age Track, including the thruster and support cards. The support cards are the black-side generators, reactors, and radiators needed for the Bernal and thruster.
 
Starship. Any spaceship with a thruster or freighter with a starship thrust triangle (Y1.D).

Vats. Human Passengers who have been induced into a state of artificial hibernation. They do not age and are protected from cancer and accidents. Humans can end up in the vats either by retirement (Y2.A) or if the number of alert Humans permitted (Y3.B) falls below the number outside the vats (e.g. during Bernal destruction). 

Vats Mutiny. In a Beehive, a Discontent Passenger who is forced into the vats (either by the Bioengineering operation or if the maximum number of alert humans per Y2.B is exceeded) immediately vat mutinies. Resolve the mutiny as beehive warfare (Y6.D) except the vats mutineer is unable to kill other Passengers or change the politics if he wins unless there are also alert mutineers to assist him. Cards sent to the vats as a result of a felony do not Vats Mutiny.
 
C. GAME SCALE.[i]
This is the same as in Colonization, except the turns are 12 years each and Bernals represent only 400 metric tonnes of habitat on Starships regardless of whether they are promoted. Each Colonist mass point equals about 15 colonists. Each fuel step on the Interstellar Rocket Diagram is 1.4%c (or 4170 km/sec, compare to 2.5 km/sec in Colonization). On this diagram, an exit velocity of 8% c (24,000 km/sec) corresponds to a fuel consumption of one. Changing speed by one step is about 0.4 milligees. For Beehives, the turns are 24-years, so that it travels at 1.3% c, and all masses are 25,000X, and each Colonist mass point is 10,000 people.
 
Y1. INTERSTELLAR SETUP
There are two ways to start a game of High Frontier Interstellar:
·       Quick Start, i.e. starting a fresh one-player game. Choose a faction, and place a disk in the polity corresponding to this faction’s color to indicate the starting starship government. Your starting Starship Stack is the crew card of this faction. For factions green or white, choose between the two available spots. Choose from one of two kinds of Starship: a Starship (Y1.A) or a Beehive (Y1.B).
·       Colonization Start (i.e. as a postscript to a High Frontier Colonization or Hermes Fall (V5) game, which can involve one or more players trying to get additional VP.) To declare a starshot and enter the Interstellar Map during or after a High Frontier Colonization game, move your rocket (using High Frontier rules) to one of the three Sol Exits (Y1.C) and convert it into a Starship (Y1.D). Any player successfully completing the Beehive Ark future can announce a starshot and become a Beehive ark (Y1.B) using the Beehive Ark quick start rules. Any player with a Beehive Starship engine can create a Beehive ark (Y1.B). Only one starshot is allowed per player. Your faction color is your starting starship government.

A. STARSHIP QUICK START. Set up eight patent stacks. The stacks are: Robonaut, Reactor, Generator, Radiator, Human Colonists, Robot Colonists, GW Thrusters (with the two Beehive Thrusters removed (Y1.B)), and Bernals. Notice that MW thrusters and refineries are not used, and the Colonist deck is divided into two: Human and Robotic. Starships cannot be solar-powered, so remove the generators that are solar-powered on their black side from their respective patent decks.[ii, iii] The freighter deck is also not used, but add the HIIPER Beam-Rider and Fission Fragment Sail into the TW Thruster Stack (purple side face-up).

·       Quick Start Funding. Take 16 WTs. You can spend this on exoplanet Searches, Inspiration and Idea Turns and convert the remaining WT into fuel.
·       exoplanet Search. Choose star sites with Living or Habitable planets in any order and make 2d6 exploration rolls (Y7.B). You must continue until you find a Living or Habitable planet or run out of sites to search. This represents finding the minimum justification required to launch such a mission. You may elect to make further exoplanet searches on any star site as required once you find a Living or Habitable planet at a cost of 1 WT per star site.
·       Inspiration. For 1 WT, you may perform one inspiration (M4.1) on any or all chosen stacks. For instance, if you do not like the TW Thruster showing, you may choose it to be moved to the bottom of the deck.
·       Idea Turns. After completing the Exoplanet search and one Inspiration, perform Idea Turns (C1) with no first player advantage, except that you may move purchased cards to either your hand or to the Starship Stack. Cards are added to the Starship Stack on their black side, except that Colonists and Crew are added on their unpromoted sides, and Bernals are added on their promoted sides. You must put one starship-capable card (Y1.D) into your Starship Stack, and it goes into the stack on its promoted (i.e. terawatt) side.
1. If you begin with the Levitated Dipole 6Li-H Fusion GW Thruster, you can immediately promote it to the Dusty Plasma or choose to double your starting fuel. You may not promote the Levitated Dipole 6Li-H Fusion GW Thruster once the mission begins due to fuel compatibility issues.
2. You must start with all the required supports for your Starship engine in the ship’s stack or on the Age Track.
3. Radiators go into the Starship Stack either on their heavy or light side, and are always 3D printed or patched to their light side. The Patch Radiator op (Y4.E) allows discarding a black card to flip a radiator to its heavy side.
4. Because of fuel compatibility problems, a Starship may use only one starship engine in its voyage.
·       Multiple Bernals. All five Bernals are identical in Interstellar. Each has a mass of 10 and needs electrical generator support to be promoted. Their privileges are inactive. A Starship can carry multiple Bernals, which increases the number of Human colonist cards it can carry on alert (Y3.B).
·       Starting Dry Mass. When you decide you are finished with Idea Turns, decommission (into your hand) all the cards you do not want to start with. (Cards in your hand represent ideas stored in the ship’s library, which can be produced later by 3D printing.) Discard unwanted Human colonists to the bottom of the Human colonist deck. Then figure out the dry mass of your Starship Stack by adding the masses of all its cards, which must include the Bernals (if any), colonists, thrusters, and thruster supports, and your hand cards (representing spare mass used for 3D printing).  Place a clear blue dry mass disk into spot on the Interstellar Fuel Strip indicting this dry mass. If the dry mass number does not appear, move it to the next lower spot (e.g. If your dry mass is 34, place the dry mass disk in the “30” spot). Your Starship can start with any number of cards, as long as the total dry mass is 80 or less.
·       Starting Tanks. Then convert all of your remaining WTs into tanks of fuel, where each ten tanks (decatank) costs 1 WT. Set up a yellow fuel figure on the Interstellar Fuel Strip in this wet mass amount. If the wet mass number does not appear, move it to the next lower spot (e.g. if your dry mass marker is on “8”, and you buy 20 tanks of fuel, place the fuel figure on “25”).
·       Choose one of the three Sol Exits (Y1.C). This sets your starting speed.
NOTE: See Y.9 for suggestions on how to start a Starship.
 
B. BEEHIVE ARK. A Beehive ark is a Starship that uses one of the two thrusters with the beehive thrust icon (S4) next to the thrust triangle. These two thrusters are functionally identical in this game.[iv]
·       Fuel and Speed. The fuel, dry mass, Sol exit, and speed of a Beehive are not tracked. Therefore it uses no fuel figure and no speed disk. It always moves 1 space per turn, unless braked (Y4.D).
·       Starting Position. Place a Map Rocket Figure on Sol.
·       Converting a Rocket into a Beehive Ark. During a High Frontier Colonization game, if a player has a promoted Bernal stack with a supported thruster with the beehive thrust icon, he may create a beehive ark by moving the stack to an industrialized synodic comet on the map, excluding generators which are solar powered on their black side.  Crew, colonists, Robonauts and Robots move to the Age track. Other cards move to the starship stack. Set up the patent decks as for a Starship (Y1.A) excluding any cards already on the starship stack or Age Track. 
·       Beehive Quick Start. This is the same as for a Starship (Y1.A), except that the GW Thruster deck consists of only the two beehive thrusters (Y1.B), and you start with 8 WTs and all 5 Bernal cards instead of 16 WT.
NOTE: The combined dry mass of your starting Beehive cards is unlimited.
 
C. SOL EXIT. A Starship may choose one of three exits to enter the Interstellar Map. In all three cases, place your standing rocket map figure in the center of the Interstellar Map, in the space labeled “Sol start.”
·       Sol Exit Neptune. This exit, using a high-thrust periapsis burn at Neptune, is on the map edge between Neptune and Uranus. Place a clear red disk into the “dead stop” leftmost spot on the Speed Track (i.e. the row on the top of the Interstellar Fuel Strip); this is your speed disk. Although this exit starts the slowest, it does not suffer the risks and limitations of the other two.
·       Sol Exit Oort. This exit is in the map corner near Sedna, and your speed disk starts where indicated on the Speed Track. If you take this exit during a Quick Start, you must make one radiation roll (Y2.B) and one event roll (Y2.C) before you start the first turn. (This represents the extra turn required get all the way out to the inner Oort Cloud.)
·       Jupiter-Sol-Jupiter Exit (Jovian Flyby). This exit is at the Jupiter flyby spot, and your speed disk starts where indicated in the Speed Track. This exit represents a close pass to Jupiter, then Sol, and then Jupiter again.[v] To represent the three close passes, before you start the first turn, roll 3d6 one dice at a time, and take the highest roll as the radiation level. Kill or decommission all cards in the ship’s stack below this radiation level. Cards in the vats are immune. These radiation rolls still apply if you take the Jupiter-Sol-Jupiter exit from a Colonization start, although you can elect to make them on the Jupiter-Sol-Jupiter exit path on Colonization map, rolling as if your net thrust is 0.
NOTE: For both the Oort and Jupiter exits, if the Starship engine shuts down due to the very first radiation roll or due to an event roll, it starts at a dead stop instead of the speed indicated. If you use any exit with a rocket containing a starship-capable card, but it is not operational (e.g. missing a required support), then it also starts at a dead stop instead of the speed indicated.
 
D. CONVERTING YOUR ROCKET INTO A STARSHIP. If at the end of a High Frontier Colonization game, you have a starship-capable rocket or freighter, you can gain extra victory points (Y7.D) by moving it (using High Frontier rules) to one of the three Sol Exit locations (Y1.C) on the expansion map. If the Starship uses Beehive thruster, convert all cards in the stack to a starship stack and use the Beehive Ark rules (Y1.B). Otherwise, choosing from both cards in your hand and in your stacks, add the crew, colonists, starship engines, Bernals, Robonauts and supports you think you will need for the trip to your Starship Stack, and save up a lot of WTs for your fuel. You cannot choose generators which are solar powered on their black side.
·       Starship Engine. A thruster or freighter card is starship-capable if it has a starship thrust triangle, i.e. a small thrust triangle to the right of the main thrust triangle. It is bordered in green (S4). A rocket with one of these thruster or freighter cards is called a Starship. You cannot use more than one Starship engine due to fuel compatibility issues - discard any others. The Starship engine starts in the Starship stack, on its purple side unless you are using the Levitated Dipole 6Li-H Fusion GW Thruster which remains on its black side for the game.
·       Supports can be shared. For instance, the generator for the Starship Engine and a promoted Bernal. All cards except the starship engine and Passengers start in the Starship stack on their black side.
·       Passengers. All Passengers from your hand start unpromoted in the Starship stack. Unless the Politics starts in the Center, all Robots must be demoted. Other Passengers start either promoted or demoted as they were in play.
·       Patent Decks. Set up the patent decks using any cards you did not choose as described at the start of the Starship Quick Start section (Y1.B).
·       Interstellar Fuel Strip. For the starshot, move your fuel figure from the Rocket Diagram to the Interstellar Fuel Strip to the left of the Interstellar Map.
·       Futures. For each Future (U1) you have completed, you may choose one Colonist Card from the unused deck to add to the Age Track. In addition, if you completed the Planet Hunt Future, choose a Star Site and make an Exploration Table Roll (Y7.B) before you start. If you completed the Star Wisp Future, make an Exploration Table Roll on a site up to 6 light years away. If the Colonization game ended because of the Footfall Future, SOS Wisps (Y4.F) may not perform Inspiration ops and you cannot recruit crew from Sol (Y7.A). If you completed the Protium Fusion Future, start with a Cube in the “Protium Fusion Breakthrough Disk” spot (Y4.B).
·       Starting Position. Place a Map Rocket Figure on Sol.
·       Dry Mass. You may also bring a Bernal Sphere, and if you have promoted a Freighter Card, you may also bring mobile factory cubes (P6). Set a Dry Mass disk and Fuel Figure in the position indicated by summing all the masses of the cards and factory cubes in the Starship Stack. A Bernal Sphere is mass 10, and each factory cube is mass 5. Note that the Interstellar Fuel Strip allows a dry mass up to 80.
·       Buy Starship Fuel. Pay 1 WT or 1 fuel tank for each ten tanks (decatank) of starship fuel.
 
Example: Your Rocket Stack with a dry mass of 25 (including its TW thruster) exits the map using Sol Exit Neptune. Place its rocket figure on the Sol spot on the Interstellar Map. Place a Speed Disk on the “Sol Exit Neptune” spot. Suppose you paid for 80 tanks of fuel. This, added to your 25 dry mass, yields a wet mass of 105. Since this is between 110 and 100, place the fuel figure on the lower number: 100.

E. AGE TRACK. Remove your Bernal Cards and your Passengers Cards from the Starship Stack, and place them into the Age Track, located along the bottom edge of the map. Each slot is one Age, where each Age = 12 years. Normally, each Passenger card shifts one to the right each turn (although Humans and Robots age faster in high radiation, and not at all in the vats). Passengers reaching 120 years of age remain in the rightmost slot.
·       Starting Age. All Passengers and Bernals start in the Age Slot indicated on the map. Humans start Age = 2, if unpromoted; Age = 4, if promoted; and Age = 5, if crew. Bernals start at Age = 1, Robots start at Age = 3, and Robonauts start Age = 6 (72 years).
NOTE: Older Humans are more likely to achieve a breakthrough or promotion, and less likely to have children or to survive bailouts, cancer, and Alzheimer’s.
NOTE: The rules make it possible to have “immortals” from either ESA or paired Domestics, or Creeper Neogens getting to age 7 plus.
 
F. STARTING PROFESSION DISK. For each Passenger on the Age Track, place a yellow, white, orange, purple, red, green, or clear disk to indicate its profession (Y3.A). The color of the disk must be something it is suited for. For each Human in the vats, place a blue disk. The number of alert Humans must not be more than two for each Bernal carried, plus one extra (Y3.B).
·       Crew. Your starting crew can start either in the vats, or with up to two disks of any color profession.

G. MULTIPLAYER INTERSTELLAR. Interstellar is designed as a single player game, but the brave can attempt to extend their Colonization game into a multiplayer Interstellar game. Each player controls one Interstellar Starship and plays with a complete set of patent cards (from another copy of High Frontier 3rd edition) rather than sharing patent cards.

Starting the Multiplayer Game. The first player to attempt the star shot decommissions the cards in the Bernal, freighter or rocket stack used to form the Starship to the bottom of the Colonization patent decks along with all other cards from their hand and stacks in play, leaving their factories and claims on the board, concluding the player’s involvement with the Colonization game. They then immediately form the Interstellar Starship (Y1.D) using their own patent decks and complete the Sol exit onto a shared Interstellar map. The player begins their first Interstellar operations at the start of the next yellow Sunspot cycle. Other Colonization players may not claim jump, attack or hijack the claims or factories of any player who leaves the game this way.
Colonization VP. Record the player’s Colonization VP at the point they leave the Colonization game. A player’s final Colonization score is calculated by adding their final Interstellar VP to their recorded Colonization VP to determine their final Colonization VP total.
Turn order and phases. Turn order is determined by the order starshots are attempted in and each player completes one of the four phases (Y2.A, Y2.B, Y2.C, Y2.D), with play then passing to the next player in turn order. Phases are interleaved with the Colonization game as follows:
Start of Yellow Cycle = Operations, Quick Start exoplanet searches
Start of Red Cycle = Move Starship, Quick Start patent deck formation and Inspiration
Start of Blue Cycle = Starship Event roll - rolled separately for each player, Quick Start idea turns
Start/Senility = Age Track, Quick Start Sol exit
When the Colonization game is completed, any players remaining in the Colonization game can join the Interstellar game without sharing their victory points between games, by spending the entire next Interstellar turn beginning at the next start of the yellow Sunspot cycle completing a Quick Start using their crew. Operations phases continue to be tracked by the Sunspot cycle, but the counter can be moved directly to the next phase when all players have concluded the current phase.
Beehives. Players with Beehives move at ½ speed such that cards advance on the age track and the Starship event roll only occurs at the end of every turn the beehive starship counter is upright. Passengers can only perform one operation each turn and can only add one profession counter per turn of school, and they cannot perform the same profession’s operation in two consecutive turns unless they are promoted, indicated by leaving the last used profession counter at the top right of the card. The Scholastics school operation only occurs on a roll of 1-3 on a 1d6, moving the profession counters to the top right of the card until the end of the phase if this fails to indicate the operation is not available.
Counters. All exploration rolls are shared and the faction coloured claim disk is placed by the appropriate player. Breakthroughs are not shared but can be replicated using Ship-to-Ship communication (see below). Use faction coloured cylinders to track starship dry mass, speed and government.
Interstellar interaction. Interaction is limited to Ship-to-Ship wisps, “No Midwife risk” avoidance, and allowing spacewalker operations be performed on any cooperating Starship in the same space if both are at a dead stop. Spacewalkers may also perform the same player to player transfer of research and breakthroughs permitted by Ship-to-Ship wisps as an operation, but only with both Starships dead stopped in the same space and if the other player has any alert Passengers.

Ending the Multiplayer Game. Each player ends their own involvement with the game per Y7.D, but other players continue. You can choose to end the game at any point with all players’ agreement or if no players are left in the game. Victory is determined by the player with the highest VPs.
Y2. INTERSTELLAR SEQUENCE OF PLAY
For your turn, perform A through D, below. Then, go to the next turn; twelve years have elapsed.
 
A. OPERATION. In any order, each Passenger which started this turn on the Age Track chooses one of school, work, or retirement. Robonauts can additionally be powered down. If hostile goo is present, Passengers may fight it.
·       School. Add profession disks onto the top right of the Passenger card up to a maximum of two (Y3.A). If coming out of the vats, this disk replaces the blue vats disk (Y4.I). Only promoted Passengers can have two profession disks of the same color and Robonauts may only have one profession disk. If the Scholastics breakthrough is achieved, the Passenger may perform one operation after placing any profession disks, as Work below, at a cost of one stress if they are Human, but not a promoted Robot (Y6.D).
·       Work. Passengers may perform one operation for each profession disk on the card (“Op” for short). If the Passenger is Human, place one stress for each operation performed unless the Passenger is a promoted Robot (Y6.D). If the Passenger is a non-human, then it doesn’t suffer any stress from performing operations. All operations must be according to the color of a disk in the Passenger’s resumé (Y3.C), as shown in the Profession Column (see far left side of the map). See also Y4. for a list of operations. Operations may be performed in any order, including before or after Operations by other Passengers. Move the profession disks from the top left to the top right of the card to track which operations have been performed.
·       Retirement into Vats (Humans and promoted Robots). Replace all profession disks with a single blue disk (Y4.I), placed at the top right of the card. This removes all stress from the card if the Neurological Breakthrough is discovered. Passengers forced into the vats from the loss of a Bernal do not get this benefit. Retirement into the Vats is a felony if attempted in a Beehive.
·       Power down (Robonauts only). Remove all profession disks. Unsupported Robonauts must power down. Send the robonaut to school to power them back up. This protects the robonaut from accident risks while they are powered down. Note that robonauts must be given a profession disk when they are 3D printed, which represents burn in and component testing with a consequent risk of failure from the Ship Event roll the turn they are added to the Age track, and as Passengers execute options in any order, an unsupported Robonaut can have its supports 3D printed before it is forced to power down.
·       Fighting Hostile Goo. Each alert Human in a Beehive can elect to attack each hostile goo on the Age Track as outlined in gray goo war (Y6.C). This incurs one stress for each attack.

Note: Passengers placed onto the Age track this turn by Parenthood, Bioengineering or 3D printing cannot elect to take an operation other than fighting hostile goo, indicated by putting their Profession disks at the top right of the card. However, human colonists placed by Parenthood are assumed to be attending school and can perform one operation (at a cost of 1 stress) if the Scholastics breakthrough has a cube on it. Non-humans moved to Age = 1 by a Dry Dock or 3D Printing operation from elsewhere on the Age track or which had their Age changed by a Service operation may perform operations as normal, before or after being moved.

Example. During an operations phase, you decant the Pantrophist out of the vats as a School Operation, giving him a White Scientist Profession. A robot biotechnician is on hand to act as a Midwife which does not use an operation. That same turn, that biotechnician can also attempt a Promote Op to promote him, but as a robot does not gain a stress from performing operations. The next turn the scientist attends school again and as a promoted Scientist is able to gain a second white scientist profession disk. The turn after is a working turn for the scientist, who performs two science operations: Promote and Breakthrough. One succeeds and the other fails, and he suffers two stress. The next turn is a school turn for him, and he earns a red disk as a spacewalker (since he has a buggy icon). After a couple more working turns, he retires into the vats.

At the end of the operations phase, move all the profession counters back to the top left of the Operations card. This allows you to track their usage during the next Operations phase.
 
B. MOVE STARSHIP. Burn fuel, adjust speed, and move your Starship (Y5). If using a beehive, move one space.
·       Local Interstellar Cloud. All the Bernal Cards are carried in a stack on the Age Track called the Bernal Stack. This stack ages and demotes as a group and Bernals in the stack are promoted individually. If the Starship enters any LIC spaces (i.e. the blue spaces) during its move, the Bernal Stack is demoted if all the Bernals in it are promoted (this indicates that the EM shields have been overloaded and are shut down). Entering additional LIC spaces during the same turn has no further effect.
·       Dust Erosion. If the Starship enters any LIC spaces and the Bernal stack already contains demoted Bernals, discard one demoted Bernal Card to the Graveyard, and decrease the dry mass and wet mass by 10 (see Y4.A example). This may result in the EM shields restarting, if the remaining Bernals in the stack are all promoted.
·       Radiation Roll (if no Bernal). If there is no Bernal, roll 1d6 to obtain a radiation level, then kill every card in the Starship Stack or on the Age Track (even in vats) that has a rad-hardness less than this level.
·      Rebuilding the starship engine. If the starship engine is decommissioned by a radiation roll, it must be 3D printed to its black side to be rebuilt, which (as a special exception) requires that the supports are already present in the starship stack or Age track. The Levitated Dipole 6Li-H Fusion GW Thruster is then functional (and cannot be promoted). The Spheromak 3He-D Magnetic Fusion must be promoted to the Colliding FRC 3He-D Fusion in order to function.
 
C. STARSHIP EVENT ROLL. Make a 1d6 event roll, and consult the Profession Column, found on the left edge of the Interstellar Map. This column contains numbered “dice” icons specifying the risks associated with certain profession disks. For every profession affected by the roll, roll the indicated risk (Y6) against all the Passengers with this color of profession. You must roll risks starting from the top and going down.
·       Meteoroid. If the event roll is a 6, then additionally discard the heaviest radiator card (if any) from the game if it is on its light side or decommission it if it is on its heavy side. Choose at random if there are several equally heavy cards.
REMEMBER: Radiator cards are always 3D printed or patched to their light side. [vi]
·       Stressed Robots. If your Robots have stress counters, on a Ship Event roll of 5 they may attempt to abort the planned mission and hijack the ship. If a roll of 1d6 is less than the number of stress counters, discard the Robot from the game and roll a 2nd Starship Event roll this turn. Multiple Robots failing this roll result in additional Starship Event rolls.
·       Glitch. If the event roll is a 4, then additionally decommission the heaviest black card (if any) on the starship stack to your hand. Choose at random if there are several equally heavy cards.

Example: A 1 is rolled. There is a red dice with the number 1 in the yellow “Engineering” Operations, superimposed on the “Alzheimer’s” icon. This means that all your Passengers with an engineering resumé suffer an Alzheimer’s Risk (Y6.B) by rolling 1d6 for each one. There is also a 1 in the orange “Business” Operations, this time on the Handgun icon, which means all business Passengers (i.e. Passengers with an orange disk) suffer a Mutiny risk. And finally, there is a 1 on the domestics, indicating that all domestics suffer a risk of accidental death (Y6.A).
 
D. AGE TRACK.
1. VATS - Remove one stress from each card with a blue disk (Y4.I). These cards do not age.
2. ALERT - Advance all other cards by one step on the Age Track (Y1.E), or two steps if using a beehive. Gray goo in a Beehive attempts to acquire factory cubes and ages all passengers in the vaults (Y4.C). If there is no Bernal, or if the Bernal Stack reaches or is at the last slot (120 years+ old), then advance the alert cards a number of steps equal to one, plus the starship speed (rounded up), three steps if using a beehive. Robots 3D printed on these Starships get 2 stress. [vii]
 
Example: The Bernal Stack is moved into the penultimate slot (at Age = 9). All of the other cards not in the vats are moved to the right by one step. But, on the next turn, the Bernal Stack moves into the final slot at Age = 10. Assuming the Starship moves at speed = 2, now the other cards are moved to the right by three steps during Aging.
 
Y3. PASSENGER PROFESSIONS \& RESUMÉS
A. PROFESSION DISKS. When each Passenger is moved to the Age Track (either from parenthood, if Human; or from 3D printing, if non-human), you must choose one or two profession disks that it is suited for. Robonauts may only have one profession disk. The color-coding indicates the specialization as follows:
·       Yellow Disk = Engineer (must have the wrench icon, Y4.A)
·       White Disk = Scientist (must have the microscope icon, Y4.B)
·       Orange Disk = Business (must have the handshake icon, Y4.C)
·       Purple Disk = Pilot (must have one or more braking icons, Y4.D)
·       Red Disk = Spacewalker (with buggy or missile platform icon, Y4.E) or Network (with raygun icon, Y4.F)
·       Green Disk = Biotech (must have the terraform icon, Y4.G)
·       Clear Disk = Domestic (must be Human, Y4.H)
·       Blue Disk = Vats (must be Human, Y4.I).  Passengers moved to the Age Track by the  Bioengineering Op always enter the Vats.

NOTE: It is possible for a Passenger through demotion or promotion to have profession counters they are not normally permitted.

Each disk permits one operation per turn of the matching profession. Promoted Passengers are permitted two disks of the same color, which permits two operations per turn of this profession.

RISK MANAGEMENT: Having two profession disks increases the relative risk to the Passenger of most profession combinations but having two disks of the same color incurs no additional risk for the second disk. There are also career combinations which will be killed by the first risk when the second risk is also deadly due to the Passenger’s age (“The cancer killed him before the Alzheimer’s got to him”) which means the overall risk to the Passenger is smaller than it initially appears.
 
B. MAXIMUM NUMBER OF ALERT HUMANS. Humans can be alert or in the vats (Y4.I). The maximum number of alert Humans is equal to one plus two more for each Bernal Card carried. Robots (both promoted and unpromoted), and Humans in vats do not count toward this limit. If the limit is exceeded, you must send the excess Humans immediately to the vats and they gain 2 stress from the process. This is a felony if you do this voluntarily except as a part of a Beehive Braking maneuver.
 
C. RESUME. By going to school (Y2.A), you may add profession disks to a Passenger for which it is suited for per Y3.A. The collection of all the professions of a Passenger is called its resumé. A Passenger’s resumé can contain up to two disks if a Human or Robot, or only one disk if a Robonaut and may only have two matching Profession disks if they are promoted. After adding any profession disks at school, you may also complete an operation of one of the Passenger’s professions if you have the Scholastic Breakthrough. 
NOTE: Once placed, Professions cannot be removed except by being decommissioned or converted to hostile goo or going to the vats, or if a Robonaut, by powering down. Professions added by the Invent operation (Y4.C) are also removed if the Robonaut is decommissioned or has its age reduced to 1 by a 3D printing or Dry Dock operations.
 
D. STRESS. Stress is represented by black disks placed upon Humans after they perform any operation. If you run out of tokens, use any other tokens to indicate stress.
NOTE: Think of stress as the mental baggage of sanity-sapping misfortunes accumulated throughout one’s life. Stress increases the chance of suicide and accidents, and may be decreased by marriage, the service operation, or being in the vats. The stress Robots get from mutiny, fighting hostile goo or being 3D printed with no Bernal or an age 9+ Bernal are especially dangerous as it can result in additional Ship Event rolls.
 
E. ABILITIES. The special ability that colonists have in Colonization is applicable to Interstellar if it is described in green font. Normally they are valid as long as the colonists are alive, even in the vats.
 
Example: The Utility Fog Halbonaut card states (in green font) that all generators in the ship’s stack have both electrical and pulsed capacities.
Y4. INTERSTELLAR OPERATIONS
Operations are performed by Passengers, either Human, Robonaut, or Robot. Passengers can only perform an operation if they have the correct profession disk: Engineer, Scientist, Business, Pilot, Spacewalker, Network, Biotech, Domestic, and Vats:
To help keep track of which operations have been performed, keep the Profession Disks on the top left of the card if it has yet to be performed, and slide each disk to the top right after the corresponding Op has been performed. Start Profession disks placed when a card is moved to the Age track by Parenthood, Bioengineering or 3D Printing on the top right to indicate these operations may not be performed this turn; move all Profession Disks to the top right if attending school - except keep one on the left if you have the Scholastics Breakthrough to be used for the operation that breakthrough provides while at school. Reset all of the disks once the Operations Phase is past.
 
A. ENGINEER OPERATIONS (Yellow Profession Disk, Wrench Icon).
·       3D Print. Move a black card (in either your hand or on the Age Track) either to the ship’s stack, or (if it’s a Robot or Robonaut) to the Age = 1 position on the Age Track. Radiators are always 3D printed to their light side. If moved from the hand to the Age Track, place one or two Profession Disks on its card per Y3.A (but it cannot perform any operations until next turn); otherwise you can replace any existing Profession disks on its card with new ones per Y3.A. If the card had already performed operations this turn, the Profession Disks must be placed in the top right corner to indicate that no further operations may be made. 3D Printing with a missing Bernal or an age 9+ Bernal stack is less effective: Robots which are 3D printed add 2 stress. 3D printing or a dry dock does not remove Robot stress - they must be decommissioned (which is a felony if promoted) or sent to the vats.
NOTE: A Robot with the engineering profession on the Age Track can rejuvenate itself to Age = 1 as a 3D Print Op.
·       Cube Nanofacture. Discard out of the game a Robonaut or Robot card from the ship’s Age Track, plus the supports it needs to be operational from the ship’s stack, and replace them with a Factory Cube. [viii] If a Human or promoted Robot is used as a support for cube nanofacture, remove their onboard reactor but do not decommission them. Unpromoted Robots used as supports are decommissioned to the bottom of the deck.
NOTE: A Robot can sacrifice itself to perform a cube nanofacture only if it is unpromoted.
·       Nano-Reconfigure Hull. By discarding 2 factory cubes or 1 Bernal Card from the ship’s stack, decrease dry mass by 10, making a dry mass adjustment (D2). If the engines are operational and you have achieved a Protium Breakthrough (Y4.B), this allows you to change the starship speed by one step (Y5). You can discard multiple factory cubes with one operation, decreasing dry mass by ten and increasing speed for every 2 discarded. You may not discard a Bernal if this results in the maximum number of alert Humans falling below the current number of alert Humans, except as a felony.
NOTE: With the exception of the nano-reconfigure hull operation, beam-core braking, or erosion of Bernal Cards (Y2.B) or restoration via dry dock, a starship's dry mass does not change once launched. New Colonists will never mass more than the resources used to feed them, so mass is always conserved.
NOTE: The minimum starship mass is six. If you reduce mass to less than six (by nano-reconfigure, erosion, etc.), the mass remains at six.
 
Example: A Starship with dry mass 35 and wet mass 50 undergoes a nano-reconfiguration. One of its factory cubes is discarded, the dry mass decreases to 25, and the wet mass decreases to 40. If the ship undergoes a second nano-reconfiguration, the dry mass goes to 15, and the wet mass to 30.

·       Drydock. This restores the Bernal Stack and all other non-human cards of your choice on the Age Track to Age = 1, and/or rebuilds one Bernal from the Graveyard. For a Starship, you must be dead-stopped at a fuel planet and if you rebuild a Bernal, you must add 10 to the wet and dry mass.  For a Beehive, you must discard 1 factory cube to perform the Dry Dock op plus 2 factory cubes if you rebuild a Bernal. 
 
B. SCIENTIST OPERATIONS (White Profession Disk, Microscope Icon).
·       Promote. To flip a card to its purple side, roll 1d6. Success if the result is less than the Age (Y1.D) of the scientist doing the promotion. Self-promotion is allowed. You cannot promote the Levitated Dipole 6Li-H Fusion GW Thruster.
NOTE: To promote a Bernal Card, it must also have an operational generator. But unlike in Colonization, it does not have to be at a dirtside to promote. You can promote Bernals without having to roll by using the EM Shield Repair op (Y4.E).
NOTE: Humans in the vats cannot be promoted. Robots cannot be promoted unless the politics is purple or as a result of mutinies (Y6.D).

·       Research. Roll 1d6, if < Age, you must discard a Factory Cube to take 1, 2 or 3 black cards of up to 5 total mass from the top of the patent decks to your hand.
·       Explore. If at a dead stop at a Star Site or LIC, make a 2d6 exploration roll (Y7.B).
·       Eureka. Place two Data Disks on any of the six breakthroughs then roll 3d6. If the roll is < age + Data Disks on the breakthrough, place a Factory Cube from the ship stack on the breakthrough and remove any disks to indicate the breakthrough has been achieved. If you do not have a Factory Cube in the ship’s stack you only place the Data Disks.
·       Endowment. Place a Data Disk on the breakthrough associated with the current politics: white = cancer cure, red = neurology, green = ecological, purple = mathematics, scholastic = orange then roll 2d6. If the roll is < age + Data Disks on the breakthrough, place a Factory Cube from the ship’s stack on the breakthrough and remove any disks. If you do not have a Factory Cube in the ship stack you only place the Data Disk.

NOTE (Breakthroughs): Each breakthrough spot can contain a cube which awards an ability specific to the breakthrough. Cubes on breakthroughs are no longer factory cubes and cannot be used for Research or Nano-Reconfigure Hull operations, Beam-core braking or affect or are discarded by gray goo risks.

If you have a Factory Cube on the Mathematics Breakthrough, then double the number of Data Disks placed by subsequent Eureka or Endowment operations on any breakthrough. Protium Fusion (the energy of the stars) is used during Nano-Reconfigure operations (Y4.A) or when braking Beehives. The Cure for Cancer reduces the Cancer risk (Y6.E). The Scholastics Breakthrough allows Passengers one operation while at school (Y2.A). The Ecological Breakthrough assists with Parenthood (Y4.H). The Neurological Breakthrough removes all stress from Passengers retiring into the vats, and makes use of the vats non-felonious in Beehives. 
NOTE: If there is no Bernal, or if the Bernal Stack reaches or is at the last slot (120 years+ old), then add one die to the Breakthrough Rolls (e.g. roll 3d6 instead of 2d6 for Endowment and 4d6 instead of 3d6 for Eureka).

Example: An alchemist scientist, aged 4 dozen years, attempts an Endowment. The politics is white and one Data Disk is placed on the Cure for Cancer breakthrough which had none previously. 2d6 is rolled for a total of 5 which is not lower than 4 + 1. This action places stress on the scientist. With no Data Disks, the scientist would have been better attempting a Eureka, with a less likely outcome, but a better result (two Data Disks) if the roll failed.
 
C. BUSINESS OPERATIONS (Orange Profession Disk, Handshake Icon).
·       Service. Halve a Human’s stress (round down) or Robonaut’s Age (round up). This can be used on Passengers in the vats.
·       Mentor. Add one permitted profession disk to any alert Human who does not have two profession disks. Promoted Passengers can have two disks of the same color. This allows the Human perform a second operation immediately if they are not in school.
·       Activism. Move the Politics one spot. It cannot end up in the center, which is reserved for Robot Emancipation.
·       Suffrage. Roll 1d6. If the result is less than or equal to the number of promoted (purple) cards on the Starship (including thruster, Bernals, and colonists both in and out of the vats), then move the Politics to the central (purple) position. This indicates that the Robot promotion (Y4.B) is legal. Once promoted, Robots are treated as Human (see Y.0B) and remain promoted even if the Politics shifts out of the center again.[ix]
·       Invent. Roll 1d6. If < Age, discard a Robot from the game to change a Robonaut's profession to any except Business.[x] This also changes the color of the profession counter on the Robonaut and a Pilot (purple) Robonaut gets Drogue brakes (Y4.D). The change to the Robonaut profession is lost if the card is powered down, decommissioned or 3D printed or its age is changed by Dry Docking. The counter placed by the Invent Op is not removed by the Service Op.

D. PILOT OPERATIONS (Purple Profession Disk, Brake Icon). Passengers with one or more brake icons on the interstellar segment of their card may acquire the Pilot Profession disk (purple). This allows them to perform brake ops. Braking slows down the speed disk on the Starship Speed Track or (for beehives) brings the ship to a stop. Braking does not require an operating starship engine.
There are five kinds of brake icons: mag-sail brake, surfer brake, drogue, drogue scoop, and beam core. Each kind can be used only once per turn, but all types can be used together in any chosen order.
·       Braking Mass Limits. As shown on the Fuel Strip, Starships in with a wet mass of 30 or more cannot use braking. Remember that Beam-riders must use brakes to slow down, while the other Starships can burn fuel to slow down.
·       Braking Speed Limits. The Starship Speed Track shows at what speeds the different kinds of brakes are effective. (The Drogue/Scoop brakes for slow speeds, the Beam-Core brakes at all speeds, the others at high speeds).
·       Mag-Sail Brakes. You may slow down a number of steps up to your current numeric speed.[xi]
NOTE: Because steps are not equal to speed, a mag-sail cannot be used to slow to a dead-stop. For instance, a Starship moving at speed 4 can slow down 4 steps, which takes it to a speed of 2.
·       Surfer Brakes. If on or adjacent to an LIC (blue space), you may slow down by one step. (This technology surfs on the energized plasma ribbon on the heliopause bow-shock).
·       Drogue Brakes. As noted on the Starship Speed Track, if you are at site hex containing a gas giant, you can slow down one step with a drogue brake. This is risky, as you must roll a crash hazard risk (skull icon, see F5) and a radiation risk (indicated by the radiation icon) as noted on the Passenger card. The radiation risk consists of rolling 1d6 for radiation level, and placing stress on all Passengers and equipment that fail. Your speed does not affect this roll.
NOTE: Performing a drogue brake at an unexplored site makes an Exploration Roll (Y7.B). The braking fails if no gas giant is located.
·       Scoop-Brake. This is identical to the Drogue Brake, except that if Scoop-Brake is used to come to a dead stop, gain one tank of S fuel. (This technology scoops and liquefies the gas giant atmosphere).
·       Beam-Core Brakes. You can slow down by one step by discarding cards or factory cubes (worth mass 5) out of the game from either your hand, starship stack or Age track with a collective mass of at least 10, and reducing the wet and dry mass by 10 (see Y4.A example). This may entail an Antimatter Risk (Y6.F).[xii]
NOTE (Fargo Tactic): It is a Felony to brake by decommissioning Humans (jargon for tossing them into the antimatter hopper).
·       Beehive Braking. Any brake operation will stop a beehive, but braking must discard five Bernals or ten factory cubes, or some combination where each Bernal is worth two factory cubes, or else the beehive does not stop. Furthermore, the beehive cannot stop unless the Protium Fusion Breakthrough (Y4.B) has been achieved. Once a beehive stops, it cannot be restarted again. Recalculate the maximum number of alert humans and force excess humans into the vats before bailing out or forming a colony after Beehive Braking or Desperation Star Dive Braking.
·       Desperation Star Dive Brake. Even if you have no Pilots, any Passenger at a Star Site can brake at any speed by plummeting through the star itself. Make a desperation die roll where only a 6 will slow your Starship one step, decommissioning all Bernals and cards in the ship’s stack, and all other results will destroy the Starship and kill all the Passengers. For beehives, a 4, 5 or 6 is a successful stop, and all others destroy the ship and Passengers.
 
E. SPACEWALKER OPERATIONS (Red Profession Disk, Buggy or Missile Robonauts, Crew, or Colonists).
·       EM Shield Repair. Promote the Bernal Stack if it has a working generator.[xiii]
·       Patch Radiator. Move a black radiator from your hand to the stack on its light side. You can choose to also discard another black card from your hand to the bottom of the patent deck to then flip the radiator to its heavy side.

NOTE: Since black cards can be freely decommissioned, you can discard a black card on the Ship Stack or Age Track using this operation to flip the radiator to its heavy side.

·       Explore. See Scientist Operations.
·       Refuel. If at a dead stop at a fuel planet, gain fuel decatanks (10s of tanks) equal to five minus the ISRU value of the performing card.
NOTE: If you have a thruster requiring S fuel, you can refuel only at a gas giant.
 
F. NETWORK OPERATIONS (Red Profession Disk, Raygun Robonauts, Crew, or Colonists). Launch an expendable wisp to a destination up to 12 spaces away. There are three types:
·       Exploration Wisp. If the destination is an unexplored star site, make an explore roll (Y7).
·       SOS Wisp. If the destination is Sol, this allows you to perform an Inspiration Event (M4.1), inspiring one card on any or all patent decks because you receive the latest in technology, entertainment, news, and psychology from Earth), or a Science Op. The Science Op, which can be a promotion, explore, research, eureka or endowment operation is successfully performed if a 1d6 roll = 1 (ignoring any requirement for an age roll). However, Earth refuses to help if the Robots are emancipated (i.e. politics in the center). This additional Science op does not add further stress to the Human performing it. The science Explore op can be used on any star system, not just those within 6 light years.

NOTE: SOS Wisps do not accumulate Data Disks on breakthroughs and so the Eureka and Endowment operations from an SOS Wisp have an identical effect if a 1 is rolled: claim a breakthrough by placing a Factory Cube from the ship stack on any Breakthrough and removing the Data Disks on it (if any).

Example: As a network operation, a raygun robot sends an exploration wisp to Sirius, which is 4 light years ahead. The exploration roll is 9, so there is nothing at Sirius. The Robot sends an SOS to Earth, requesting an exploration of Procyon, which is 6 light years beyond Sirius. If a 1 is rolled, then a 2d6 exploration roll for Procyon is made.

·       Ship-to-Ship Wisp. If the destination is another Starship (assuming a multi-player game) with wisp capacity and neither ship is Red Polity, the exchange of information removes one stress from each card on both ships or can be used to replicate one known Breakthrough or three researched cards of up to 5 mass total from a destination ship who agrees to share this information. The Network operator must use a Factory Cube to claim the Breakthrough or discard a Factory Cube to add researched cards to their hand and draws the matching patent cards from their own patent decks, shuffling the patent deck other than the top card afterwards. Cards discarded from the game cannot be acquired this way.
NOTE: The 6-lightyear range of the SOS Wisp is marked on the Interstellar Map.[xiv]
 
G. BIOTECH OPERATION (Green Profession Disk, Terraform Icon)
·       Bioengineering. If in a Beehive or creating a colony (Y7.C), draw a random Colonist Card into the vats at Age = 1. In a Beehive immediately resolve a Vats Mutiny (Y1.B) unless you have the Neurological Breakthrough and politics is Red. The Colonist must stay in the vats for the remainder of the turn if he survives the mutiny. The Bioengineering operation may only be used on a non-Beehive Starship as a part of ending the game. [xv]

Example: The Svalbaard Caretaker biotechs use the Bioengineering operation to draw the Microgravity Pantropists on a Left Wing Commune beehive without any alert Orange passengers and only one other badge on an alert Passenger The Microgravity Pantropists immediately mutiny, rolling a 2 on 1d6 but because of their 1 vote they are overwhelmed and killed by the loyalists with 4 badges (no need to roll). If Siren Cybernetics had been drawn instead, no mutiny would occur because they are content to go into the vats. Drawing the Islamic Refugees with a roll of 2 would also result in the Islamic Refugees surviving in the vats but with 5 White votes and an alert co-conspirator, the Svalbaard Caretakers will change the politics to White and begin killing the other passengers unless the loyalists can roll a 6.

·       Wet Nano Lab. Discard a Graveyard card from the game to perform a Science or Engineer op
·       Promote. See Scientist Operations.
 
H. DOMESTIC OPERATION (Clear Profession Disk, “Humans On Board” Icon)
·       Marriage. Remove one stress each from any two alert Humans not in the vats. Promoted Robots can have stress removed this way only if the politics is Purple. You cannot perform the Marriage op if there is only one alert Human.
·       Parenthood. A domestic may attempt to become a parent by rolling 1d6. If the result is greater than the parent’s Age (Y6.A), then draw a random Human colonist from the colonist deck, if any remain. It starts at Age = 1 and immediately attends School for the turn to get one or two profession disks. If the maximum number of alert Humans (Y3.B) is exceeded, then send either the parent or child to the vats instead of school, resulting in a Vats Mutiny on a Beehive.
NOTE: If the Ecological Breakthrough has been attained, then the roll for Parenthood Success is 2d6 rather than 1d6.  
NOTE: The Parent operation may only be used to replace dead crew and Human colonist cards unless in a Beehive Starship. Keep Humans killed out of the game in the Graveyard, and discard one card from the game from this stack if a child is born. Furthermore, on-board nuclear devices (P5) are inactive on cards born by parenthood (you can’t give birth to nukes). In a Beehive, there is no requirement to discard a card from the Graveyard to perform the Parenthood operation.
·       Self-Promote. Roll 1d6. If the result is less than the Human’s Age (Y1.D), then flip its card to its purple side.
 
I. VATS OPERATION (Blue Profession Disk, Humans only)
Humans with a blue profession disk are in the vats, i.e. in a hibernation mode. Passengers in the vats lose all their other profession disks, cannot be promoted, do not age, and lose one stress per turn. If they have an on-board nuclear reactor (R3), it is usable while they are in the vats.
·       Decant. Humans emerge from the vats by decanting. Replace the blue disk with one or more suitable profession disks (Y3.A). This is disallowed if the max number of alert Humans (Y3.B) is exceeded.
·       The “No Midwife” Risk. If a Human is decanted with no other alert Robotic biotechnician or Human on board or (in a multiplayer game) in the same space, then roll 1d6. The decanted Human dies if 1d6 is more than its rad-hardness. You can attempt a flyby of Earth by entering the Sol space to avoid this risk as Earthbound medical staff can provide remote operation of the necessary systems during the small window available.
·       Legality. Using the vats is a felony in a Beehive if voluntary or results in a Vats Mutiny if involuntary, see Interstellar glossary.
Y5. STARSHIP MOVE
Except for Beehives, Starships burn fuel and move per Y5.A and Y5.B.
 
A. INTERSTELLAR BURN FUEL. The first number in the starship thrust triangle (Y1.D) or the Beehive thrust icon (Y1.B) is the starship thrust. This is the maximum number of steps of speed adjustment you can make each turn. The second number is your fuel consumption. This is the number of steps of fuel consumed for each step of speed adjustment (rounding up, in the case of fractions). Follow these two steps:
1. ADJUST SPEED DISK - The max number of steps of speed adjustment on the Speed Track = Thrust. This advances or retards your speed disk (Y1.C).
2. BURN FUEL - For each step of speed adjustment, burn fuel steps = Fuel Economy.
·       Ballistic Starship. If the Starship thruster or its supports are non-operational, you cannot burn fuel and can change speed only by braking. You may still choose turns at intersections.
·       Pushed Beam-Rider. A mass beam generated from Sol pushes the Beam-Rider Starship. Its thrust triangle states: “1•0 no decel,” which means it can accelerate by one without expending fuel, but cannot use its thruster to decelerate (relying on brakes to stop the Starship).
NOTE: In Interstellar, the starship thrust and fuel consumption are never modified (i.e. are unmodified by wet mass modifiers, solar power modifiers, or thrust- or fuel consumption-modifying supports).
 
Example: A Daedalus thruster has a starship triangle of 7•2. Starting from a dead stop, it could accelerate 7 steps on the Speed Track, bringing it up to 2 spaces a turn (9.5% the speed of light). This costs 7 X 2 = 14 steps of fuel. On its next turns, if it does nothing, it continues to travel at 2 spaces each turn. Upon reaching its destination, at this high speed it will have trouble stopping, even with really good pilots and hull-configure architects.
 
B. INTERSTELLAR MOVE. The Speed Track is located in the top left corner of the map. The red speed disk on this track indicates how fast your Starship travels, from a “dead stop” to 15% the speed of light. You must move it the number of spaces indicated, with no U-turns. At any intersection, choose which direction to go (but not back the way you came).
 
C. HALF-SPEED. If your speed disk is in a space labeled “1/2” on the Speed Track, each time your Starship moves a space, lay its map figure on its side. For its next move, it stands up instead of moving one.[xvi]
 
Example: If you start your turn at ½ speed laid down, then on the next move you must stand up instead of moving (regardless if you have speeded up, slowed down, or stayed the same speed).
 
D. HAZARD SPACES. Entering a space marked by a red hazard skull is a Crash Hazard. However, if the hazard is a flare star, and you previously successfully explored it without busting it (for instance, you sent a wisp and rolled a 2– 6 in the Exploration Roll), then the flare star is inactive and safe to enter. (This security comes from having explored the system and knowing its flare cycle.)
NOTE: You cannot use “Failure is Not an Option” in Interstellar to avoid Interstellar Hazards.
Y6. RISKS AND DEATH
If a risk is called for by the Profession Column, roll 1d6. Risks are always resolved starting at the top of the chart (engineer) and moving down!
 
A. ACCIDENT RISK (slipping human icon). If Human, the unit is killed if the 1d6 roll result is less than or equal to the stress. If non-human, the unit is decommissioned if the 1d6 roll result is less than or equal to its Age. If a stressed non-human, check the roll against both Age and stress.
 
B. ALZHEIMER’S RISK (human head icon). Demote if the 1d6 roll result is less than the Passenger’s Age. If the Passenger has two Profession disks of the same color, remove one.
·       Time Separated Twinning (optional): This rule allows Humans (but not promoted Robots) one reroll for each failed Alzheimer’s Roll. It assumes that radiation-induced dementia (the leading cause of death in a Starship) is reduced by a special medical procedure.[xvii]
 
C. GRAY GOO RISK (biohazard icon). Add one stress to every alert Human. The performer is killed if 1d6 > rad-hardness. Then, roll another 1d6 and add the number of Factory Cubes carried (if any, see Y4.A). If the result of the second roll is greater than the number of badges (combined for all alert colonists), discard one Factory Cube, and advance everything in the vats by one in Age (or kill it if in vats at Age = 10).
·       Badge Icon. The badge icon indicates the discipline/loyalty level of the Passenger, useful during bailouts and beehive warfare, as well as gray goo outbreaks.
·       If the Starship Government (Y1) is in the Red Polity, then add one badge to each Human. If in the White Polity, add one to the rad-hardness of the scientist or biotechnician who caused the outbreak.
·       Beehive Gray Goo War. If in a Beehive and gray goo occurs, choose the oldest Robonaut or Robot on the Age Track to convert into a hostile gray goo. (If there are none, then the gray goo has no effect). Remove any profession counters on the card. The goo is unaffected by risks other than radiation rolls but can be killed by colonists attacking it (see below).
·       Aging the Goo. When each goo ages, it acquires one Factory Cube carried by the starship (oldest goos acquiring first) and advances the Age of all units in the vats by one (any Passengers that cannot be advanced because they are in the Age = 10 slot are killed). If there are no Factory Cubes to acquire, the goo cannibalizes the Bernal stack: either discarding a demoted Bernal to the Graveyard and converting it into 2 Factory Cubes, acquiring one, or demoting the Bernal stack if all Bernals are promoted.  Do not make a mass adjustment due to Bernal loss this way and discard the goo to the bottom of the patent deck if it fails to acquire any Factory Cubes the turn it is converted. The cubes acquired by the goo should be placed on the goo card and cannot be used for any actions until the goo is killed; likewise the goo card cannot be used for any actions (such as discarding or decommissioning) while it remains on the age track.
·       Fighting the Goo. Each alert Human can attack each goo once per operations phase (in addition to their normal operations) but adds one stress for each attack. The attack kills the goo if 1d6 < than the goo’s rad-hardness (low rad-hardness robonauts are converted into more dangerous goo) minus the number of cubes sitting on the goo, adding 1 to the goo rad-hardness if politics is White. If the roll is higher than the twice number of badges the Human has, they must check for an accident risk, adding one badge if the Politics is red. If the goo is killed, discard it to the bottom of the patent deck and return half the cubes (rounded up) to the starship stack as Factory Cubes - the remainder are discarded.  [xviii]

D. MUTINY RISK (revolver icon). A Human is content if it has an election button with a color matching the Politics on the Starship Government (Y1).  Crew are content if the Politics matches their color (and they have no ballot box icons). Robots are content if the Politics is in the Centrist (purple) position. Include Passengers in the vats when determining mutiny outcomes but they are not affected by the resulting stress.
·       Content. If the mutineer is content, nothing happens.
·       Discontented majority. If Discontent, and the ballot box icons (R3) of the contented are less than the Discontented, move the Politics to the Polity of the mutineer, and all alert Humans suffer stress equal to the number of steps the Politics moved. This includes the mutineer and promoted Robots (See Robotic Mutiny, below).
·       Discontented minority: If the mutineer is Discontent and the ballot box icons of the contented are greater than or equal to the Discontented, then the Politics does not move and all the alert Discontented suffer two stress.
·       Suicide. If the mutineer is alone (the only Human out of the vats), he is killed if 1d6 is less than his stress unless the Politics is white. Promoted Robots can commit suicide. If he survives, and is Discontented, then he moves the Politics and suffers stress per the above paragraph.
·       Robotic Mutiny. If the mutineer is an unpromoted Robot, it is first promoted and then it mutinies as if Human. This represents what happens if a Robotic computer spontaneously becomes conscious.
·       Robonauts and those in the vats do not suffer any effects from mutinies.
·       Beehive Warfare. If in a Beehive and the mutineer is Discontent, instead of the above, war breaks out. If there is only one alert Passenger, then they attempt Suicide, as above. Otherwise, the alert Passengers are divided into two camps: the mutineers (all alert Passengers aligned with the polity of the mutineer, along with the vats mutineer if a Vats Mutiny), and the loyalists (everyone else alert). The mutineers roll one dice and add their ballot box icons, and the loyalists roll one dice and add their badges. The high roller wins the war; ties have no effect. Kill off a number of losing Passengers equal to the winning die roll less the losing die roll, including the vats mutineer if a failed Vats Mutiny. You choose the victims. If the mutineers win, the Politics is changed to that of the mutineer, and all surviving alert humans suffer stress equal to the number of steps that the Politics changes; if they draw or lose the mutineers suffer two stress.
NOTE: If a mutiny shifts the Politics to purple, then the Robots are legally promotable (Y4.C). If a mutiny occurs during the Red Polity, then add one badge to each alert Human.[xix]
 
E. CANCER RISK (crab icon, Humans only). Killed if the 1d6 roll result is less than Age, or if a 2d6 roll result is less than Age if a cure for cancer breakthrough cube is attained.
 
F. ANTIMATTER RISK. Because beam-core braking (Y4.D) uses stored antimatter to consume the braking mass, rapid braking could lead to an antimatter explosion. After the Starship moves in a braking turn, roll 1d6. If the roll result is less than the number of steps decelerated that turn by beam-core braking, the ship is vaporized and Passengers must bail out.
 
G. BAILOUT RISK. If the Starship is dead-stopped at a world, or destroyed while drogue-braking or scoop-braking, each Passenger may opt to bail out. (This simulates parachuting in high-gee liquid-immersion crash suits.) If the Starship has no Spacewalkers, then a bailout is the only option to start a colony.
·       Risk. If a Passenger bails out, he is killed if the result of a 1d6 roll is less than his Age (Y1.D) minus the number of his badges (Y5.A). Roll for each Passenger who bails.
 
Example: An Age = 5 Passenger with 2 badges bails out. He dies if the roll is 1 or 2.
 
·       Vats. If a Passenger in the vats bails out, he additionally makes a “No Midwife” risk to decant (Y4.I).
·       Exploration. If the Passenger survives a bailout into an unexplored system, make an Exploration Roll per Y7.B.
 
H. KILLED. If a Human dies on a Starship, discard them to the Graveyard. If a Human dies aboard a Beehive, discard to the bottom of the Colonist Deck. If a black card (i.e. Robonaut or unpromoted Robot) or promoted Robot is killed, decommission to your hand, except a hostile goo is decommissioned to the bottom of the Patent deck and half the cubes (rounded up) are moved to the ship’s stack as Factory Cubes, the remainder being discarded.
 
I. DEMOTION. Flip a promoted Passenger or Bernal to its unpromoted side, or kill it if it is already unpromoted or is a crew.
·       Passengers. Passengers are demoted by Alzheimer’s (Y6.B), which afflicts both Humans and Robots, but not Robonauts.
·       Bernals. Bernals are demoted (as a stack) or killed (individually) if the Starship flies into the LIC (Y2.B).
Y7. INTERSTELLAR EXPLORATION AND VICTORY
A. STAR SITES AND LOCAL INTERSTELLAR CLOUDS.[xx] The ovals on the map are called spaces. A red space indicates a Star Site. The blue space indicates a Local Interstellar Cloud (LIC). To explore either one, either send a wisp (at a range up to 12 spaces, per Y4.F) or come to a dead stop on its space.

NOTE: A Starship can return to a dead-stop at Sol to retry a Sol Exit, to refuel or drydock (as operations), or to replace all colonists and crew lost to the Graveyard by restoring these to the starship stack in the marked spaces on the Age track (Y1.E) as a Business operation called Recruit.
NOTE 2: A Starship can avoid the “No Midwife” risk if they are in the Sol space (regardless of the speed they are travelling at). 

B. EXPLORATION TABLE ROLL. Each Star Site has an Exploration Table listed on the map. During a Wisp or Exploration Operation, roll 2d6 and place a Claim Disk of your color on the number rolled. Thus, the position of this disk indicates the planets discovered. A failure busts the entire site with a black disk.
·       A gas giant is a gas giant or ice giant where S fuel isotope (e.g. helium-3) is available during a refuel operation; also drogue and scoop-braking may be performed.
·       A fuel planet is a Super-Earth, hot Earth, gas giant moon, cometary halo or wandering comet where M, V, and D fuel isotopes are available during a refuel operation (Y4.E).
·       A habitable planet is an aqueous world where a Human migration colony can be founded which ends the game (Y7.D). Each surviving Human (including in vats) at a colony at a habitable planet is worth 1 VP per mass point.
·       A living planet means that the habitable planet is alive with ET lifeforms, which doubles your colonizing VP.
NOTE: As indicated by the brackets, some rolls indicate multiple planets are found.
 
Example: A 3 is rolled while exploring Tau Ceti, so a Claim Disk at “2– 3” is placed. As indicated by the bracket, a living planet, a habitable planet, a gas giant, and a fuel planet have been discovered.

Early Exit (optional). An Interstellar Quick Start may begin before the cataloging of all nearby exoplanets is complete, and therefore without knowing whether habitable planets exist at all. In this instance, any exoplanet search at the start of the game rolls the first dice roll of the 2d6 exploration roll and puts the dice face up next to the star site. The search stops when a roll = 1. The second dice is rolled only when the site is explored during the game (or when using a free exoplanet search provided by a future). Exploration rolls using an SOS Wisp are performed using gravitational lensing which provides much higher accuracy than the initial search - the SOS Wisp roll is to see whether the gravitational lens for a particular star site has come online.

C. EXPLORING LIC. Exploring in the LIC (Y7.A) is successful if a 2, 3, or 4 is rolled on 2 dice (2d6), indicating a fuel planet (Y7.B) is found (among the Oort and Nomad Bodies in the Local Interstellar Cloud). If successful, leave a claim disk on top of the LIC Site. A failure busts the targeted LIC space with a black disk.
 
D. ENDING THE GAME. You can choose to end the game at any time, even part way through a turn. If you are dead-stopped at a Habitable or Living Planet, you may form a colony for victory points. To form a colony, you must “bail out” if you do not have any alert spacewalkers and may optionally perform a bioengineering op (Y4.G) with each of your biotechs and a parenting op with each of your domestics (Y4.H), then count the VPs from the resulting colony towards your victory point total. (The new Colonists created this way represent the first generation born on the New World.)
·       The game also ends if you fly off the map, or cannot move any further.

Legacy Game. You can choose to start a new Quick Start game at a dead stop at any colony you found at a Habitable or Living planet instead of Sol, retaining all the star site exploration (but not LIC) and breakthrough results from the previous game. You cannot found additional colonies in any site a colony is already present at and Sol Exits are only available at Sol.
 
E. INTERSTELLAR VICTORY CONDITIONS. You receive Victory Points (VPs) for each Claim Disk (Y7.B), plus one VP for each Human mass point in a colony (Y7.D). If the system contains a living planet, score 2 VPs instead of 1 VP for each colony mass point. Note that Robots do not count unless promoted.
·       0– 3 VPs = Major loss discouraging future attempts, leading to near-term human extinction.
·       4– 6 VPs = Loss. Negative growth leading to eventual extinction.
·       7– 12 VPs = ZPG. Colony birth rate equal to colony death rate.
·       13– 24 VPs = Linear Victory with constant growth of human populations.
·       25+ VPs = Exponential victory with an unstoppable human presence in the galaxy.
Y8. INTERSTELLAR EXAMPLE MISSIONS
A. SEED STARSHIP VARIANT (Daniel Eliot Boese). Starting TW thruster: Antimatter sail with 20 tanks of fuel. Passengers: two or three Humans and/or Robots with a maximum mass of 7. Maximum dry mass, including thruster: 9 (any heavier and the ship will not have enough acceleration/deceleration to reach 6.5% light-speed).

Mission profile: Use the Jupiter flyby exit. Accelerate 3 steps to speed = 2; start decelerating before your destination (so that the sail has time to decelerate 5 steps to a full stop). Braking can also be used.

Special Rules. Humans start in the vats as infants (Age = 1). Because this lightweight ship carries no Bernal, a radiation roll is made every turn in the LIC. However, roll separately for each Human (exception to Y2.B).

Risks. The ship has no radiators, and the Humans in vats are too young to degrade from Alzheimer’s. Therefore, if nobody is alert during the cruise, the only risk is radiation. There is the radiation roll for the Jupiter exit, the radiation roll each turn in the LIC (because there is no Bernal), and at the destination, there is a radiation roll for the first Passenger to be awoken, since there is no Midwife. Therefore, use only the most radiation-resistant Passengers, and travel only to the closest destinations (e.g. Alpha Centauri). At rad-harness = 6, the antimatter sail itself is radiation-proof.

Robots. If a raygun Passenger is kept alert, then exploration wisps can be sent (e.g. to Proxima Centauri). An engineering Robot can keep all the Robots rejuvenated. A biotech Robot can act as a Midwife, or (even if all the Humans have died) use bioengineering to create new Humans (presumably from stored gametes or even digitally-stored DNA).

Complications: If there's no good planet at the destination, the Passengers will need to refuel for another destination (assuming a fuel planet is present, otherwise they are marooned).
B. EXAMPLE NASA MISSION (Phil Eklund).
No exoplanet search variant rule used aka “the Eklund end run”. This allows a Quick start with 20 WT instead of 16.

Starting TW thruster: Colliding FRC 3He-D Fusion ( 2-1/2 ). Supported by Ultracold \& Fusor Reactors, Multiphase Generator, and Maragoni Radiator.

Starting Crew: Alert: NASA (science), Svalbard Caretakers (science), Babbage Halbonauts (engineers), Security (network), Ablative Laser (spacewalker), Nanobots (spacewalker), Lorentz Propelled Microprobes (network).
In Vats: Juiced Cosmonauts, Vatican Observers, Malcolm, Islamic Refugees.
Mass: 40 dry mass with 3 steps of fuel. Two Bernals are carried.
Starting Governance: Right Wing Ending Governance: Right Wing.
Starting Trajectory: Sol exit Neptune.
Mission Highlights:
Year 12 (1): The Ablative Laser is converted into a factory cube allowing the heavy science team to make a Math Breakthrough, and a Wisp busts Alpha Centauri. Tragically, the cosmonauts die in the vats from brain rot.
Year 36 (3): Two more breakthroughs have been accomplished: Cure for Cancer and  Ecological, using the remaining robonauts, so the librarians are sent to the vats. The security system breaks down. A fuel planet is discovered at Barnard’s, so the decision is made to head to 61 Cygni by way of Barnard’s. Because of the small amount of fuel carried (30 tanks), the ship is at its maximum speed of 4% c (1/turn).
Year 48 (4): The NASA astronauts are dead at 100 years old. They served well, first as scientists, then as entrepreneurs. The caretakers deteriorate in the vats, and the decision is made to revive and promote them. A flurry of Robotic engineering activity as the ship flies through the LIC.
Year 72 (6): The Halbonauts have shut down which prevents the security system from being further repaired. Various mutinies defused because of the strength of the Right Wing voters. The caretakers and pilgrims marry, which proves to be a highly successful union.
Year 84 (7): Despite the ravages of Alzheimer’s, the couple conceives baby Pantrophists. Gray goo is defeated by the 6 active badges, and the Robots are emancipated.
Year 96 (8): The pilgrims die of cancer, and their spouses the caretakers go into the vats at age 120. One Robot fails, so the ship is controlled by the emancipated Navigators, now downgraded to just a mechanical Security System.
Year 168 (14): The Starship passes the Barnard System on its way to 61 Cygni. The security system failed at turn 10, leaving nobody alert. The caretakers have gradually wasted away in the vats, dead by turn 13. Malcolm dies in the vats on this turn, leaving only two Humans left, both in the vats and nobody at the wheel.
Year 192 (16): Approaching LIC, the refugees are awakened. But they fail to survive the “no Midwife” risk. Only the pre-teen Pantrophists are left, and they (ironically) are scientists without any skills to 3D print the Robots needed to keep the EM shields operational. Earth is out of SOS range. So, the decision is made to leave the youngsters in the vats. Obviously, we should have stopped at Barnard’s fuel planet to rebuild the hull and refuel.
Year 324 (27): The second and final Bernal is destroyed in the Bow Shock of 61 Cygni, lowering the dry mass to 20.
Year 372 (31): As the ship approaches 61 Cygni, the only systems that remain functional were the Pantrophists in the vats. The starship engine has lost all its supports due to glitches, and there are no engineers to 3D print replacements. Because of the midwife catastrophe, conditions at 61 Cygni are unknown. The desperate plan is to revive the pre-teens as biotechnicians, self-promote to Sybonts (this step is essential, because only the promoted side has the ability to for magnetic then drogue braking, assuming a gas giant is found), then train for piloting skills for the braking. In a Hollywood finish, the rapidly aging Sybonts barely survive gray goo and two accidents, as well as the drogue brake hazard and radiation. They discover both a gas giant and a habitable planet, using the easily-missed rule that a drogue brake attempt forces an exploration roll.
Year 384 (32): A second drogue brake allows the starship to come to a complete halt before it leaves the system (at ½ speed). The Sybonts survive the bailout risk that not having an alert spacewalker entails, and form a colony.
Loss, Negative Growth. One mass-3 Human landed at a habitable planet on 61 Cygni.
Ending Score = Human Mass 3 + 2 claims = 5 VP.
C. CONSERVATION OF MASS AND SPECTRAL TYPE (Andrew Doull)
One of the less obvious but significant improvements in the 3rd edition of High Frontier is that more attention has been paid to tracking mass correctly, by redesigning the fuel tracker to be a strip and changing card text, so that it is not possible to get mass for free by e.g. promoting cards in space. During the 3rd edition design process I was reasonably assured that I could persuade Phil to accept a suggestion if I included the phrase ‘conservation of mass’.
High Frontier Interstellar is more forgiving about tracking mass, in order to simplify Parenthood and allow a “cube economy” to exist on the starship. Per the footnotes, it is possible to get free mass from nowhere by converting a low mass robot or robonaut into a mass 5 Factory Cube. However in making cubes a fungible commodity, a central tenant of High Frontier is lost - that of differing asteroid spectral types giving access to different technologies due to composition and isotope availability.
For players looking for more difficult Interstellar game, if there is such a player, it is possible to play a version of the game which is more careful about preserving conservation of mass, and (for want of a better term) conservation of spectral type. The following rule changes should be adopted:
Factory cubes cannot be created by or used for operations. This prevents Cube Nanofacture, Eureka, Endowment and Research operations and only permits Bernals to be used for Nano-Reconfigure Hull operations. Hostile goo do not produce cubes when cannibalizing Bernals and rebuilding Bernals during a Dry Dock in a Beehive (as an exception) does not require cubes.
Scientists get a new Research operation as follows: ‘Discard a black card to draw a black card of matching spectral type from any patent deck. Draw a second card of matching spectral type if both have less mass than the discarded card.’
(Optional) Parenthood and Research operations are only permitted when the discarded card is of greater than or equal mass to the newly drawn card.
Challenge mode: Complete a ‘mundane’ Interstellar game, with the above rules, the optional rules in point 3 above and under footnote [iii] and where only the white side of colonists are permitted (and therefore no robots). You are not expected to win.
D. GENERATION SHIP - A MULTIPLAYER SINGLE SHIP VARIANT (Andrew Doull)
The Generation ship is a Beehive Ark using the Beehive Quick start rules with the following modifications:
The ideas turns are resolved in player order with each player beginning with 5 WT. If a player acquires and places the starship engine or its supports in the starship stack, they may also place the crew card of their faction for free - the player who first does this is in power at the start of the game. Note: Players are free to keep non-colonist cards in their hand.
When acquiring colonists using either ideas turns or Parenthood or Bioengineering operations, instead of picking a random colonist draw the top 3 cards on the colonist deck and choose one, returning the other two to the bottom of the deck in any order.
Each player is responsible for choosing the operations of colonists loyal to their faction during the Operations phase, resolved in player turn order. Player Purple is also responsible for choosing promoted robot operations. The first player for the turn (Captain) and the role of the player responsible for choosing the operations of Robonauts (Executive Officer), unpromoted robots (Roboticist) and colonists (Councillor) without a loyal faction in the game rotates in a round-robin fashion at the end of the turn: these additional roles are assigned out at the start of the game clockwise from the left of the in power player, who begins as Captain. Exception: Any card or cube placed in the starship stack or age track is controlled by the player who placed it until the end of their turn.
Any operation which affects a card or uses a cube which the player controlling the card or cube does not agree to is a felony if a human or robot passenger, or only permitted during Anarchy otherwise. The starship stack and cubes are controlled by the in power player. Bernals are controlled by the faction controller for that Bernal. Desperation actions are only permitted with a majority vote, with the in power player acting as the tie breaker.
Decommissioned cards: Any decommissioned black card goes to the hand of the player who controls it that turn - during Anarchy, cards decommissioned from the starship stack go to the Captain. If a crew card is decommissioned, flip it and put it on the bottom of the colonist deck.
Victory conditions: If the game concludes with 13+ VP achieved, the player with the highest mass of loyal humans is the winner.
Y9. INTERSTELLAR STRATEGY ADVICE
By Francisco Colmenares and Sam Williams

A. General Guidelines. (independent of configuration):
·       Any starship thruster can get you through the game; consider it part of the challenge. I recommend just taking the top card and obtaining its supports (assuming it’s just a single card or two). The Beehive start will allow you to choose between two thrusters. Otherwise it’s not worth the WTs to fish for a specific thruster. As a house rule, you can try choosing a specific thruster (this adds to the replayability of the game).
·       You should try to keep Robotic biotechs or Humans alert in your Starship, because decanting without Midwives is hazardous. However, unless you can get the Starship up to at least speed of 2, expect to have all your Humans in vats and only Robots alert.
·       If you don’t have any Robotic engineers, keep at least some of your Spacewalker capable Robonauts powered down. The dreaded Ship Event Roll of 2 will kill human engineers with cancer and Robonauts through accidents, and risks leaving you defenseless for the next LIC you encounter and unable to refuel if you stop without finding a living or habitable planet.
·       During your idea turns, it’s more fruitful to spend the majority of your WTs rifling through the colonist deck. Good crew composition is your number one priority, even if you don’t end up taking all the colonists with you. Consider the colonists you draw to be a pool of candidates you are evaluating to take with you.
·       If you don’t have a scientist along, you will need to have your crew act as a scientist and research all the Robonauts and supports you think you will need for your entire trip. Alternately, make your crew an engineer-businessman, 3D print a Robonaut then Invent it to become a scientist (this requires discarding a robot). You can also possibly gain a breakthrough from SOS wisps to Earth, but this is quite chancy.
·       Fret over your crew, but not their political affiliations. You will need to handle mutinies as they arrive.
·       Don’t fret over fuel too much, which is very expensive. For high acceleration, fuel thirsty starship engines, take the Sol Neptune exit and spend several turns dead stopped refueling at Sol to get the additional fuel you’ll need. There will also be opportunities to refuel, plus alternatives to increase/decrease your speed without burning fuel (e.g. protium breakthrough, pilots).
·       Do not consider a voyage without a raygun Robonaut or other Networker unless your destination is close. The ability to explore at a distance ensures you don’t make fruitless stops and risky braking at a site with no habitable planets or fuel.
 
B. Mission Priorities. (listed in order of urgency):
1.     Keep your Bernals promoted, and promote or self-promote Humans so as to be robust against the threat of Alzheimer’s. Domestics and Biotechs are good second professions to permit you to do this.
2.     Speed up the ship, but leave enough fuel to stop it again, or take other braking measures (engineering cube nanofacture, pilot braking).
3.     Conserve your Humans, and vat them as soon as they have more than 3 stress.
4.     Cure cancer (science breakthrough),
5.     Use 3D printing to keep your Robots and Robonauts young and risk free.
6.     Keep one or more age 1 or 2 Humans in the vats to act as parents to replenish your Passengers as they are killed.
7.     Drydock before the Bernal reaches 120 years old.

INTERSTELLAR FOOTNOTES
[i] The game starts with a Kardashev I civilization, when individuals have terawatts of energy available. (Today’s global energy output is about 15 terawatts.) Operations are undertaken by reproducing machines, and products become cheap and plentiful. Due to the “tyranny” of the rocket equation, energy is cheap and mass expensive for interstellar travel. Therefore, a useful starship engine must transform an appreciable fraction of valuable mass into the kinetic energy of expelled propellant. (I do not believe in postulated “reactionless” drives; every vehicle in all of history has followed Newton’s action-reaction principle.) Electric engines are hopeless because of very low specific power and low-thrust. Only nuclear annihilation drives, transforming mass to energy, have any chance for the stars. The fuel economy of these engines is proportional to the square root of the percentage of mass converted to energy. This percentage is theoretically 100% using antihydrogen fuel, about 1% for fusion fuels, and 0.1% for fission fuels. This assumes all the fuel mass is available for thrust generation, which is far from the case for real engines. Especially fission, where the energy liberated in the fission process appears as heat in the fuel rods. I do not include any fission starship engines: the Medusa has been converted to D-D fusion and even the Zubrin Drive has insufficient Isp. Starship engines in the game using antimatter are the antimatter sail and the H-B magnetic-inertial (this last is really a fusion engine initiated by antimatter). Antimatter drives have severe problems coupling the annihilation products to the propellant. Even using an extremely dense propellant, such as lead, most of the reaction energy is lost. Fusion fuels, even with two orders of magnitude less energy than antimatter, are ultimately superior for reasons of propellant coupling. Furthermore, fusion fuels can be mined, whereas antimatter normally must be made.
 
[ii] The diamonoid electrodynamic tether generates electricity at the expense of a small amount of the starship’s momentum as it is dragged through a changing electric field. How well it works depends upon how much the electric field alters between the stars. Part of the display of a comet may be due to the charge it accumulated in interstellar space and discharges as it enters heliocentric space.

[iii] Solar powered Robonauts are instead beam-powered by base stations installed on the starship. Optionally, you can require that Robonauts with no support requirements instead have the support requirements of the white side of the card, except Nanobot which acts as a factory cube during gray goo risks and war.
 
[iv] A beehive ark is of vastly greater mass than other starships. Each mass point is one million tons (instead of 40), and each colonist mass point is 10,000 people. So, imagine that each card and unit in a beehive stack is a large number of units. The beehive population supposedly lives in an underground colony on a comet propelled by expelling the mass of the comet itself. Practically, this has many problems. Suppose the comet has the mass of Halley’s Comet, 200 billion tonnes, and uses a beehive engine with an exit velocity of 0.5% lightspeed. To accelerate it to 1 space per turn, or 1.3% lightspeed in the beehive scale, the rocket equation requires that 93% of the cometary mass be expelled (so that it ends up with only 14 billion tonnes). At this speed its kinetic energy is a breathtaking 10^26 newtons. It would require at least 33 Medusa engines to obtain the 55 giganewtons of thrust required to obtain this speed in a 24-year turn. To stop the beehive at its destination requires 93% of the remaining mass be expelled, ending up with less than a billion tonnes.
 
[v] All three close passes use the Oberth effect to accelerate the Starship. A solar Oberth flyby is also useful, utilizing the 230km/sec orbital velocity of Sol around the center of the galaxy. 

[vi] In a beehive, the radiators additionally represent sulfur lamps used as the energy source for the lichens.
 
[vii] The aging of the Bernal Stack represents the irreversible erosional loss of hull material from the impingement of dust at relativistic speeds. When the hull has suffered 120 years of erosion, the increased radiation causes the passengers to age faster. Only with a drydock can the hull be restored.
 
[viii] As each cube is treated elsewhere as mass 5, the Cube Nanofacture operation seems to violate the law of conservation of mass by adding mass to the starship. To avoid this violation, I have assumed that building the cube also reduces the mass of the other components on the ship by a corresponding amount. Because the mass of each card is assumed to change mass with redesign during the voyage, the dry mass of the starship is equal to the mass of its cards only when it embarks.
 
[ix] If the Politics is in the center position, this emancipates the robots (i.e. robotic promotion is legal). Just like Lincoln’s Emancipation Proclamation of 1863, this does not actually change the status of any of the enslaved. It just allows scientists or biotechs to promote Robots to full consciousness. Robots can also self-promote during a mutiny. This is spontaneous, accidental, and (unless the Politics is in the center) illegal.

[x] What role do businessmen perform?  Technically, everyone with a product or a service is a businessman, at least in a non-slave starship. So doctors, engineers, scientists, pilots, biotechs, emancipated robots, spacewalkers, domestics, networkers are all business entities. But the business profession is envisioned as facilitating all this, as well as keeping the crew sane with products, entertainment, and services. It also includes artists such as novelists, musicians, and holodeck painters. The business profession is something of a luxury when dealing with tiny starship crews, where every kilogram matters. They would be more useful in a Beehive. Nevertheless, even in a starship the business of keeping the crew sane and stress-free for decades and centuries is an important one. However, aesthetics and art appreciation might not be available for AI or emancipated robots because art and culture are quirks of the origins of consciousness, a theme investigated in my Neanderthal game.  These quirks are not likely duplicated in the origins of AI consciousness.  So the business class might be handicapped with AI crews.

[xi] Because it uses the interstellar plasma as a gradual friction force, the mag-sail works best between the stars. Further, the slowing force varies with the square of the velocity, so the mag-sail is only effective at higher speeds. Mag-Sail Brake rules research courtesy David Harris.
 
[xii] Stuff tossed into the antimatter hoppers is gone forever. This is not because the idea is lost, but because there are some things (special materials, wiring harnesses, etc.), which cannot be recovered by 3D Printing.
 
[xiii] The Electromagnetic (EM) Shields deflect charged space hazards such as Corona Mass Ejections (CMEs) and other plasmas. A promoted Bernal means the shield is operational, while an unpromoted one means the shields are shut down. If the shields are down, and for neutral dust, the ship relies on physical shielding: layers of c-nanotube composite and aerogel. Aerogel is mostly gas, weighs next to nothing, and particles will vaporize upon contact like meteors. Some of the aerogel and C60 will vaporize also. The vaporized gaseous matter will be deflected by the protective magnetic field. Ionizing bow lasers and shield vehicles are also used.
 
[xiv] Since each space on the Interstellar Map is half a light year, the speed of light is 24 spaces per 12-year turn. I cut this in half to allow for time spent for two-way communication between source and destination. A trade or SOS wisp can move at light-speed, because while in transit, it is not a physical thing, but only a code that can be translated into a message or product at the receiving end. Exploration wisps move slower (perhaps 50% light-speed).
 
[xv] This represents the next generation, enabled by a terraformed ecosystem utilizing lichen dusted with miRNA for genetic modification. miRNA is small non-coding RNA ubiquitously expressed in unicellular to multicellular eukaryotes.
 
[xvi] A beehive or half-speed starship travels about 1.3%c.
 
[xvii] Alzheimer’s and other complex diseases such as type 1 and 2 diabetes, autism, bipolar disorder, and allergies, show significantly higher concordance in monozygotic twins than in dizygotic twins or parent-child pairs. These diseases can be controlled through time-separated twinning, a strategy involving the collection and fertilization of human oocytes followed by several rounds of artificial twinning. If preimplantation genetic screening reports no aneuploidy or known Mendelian disorders, one of the monozygotic siblings would be implanted and the remaining embryos cryoconserved. Once the health of the adult monozygotic sibling is established, subsequent parenthood with the twins in the vats could substantially lower the incidence of hereditary disorders with Mendelian or complex etiology. Dr. Alexander Churbanov
 
[xviii] The gray goo problem, originally formulated in macroscopic form by John von Neumann and in nano-form by Eric Drexler, describes a doomsday scenario of out of control self-replicating robots which have escaped from the lab. Self-replicating robots are essential for various operations, such as refining fuel isotopes and hull reconfiguration.
 
[xix] I belong to the Julian Jaynes society and Dr. Jaynes postulates the origins of human consciousness as a breakdown of a former mode of authoritarian decision-making. This breakdown has led to "quests for authority", "the need for authorization", "the power of suggestion", and many other uniquely human tendencies. In this view, human politics tends toward the authoritarian corner of the Nolan Chart (the basis for the starship political diagram) as an artifact of how consciousness came to be. However, a conscious computer, having an entirely different origin of consciousness, would not share this purely biological ilk. Furthermore, if the machine happened to be immortal, the human basis for morality would be removed. Under these conditions, I speculate that a suddenly conscious machine would tend toward the center (or perhaps upper?) part of the Nolan in its dealing with other consciousnesses.
 
[xx] Several thousand years ago Sol blundered into the “Local Interstellar Cloud” (LIC), one of several nearby clouds composed of warm, low density (~0.3 atoms/cm3) dust and hydrogen blowing at us from the direction of the Scorpius and Centaurus constellations. The difficulty with this cloud is that the material is electrically neutral and thus passes undeflected through the ship’s primary EM Shield. Once safely outside the LIC, the starship enters the “local bubble”, where the only material is ultra-low density hot plasma (<0.001 atoms/cm3).
 
Z. PATENT CARD DESCRIPTIONS
Phil Eklund and Dr. Noah Vale
Note: Temperatures are listed in degrees Kelvin (K), (where 0 K is absolute zero, and water boils at 372 K).  To convert degrees K into degrees C, subtract 272.  Plasma temperatures are listed in kilo-electron volts (keV). To convert keV into degrees K, multiply by 11604000.
 
Z1. CREW CARDS
Crew Card – Your starting crew represents eight hardy specialists, plus their consumables and life support. They are housed in 16 tonne inflatable Bigelow habitation modules (1760 m3), made of Vectran (a “bulletproof” textile). Paired modules, 45 meters apart, are rotated at 5 RPM to provide 0.6 G of artificial gravity.  Food and atmospheric conditioning is provided by crops that grow without soil but have their roots misted with nutrients daily.  A plot 25 meters across provides all the foodstuffs for the year. Waste heat from plant evaporation requires low temperature radiators. A 10 tonne life support module requires 12 kWe, and communications from Ka band antennas require another 0.2 MWe.  A charged plasma sustains a high electrical potential (10 GeV) about the hab unit for protection against most galactic cosmic rays. When a charged particle passes through this magnetic field, its path curves to avoid the occupants. If a solar storm erupts, the crew must evacuate into a small (8m dia)  storm shelter.  The shelter is shielded by 100 kg/m2 of polyethylene (12 cm thick), plus water propellant and graphite.
 
Each of  the crew cards includes a H2-O2 chemical thruster.  The combustion of the cryogenic fuels hydrogen and oxygen produces an ideal specific impulse of 528 seconds. The product is water, which is exhausted through a converging-diverging tube called a De Laval nozzle. An example is the Space Shuttle main engine, with a specific impulse of 460 seconds (fuel economy of 0.25). It uses a nozzle with a 180:1 area ratio, regeneratively-cooled with liquid hydrogen.  The chamber temperature is 3500K, and the chamber pressure is 2.8 MPa.  The engine has a thermal efficiency of 98%, a mixture ratio of 5.4, and a frozen-flow efficiency of 55%.  A 2000 MWth chamber generates 440 kN of thrust and a thrust to weight ratio of one gravity.  Space Transportation Systems, American Institute of Aeronautics and Astronautics, New York, 1978.
Z2. THRUSTERS 
Ablative nozzle thruster – This hemispherical nozzle, made of silicon carbide, is designed to intercept thermalized x-rays. A thin film of the SiC is vaporized and ejected for thrust.  If the fuel is a hard gamma emitter (e.g. positronium or antimatter), the radiation must be thermalized (i.e. softened) by surrounding the fuel pellet with a layer of lead. The lead absorbs the gammas which are re-emitted as softer x-rays better absorbed by the SiC.
 
Ablative plate thruster – A nozzle can be plate-shaped.  Pulsed energy is intercepted, and a thin film of the plate is vaporized and expelled.  This expellant is a type of collimated propellant. Project Orion uses a shock-absorbed plate, tapered so as to absorb nuclear energy evenly.  
 
Cermet NERVA thruster – A heat exchanger cooled by water or liquid hydrogen can capture the neutronic energy of a dirty nuclear reaction. It is made of graphite and cermet (ceramic-metal) composites, jacketed by a beryllium oxide neutron reflector. The exchanger is similar to NERVA but has a low pressure chamber with thin foil or advanced dumbo fuel elements using cermet substrates; the fuel elements can be fission, fusion, or antimatter.  The chamber temperature is limited to 3100K for extended operational life of the solid fuel elements.  At this temperature, the system takes advantage of the disassociation of molecular H2 to H, with significant increases in specific impulse being achieved with chamber pressures below 1 MPa (10 atm). A propellant tank pressurized to 2 atm expels the LH2 coolant into the matrix without the need for turbopumps.  This open-cycle coolant is expanded through a hydrogen-cooled nozzle of refractory metal to obtain thrust. An 940 MWth matrix at 96% thermal efficiency, 76% frozen-flow efficiency (mainly H2 dissociation, including recombination in the nozzle), and 96% nozzle efficiency yields a power density of 340 MW/m3, a thrust of 134 kN, and a specific impulse of 1 ks.   “Operating Characteristics and Requirements for the NERVA Flight Engine,” Altseimer, J.H., et al., AIAA Paper 70-676, June 1970.  (The classic nuclear-thermal design, reconfigured for low pressure fission, fusion, or antimatter fuel elements arrayed dumbo-style.)
 
Colliding Beam H-B Fusion thruster –Hydrogen and boron-11 can be brought to fusion by energetically colliding a tangential beam of H with 11B in a FRC (field reversed configuration) plasma.  If in a magnetic mirror configuration, the helium-4 reaction products are exited for thrust.  The Q is 2.63. With open-cycle cooling, a specific impulse of 40 ks is attained. The efficiency of 83% assumes that some means of controlling the bremsstrahlung radiation is found.

De Laval nozzle thruster – The familiar converging-diverging shape of the de Laval nozzle is designed to accelerate a propellant flow to supersonic speeds. Its operation relies on the different properties of gases flowing at subsonic and supersonic speeds. As the nozzle constricts, the speed of a subsonic flow of gas will increase to maintain a constant flow rate. The gas flow through a de Laval nozzle is isentropic (gas entropy is nearly constant). At subsonic flow the gas is compressible; sound, a small pressure wave will propagate through it. At the throat, the gas velocity locally becomes sonic (Mach number = 1.0), a condition called choked flow. As the nozzle cross sectional area increases the gas expands and the gas flow increases to supersonic velocities where a sound wave will not propagate backwards through the gas as viewed in the frame of reference of the nozzle (Mach number > 1.0).  The nozzle illustrated has an area ratio (ratio of throat area to exit area) of 100:1, and a nozzle efficiency of 90%. It is regeneratively-cooled by passing liquid hydrogen coolant through channels surrounding the nozzle wall.  The heated hydrogen is then injected into the rocket as propellant.
 
%Electric sail – The solar wind dynamic pressure varies but is on average about 2 nPa at one AU. An electric sail generates thrust from this stream of particles in a manner similar to a mag sail, except that electric rather than magnetic fields are used.  The electric sail geometry employs hundreds of long (e.g., 100 km), thin (e.g., 20 microns) conducting tethers (wires). The entire sail rotates with a period of 20 minutes to keep its wires in positive tension. A solar-powered electron gun (typical power a few hundred watts) is employed to keep the spacecraft and the wires in a high (up to 20 kV) positive potential. The electric field surrounds each wire a few tens of meters into the surrounding solar wind plasma. Therefore the solar wind ions "see" the wires as rather thick obstacles. It is this multiplication factor that allows sails using the solar wind to outperform sails using photon pressure, which is 5000 times stronger. The positively-charged tethers repel solar wind protons, thus deflecting their paths. Each 100 km tether, massing but a kilogram, generates 0.01 N of thrust this way. Simultaneously it also attracts electrons from the solar wind plasma, which are neutralized by the electron gun. Potentiometers between each tether and the spacecraft control the attitude by fine-tuning the tether potentials. Additionally, the thrust may be turned on or off by simply switching on or off the electron gun. To make the design robust against meteoroids, each tether is composed of multiple wires with redundant interlinking. One limitation of the electric sail is that since it uses the solar wind, it cannot produce much thrust inside a magnetosphere where there is no solar wind. Unlike photon sails, whose propelling thrust varies as the inverse square distance from the Sun, the electric sail thrust force decays as (1/r)^{7/6}.
%Electric Sail - AIAA Journal of Propulsion and Power, 20, 4, 763-764, 2004, Janhunen, P. and A. Sandroos, Simulation study of solar wind push on a charged wire: solar wind electric sail propulsion, Ann. Geophys., 25, 755-767, 2007).
 Hall Effect thruster – This ion rocket accelerates ions using the electric potential maintained between a cylindrical anode and negatively charged plasma which forms the cathode. To start the engine, the anode on the upstream end of the thruster is charged to a high positive potential by the thruster's power supply. Simultaneously, a hollow cathode at the downstream end generates electrons. As the electrons move upstream toward the anode, they encounter a magnetic field produced by powerful electromagnets. This field traps the electrons, causing them to form into a circling ring at the downstream end of the thruster channel. The Hall thruster gets its name from this flow of electrons, called the Hall current. This gyrating current collides with a stream of magnesium propellant, creating ions. As the propellant ions are generated, they experience the electric field produced between the anode (positive) and the ring of electrons (negative) and exit as an accelerated ion beam. A significant portion of the energy required to run the Hall Effect thruster is used to ionize the propellant, creating frozen flow losses.  This design also suffers from erosion of the discharge chamber. On the plus side, the electrons in the Hall current keeps the plasma substantially neutral, which allows much greater thrust densities than ion drives.   
Ion drive thruster –An electrostatic particle accelerator in space is effectively an electric rocket. The illustrated design uses a combination of microwaves and spinning magnets to ionize the propellant, eliminating the need for electrodes, which are susceptible to erosion in the ion stream.  The propellant is any metal that can be easily ionized and charge-separated. A suitable choice is magnesium, which is common in asteroids that were once part of the mantles of shattered parent bodies, and which volatizes out of regolith at the relatively low temperature of 1800 K.  The ion drive accelerates magnesium ions using a negatively charged grid, and neutralizes them as they exit.  To reduce grid erosion, C –C grids are used. Since the stream is composed of ions that are mutually repelling, the propellant flow is limited to low values proportional to the cross-sectional area of the acceleration region and the square root of the voltage gradient.  Decoupling the acceleration from the extraction process into a two-stage system allows the voltage gradients to reach 30 kV without vacuum-arcing, corresponding to exit velocities of 80-210 km/sec. A 60 MWe system with a thrust of 1.5 kN utilizes a hexagonal array, 25 meters across, containing 361 accelerators. Frozen flow efficiencies are high (96%). To boost the acceleration (corresponding to the “open-cycle cooling” game rule), colloids are accelerated instead of ions. Colloids (charged sub-micron droplets of a conducting non-metallic fluid) are more massive than ions, so a colloidal thruster boosts thrust at the expense of fuel economy.
 
Mag sail– A magnetic sail maneuvers by reaction with the protons of the solar wind.  At 1 AU, this wind comprises several million protons per cubic meter, spiraling away from the sun at 400 to 600 km/sec (256 µwatts/m2).  When such charged particles move through a magnetic field formed by the mag sail, a tremendous loop of wire some 2 km across, they are deflected. An unloaded Mag Sail this size has a thrust of 100 newtons (at 1 AU) and a mass of 20 tonnes. The wire is superconducting whisker, at 10 kg/km, connected to a central bus and payload via shroud lines. The loop requires a simple multi-layer insulation and reflective coatings to maintain its superconducting temperature of 77 K. Because the sail area is a magnetic field, which has no mass, a mag sail has a superior thrust/weight ratio than photon sails.  Just as with photon sails, lateral motion is possible by orienting the sail at an angle to the thrusting medium.  A magnetic sail can also develop thrust from planetary and solar magnetospheres, which decrease as the fourth power of the distance from the magnetosphere source. Field strength is typically 10 µT (in Earth’s magnetosphere) or less in the solar magnetosphere.  The mag sail illustrated has its thrust augmented by a spinning disk photon sail attached to its staying lines.  The sail is maneuvered using photonic laser thrusters (propellantless thrust derived from the bouncing of laser photons between two mirrors). 
 
Magnetic nozzle thruster – High specific impulse thermodynamic rockets benefit from a nozzle that is not limited by the melting point of its material. Magnetic nozzles direct the exhausted flow of ions or a conductive plasma by use of magnetic fields instead of walls made of solid materials (see de Laval nozzle). If superconducting coils are used, these must be thermally shielded to remain in the superconducting range. The design illustrated operates at a throat magnetic field strength of 25 T and a nozzle efficiency of 77%.
 
Mass driver thruster – An electrodynamic traveling-wave accelerator can be used as either a thruster or a payload launcher. Either system uses a lightweight bucket, banded by a pair of su­perconducting loops act­ing as armatures of a linear-electric guideway, loaded with of regolith (or anything else handy).  The thruster illustrated accelerates the payload at 50,000 gee's, utilizing 3.8 GJ of electromagnetic energy stored inductively in supercon­ducting coils.  The trackway length is 390 meters. The 36kg of payload is ejected at 11 km/sec every 30 seconds while the bucket is decelerated and recovered. Efficiency is 85%, although cryogenic 77°K radiators are needed to cool the superconductors. A mass-driver optimized for materials transport rather than for propulsion uses a higher ratio of payload mass to bucket mass. With a 54% duty cycle, this system can launch 10 kt/yr. of factory products or stones. Coupled with a pointing accuracy in the tens of microradians, the latter can launch payloads or threaten enemy bases at destinations millions of kilometers away. The ejected mass velocity is equal to the Earth escape velocity (11 km/sec), making it feasible for a terrestrial mass driver to launch payloads up the side of a convenient equatorial mountain. Imparted with a launch energy of 76 GJ, a one tonne payload the size and shape of a telephone pole with a carbon cap would burn up only 3% of its mass and lose only 20% of its energy on its way to solar or Earth orbit.   The High Frontier: Human Colonies in Space, Gerard K. O’Neill, 1977.
 
%Metastable helium thruster – Metastable helium is the electronically excited state of the helium atom, easily formed by a 24 keV electron beam in liquid helium.  If the spin-orbit decay could be suppressed by a coherent laser pump, its theoretical lifetime would be eight years (as ferromagnetic solid He*2 with a melting temperature of 600 K). Spin-aligned solid metastable helium could be a useful, if touchy, high thrust chemical fuel with a theoretical specific impulse of 3.2 ks. http://stinet.dtic.mil/cgi-bin/GetTRDoc?AD=B088771&Location=U2&doc=GetTRDoc.pdf
J.S. Zmuidzinas, "Stabilization of He2(a 3Sigmau+) in Liquid Helium by Optical Pumping", unpublished (1976) 
Mirror steamer thruster – Water is an attractive volumetric absorber for infrared laser propulsion. Diatomic species formed from the disassociation of water such as OH are present at temperatures as high as 5000 K, and can be rotationally excited by a free electron laser operating in the far infrared. The OH molecules then transfer their energy to a stream of hydrogen propellant in a thermodynamic rocket nozzle by relaxation collisions. Beamed heat can also be added by a blackbody cavity absorber.  This heat exchanger is a series of concentric cylinders, made of hafnium carbide (HfC).  Focused sunlight or lasers passes through the outermost porous disk, and is absorbed in the cavity.  Heat is transferred to the propellant by the hot HfC without the need for propellant seeding.  The specific impulse is materials-limited to 1 ks.
“Solar Rocket System Concept Analysis”, F.G. Etheridge, Rockwell Space Systems Group.  (I resized the Rockwell “Solar Moth” design for 3 kN thrust).

MPD T-wave thruster – Impulsive electric rockets can accelerate propellant using magnetoplasmadynamic traveling waves (MPD T-waves).  In the design shown, superfluid magnetic helium-3 is accelerated using a megahertz pulsed system, in which a few hundred kiloamps of currents briefly develop extremely high electromagnetic forces. The accelerator sequentially trips a column of distributed superconducting L-C circuits that shoves out the fluid with a magnetic piston.  The propellant is micrograms of regolith dust entrained by the superfluid helium. The dust and helium are kept from the walls by the inward radial Lorentz force, with an efficiency of 81%. Each 125 J pulse requires a millifarad of total capacitance at a few hundred volts.  Compared to ion drives, MPDs have good thrust densities and have no need for   charge neutralization. However, they run hot and have electrodes that will erode over time. Moreover, small amounts of an expensive superfluid medium are continually required.
 
n-6Li microfusion thruster – Rockets that fly by using atomic explosions, such as Project Orion, require huge shock absorbers, due to the fact that the minimum explosive yield for fission bombs is about a quarter kiloton.  The pulse size can be brought down to microfission levels by the use of exotic particles. The isotope of lithium 6Li can be brought to spontaneous microfission by interaction with particles with very large reaction cross sections such as ultracold neutrons. No “critical mass” is required. This is a clean reaction, with charged particles (T and He) as products, each at about 2 MeV.  The system illustrated uses a 5 meter magnetic nozzle to transfer the microexplosion energy to the vehicle.  This maintains the advantage of magnetic impulse transfer indicated by the MagOrion concept (combination of Orion and the magnetic sail). A fuel reaction rate of 30 mg/sec produces 2000 MWth.  At a pulse repetition rate of one 224 GJ detonation every 2 minutes, the thrust is 5 kN at a 16 ksec specific impulse.  A hydraulic fixture oscillates at a tuned frequency to provide a constant acceleration to the spacecraft. The combined frozen-flow and nozzle efficiencies are 25%, and the thermal efficiency is 80%.
Ralph Ewig’s “Mini-magOrion” concept, modified for n-Li6 fission, http://www.andrews-space.com/images/videos/PAPERS/Pub-MMOJPLTalk.pdf
 
Photon heliogyro sail – A photon sail consisting of multiple spinning blades is called a heliogyro. Its blades are rigidized by centrifugal force and pitched to provide attitude control, much like a helicopter. Although a spinning design does not need the struts of a kite sail, the centrifugal loads generated must be carried by edge members in the blades. Moreover oscillations are created when the sail’s attitude changes, which need to be restrained by transverse battens. Small sail panels prevent wrinkling from the curvature in edge members between the battens. For these reasons, the heliogyro has no mass advantage over a kite sail, but it has the advantage of easier deployment in space. The reference design at 1 AU generates 100 newtons (maximum) from 4 banks of 48 blades each. Each blade has a dimension of 8 x 7500 m. The sail film is 1 um thick with reflective and emissive coatings. Each bank is fixed to a hub so the members co-rotate. The combined film masses 7 tonnes alone, and with the supporting cables mass 20 tonnes.
Scaled up from the JPL Halley Rendezvous design: Jerome Wright, Space Sailing, Gordon \& Breach Science Publishers, Amsterdam, pp. 82-88, (1992).
 
Photon kite sail – The simplest way to hold a sail out to catch sunlight is to use a rigid structure, much like a kite. The columns and beams of such a structure form a three-axis stabilization, so-named because all three dimensions are rigidly supported. Kite sails are easier to maneuver than sails that support themselves by spinning. By tilting the sail so that the light pressure slows the vessel down in its solar orbit will cause an inward spiral towards the sun.  Tilting the opposite way will cause an outward spiral.  The kite sail shown has a has a mast, 4 booms, and stays supporting a square sail 4 km to a side. At 93% reflectance, it develops a maximum thrust of 100 newtons at 1 AU. Control is provided by 4 steering vanes at the ends of the booms of 20,000 m2 area each. It has an unloaded mass of 16,000 kg and 1 g/m2 unloaded sail loading. A sail this light must be quite thin (300 nm aluminum film) and perforated with holes the size of the wavelength of visible light. (It is not necessary to have a full surface to reflect light). The perforated microstructure is formed by DNA scaffolding, which is then coated with aluminum and the DNA baked off. The kite sail is thermally limited to 600K, and cannot operate in an Earth orbit lower than 1000 km due to air drag. Its thrust can be increased tenfold by the illumination of the 60 MW laser beam which is standard in this game. Operating at 50 Hz, this beam boils off water coolant replenished through capillary action in the perforated film. Tiny piezoelectric robot sailmakers repair ablated portions of the sail using vapor-deposited aluminum.

Twice the size of Garvey’s “Large Square Rigged Clipper Sail”, and adding the perforation feature: J. M. Garvey, "Space station options for constructing advanced solar sails capable of multiple mars missions", AIAA Paper 87-1902, AIAA/SAE/ASME 23rd Joint Propulsion Conference, San Diego, California, June 29-July 2, 1987.
 
Ponderomotive VASIMR thruster – VASIMR stands for variable-specific-impulse magnetoplasma rocket.  This electric rocket has two unique features, the removal of the anode and cathode electrodes (which greatly increases its lifetime compared to other electric rockets) and the ability to throttle the engine, exchanging thrust for specific impulse. A VASIMR spacecraft would use low gear to climb out of planetary orbit, and high gear for interplanetary cruise. Other advantages include efficient resonance heating (80%), and relatively low current and high voltage power conditioning, which saves mass. Propellant (typically hydrogen, although many other volatiles can be used) is first ionized by helicon waves and then transferred to a second magnetic chamber where it is accelerated to ten million degrees K by an oscillating electric and magnetic fields, also known as the ponderomotive force. A hybrid two-stage magnetic nozzle converts the spiraling motion into axial thrust at 97% efficiency.  
 “The Physics and Engineering of the VASIMR Engine,” F.R. Chang-Diaz, et al., AIAA conference paper 2000-3756, 2000. Franklin Chang Diaz, MIT
 
Pulsed plasmoid thruster – A plasmoid is a coherent torus-shaped structure of plasma and magnetic fields. “Kugelblitz” (ball lightning) is a terrestrial example of a plasmoid (one of my mentors, Dr. Roger C. Jones of the University of Arizona, has demonstrated the physics of this).  An plasmoid rocket creates a torus of ball lightning by directing a mega-amp of current onto the propellant. Almost any sort of propellant will work.  The plasmoid is expanded down a diverging electrically conducting nozzle. Magnetic and thermal energies are converted to directed kinetic energy by the interaction of the plasmoid with the image currents it generates in the nozzle. Ionization losses are a small fraction of the total energy; the frozen flow efficiency is 90%. Unlike other electric rockets, a plasmoid thruster requires no electrodes (which are susceptible to erosion) and its power can be scaled up simply by increasing the pulse rate. The design illustrated has a 50 m diameter structure that does quadruple duty as a nozzle, laser focuser, high gain antenna, and radiator. Laser power (60 MW) is directed onto gap photovoltaics to charge the ultracapacitor bank used to generate the drive pulses. R. Bourque, General Atomics, 1990
 
Vortex confined thruster – The hotter the core of a thermodynamic rocket, the better its fuel economy. If it gets hot enough, the solid core vaporizes. A vapor core rocket mixes vaporous propellant and fuel together, and then separates the propellant out so it can be expelled for thrust. Energy is efficiently transferred from fuel to propellant by direct molecular collision, radiative heat, and direct reaction fragment deposition.  The open-cycle arrangement illustrated accomplishes this by spinning the plasma mixture in a vortex maintained by tangential injection of preheated propellant from the reactor walls. The denser material is held to the outside of the cylindrical reactor vessel by centrifugal force.  The fuel is subsequently cooled in a heat exchanger and recirculated for re-injection at the forward end of the reactor, while the propellant is exhausted at high velocity. For fission reactions, the outer annulus of the vortex is high-density liquid uranium fuel, and the low-density propellant is bubbled through to the center attaining temperatures of up to 18500 K.  A BeO moderator returns many reaction neutrons to the vortex.  Maintaining a critical fuel mass given the turbulent flow of water or hydrogen propellant requires advanced prompt feedback actuators. Since the core has attained meltdown, reaction rates must be maintained by fuel density variation rather than with control rods or drums.  For antimatter reactions, swirling liquid tungsten (about 4 cm thick) is used instead of uranium, for absorbing anti-protons.  For fusion reactions, it is the propellant that is cooler and higher in density, and thus it is the reacting fuel ball that resides at the center of the vortex.  An 1880 MWth vortex generates 560 MW of waste heat in the heat exchanger.  
 

Z3. ROBONAUTS
Ablative laser robonaut – A rocket can be driven by high-energy, short-duration (<10-10 sec) laser pulses, focused on a solid propellant. A double-pulse system is used, in which one laser pulse ablates material and a second laser pulse further heats the ablated gas. A low Z propellant such as graphite obtains the best fuel economies (4 ksec).  (Unfortunately, ice is not a suitable medium due to melting and “dribbling” losses.) Primary and secondary mirrors focus the laser pulses at irradiances of 3 × 1013 W/cm2. The mass-removal rate is 3.0 μg per laser pulse.  Powered with a 60 MW laser beam, an ablative laser thruster has a thrust of 2.4 kN and, with a fuel tuned to the firing sequences and an efficient double-pulsed shape, an overall efficiency of 80%.
 “Specific impulse and other characteristics of elementary propellants for ablative laser propulsion” Dr. Andrew V. Pakhomov, Associate Professor at the Department of Physics, UAH, pakhomov@email.uah.edu

Cat fusion z-pinch torch robonaut – A plasma torch driven by catalyzed fusion may be used in the refining of ores or the recycling of waste materials.  Fusion occurs in a zeta-pinch, a high-density fast-pulsed plasma focusing device, using a vanadium-gallium (V3Ga) superconductors and aluminum stabilizers.  The confining force of the zeta-pinch is a “self-generated” magnetic field (a field set up by electric currents in the plasma itself).  The megamp electric current (I) is in the zeta direction, and the resulting magnetic field (B) is in the theta direction.  A variety of hypothetical exotic particles catalyze the D-T and D-D fusions. Catalyst escape is minimized by the Z-pinch configuration, but is enhanced downstream by high temperature staged cascades of modulated sorting nanorotors, which are themselves continually destroyed and rebuilt.
 
D-D inertial fusion robonaut – A “target” of fusion fuel can be brought to ignition by “inertial confinement”: the process of compressing and heating the fuel with beamed energy arriving from all sides.  A snowflake of deuterium, the “heavy” isotope of hydrogen, can be imploded and fused with a combination of lasers and deuterium particle beams.  The illustrated design uses combined input beam energy of 38 megajoules, arrayed in a ring surrounding the ejected iceball target.  This energy operates at 1 Hz to blast a 2 gram ice pellet ejected each second. The outside 99% of the pellet is ablated away within 10 ns, super-compressing the deuterium fuel at the core to a density of a kilogram per cubic centimeter.  The T and 3He products are catalyzed to undergo further fusion until all that remains is hydrogen, helium and some neutrons.  (Neutrons comprise 36% of the reaction energy.)  Fractional burn-up of the fuel (30%) is twice that of magnetic confinement systems, which implies a 40% higher fuel economy. The energy gain factor (Q) is 53. For a 500 MWth reactor, 320 MW of charged particles are produced, which can be used directly for thrust or metals refining. About 105 MW of fast neutrons escape to space, but another 75 MW of them are intercepted by the structure.  About two thirds of this energy must be rejected as waste heat, but the remainder is thermally used to generate electricity or to breed tritium to be added to the fuel to facilitate the cat D-D pellet ignition. When used as a rocket, an ablative nozzle, made of nested layers of whisker graphite whose mass counts as propellant and shadow shield, is employed.   “A Laser Fusion Rocket for Interplanetary Propulsion,” Hyde, R., 34th International Astronautical Conf., AIF Paper 83-396, Budapest, Hungary, Oct. 1983.  (To keep radiator mass under control, I reduced the pellet repetition rate from 100 Hz to 1 Hz).
 
Electrophoretic sandworm robonaut – The mining vehicle depicted, called a sandworm, is designed for regolith devolatilization.  It consists of a Schaufelrad (shovel-wheel) and conveyor belts to transport material to a central hopper, which holds a soil pressurizer, grinding mill and heater, solid - vapor separator, volatiles collector bag, tailings disposal, and gas cleaner / reheater / repressurizer. A solar concentrator is pivoted to concentrate sunlight onto a heat engine target. The term “sandworm” is inspired by the huge worms that appear in the novels of Frank Herbert, which filter huge amounts of sand for tiny amounts of valuable “spice”. Here, the “spice” is helium 3, a substance not found on Earth, but present in tiny amounts in asteroidal and lunar regolith. Helium 3 is necessary for the clean 3He-D fusion reaction, and thus in the future may become more valuable than oil.  After the irons are magnetically removed, the remaining regolith is processed in a high-voltage zone electrophoretic tank. (Electrophoresis beneficiates charged particles in an electric field according to the magnitude and sign of the electrophoretic mobility.)  A 30 tonne sandworm of the size depicted operates with 350 kWe derived from beamed power (using a 12 MW beam). It would process 9 million tones of regolith a year, yielding 110 tonnes of water (propellant), 200 tonnes of hydrogen (propellant), and 33 kg of helium 3 (fuel). Based on Gajda NASA academy model Mark 2 and 3 sandworms.
 
Explosive-pumped gas dynamic laser robonaut –  The laser illustrated is similar to the MHD excimer laser, it uses explosives rather than solar heat for the expansion to pump the laser. This results in very high pulse peak power output and high efficiency (30%) applicable in space weapons. A typical wavelength is 10.5 µm (CO2, N adiabatic expansion).  After heating, the gas is expanded through a supersonic nozzle, which freezes the upper laser level in the gas, which exits through a duct.  After a certain period, the lower vibrational state relaxes, but the frozen higher states do not, achieving a population inversion. The explosive source can be microfission, fusor, antimatter, fusion focus, or fusion mirrors. Winterberg has proposed hexogen as an explosive source, surrounding an argon rod to produce an UV argon ion laser. To reduce depopulation of the upper laser level during the expansion by super radiance, the argon is doped with a saturable absorber, acting as an “antiknock” additive. In this way megajoule laser pulses can be released within 10 nanoseconds. F. Winterberg, Pure Nuclear Fusion Bomb Propulsion, 2008.
 
Flywheel mining tractor robonaut – A fleet of a half dozen wheeled mining haulers.  Each is capable of scooping and hauling 120 tonnes of iron fines or volatiles per year over rugged terrain, using a 6 MW (8500 horsepower) flywheel motor.  The flywheel is recharged via a microwave rectenna. Acting as a long-range rover, its dynamic active neutron spectrometer homes in on hydrogen signals indicating ice deposits or crystals. An auger digs through the regolith, which uses an impact grinder and screens to break up the agglutinates.  Starting with a 7 tonnes/hr throughput of regolith, a one Tesla magnetic separator can pick-out 11 kg/hr of free iron, titanium ilmenite grains, and magnetic oxides of iron, cobalt, and nickel. If volatiles are required, a large solar concentrator will be needed to roast the ice crystals out of the regolith. Dave Dietzler  www.moonminer.com
 
Free electron laser robonaut – An intense and rapidly alternating set of static magnetic fields, called a "wiggler", can efficiently convert the energy in a relativistic electron beam into coherent photons. This is the basis of the free electron laser (FEL). Shown is a 125 MWe FEL with an 80m electron accelerator. The acceleration may either be continuous, using resonant electrical cavities powered by high frequency electrical power, or it may be pulsed with a set of microwave "transformers" that use the electron beam as the effective secondary winding. The wavelength is tunable in the 300 nm repetitively pulsed range. The electrons are decelerated, and recirculated.  The FEL's conversion of electrical energy to light energy is remarkably efficient (40% overall).   
 
H-B cat. inertial robonaut – The fusion of hydrogen and boron 11 is a clean reaction, releasing only 300 keV alpha particles, which can be magnetically directed. However, the H-B fusion will not proceed at temperatures less than 300 keV unless catalyzed using exotic particles. One possibility: replace the electrons in H-B atoms with stable massive leptons such as magnetic monopoles or fractionally-charged particles (the existence of these is hypothetical). The resulting exotic atoms can fuse at “cold” temperatures, allowing the exotic catalysts to be recycled. A second possibility is to use antiproton-catalyzed microfission to initiate the H-B fusion. If a hundred billion antiprotons at 1.2 MeV in a 2 nsec pulse are shot at a target of three grams of HB: 235U in a 9:1 molar ratio, the uranium microfission initiates H-B and releases 20 GJ of energy.  Operating at a fifth of a hertz, hydrogen and boron 11 reacting at a rate of 145 mg/shot produces 2000 MWth. A shell of 200g of lead about the target thermalizes the plasma from 35 keV average to 1 keV, low enough that this radiation can be optimally transferred to thrust using a magnetic or ablative nozzle at 80% efficiency. The ejected mass per shot is 2.4 kg. The exotic catalysts are recycled.  Catalyzed fusion enjoys an excellent thermal efficiency (86%) and thus a good thrust/weight ratio (34 milli-g), making it one of the best engines in the game.  The specific impulse ranges between 8 and 16 ksec, depending whether spin-polarized free radicals are used as the hydrogen fuel.   “Antiproton-Catalyzed Microfission/Fusion Propulsion Systems for Exploration of the Outer Solar System and Beyond”, G. Gaidos, et al., Pennsylvania State University, 1998.  (I used the ICAN-II spacecraft design, modified from cat D-T to cat H-B fuel, and scaled way down from 1 Hz to 0.2 Hz, and 302 GW to 2 GW.)
 
He-Ar nuclear-pumped laser robonaut – A laser of helium and argon gases uses the transmutation of helium 3 in a neutron flux (i.e. the 3He (n,p) D reaction) as the energy source to create inverse populations of energy levels in the gases. These lasing populations surrender their energy as a laser beam by the special optical system illustrated. The neutron flux can be provided a fast breeder reactor, or cerium microfission bomb. The He-Ar lazing gases are contained in a battery of rods surrounding each shot. If the rods are sacrificed in a microfission bomb burst, the output beam wavelength is in the X-ray range, which has a low beam divergence and a high energy per photon.
 
Helical railgun robonaut – The traditional railgun is a single turn, contact, linear motor limited by the inductance of the rails.  In high acceleration, the necessary high current and brush contact heats the rails greatly limiting their life.  By twisting the rails in a helical fashion (with matching turns on the armature and projectile) higher accelerations can be achieved at a tiny fraction of the current. In its prospecting mode, the Helical Railgun fires a 1 ton metal projectile at a potential mineral deposit.  By placing it a highly elliptical suborbital trajectory, the projectile collides at 10 -70 kps from a fraction of the launch velocity. The resulting plume is analyzed spectrographically to determine the contents.  The small crater can then be used as the start of any open pit mining operation. In its thruster mode, its capacitor banks charge for days before firing a single, high impulse, dirt projectile along with the armature, which is not recovered. Engel T., Nunnally W., Neri J. Research Progress in the Development of a High-Efficiency, Medium-Caliber Helical Coil Electromagnetic Launcher. 2004. Inspired by the LCROSS mission.
 
Kuck mosquito robonaut - As icy dormant comets or D-type asteroids are warmed by the sun, they accumulate an outer anhydrous lag layer.  An in-situ mining robonaut called the Kuck mosquito is designed to drill through this layer, inject steam, and pump out the water in the core. Some of the water is electrolyzed for fuel for a small H2-O2 chemical engine. Thermal lances are used to melt into the substrate and gain a secure foothold.  The targeted bodies must have ice in a cometary matrix of not less than 30%. There is a danger of catastrophic fracture, if the subsurface mantle layer is too weak to resist the tensile forces generated by the pressurization. Dave Kuck, “The Exploitation of Space Oases”, Princeton Conference on Space Manufacturing, Space Studies Institute, 1995.
 
Lorenz propelled microprobe robonaut –A swarm of tiny probes can be sent to distant sites by exploiting the force that results from a charged particle moving perpendicular to a magnetic field line (in this case the Earth’s).  Each probe would consist of a long electrically charged (+10V) microfilament (10μm radius), essentially acting as part of a Mag Sail, which slowly accelerates it for years towards the target site. Two simple low-powered electrostatic micro-thrusters provide small course corrections.  The anode is a faraday cage charged to -190V to protect the probe’s electronics. The cage is shed upon arrival. Integrated into the probe’s chip are a solar panel, sensor, and microprocessor. Each sensor in the swarm would fulfill a different purpose (e.g. detect hydration, temperature, chemical tests) and transmit its results.  The probes are sturdy enough to survive a direct impact on the target body, allowing surface sampling to take place.  Even if up to 90% of probes are lost or malfunction, enough will be available to provide a detailed analysis of the site. Atchison J., Peck M. A millimeter-Scale Lorentz-Propelled Spacecraft. 2007. And http://www.popsci.com/military-aviation-space/article/2007-08/mmmm-space-chips. 

%MagBeam robonaut – A current can be propagated through space in a self-focusing plasma beam called a Birkeland “MagBeam” current. Differential motion between the ions and electrons in the MagBeam produce currents and magnetic fields such that the magnetic field in which the plasma is born stays with the plasma, making its own transmission line.  The magnetic field continuously expands, keeping the ion stream focused. If produced by a helicon plasma source on a power station, it can be intercepted for mineral processing or thrust by a spacecraft equipped with a small amount of gas for propellant such as argon or xenon, a power source and a set of electromagnets to produce a mini-magnetosphere magnetic sail. The intercepted beam ionizes the Argon in the sail, which is accelerated for thrust. Parameters to achieve a delta-v of 20 km/sec: beam of density of 2 X10^13 /cm3, a speed of 30 km/sec, robonaut mass of 10 tonnes, propellant mass of 7 tonnes, a 300 mW thruster with an Isp of 4 ks, and an interaction time of 4 hours. The disadvantages of the system as a thruster is that the power station must have kilotonnes of capacitors to store the required energy, the beam is of limited range permitting only hours of acceleration time, and there must be another such station at the destination to decelerate the rocket. G. A. Landis, "Interstellar Flight by Particle Beam," 2004.

MET steamer robonaut - This device works by generating microwaves in a cylindrical resonant, propellant-filled cavity, thereby inducing a plasma discharge through electromagnetic coupling. The discharge performs either mining or thrusting functions. In its mining capacity, the head brings to bear focused energy, tuned at close quarters by the local microwave guides, to a variety of frequencies designed to resonate and shatter particular minerals or ice. In its electrothermal thruster (MET) capacity, the microwave-sustained plasma superheats water, which is then thermodynamically expanded through a magnetic nozzle to create thrust. The MET needs no electrodes to produce the microwaves, which allows the use of water propellant (the oxygen atoms in a steam discharge would quickly dissolve electrodes).  MET steamers can reach 900 seconds of specific impulse due to the high (8000 K) discharge source temperatures, augmented by rapid hydrogen-oxygen recombination in the nozzle. Vortex stabilization produces a well-defined axisymmetric flow. However, the specific impulse is ultimately limited by the maximum temperature (~ 2000 K) that can be sustained by the thruster walls.  The illustration shows a microwave plasma discharge created by tuning the TM(011) mode for impedance-matched operation.  This concentrates the most intense electric fields along the cavity axis, placing 95% of the energy into the propellant, with less than 5% lost into the discharge tube walls.  Regenerative water cooling is used throughout. For pressures of 45 atm, each unit can produce 30 newtons of thrust.  The thrust array contains 400 such units, at 50 kg each.  “Development of a High Power Microwave Thruster, with a Magnetic Nozzle, for Space Applications.”  John L. Power and Randall A. Chapman, Lewis Research Center, 1989.
 
MITEE arcjet robonaut – A working fluid such as hydrogen can be heated to 12,000 K by an electric arc. Since the temperatures imparted are not limited by the melting point of tungsten, as they are in a solid core electrothermal engine such as a resistojet, the arcjet can burn four times as hot. However, the thoriated tungsten electrodes must be periodically replaced.  When used as an electrothermal thruster, the arcjet operates at low pressures allowing the propellant to disassociate to monoatomic hydrogen, increasing specific impulse to 2 ks with frozen-flow efficiencies of 52%. The thrust is boosted by hybridization with a MITEE reactor, an assembly of 37 individual beryllium pressure tubes each containing an inner 242Am core surrounded by an outer 7Li hydride moderator.  Cold hydrogen propellant flows radially inwards through a 1cm thick fuel region where it is preheated to over 3200K before entering the arcjet. When used for mining beneficiation, regolith or ore is initially processed with a 1T magnetic separator and impact grinder (3.5 tonnes), before being vaporized or arc welded by the MITEE arcjet. Powell J., Paniagua J. Phase I Final Report – Lightweight, High Specific Impulse (1000 Sec) Space Propulsion System. 1999.
Powell G, Paniagua J. The MITEE Family of Compact Ultra Lightweight Nuclear Thermal Propulsion Engines for Planetary Space Exploration. 1999.
Artwork: Powell G. Compact Nuclear Rockets. Scientific American 1999.
 
Nanobot robonaut – An army of nano regolith and ore scavenging machines can be used for ore beneficiation. Molecular assemblers will use nano-structures much like enzymes to work with reactive molecules.  Mechanical nanobots based on biomimetic soft nanotechnology might actually incorporate biological protein molecules (enzymes and ribosomes) to do some of their work.  The nanobot illustrated uses legs driven by a harmonic drive. Harmonic drives use a pair of dislocations driven by a wave generator moving along the inner interface of a nanotube to deform a flexspine.  Each rotation of the wave generator turns the flexspine by a two-tooth increment relative to the surrounding drill spline.  A sorting rotor selects for a desired atomic shape, such as helium 3.  These features are driven by a nano-electrostatic motor, that operates on the principle of a Van der Graaff generator worked backwards. Nanosystems, K. Eric Drexler, 1992.
 
Neutral beam robonaut – Beams of ions are easy to accelerate, but they must be neutralized in space, otherwise their mutual electric repulsion would quickly diffuse the beam. (Also arcing would destroy the robonaut.) The ions accelerated are negatively charged, which obtains higher neutralization efficiency than positive ones. Neutral beams of hydrogen or deuterium, operating in a pulsed mode up to 800 MeV, pack more punch than lasers, which makes them useful in mining or combat. They can also be injected into fusion reactors both to heat the plasma and replenish the burned fuel.  If a plasma vortex is needed, the neutral beam is introduced into the chamber off-axis.  The design illustrated uses a Dudnikov high-brightness H- source with a radiofrequency quadrupole (RFQ) injector.  A stage of resonance-coupled Alvarez drift-tube linacs (DTLs) boosts the beam to 85 MeV.  A stage of superconducting linacs increases the energy of the ions to 600 MeV without unduly increasing the emittance of the beam (10-8 m·rad).  The beam is very precisely focused on a target by a magnetic beam-steering optics, and then passed through a laser photodetachment neutralizing cell in order to remove the extra electron.
 
Nuclear drill robonaut – Shaft mining and boreholes are cornerstones to any effective ISRU operation.  However, the power and mass requirements increase rapidly as the mine extends to greater depths and to harder rocks.  Given a melting point of basalt on the order of 1500 K, 4.3kJ/cm3 of energy can be used to melt the rock (compared to 2-3 kJ/cm3 required for a rotary drilling).  Using a heat-pipe, thermal energy from a nuclear reactor to the drill head, keeping it above the melting temperature of the rock.  The heat pipe cavity is lined with a capillary structure which transports the working liquid continuously from the cool end to the heat source end.  A closed evaporative/condensation cycle allows considerable heat transfer even though the difference in temperature between the hot and cold end is minimal.   Venting the hot liquid can allow the drill to operate as a thruster. Under the thermal stress the rock face walls rock would crack, but by applying mechanical pressure through of the tunneling device, the melted rock at the drill head is then forced into the cracks, reinforcing the tunnel walls (lateral extrusion).  The pressure is provided by 6  hydraulic pusher pipe capable of exerting 80,000 kN of force. Armstrong D., McInteer B., et al.  Method and Apparatus for Tunneling by Melting. United States Patent. 1972.
  
Phase-locked diode laser robonaut – Using a solid state chip, results in ⅓ the heat per watt of output power by matching the heat emission spectrum with the absorption lines of the diode.  Diode lasers however are typically limited in power to the 100 mW range or must accept a wide beam divergence (>10°) that comes with a large aperture.  With internal fibres and reflective zigzag slabs heat can be more evenly distributed across the chip allowing higher powers (10s of kW) per beamlet.  To reach the megawatt range multiple beamlets must be combined.  Active phase-locked coherent beam combination uses the constructive interference of numerous polarized beamlet elements.  With this careful synchronisation of each element seed as it is pumped to high power, beam quality stays constant even as elements scale into the hundreds. maintained using hundreds of elements.  These lasers are electronically steerable requiring no mechanical components. Goodno G., Asman C., et al. Brightness-Scaling Potential of Actively Phase-Locked Solid-State Laser Arrays. 2007.
 
Quantum cascade laser robonaut – Mineral grains blasted from the substrate can be photoelectrically charged with solid-state IR lasers such as quantum cascade lasers. Once photoelectrically separated, it is possible to process them electrically, thermally and with chemicals to extract metals and oxygen. The quantum cascade laser shown uses carbon buckytubes as the gain medium in a Fabry-Perot resonator. The layers of buckytubes form a series of quantum wells, down which electrons cascade. Each quantum-cascade laser electron tumbles down thousands of wells, producing a photon at each step. This electronic waterfall significantly boosts wall plug efficiency (30%), enabling quantum cascade lasers to emit gigawatts of peak power in pulsed operation and tens of megawatts in continuous wave. The wavelength is altered by modifying the buckytube temperature. For long range work, a set of laser expanders are used for precise beam pointing and to decrease the spot size at great distances.

Rock splitter robonaut –The Bureau of Mines has developed a crawler-mounted excavation machine combining a percussive drill, drill feed, radial-axial rock splitter, and loader on a single boom.  The system weighs 6 tonnes with a power requirement of 50 kW and a production rate of 10 tonnes/hour.
 
Solar-pumped MHD exciplex laser robonaut – An UV laser can be pumped by short-lived pseudo-molecules called excimers, which are stable only in the excited state. The excimers are formed in a cesium-xenon solar-sustained plasma, heated to 3000 K by sunlight amplified a thousand times, that acquires supersonic speeds at 20 atm in a nozzle.  The nozzle is part of a short-circuited Faraday MHD duct, which decouples the electron and gas temperatures in its 1 T magnetic field. The system illustrated is a pulsed, solar-pumped, high-pressure CsXe excimer laser with a lasing volume of 40 liters. Each laser pulse is 15 KJ of 0.8944 µm UV light.  A frequency of 4 kHz and a solar collector 540 meters in diameter is needed to produce a 60 MW cw beam. The efficiency is 19%, requiring tons per second of gas flow to remove heat from the lasant. This system does not need the economies of scale that a photovoltaic/microwave solar power satellite would entail.  The receiving antenna can be small.  It could beam its modest power to spacecraft, or to remote areas on Earth that need energy. http://ntrs.nasa.gov/archive/nasa/casi.ntrs.nasa.gov/19880013813_1988013813.pdf
 
Tungsten resistojet robonaut – Tungsten, the metal with the highest melting point (3694 K), may be used to electric-resistance heat ore for smelting or propellant for thrusting.  In the latter mode, the resistojet is an electro-thermal rocket that has a specific impulse of 1 ksec using hydrogen heated to 3500K. The frozen flow efficiency (without hydrogen recombination) is 85%.  Internal pressures are 0.1 MPa (1 atm). To reduce ohmic losses, the heat exchanger uses a high voltage (10 kV) low current (12.5 kiloamp) design.  The specific power of the thruster is 260 kg/MWj and the thrust to weight ratio is 8 milli-g.  Once arrived at a mining site, the tungsten elements, together with wall of ceramic lego-blocks (produced in-situ from regolith by magma electrolysis) are used to build an electric furnace. Tungsten resistance-heated furnaces are essential in steel-making. They are used to sand cast slabs of iron from fines (magnetically separated from regolith), refine iron into steel (using carbon imported from Type C asteroids), and remove silicon and sulfur impurities (using CaAl2O4 flux roasted from lunar highland regolith).
Wakefield e-beam robonaut – An electron beam has many uses.  It can bore holes in solid rock (mining), impart velocity to reaction mass for thrust (rocketry), remove material in a computer numerical control cutter (finished part fabrication), or act as a laser initiator (free electron laser). A wakefield accelerator accelerates the electrons by using a brief (femtosecond) laser pulse to strip them from gas atoms and to shove them ahead.  Other electrons entering the electron-depleted zone create a repulsive electrostatic force.  The initial tight grouping of electrons effectively surf on the electrostatic wave.  Wakefield accelerators a few meters long exhibit the same acceleration as a conventional rf accelerator many kilometers in length. In a million-volt-plus electron beam the electrons are pushing lightspeed, so the term relativistic electron beam is used. When used as an electrothermal rocket, it is similar in principle to the arcjet, but far less discriminating in its choice of propellant. Dr Bussard (of interstellar ramjet fame) calls relativistic e-beam-heated systems "QED" (Quiet Electric Discharge) engines.  

Z4. REFINERIES
Atmospheric ISRU scoop refinery – A vehicle with an air intake coupled to Rankine-cycle liquefaction gear, is capable of scooping the atmosphere of planets such as Earth, Venus, Mars, Saturn, and Titan for use as propellant. A typical scooping orbit is a quasi-elliptical with a perigee well within the atmosphere. The system shown requires 54 MWe to overcome atmospheric drag with an electric rocket and 6 MWe to compress and store 720 kg/hr of liquid nitrogen or other volatiles.  The thermal efficiency is 22% at a turbine inlet temperature of 1400 K.  Reichel, Smith, and Hanford.  Electric Propulsion Development
 
Atomic layer deposition refinery –Two or more gas phase chemicals (precursors) react with a surface in a one-at a time sequential manner (essentially CVD at an atomic scale by breaking it into two steps).  Each cycle can deposit a layer as thin as 10pm (10-12 m). A pulse of an inert purging gas after each precursor pulse removes any excess from the chamber preventing the CVD deposition.   The final film thickness is dependent only on the number of cycles.  Because layers are deposited atom by atom, the process is slow for manufacturing anything substantial in size.  ALD is used primarily to manufacture the microchips required for Robonauts and ET Factories. Thin film solar cells can also be produced using local amorphous silicon.  While these solar cells are of lower efficiency than their thicker cousins, they have a higher power density.  They also are more sensitive to radiation damage, rapidly degrading in the presence of ionizing radiation. The circular solar cell sail produced by this refinery takes two years to manufacture and masses 80 tonnes (including structural supports). Specifications include a 290m radius and 10.25μm thickness (too thick to provide thrust).  At 1AU it provides 34 MWe with 9.5% efficiency (output increases to 70 MWe at Venus’ 0.7AU orbit).
Levitin G., Hess D. Surface Reactions in Microelectronics Process Technology. 2011. And Wyrsh N., Domie D., et al. Ultra-light amorphous cilicon cell for space applications. 2006.
 
 
Biophytolytic algae farm refinery  Biophytolysis is the use of microorganisms to break down and refine low grade ores and volatiles. Thermophyllic bacteria are used to extract nickel, zinc, and cobalt.  Sulfur-metabolizing bacteria obtain gold, copper, and uranium.   In either case, weak solutions of acids are dripped through the ore and a bacterial liquor forms that is then electrolytically or chemically processed. Cyanobacteria and green algae, bioengineered for radiation and O2 tolerance,  will oxidize water at room temperatures, producing hydrogen and oxygen.  This process is carried out by photosynthetic enzymes, which split water to obtain electrons, excite these electrons with photons, and eventually use these electrons to reduce 2H+ to H2. Molecular complexes involved in mediating electron flow from water to carbon-fixing or hydrogen-production reactions make up the photosynthetic electron-transport chain found in the thylakoid membranes of cyanobacteria and green algae. Chaff from the algae farm is used as an organic substrate.  Biophytolytic processes do not require much energy, and have a theoretical efficiency of 40%, but they are slow.
References: Prospecting for Lunar \& Martian Resources. G. Jeffrey Taylor and Linda Martel (Hawaii Institute of Geophysics and Planetology, University of Hawaii, 2001. www.mines.edu/research/srr/2001abstracts/Taylor.PDF
 
Carbonyl volatilization refinery – Gaseous carbon monoxide (CO) is the catalyst for carbonyl volatilization, by far the easiest way to refine metals in space. Solar-heated ores of nickel, iron, cobalt, and other metals react with CO to form gaseous carbonyls, which are then vapor-deposited via the CVD molding or CVI foam stereolith processes to form finished materials. The foamed metal composites exhibit tremendous directional strength and stiffness, and can produce shape-welded vessels many tons in mass.  The CO can be recovered by another heating cycle or replenished by heating (to 1300 K) almost any known type of asteroidal material.  The temperatures and pressures involved are particularly mild for nickel refinement via the Mond process: Ni(CO)4 carbonyls form at just 530 K. The residue from carbonyl extraction of native ferrous metal alloys in M type asteroids is very rich in cobalt and platinum group metals, which are far more valuable than gold. The cobalt may in turn be separated from the platinum-group metals by very high-pressure extraction with CO, by extraction with CO-H2O mixtures as the carbonyl hydride, or by wet chemical techniques.   “Asteroidal Resource Opportunities suggested by Meteorite Data”,  John S. Lewis and Melinda L. Hutson, University of Arizona.
 
Carbochlorination refinery – Metal sulfates may be refined by exposing a mixture of the crushed ore and carbon dust to streams of chlorine gas. Under moderate resistojet heating (1123 K) in titanium chambers (Ti resists attack by Cl), the material is converted to chloride salts such as found in seawater, which can be extracted by electrolysis. The example shown is the carbochlorination of Al2Cl3 to form aluminum.  Al is valuable in space for making wires and cables (copper is rare in space).  The electrolysis of Al2Cl3 does not consume the electrodes nor does it require cyrolite. However, due to the low boiling point of Al2Cl3, the reaction must proceed under pressure and low temperatures.  Other elements produced by carbochlorination include titanium, potassium, manganese, chromium, sodium, magnesium, silicon and also (with the use of plastic filters) the nuclear fuels uranium 235 and thorium 232. Both C and Cl must be carefully recycled (the recycling equipment dominates the system mass) and replenished by regolith scavenging.  Dave Dietzler
 
CVD molding refinery – Metal-walled structures and complex metal parts are readily produced through chemical vapor deposition (CVD). The beauty of CVD molding is that no machining is required for the final product. In particular, nickel is the easiest ferrous metal obtained from carbonyl volatilization, and is strong, ductile, corrosion-resistant, versatile, and common in space The CVD process distributes nickel carbonyl vapors over a mildly solar-heated surface, where they decompose and deposit structurally-sound nickel on a mandrel. Deposition occurs with mild temperatures (450 K) and high purity (less than .02% carbon impurities). A vapor stream of 90% Ni(CO)4 carbonyl and 10% CO creates forms and shapes with phenomenal leveling and corner-filling.  If the fines are mixed nickel and iron, as from a type M asteroid, the carbonyls formed will be mixed, and a Fe-Ni alloy will be deposited. Synthesis of iron carbonyls is not as simple as for nickel, requiring higher pressures and a carbon dioxide-water vapor mix. The unreacted gases must be recycled, so as not to waste the iron carbonyl. However, due to the near instant disassociation of escaped carbonyl in a vacuum, some replacement of CO is needed.
http://www.space-mining.com/beneficiation.html By William C. Jenkin
 
Electroforming refinery – Metal thin-walled structures can be manufactured in a bath of electrolyte-plating solution by depositing metal onto a mandrel having the inverse contour. A robonaut monitors the electrical current densities as a function of metal deposition rate. Extraterrestrial metals most commonly deposited by this electroforming technique include nickel and iron.  Mandrels are prepared from aluminum-coated cast or spun basalt. The need for an electrolyte-plating solution requires the electroforming unit to be pressurized and operated only in an accelerated frame. The anode plate is consumed during the forming process, but lunar or asteroidal iron and titanium are widely available for this purpose. The electrolyte is recycled. Proceedings of the 1980 NASA/ASEE Summer Study, Editor Robert Freitas, Jr., of Space Initiative/XRI, Santa Clara, California
 
Femtochemistry refinery – During a chemical reaction, bonds within a molecule pass through an unstable transition state lasting on the order of a picosecond (10-12).  By utilizing femtosecond (10-15) laser pulses it is possible to monitor molecules as they undergo the transition state and even influence the pathway they take.  Using femtosecond pulses with sub-angstrom (0.1nm) lasers, individual bonds can be targeted. The result is that otherwise unfavorable or even impossible reactions can become possible and high yield.  Types of catalysts for certain reactions become more variable and easier to produce.  Furthermore, catalysts are more likely to be recovered and exotic catalysts (which are ordinarily consumed) can often be re-used for hundreds of reactions.  Antimatter transmutation, once restricted to alchemy, uses antimatter and other exotic catalyst to convert element to another.  The easiest reaction occurs when a heavy metal is bombarded with anti-protons, yielding a small amount of radioactive higher order elements.  Through careful control of this reaction, a small chain reaction can be initiated resulting in substantial yields.  Light isotope production using antiprotons are currently being investigated.
Zewail Ahmed. Femtochemistry. Past, present and future. 2000
 
Fluidized bed refinery – In a fluidized bed refiner, a fluid (gas or liquid) is passed through a granular solid ore (such as regolith) at high enough velocities to suspend the particles and cause them to behave as though they were also a fluid. The heat and agitation increases the reaction rate. Ilmenite fines (grains of FeTiO3 electrostatically separable from lunar regolith) may be reduced in a fluidized bed, using either hot hydrogen (1270 K), or carbon monoxide, to obtain iron fines and titanium dioxide particles. In either case, titanium is obtained from the reactants by acid leaching, carbonyl CVD, or vacuum high temperature heating, followed by electrolysis to Ti metal in FFC Cambridge cells. A variety of machine tools are used to finish the Ti part, including a grinder, e-beam welder, drill press, small lathe, and small rolling mill. 
 
Foamglass sintering refinery – Natural glass formed in the heat of asteroid impacts can be easily separated with electrostatic beneficiation.  Formed in the absence of water vapor, this anhydrous glass is less brittle and has compressive tensile strength.  Foamglass (or cellular glass) is a lightweight yet strong extremely finely pored glass that can be used in place of metal and ceramics for structural components (such as long booms to separate crew from a reactor).  Complex structural components and prototypes can be sintered.  Sintering is a process whereby fine powder is heated below its melting point causing them to adhere to each other.  A point heat source builds the product layer by layer.  Given high purity starting materials, the resulting product is of extremely high and uniform quality with absolute control of the porosity.  Nearly any ceramic, plastic or metal can be sintered.
 
Froth flotation refinery – This highly versatile method uses air bubbles to selectively adhere to specific mineral surfaces within a mineral/water slurry.  The particles with air-bubbles then float to the surface and can be skimmed off.  Sulfites, silicates and metallic ores (including zinc, iron, nickel and tungsten) can all be separated and concentrated and recovered at over 95% efficiency per run.  The hydrophobicity which selectively attracts the air-bubbles can be natural or induced through chemical treatments with non-polar oils.  Through variations in pH, surfactants, wetting agents, activators, depressants and various other reagents, one can selectively float a wide variety of compounds and minerals.  The limiting factor in a space based froth flotation operation is the vast amounts of water required, even with waste water recycling.  While non-water based solutions are possible (e.g. ammonia, hydrocarbons), they are not as well studied.
 
Ilmenite semiconductor film refinery – The common lunar mineral ilmenite (FeTiO3), could produce 10% efficient PV cells (compare to 25% efficient cells from silicon). Although actual efficiencies are much lower, the Shimizu Corp is studying how to ISRU-produce a girdle of photovoltaic cells completely around the lunar equator.  The day side of Luna produces TW of power, which is beamed to Earth from the near side. Energy Demand and Climate Change: Issues and Resolutions, By Franklin Hadley Cocks
 
Impact mold sinter refinery – Sintering is a method for making objects by heating powdered material below its melting point until the particles adhere to each other. Metal grains, produced perhaps by CVI or CVD space vacuum processes which suffer no surface contamination, are sequentially blown into an impact die made of sintered regolith ceramics. As these grains accumulate on the developing workpiece, they are sintered by the energy of impact and coalesce by cooling. Insertable shields are used to create voids and internal patterns. As it is made, the part is actively inspected by scanning electron microscopes or optical sensors which guide the beam to areas where the surface is rough, appears too porous, or has not adequately been filled. The parts then move to an inspection station for trimming by a high-energy laser, and assembly using an e-beam welder. Such an assembly line can produce at 180 kg/hr, with a specific power of 0.5 MW/(ton/hr of product), and 5 ton machinery/(ton/hr of product).  The parts will typically be made of a nickel/carbon fiber system sintered with minor amounts of aluminum. The reaction creating nickel aluminide is highly exothermic, so much of the heat of reaction supplies the sintering energy required. A part made of nickel aluminide has a stiffness and a strength to weight ratio far superior to titanium, for temperatures well over 770 K.  Iron-nickel fines can also be impact mold sintered. George Hansen, Metal Matrix Composites Company
 
Ionosphere lasing refinery – Sunlight maintains population inversions in the ionospheres of Venus, Mars, and (possibly) Saturn. If two orbiting mirrors are arranged so the path between them intersects this portion of the atmosphere, the atmosphere itself can lase.  A two satellite system sized as shown, 1000 meters apart, orbiting at 8 km/sec and interacting with a grams worth of excited molecules, can achieve about 5 kJ per pulse at 8 kHz (average power 40 MW).  This is a significant power level, for propulsion, refining, and weapons purposes. D. Deming and M. Mumma, "Modeling of the 10 micrometer natural laser emission from the mesospheres of Mars and Venus", NASA TM-85045, NASA/Goddard, Greenbelt, MD 20771 (June 1983).
 
ISRU carbothermal refinery – An ISRU (In-Situ Resource Utilization) alternative to carbochlorination for the refining of aluminum and oxygen is solar carbothermal, the solar heating of alumina ores in the presence of carbon. This process needs no chlorine, which is rare in space, and also uses solar heat instead of electricity, which uses less energy than electrolysis since the energy comes directly from solar radiation without the inefficiencies of conversion to electricity and power conditioning.  The aluminum and oxygen produced can be used to fuel Al-O2 chemical boosters and suborbital hoppers, which burn fine sintered aluminum dust in the presence of liquid oxygen (LO2). Unlike pure solid rockets, hybrid rockets (using a solid fuel and liquid oxidizer) can be throttled and restarted. The combustion of aluminum obtains 3.6 million joules per kilogram.  At 77% propulsion efficiency, the thrust is 290 kN with a specific impulse of 270 seconds. The mass ratio for boosting off or onto Luna using an Al-O2 rocket is 2.3.  In other words, over twice as much as much fuel as payload is needed. 
 
ISRU Sabatier refinery - The Sabatier reactor is a small lightweight steel cylinder that has a mixing chamber and a chamber filled with a nickel catalyst. When charged with hydrogen and atmospheric carbon dioxide, it produces water and methane.  (The similar Bosch reactor using iron catalyst produces elemental carbon and water.) A condenser removes the water vapor from the products of the Sabatier reaction.  The condenser is a simple pipe with outlets on the bottom to collect water; natural convection on the surface of the pipe is enough to carry out the necessary heat exchange. Electrolysis of the water recovers the hydrogen for reuse. These ISRU (In-Situ Resource Utilization) reactors create closed hydrogen and oxygen cycles for life support on bases with access to CO2 atmospheres such as Mars bases or Venus aeroxities. The methane and oxygen produced can be used to power chemical rockets for the ascent stages. The CH4/O2 fuel, with a specific impulse of 380 sec, is the highest-performing space-storable chemical propellant that can be easily manufactured in space. In the absence of an atmosphere, ISRU units can produce H2O, H2, and CO by steam-heating crushed carbonaceous asteroidal material to 1300 K in a closed vessel. In either case, the elemental carbon produced is used to make nanotubes through electrophoretic techniques (the migration in solution of charged colloids).  Nanotubes have higher electrophoretic mobilities than do particles, and move toward the negative electrode aligned along the electric field.   Carbonaceous asteroids, C.R. Nichols, Bose Corporation
Laser-heated pedestal growth refinery – This crystal growth technique uses a 10-100kW CO2 or YAG diode laser as a heat source with the nutrient suspended in a floating zone.  The liquid phase undergoes high speed convection resulting in a high yield crystal fiber.  These very pure crystals can exhibit unique properties such as very high melting points when doped with rare-earth metals. Of particular interest is the growth of organic-inorganic gold-based composite materials for batch production of metamaterials. A metameterial is an engineered material which exhibits properties impossible in nature.  By employing periodic features smaller than the wavelength of light passing through them the electromagnetic response can be controlled resulting in a zero or negative refractive index.  Metamaterial “super-lenses” can focus beyond the diffraction limit of light for sub-wavelength antennas, microwave cloaking, and inductive power transmission over longer distances (decaying on the 3rd instead of 6th power of distance used in induction coils).
 
Magma electrolysis refinery – This refining unit melts regolith with solar energy, and passes electricity through the melt.  This liberates oxygen at one electrode, and reduces the material to a lower oxidation state at the other. A flux material is used to reduce the melting temperature of the regolith to around 1600 K. A 80 tonne magma electrolysis unit with a throughput of 5000 tonnes of regolith per year and a 3.5 MWe power source could produce 2000 tonnes of ceramic and silicon blocks or heatshields, 1000 tonnes of oxygen, and hundreds of tonnes of iron, magnesium, and silicon.  The silicon can be zone-refined to high purity for solar panels.  Zone-refining does not require chemicals that must be upported from Earth and will be done more easily in the low gravity and vacuum of space than on Earth (where it must be done in inert gas-filled chambers and rods can't be too massive lest they fall apart at the molten zone). David Dietzler www.moonminer.com
 
Von Neumann - Santa Claus refinery – The great mathematician John Von Neumann (pronounced von noi-man) visualized a set of nanomechanical assemblers, operating on local energy sources and material inputs, which use sensors integrated by a central processing unit to perform a wide range of mechanosynthetic operations including mining.  Assemblers are molecular machines capable of being programmed to build stuff from raw materials such as those readily gleaned from carbonaceous regolith. Electrophoresis and magnetoplasmadynamic systems separate this regolith into its basic elements. The basic building blocks of Von Neumann machines are rods of interlocked diamondoid fibers. (Diamondoid structures comprise a wide range of polycyclic organic molecules consisting of fused, conformationally rigid cages.) The system includes internal radio communications, fuzzy logic design data storage, and a robot that can disassemble faulty engines and drills, test them, and replace them with new diamondoid components made on site. This device is called a Santa Claus machine, because of its capacity to make anything you want, including copies of itself. However, in practice self-reproduction (of disassembled parts) is vastly easier than self-assembly of parts.  The machine illustrated (like the 2009 RepRap open-source project) can self-reproduce, but relies on humans to assemble itself. Von Neumann, John: Theory of Self-Reproducing Automata, A. W. Burks, ed., Univ. of Illinois Press, Urbana, Illinois, 1966
 
Z5. GENERATORS
AMTEC thermoelectric generator – The alkali metal thermoelectric converter (AMTEC) is a thermally regenerative electrochemical device for the direct conversion of heat to electrical energy using high-voltage multitube modules. These modules accept a heat input (solar or nuclear) at 900–1300 K and reject it at 400-700 K, producing direct current with an efficiency of 45% and no moving parts. The molten alkali metal (sodium or potassium) is driven around a closed thermodynamic cycle between the heat source and the heat sink, in a similar fashion to the Rankine MHD heat engine. However, instead of the MHD unit, the AMTEC cycle expands the alkali metal vapor through a solid electrolyte (sodium beta-alumina) which causes it to ionize. The isothermal expansion of the alkali vapor is thus converted directly into electricity, with power densities of 100 kW/m2 and 2 ton/MWe. “Direct Thermal-to-Electric Energy Conversion for Outer Planet Spacecraft”, M. A. Ryan and J.-P. Fleurial, Jet Propulsion Laboratory, 2002
 
Brayton turbo generator – A solar or nuclear heat source can be used to generate electricity via a closed-loop Brayton cycle. Unlike the Rankine cycle, in which the working fluid changes phase, the Brayton cycle uses an inert gas such as helium which is expanded through a power-producing turbine, after which it is circulated through a radiator for cooling and reuse.  Typically, heated helium enters the turbine at 1700 K and 24.5 atm. A mass breakdown: 3600 rpm turbine with a diameter or 2.6 meters: 4.5 tonnes, alternator-generator: 12 tonnes, recuperator: 12 tonnes, compressor: 10 tonnes, transformers: 20 tonnes. The liquid fluids used in the Brayton cycle have heat-transfer coefficients approximately 50 times lower than the gaseous fluids used in the Rankine cycle. As a result, the closed Brayton cycle has an inherently lower thermal efficiency (19% vs. 22%).
 
Buckyball C60 photovoltaic generator – The C60 “Buckyball” molecule is a fullerene carbon allotrope, about a nanometer across, composed of 60 carbon atoms arranged in a hollow sphere. It has semiconducting and magnetic properties, up to its Curie temperature around 500 K.  In its amorphous form, Buckyball C60 is a semiconductor with a bandgap of 2.5 eV. By intercalating dopants between the Buckyballs, the conductivity can be increased. The organic photovoltaic device illustrated uses charge-generating layers of copper phthalocyanine (CuPc)/fullerene (C60) over a light-absorbing rubrene antenna. Radiation absorbed by the antenna is transferred into the charge generating layers via surface plasmon polaritons. The antenna tunes the cavity to absorb light strongly, improving the quantum efficiency to 85%. Membrane photovoltaic C60 films, centrifugally-tensioned and supported by wires, have a specific area of 1.5 kg/m2. T.D. Heidel, J.K. Mapel, K. Celebi, M. Singh, M.A. Baldo, 'Analysis of surface plasmon polariton mediated energy transfer in organic photovoltaic devices', Proc. of SPIE, 6656, 66560I1-8, (2007).
 
Cascade photovoltaic generator – A photovoltaic device is one that generates a voltage when radiant energy falls on the boundary between dissimilar substances.  A high efficiency example is the multijunction cascade cell, made from combinations of elements from the third and fifth columns of the periodic table. Three junction cells arranged in tandem atop one another achieve 50-percent conversion efficiency at 100 times solar concentration and at 350 K. For space use, radiation resistance has been improved by technologies such as introducing of an electric field in the base layer of the lowest-resistance middle cell, and EOL current matching of sub-cells to the highest-resistance top cell. If light is waveguide directed down woven optical fibers (200 um diameter) coated with indium-tin oxide, the absorption range is greatly increased. The fiber bundles act as their own radiator. Other nano-technology enhancements include double-hetero wide band-gap tunnel junctions, precise lattice-matching to Ge substrates, and 1.96 eV AlInGaP top cells. Cascade photovoltaics so enhanced attain 2 ton/MWe and 1.4 kg/m2.  “InGaP/GaAs-based multijunction solar cells”, Tatsuya Takamoto, Minoru Kaneiwa, Mitsuru Imaizumi, Masafumi Yamaguchi,  Progress in Photovoltaics, Vol. 13, pp 495-51122 Aug 2005

Cascading Thermoacoustic generator - Two pairs of traveling wave stages in a linear topology can create acoustic power from thermal power with high efficiency (20%).  In a closed-cycle, hot coolant from a radiator enters a standing wave stage, which supplies power to a traveling-wave stage., which produces  60 MW of acoustic power. D.L. Gardner & G.W. Swift, 2003.

Casimir Battery generator - The Casimir-Polder force is the relativistic retarded van der Waals force between two metal plates.  The force per unit area between the plates goes to zero as alpha, the fine structure constant, goes to zero. In the nanotech battery application shown, “vacuum” energy is stored in an aluminum spiral just 1 or 2 atoms thick and a few nm apart. The coil is positively charged, so that the electrostatic repulsion between each coil loop balances the vacuum fluctuation attraction.  Robert Jaffe of MIT, 2005.

Catalyzed fission scintillator generator - The output of a matter-antimatter reaction is difficult to convert into electricity. One way is to direct a stream of antiprotons onto a conical receptor consisting of a uranium-coated scintillator. The collisional fissions emit light, captured by high temperature CdWO4 photovoltaic cells in the next layer down. These PVs operate in the 350nm (5-eV) range, converting 10% of the antiproton beam into electricity. Radiators, coupled to the scintillator is a pulsed mode, reject 21% of the energy. Howe estimates the specific weight at 6.6-tonnes/MWe, although the game uses a more optimistic value of 2-tonnes/MWe. Steven Howe, 2012.
 
Diamonoid electrodynamic tether generator - When a conductive tether is deployed from a spacecraft and cuts a planet's magnetic field, it generates a current, and thereby converts some of the spacecraft's kinetic energy to electrical energy. As a result of this process, an electrodynamic force acts on the tether and attached object, slowing their orbital motion. The tether's far end can be left bare, making electrical contact with the ionosphere. Functionally, electrons flow from the space plasma into the conductive tether, are passed through a resistive load in a control unit and are emitted into the space plasma by an electron emitter as free electrons. Megawatts of high-current tether power are attainable. The energy is stored in a magnetically insulated spacecraft acting in the ultrahigh vacuum of space as a gigavolt capacitor. Winterberg, 2008.

Dusty plasma MHD generator –This generator produces electricity by decelerating an ion beam of fission products.  This bypasses the Carnot cycle and thus achieves twice the efficiencies of heat engines. The ions are emitted by fissile 80nm dust particles suspended by magnetic fields. Dust increases the surface area enough to allow for effective radiative cooling. As the particles naturally ionize as fission occurs, electrostatic suspension is a simple process. Deceleration is accomplished by a series of MHD electrodes, generating 60 MWe DC power at 46% efficiency. Rodney A. Clark and Robert B. Sheldon, 2005.

Ericsson engine generator–The Ericsson thermodynamic cycle is similar to the Stirling cycle in that both generate electricity efficiently using external combustion with regenerators. Both cycles can run off either a solar or nuclear heat source. The game design uses the latter, generating 60 MWe at 3000 rpm and 30% efficiency.
 
Flywheel compulsator generator – A low-density disk spun in a vacuum to up to 80000 rpm can store considerable energy in its angular momentum.  Twin wheels, each 1.5m in diameter and 1.2m thick and made of graphite stiffened with carbon whiskers, together weigh 5 tonnes.  They spin on a single axle suspended by superconducting YBCO magnetic bearings. The system includes rectification, filtering, and inversion electronics, as well as fuzzy logic dampening of shaft vibration.  A compulsator (short for compensated pulsed alternator) provides the power supply. The maximum energy stored is 200 GJ, with a peak load of 600 MWe.  The specific power and energy are 7.6 kW/kg and 2.5 MJ/kg.
 
Granular rainbow corral generator– A cloud of micron-sized reflective particles are shaped into a specific surface by light pressure, allowing it to form a very large and lightweight aperture of an optical system. Flying in a formation as part of a so-called “granular spacecraft”, it reflects sunlight like a rainbow. The rod-like particles are corralled in three dimensions by use of a strongly focused Gaussian laser beam. The opto-mechanical interactions are at the grain level, trapping the low index particles within an optical vortex. Marco Quadrelli, 2013.
 
H2-O2 fuel cell generator – Regenerative alkaline fuel cells convert the chemical energy of fuels directly into electricity, using hydrogen as the fuel and oxygen as the oxidant.  At the anode, hydrogen gas combines with hydroxide ions to produce water vapor plus free electrons.  The specific energy of the fuel alone is 13.5 MJ/kg.  At the cathode, oxygen and water plus returning electrons from the circuit form hydroxide ions that are recycled back to the anode. Either proton exchange membranes (PEM), or microbes that transfer electrons to the electrode as they metabolize, are used to produce the current. Titanium-crusted carbon nanotubes meet the two key requirements for efficient hydrogen storage: the abilities to latch on to hydrogen molecules in adequate numbers and to relinquish the hydrogen readily when heated.  A 200 GJ fuel cell stack has a peak load of 600 MWe, and a volume of 300 m3.  The specific power and energy are 15 kW/kg and 5 MJ/kg.  Operating temperatures are 400K with a thermal efficiency of 70%. 
 
In-core thermionic generator - Thermionic conversion systems may be conceived as an "electron boiler" in which electrons are thermally boiled off a heated emitter cathode, and collected on an anode surface, delivering DC electrical power to an external load. In-core thermionic systems are composed of converters which are directly attached to individual fuel elements within the reactor core (fast fission or fusion). Anode cooling is by a loop which circulates a liquid metal coolant to an external radiator. The baseline system uses single-layer BaCs converters and collector anodes coated with SiC.  Thermoelectric-electromagnetic (TEM) pumps are used to pump the sodium to the radiators and back. When operating at a 1300 K emitter temperature, the cycle conversion efficiencies are 15%, and the output is 125 MWe.  Specific power including the sodium heat pipes is 2200 kg/MWe.
 
JTEC H2 thermoelectric generator – Most heat engines use a Stirling, Brayton, or Rankine cycle operating between a hot source and a cold source. These are mechanical devices which uses circulating fluids and pistons or turbines to express their energy, typically with about 30% efficiency.  In contrast, the JTEC (Johnson Thermo-Electrochemical Converter) is all solid-state, generating electricity without moving parts at efficiencies of about 60%.  The JTEC generator utilizes the electrochemical potential of hydrogen pressure applied across a proton conductive membrane (PCM). On the high-pressure side of the PCM, hydrogen gas is oxidized resulting in the creation of protons and electrons. The pressure differential forces protons through the membrane causing the electrodes to conduct electrons through an external load. On the low-pressure side, the protons are reduced with the electrons to reform hydrogen gas. The PCM and a pair of electrodes form a membrane electrode assembly similar to those used in fuel cells. The JTEC uses two such assemblies, one coupled to a high temperature (up to 1400 K) heat source and the other to a low temperature radiator. Hydrogen circulates within the engine between the two MEA stacks via a counter flow regenerative heat exchanger. Lonnie Johnson www.johnsonems.com
 
Magnetoshell plasma parachute generator–The magnetoshell parachute, deployed on a 50m tether during an aerobrake maneuver, forms a 500 Gauss magnetic dipole field.  A low temperature magnetized plasma is injected into that field. Instead of deflecting gas like an aeroshell, or plasma like a magnetic decelerator, it captures hypersonic neutral gas through collisional processes.  The momentum of the charge-exchanged gas is absorbed by the magnetic structure, while the ionized gas fuels and heats the plasma. Assuming a Neptune aerobrake at 2000km altitude, and a maximum atmospheric density of 3.5X1018 molecules/m3 at an incoming velocity of 26.7 km/s, the neutral molecular weight is 2.5 amu and the directed neutral energy is 9.4 eV. This generates 10 MWe while decelerating the spacecraft. David Kirtley, NASA, 2013.
 
Marx capacitor bank generator - Many impulsive drives (Z-pinch, fusion focus) require a large bank of capacitors to be charged over several microseconds and discharged much more quickly with very little loss. The circuit known as a Marx Generator generates a high voltage pulse by charging capacitors in parallel to a given voltage, then discharging them in series by spark gap or plasma switches. Diodes prevent ringing between the capacitive and inductive portions of the circuit. A bank of Marx capacitors has 0.53 Amps/m and stores 54 kJ/kg at a few hundred kilovolts.
 
MHD open-cycle generator – Magnetohydrodynamics (abbreviated MHD) is the control of plasmas using magnetic fields. A MHD electric generator has the advantages of high direct energy conversion efficiencies (90%), no moving parts, and instant turn-on. Installed on the output of a rocket nozzle or vapor core reactor, it magnetically expands and cools the exhausted plasma, extracting electrons with a large grounded collector plate.  MHD can also convert laser energy into electricity. For the open-cycle design shown, an expander spreads out into a fan shape with a radius of many meters. The positive ions (at energies in the region of 400 kilovolts for fusion end loss plasmas) are collected on a series of high voltage electrodes, resulting in the direct transfer of kinetic energy to a direct current.  The electrodes must be cleaned to prevent build-up of conductive deposits that can cause shorts. The process is reversible, with an input of electricity the fuel economy of the rocket exhaust (such as for chemical rockets or arcjets) can be doubled. Room-temperature superconducting magnets with a high current density create a crossed field region (4 T) that accelerates the plasma with Lorentz forces. Because of material limits, the duct temperature is limited to 2500 K. The plasma is usually seeded with an alkali metal (e.g. magnesium) for conductivity, which must be recovered and recycled from the effluent stream. The net power density is 350 kg/MWe. Development of Liquid-Vapor Core Reactors with MHD Generator for Space Power and Propulsion Applications, Samim Anghaie, University of Florida, 2002
 
Microbial fuel cell generator – Bioengineered “organic-electronic” crops grown in space transfer organic material through their roots into the sediment, a process called rhizodeposition. A microbial fuel cell (MFC) uses bacteria as a catalyst to convert the chemical energy of this sediment directly into electricity. The bacteria in the anaerobic sediment will use the MFC anode as an insoluble electron acceptor. Using nano-wires, the anode collects 95% of the electrons originating from the microbial metabolism. The protons flow through a proton or cation exchange membrane to the cathode, where they are reduced with oxygen into water. For rhizodeposition feeding rates of .3 kg/liter/day, MFCs achieve powers of a kW/m2 of electrode surface. This power is stored in “living capacitors”, which utilize entrained nanotubes for high surface area, enabling near instantaneous charging and no degradation.  The dielectric is whisker nMOS, and the leads are vapor deposited gold.  Up to 4 GJ of solar energy are stored in 20,000 cells, each at 390 K and 30V. A plot 470m across (18 hectares) at 1 AU can generate 60 MWe.
 
Nanocomposite thermoelectric generator–When current flows between a junction of two different metals, heat is generated at the upper junction and absorbed at the lower one, known as the Peltier effect.  A unit-less figure of merit known as the ZT score reflects the overall efficiency, with 1 the standard for most compounds and 4 performing similar to mechanical devices.  Nanocomposites, particularly Ytterbium, Perovskite-type oxides, Si-Ge-doped buckyballs and Copper Aluminate compounds have shown promise of ZT > 5 at temperatures approaching 1400 K.   Complex semi-organic quasi-crystals may achieve ZT scores on the order of 20, allowing efficiencies > 90%. The thermoelectric effect can also be used in reverse to convert waste heat to electricity.  Specific area is 34 kg/m2 at 1275 K. Su L., Gan Y. Advances in Thermoelectric Energy Conversion Nanocomposites. 2011.
 
O’Meara LSP paralens generator – A laser can project a beam only so far before it starts to spread. If a lens is inserted into the beam before the spreading starts, then the energy in the beam is captured and refocused to form a completely new beam. A lens is a gossamer structure that has the ability to focus electromagnetic radiation, typically optical, infrared or microwave radiation.  Illustrated is a 55m diameter O’Meara para-lens, consisting of alternating layers of nothing and Kapton plastic with a thickness chosen to add a half-wave of phase to the laser light.  Its parameter is 9 kg/m2. Injecting the beam into a specially-shaped rocket nozzle that focuses the beam with a laser absorption efficiency of 90%, the rocket fuel efficiency can be doubled.   
Robert L. Forward, Advanced Propulsion Concepts Study-Comparative Study of Solar Electric Propulsion and Laser Electric Propulsion (June 1975). Final Report on JPL Contract 954085, Subcontract under NASA Contract NAS7-100, Task Order RD-156.
 
Palmer LSP aerosol lens generator – The lightest possible lens is formed from a cloud of glass beads or aerosol droplets forming a 3-D pseudo-holographic Fresnel lens. If these droplets have a highly nonlinear index of refraction, they can be "organized" by a structured laser beam that interacts with the nonlinear optical index of the beads to put forces on the beads that "trap" the beads into fresnel-zone-like three-dimensional holographic-grating lens structures. A typical droplet field is 80 meters across, with a parameter of 0.1 kg/m2 and a transmission efficiency of 96%.  The aerosol (vacu-sol?) lens can improve the fuel efficiency of a rocket by focusing the laser light into its supersonic exhaust stream in the nozzle. This creates a laser sustained plasma (LSP), a 15,000 K stationary region which transfers energy to the propellant via the reverse bremsstrahlung process.
J. Palmer, "Aerosol Lens", J. Optical Society of America, Vol. 73, p. 1568 ff, 1983.
A. Mertogul and H. Krier. Two-temperature modeling of laser sustained hydrogen plasmas. Journal of Thermophysics and Heat Transfer, 8:781–790, 1994.
 
Photon tether rectenna generator – Microwaves beamed from a remote satellite are efficiently converted into DC power by using a special antenna called a rectenna.  This is an assemblage of little dipole antennas connected to a network of semiconductor diodes and filter circuits that rectify AC into DC power. Using nanotube Schottky barrier diodes as the rectifier allows high conversion efficiencies (90%). The thin film rectenna illustrated installs these diodes on a etched layer of Kapton film, and operates at 30 GHz at a specific power of 700 kg/MW. However, the range is quite limited for receiving and transmitting antennas of reasonable size. For instance, a transmitter in LEO would need to be 10 km in size to send 30 GHz power to a 100 m rectenna in GEO.

Radioisotope stirling generator– The radioactive decay of plutonium provides 0.5 kWth/kg. A hundred tonnes of plutonium produces 15 MWe using a Stirling engine operating at 30% efficiency.
 
Rankine MHD generator - As contrasted to the Brayton cycle, the Rankine heat engine cycle uses a working fluid that changes phase. This gives it lower cycle temperature ratios, and thus lower masses and higher efficiencies (22%). Rankine MHD is a closed-cycle version of the MHD generator. Magnetohydrodynamic (MHD) conversion systems produce electrical power the same way as conventional turbine generators, except that the rotating magnet is replaced by a conductive ionized plasma, which passes through a channel surrounded by a magnetic field. The power generated is proportionate to the channel volume, plasma velocity, and the field strength of the surrounding magnets. The ultrahigh temperature system shown uses a disk MHD generator (Hall type with 4T magnetic field) surrounding the output stream of a reactor.  For a vapor-core fission reactor, the working fluid might be uranium tetrafluoride (UF4) plus KF at 4000 K and 40 atm.  After being expanded to 0.08 atm in the MHD diffuser, the UF4 /KF is condensed and separated in radiators operating at 2100 K, and recycled, with losses proportionate to channel area. The steady state parameters are: a neutron flux of 1015 n/cm2-sec, a nuclear-enhanced electric conductivity of 60 mho/m, a magnetic field of 4 Tesla, and a specific power of 1.5 ton/MWe. This generator can operate in a burst mode when powering pulsed lasers or electric engines. When pulsed power is not needed, the nuclear core remains at a subcritical fuel density. Then when a pulse of power is needed, the active volume of the core is decreased to criticality, using pulsed MHD magneto-induction. 
Samim Anghaie, University of Florida, “Development of Liquid-Vapor Core Reactors with MHD Generator for Space Power and Propulsion Applications”, 2002
 
Rankine multiphase generator - This thermal-to-electric closed-cycle heat engine is designed to convert “waste heat” bremsstrahlung radiation into electricity. The unwanted radiation is absorbed by and ionizes a high-temperature working fluid. Two twisting electromagnetic forces out of phase with each other form vortices that axially and radially expand the ionized gas. The F= q (v Å~ B) force longitudinally expands and cools the gas and the F= i (L Å~ B) magnetic force transfers its energy to a multiphase electrical system where electricity is harvested by rectifiers. Thus the vortex acts as an MHD “turbine” extracting heat energy in a cycle similar to the Rankine MHD generator. The claimed almost-Carnot efficiency of 85%, unheard of in a closed-cycle, has not been demonstrated. The 15% low quality heat passes to a heatsink and ultimately to a medium temperature radiator. An inlet temperature of 4000K generates 60 MWe and requires a 600K radiator of 1.5m2 area. Moacir Ferreira Jr.
 
Stirling generator - The Stirling cycle uses a closed-cycle reciprocating engine and a high-pressure single-phase gaseous working fluid, often hydrogen or helium. The fluid may be heated by solar or nuclear energy. The engine is designed to compress the working fluid in the colder side of the engine and expand it in the hot side, resulting in a net conversion of heat into rotary motion. The free piston Stirling converter illustrated is optimal for space applications due to the absence of wear mechanisms. A Stirling engine heated by a solar mirror 425 m in diameter at 1 AU, can drive a generator with an output of 60 MWe of alternating current. If carbon-carbon composites are used, the hot side of the Stirling cycle can reach temperatures of 2000 K, with a thermodynamic efficiency of 30%. This efficiency is superior to all other heat engines.     
 
Superconducting adductor generator – Energy can be stored inductively in the magnetic field of coils chilled to superconducting temperatures. A toroidal geometry lessens the external magnetic forces and reduces the size of the mechanical support needed. A 25 tonne adductor with a volume of 150 cubic meters stores 6 GJ in a magnetic field of 10 T.  Its parameter is 40 MJ/m3.at 60 MWe.  Kamiyama 1994
 
Thermophotovoltaic generator – A solar thermophotovoltaic (TPV) system includes both a photon element and a heat element, so it can run off of both light and heat.  The photon element includes a filtered blackbody-based converter, bandpass/infrared (IR) reflector filters, and monolithic two-junction two-terminal TPV converters: GaSb (top cell)/InGaAsSb (bottom cell).  The heat element has three diamond nanofactured layers: silicon germanium (SiGe), lead tellurium (PbTe), and bismuth telluride (Bi2Te3). When equipped at 1 AU with a lightweight solar concentrator 300 meters in diameter, solar TPV generates 60 MWe of DC power.  With an 1800 K heat source, both high cascade efficiency (62%) and high output power density (about 2 W/cm2) are realized.  Other parameters are 1 kg/m2 specific area, and 1400 kg/MWe specific power. Sang Choi, Nano-BEAMS Lab, NASA Langley Research Center, 2003.

Triggered decay nuclear battery generator–The nucleus of an isomer—a long-lived excited state of an atomʼs nucleus—holds an enormous amount of energy. If  this energy is suddenly released by a trigger, rather than a slow decay over time, it would be the basis for a powerful nuclear battery. This design uses the radioactive isomer 177mLu, triggered into fission by x-rays.  1000 TJ/kg are released. The radioactive material sits atop a device with adjacent layers of P-type and N-type silicon, so that ionizing radiation directly penetrates the junction and creates electron-hole pairs for the generation of electricity. http://phys.org/news/2013-03-revolutionary-nuclear-battery-closer.html\#jCp
Z-pinch microfission generator – Electrodynamic zeta-pinch compression can be used to generate critical mass atomic bombs at very low yields. These detonations can be used to generate impulsive power or thrust. Exotic fission material (245Cm) is utilized to reduce the required compression ratio. The explosion of each low yield (335 GJ) atomic bomb energizes and vaporizes a set of low mass transmission lines, used to pump either another high current Z-pinch, or a bank of nanotube-enhanced ultracapacitors. Each bomb uses 40 g of Cm fissile material and 60g of Be reflector material, with an aspect ratio of 5. A DT diode is used as a neutron emitter. The mylar transmission lines have a mass of 15 kg, and are replaced after each shot. The design illustrated is rated for a shot every 5.5 minutes, equivalent an output of 1000 MWth. If utilized for thrust, this provides 7.7 kN at a specific impulse of 17 ksec.
“Mini-MagOrion Micro Fission Powered Orion Rocket”, 2002 Ralph Ewig \& Dana Andrews, Andrews Space \& Technology
 
Z5. REACTORS
Antimatter bottle reactor – If a way to produce antimatter fuel cheaply is found, it can be stored as levitated antihydrogen ice. A few micrograms can be removed from storage by first using UV to drive off the positrons, and then accelerating the antiprotons out with electric fields. A pulse of 5 µg of fuel (3 X1018 antiprotons) is then collided with 60 g of heavy metal propellant (9 X 1024 atoms of lead or depleted uranium) in a magnetic bottle. Each antiproton annihilates a proton or neutron in the nucleus of a heavy atom.  The use of heavy metals helps to suppress neutral pion and gamma ray production by reabsorption within the fissioning nucleus. If regolith is used instead of a heavy metal, the gamma flux is trebled requiring much more cooling.  Each pulse contains 900 MJ of energy, and at a repetition rate of 2 Hz, a power level of 1800 MWth is attained. Compared to fusion, antimatter rockets need higher magnetic field strengths to ensure adequate containment: 16 T in the bottle and 50 T in the throat. After 7 ms, when the plasma reaches 6 keV in temperature and 350 atm in pressure, this field is relaxed to allow the plasma to escape. Compared to fusion, more power is lost to bremsstrahlung X-rays due to the higher temperatures and pressures. Furthermore, the short-lived charged products of the antiproton reactions (charged pions and muons) must be exhausted quickly to prevent an increasing amount of reaction power lost to unrecoverable neutrinos. About a third of the reaction energy is X-rays and neutrons stopped as heat in the shields (some of which is recoverable in a Brayton cycle), another third escapes as neutrinos, and the final third is charged fragments directly converted to thrust or electricity in a MHD nozzle. “Concepts for the Design of an Antimatter Annihilation Rocket,”  D.L. Morgan,  J. British Interplanetary Soc. 35, 1982.  (For use in this game, to keep the radiator mass within reasonable bounds, I reduced the pulse rate from 60 Hz to 2 Hz.)  Antiproton Annihilation Propulsion, Robert L. Forward, University of Dayton, 1985.
 
 
D-D magneto-inertial fusion reactor –In theory, deuterium fuel can be compressed to fusion conditions by an imploding metal liner if the implosion is uniform, intense, and accomplished with great precision.  If a large magnetic field in the target suppresses thermal transport, the imploding power becomes low enough to use magnetic implosion. The metal shell could subsequently be used as a propellant. An oscillating compression coil magnetically implodes a plasma liner used to bring a target FRC plasmoid to fusion conditions. The reciprocating nature of this engine also provides efficient direct electrical power needed for target plasmoid formation and heating. By using sprayed on Lithium liners, a repetition rate of 0.1 Hz and a power of 510 MWth is attained, at a Q of 200. Uses a 350 kW solar-powered initiator. John Slough, MSNW, 2012

D-T fusion tokamak reactor – Of all the fusion reactions, the easiest to attain is a mixture of the isotopes of hydrogen called deuterium and tritium (D-T). This reaction is “dirty”, only 20% of the reaction power is charged particles (alphas) that can be magnetically extracted with a diverter for power or thrust. The remaining energy (neutrons plus bremsstrahlung and cyclotron radiation) must be captured in a surrounding jacket of cold dense lithium plasma. The heated lithium is either injected into the nozzle as open-cycle coolant, or recirculated through a heat engine (to generate the power needed for the microwave plasma heater). The fusion reactor shown uses magnetic confinement. The donut-shaped tokamak uses eight poloidal superconducting coils, weighing 22 tonnes with stiffeners and neutron shielding, to produce a 30T magnetic field. The pulsed D-T plasma, containing a current of tens of megamps, is superheated by 50 MW of microwaves or colliding beams to 20 keV.  The Q (gain factor) is 40. Closed field line devices such as tokamaks can ignite and burn, in which case the Q goes to infinity and microwave heating is no longer needed.  However, since ignition is inherently unstable (once ignited, the plasma rapidly heats away from the ignition point), the reactor is kept at slightly below ignition.  Fuel is replenished at 24 mg/sec by gas puffing to maintain a plasma ion density of 5 X 1020/m3 and a pressure of 26 atm. At a power density of 250 MWth /m3, the lithium-cooled first wall has a neutron loading of 1 MW/m2 and a radiation loading of 5 MW/m2.  More advanced vortex designs, which do away with the first wall, separate the hot fusion fuel from the cool lithium plasma by swirling the mixture. The thermal efficiency is 50% in open-cycle mode.   “A Spherical Torus Nuclear Fusion Reactor Space Propulsion Vehicle Concept for Fast Interplanetary Travel,” Williams, Borowski, Dudzinski, and Juhasz, Lewis Research Center, Cleveland, Ohio, July 1998.  (The tokamak used in High Frontier is a smaller lower tech version of the Lewis design.  Because it uses D-T instead of 3He -D fuel, it requires far more open-cycle and radiator cooling.)
 
Fast-neutron fission reactor - When struck by a thermal neutron, a fissile nuclide splits into two fragments plus energy. For example, the fission of the 235U atom produces 165 MeV of energy plus 12 MeV of neutral radiation (gammas and a couple of fast neutrons).  The fast neutrons must be thermalized by a low Z moderator (a surrounding blanket of about 80 cm of D2O, Be, liquid or gas D2, or CD4), which returns enough thermal neutrons to the core to sustain the chain reaction.  (Thermal neutrons diffuse through the reactor like a low pressure gas.)  Alternatively, a molybdenum neutron reflector can be used. Much of a reactor’s mass is constant, regardless of power level. Therefore, nuclear power sources are more attractive at higher power levels. The 650 MWth system illustrated is dual mode, which can either generate electricity, or directly exhaust coolant for thrust.   It uses a fast reactor with fuel tubes interspersed with cooling tubes.  The coolant is lithium, which for electrical power is passed to a potassium boiler at 1650 K. The potassium vapor is passed to a static (AMTEC) or dynamic (turbine) heat engine for power generation (60 MWe), or heats hydrogen in a heat exchanger for thrust (125 kN at a specific impulse of 1 ks).  The thermal efficiency is 19% if closed-cycle (for power generation) or 94% if open-cycle (for thrust).
 
Fission-augmented D-T inertial fusion reactor – A hypothetical hybrid fission-fusion reactor may use D-T fuel pellets surrounded by a fissionable blanket Th-232 to produce energy sufficiently greater than the input (laser) energy for electrical power generation. The principle involved is to induce inertial confinement fusion (ICF) in the fuel pellet which acts as a highly concentrated point source of neutrons which in turn converts and fissions the outer fissionable blanket. LLNL, 2008.
http://nextbigfuture.com/2008/12/proposed-laser-ignition-fusionfission.html
 
Free radical hydrogen reactor – Free radicals are single atoms of elements that normally form molecules.  Free radical hydrogen has half the molecular weight of H2 and, if used as propellant, doubles the specific impulse of thermodynamic rockets.  Alternatively, if recombined, its specific energy (218 MJ/kg) produces a theoretical specific impulse of 2.13 ksec.  Monatomic H is produced in situ in a solid H2 matrix by particle bombardment, cooled by VUV laser chirping, and finally trapped in a hybrid laser-magnet as a Bose-Einstein gas at ultracold temperatures. Free radical hydrogen is confined in a Pritchard-Ioffe trap to keep its mobile spin aligned.  Confinement is provided by the interaction of the atomic magnetic moment with the inhomogenous magnetic field. The trapping density is >1014 atoms/cc   (much higher than Penning traps).  Spin-vector polarization increases the fusion reactivity cross-sectional area of heavy hydrogen by 50%, increasing its utility as a fusion fuel, and neither ionization nor atomic collisions will depolarize the free radicals.
 
H-B fusion reciprocating plasmoid reactor – The operation of this fusion engine is comparable to that of an internal combustion engine. The fuel to be combusted is a 25 mg pellet of decaborane (H14B10), a solid at room temperature. This is magnetically converted into a hydrogen-boron plasmoid in a field-reversed configuration (FRC), and injected into a compression/burn chamber. The compression stroke is driven by a piston sheath coupled to a 100 kHz axial magnetic field. This stroke ignites the plasmoid at 300 keV. The sheath plasma forming the piston is lithium, water, or scooped atmosphere propellant. After being superheated, both the fusion products and the sheath propellant are expanded for thrust or energy in a magnetic nozzle. Electrical energy for the compression is picked up via MHD coils in the exhaust. The high energy density, direct propellant coupling, magnetic insulation, and low fusion gain allow for a vastly lighter engine than other magnetically-confined fusion systems such as spherical Tokamaks. With open-cycle cooling and an air scoop, the thrust to weight ratio can be above unity, allowing a ramjet version to enter orbit from the Earth’s surface. Although the H-B reaction is aneutronic, collisions between the ions and electrons lose half the energy to bremsstrahlung X-rays. The sheath, acting as open-cycle coolant, intercepts many of these X-rays, allowing a thermal efficiency of 85%. At 100 kHz and 2.5 kg/sec, 3 GWth is generated with an overall efficiency of 65%. John Slough, “Earth to Orbit based on a Reciprocating Plasma Liner, Compression of Fusion Plasmoids,” University of Washington, 2007.
 
3He-D mirror cell reactor – Helium 3 is an isotope of helium, and deuterium (abbreviated D) is an isotope of hydrogen.  The 3He – D fusion cycle is superior to the D-T cycle since almost all the fusion energy, rather than just 20%, is deposited in the plasma as fast charged particles.  Magnetic containers with a linear rather than toroidal geometry, such as steady-state mirrors, have superior ratios of plasma pressure to magnet pressure (ß >30%) and higher power densities necessary for reaching the high (50 keV) 3He – D operating temperatures.  The mirror design shown is a tube of 11 T superconducting magnetic coils, with choke coils for reflection at the ends.  The magnets weigh 12 tonnes, plus another 24 tonnes for 60 cm of magnet radiation shielding and refrigeration.  A mirror has low radiation losses (20% bremsstrahlung, 3% neutrons) compared to its end losses (77% fast charged particles).  These losses limit the Q to about unity and prevent ignition. (This is not a problem for propulsion, since reaching break-even is not required to achieve thrust. The plasma is held in stable energy equilibrium by the constant injection of auxiliary microwave heating.)  The Q can be improved by a tandem arrangement: stacking identical mirror cells end to end so that one’s loss is another’s gain.  The exhaust exiting one end can be converted to power by direct conversion (MHD), and the other end’s exhaust can be expanded in a magnetic flux tube for thrust.  Mirrors improved by vortex technology, called field-reversed mirrors, introduce an azimuthal electron current which creates a poloidal magnetic field component strong enough to reverse the polarity of the magnetic induction along the cylindrical axis.  This creates a hot compact toroid that both plugs end losses and raises the temperature of the scrape-off plasma layer fourfold (to 2.5 keV), corresponding to a specific impulse of 32 ksec. Mirrors, like all magnetic fusion devices, can readily augment their thrust by open-cycle cooling.   “Considerations for Steady-State FRC-Based Fusion Space Propulsion,” M.J. Schaffer, General Atomics Project 4437, Dec 2000.
 
H-6Li fusor reactor – A Farnsworth-Bussard fusor is little more than two charged concentric spheres dangling in a vacuum chamber, producing fusion through inertial electrostatic confinement. Electrons are emitted from an outer shell (the cathode), and directed towards a central anode called the grid.  The grid is a hollow sphere of wire mesh, with the elements magnetically-shielded so that the electrons do not strike them.  Instead, they zip right on through, oscillating back and forth about the center, creating a deep electrostatic well to trap the ions of lithium 6 and hydrogen that form the fusion fuel. With a one meter diameter grid and a fuel consumption rate of 7 mg/sec, the fusion power produced is 360 MWth.  Half of this energy is bremsstrahlung X-rays, which must be captured in a lithium heat engine. The other half are isotopes of helium (3He and 4He), each at about 8 MeV. (Overall efficiency is 36%). Since both products are doubly charged, a 4 MeV electric field will decelerate them and produce two electrons from each, producing an 18 amp current at extremely high voltage.  An electron gun using this 4 million volt energy would emit electrons at relativistic speeds.  This beam dissipates quickly in space, unless neutralized by positrons or converted into a free electron laser beam.   “Inertia-Electrostatic-Fusion Propulsion Spectrum: Air-Breathing to Interstellar Flight,” R W. Bussard and L. W. Jameson, Journal of Propulsion and Power, v. 11, no. 2, pp. 365-372.   (Philo Farnsworth, the farm boy who invented the television, spent his last years in a lonely quest to attain break-even fusion in his ultra-cheap fusor devices. His ideas are enjoying a renaissance, thanks to Dr. Bussard, and working fusion reactors are making an appearance in high school science fairs.  On the theory that the fusor is power-limited, I have scaled down Bussard’s 10 GW design to 360 MW.)
 
Lyman alpha trap reactor – This antimatter factory uses a 200 GeV proton accelerator to smash a stack of multiple thin targets surrounded by wide angular arrays of multiple lenses with different velocity acceptances.  At a beam power of 0.2 MWj and a current of 3.2 µamps, one in ten incident protons becomes an antiproton.  The shower of antiprotons is entrained with positrons to form first atomic, then molecular, antihydrogen. This formation process is enhanced 100-fold by an optical laser traveling opposite to the beams of antiprotons and positrons, as shown in the illustration. This antimatter beam is cooled by a radio frequency quadrupole (RFQ) decelerator, trapped with Lyman alpha laser beams, and tickled into forming an iceball in a Lyman-α light trap.  The iceball is grown into a charged microcrystal levitated by electrostatic forces.  The temperature is kept below 1°K by lasers to keep sublimation pressure low.  The energy of the unused exiting proton stream is recovered via MHD or uranium enrichment. Solar energy is obtained through solar-pumped iodine gas dynamic lasers. 10 µg/year of antimatter is produced.

Metallic hydrogen reactor –The 25 GPa pressure required for metallic hydrogen can be greatly reduced both by lithium doping and by entrapping it within cryogenically frozen solid hydrogen.  SiH4, even at moderate pressures forms a sub-lattice of metallic hydrogen.  Deuterium is easier to metallize than hydrogen, requiring less than a third the pressure and solid D2 has been achieved at room temperatures. When the metastable metallic hydrogen lattice is heated above 1000 K at 40 bar pressure, the metastability breaks down and the hydrogen lattice recombines into hydrogen gas, releasing 216 MJ/kg in the process (LOX-H2 has an energy density of 10 MJ/kg).  Pure metallic hydrogen could yield an Isp of 1700 sec but would result in reaction chamber temperatures above 6000 K.  The design burns 5 kg/sec of metallic hydrogen for 1100 MWth.
SIlvera I. and Cole J. Metallic hydrogen: the most powerful rocket fuel yet to exist. 2010.
 
Mini-Magnetosphere rf Paul trap reactor – The natural environment of space can be used as a low energy accumulator to store a variety of exotic particles, including antimatter. This device turns a spacecraft into a Bussard ramjet, (albeit one that scoops its fuel rather than its propellant, a much simpler task).  The “scoop”, a huge (100 km) magnetic bubble called a “mini-magnetosphere”, is inflated with conductive ionized gas or plasma. Field lines are generated by onboard solenoids, powered by 3 kW photovoltaics. A reservoir of helium replenishes ions that leak from the plasma. The field lines of the mini-magnetosphere direct solar wind onto a heavy metal target. The resulting spray of photons and particle-antiparticle pairs contain a small fraction (<5%) of proton-antiproton pairs, as well as other exotics.  These are magnetically focused, debunched and cooled using laser chirping and radio frequency (rf) fields, and stored along with positrons in a low energy antiproton Paul trap (see illustration).  Neutral antimatter hydrogen and other exotics are condensed as electrostatically-suspended ice pellets. A few micrograms per year of antimatter ice condensates are produced, at densities approaching 1023 atoms/cc. The mini-magnetosphere can also act as a sail, deriving thrust from the solar winds.  Robert M. Zubrin, "The use of magnetic sails to escape from low earth orbit", Journal of the British Interplanetary Society, Vol. 46, pp. 3-10, (1993).
 
Nuclear-pumped excimer flashlamp reactor – A fission or fusion reactor using fissile microspheres mixed with an excimer fluorescer gas as an aerosol.  This aerosol is driven to an excited state by nuclear radiation, and then emits incoherent narrow band photons. Conical waveguides transport the photon energy out of the cylindrical reactor to fiberoptic bundles, and then to banks of PV cells, which convert it to electricity at 30% efficiency.  This two-step method enables non-Carnot-limited electric power generation from fission fragments, whose transport length (microns) is far too short for direct energy conversion such as MHD.  The narrow band enables a high photovoltaic efficiency of 90%, and the excimer fluorescence efficiency is 35%.  Radiation damage to the crystalline PV is alleviated by periodic thermal annealing. Alternately the VUV photons can be transmitted to a gaseous laser medium in close proximity to the reactor core. Critical dimensions for the aerosol reactor: 2.45m core length and diameter, 9 cm fuel cell radius, 235U fuel density of 1 mg/cm3, or 2.5 X10^18 atoms/cm3, Be Moderator/reflector radius is 12.5cm and reflector thickness is 20cm (keff = 1.11). M.A. Prelas, 1990.
 
Optoelectric nuclear battery reactor – This lightweight, low-pressure, high-efficiency battery converts nuclear energy into light, which it then uses to generate electrical energy. A beta-emitter such as technetium-99 or strontium-90 is suspended in a gas or liquid containing luminescent gas molecules of the excimer type, constituting a "dust plasma." This permits a nearly lossless emission of beta electrons from the emitting dust particles. The electrons then excite the gases whose excimer line is selected for the conversion of the radioactivity into a surrounding photovoltaic layer. The surrounding weakly ionized plasma consists of gases or gas mixtures (such as krypton, argon, and xenon) with excimer lines such that a considerable amount of the energy of the beta electrons is converted into this light. The surrounding walls contain photovoltaic layers with wide forbidden zones such as diamond. This converts the optical energy into electrical energy.
 
Pebble bed fission reactor – This is a graphite-moderated, gas-cooled, nuclear reactor that uses spherical fuel elements called "pebbles". These tennis ball-sized pebbles are made of pyrolytic graphite (which acts as the moderator), interspersed with thousands of micro fuel particles of a fissile material (such as 235U). In the reactor illustrated, 360,000 pebbles are placed together to create a 120 MWth reactor. The spaces between the pebbles form the "piping" in the core for the coolant, either propellant or inert He/Xe gas. The design illustrated can is dual mode. It can operate either as a generator for 60 MWe of electricity, or act as a solid-core thruster using hydrogen propellant/coolant expelled at a specific impulse of 1 ksec. When used as a thruster, it offers a slight increase in specific impulse but significant acceleration benefits over traditional fission reactors.  Moreover, the high temperatures (up to 1900 K) allow higher thermal efficiencies (up to 50%).
 
Penning trap reactor – Exotic fuel, produced in a superconducting supercollider, can be cooled by a radio frequency quadrupole (RFQ) and stored and transported into space in liter-sized Penning trap thermos bottles. For instance, a thousand such traps, each weighing 80 kg, would hold 1017 antiprotons (140 ng), enough fuel for a delta-v of 100 km/sec.  Other fuels include metastable helium, ultracold neutrons, and free radical hydrogen.  A Penning trap uses a combination of laser cooling and electromagnetic fields to store particles. Each is able to store more particles (1014) then rf Paul traps, and also does not use dynamic rf fields which can heat the trapped fuel.  The Brillouin limiting factor for Penning traps is 1011 antiprotons/cc. 
 
Positronium bottle reactor – Positronium (Ps), comprised of an electron and a positron, is the smallest possible “atom” as both components are structureless point-like leptons (massing 1.82× 10-30 kg compared to 1.67 × 10-27 kg for hydrogen) .  Depending on the spin state, para-Ps has a lifetime of 0.125 ns while ortho-Ps has a lifetime of 142 ns before decaying into two or three gamma-rays respectively.  Ps can be stabilized by pinning the electron and positron in a magnetic field while a crossed electric field stretches the atom apart by up to 400 nm.  This increases the potential barrier that the leptons must tunnel through to annihilate one another, extends their lifetimes to over a year.
A Ps beam is created by impacting a beam of positrons onto a tungsten target with an efficiency of 10-5.  The positrons themselves are produced via the natural decay of 22Na which has a half-life of 2.6 years before emitting a gamma ray and positron and decaying into 22Ne.  The 22Na is produced by dissolving proton irradiated aluminum into an a sodium enriched hydrochloric acid salt. The resulting positronium beam is stabilized within the crossed electric-magnetic field and directed towards 1 cm × 1 cm × 11 μm quantum dot nanochips where each quantum dot within the chip can suspend up to 1000 Ps are suspended in potential wells.  Up to 1011 Ps atoms may be stored per chip which is kept at 25 mK.  At full capacity, 15 billion chips may be produced and stored. Unfortunately, due to the nature of production and storage, rapid explosive annihilation is not possible but rather must occur at a steady rate via excitation with UV lasers.
Edwards K., Propulsion and Power with Positrons. 2004. AND Ackermann J., Schertzer J., Schmelcher P. Long-lived states of Positronoium in Crossed Electric and Magnetic Fields. 1997.
 
Project Orion reactor – This fabled technology converts the impulses of small nuclear detonations into thrust.  The small shaped-charge bombs each have a mass of 250 kg (including propellant) and a yield of a quarter kiloton (1 terajoule). The fissile material is curium 245, with a critical mass of 4 kg, surrounded by a beryllium reflector.  The soft X-rays, UV and plasma from the external detonation vaporize and compress the propellant to a gram per liter, highly opaque to the bomb energies at the temperatures attained (67000 K). The propellant, a mixture of water, nitrogen, and hydrogen, interfaces with a pusher plate “nozzle”, which can be either solid or magnetic.  Shown is a solid plate, which tapers to the edges (to maintain a constant net velocity of the plate given a greater momentum transfer in the center). Pressure on the plate reaches 690 MPa in the center.  The impulse shock is absorbed by a set of pneumatic “tires”, followed by gas-filled pistons detuned to the maximum detonation frequency of once per 11.5 seconds. This corresponds to 56 GWth of blast energy, of which 7% is utilized for thrust. The shock plate system becomes a useful shield if pointed towards the enemy.    Ted Taylor’s classic design, optimized for low yield bombs and 2 ksec specific impulse: “Project Orion”, George Dyson, Henry Holt and Company, 2002. 
 
Project Valkyrie reactor – This antimatter beam core rocket uses a high temperature (100K+) superconducting magnetic nozzle to direct charged pions resulting from the annihilation of equal amounts of protons and antiprotons to produce thrust. These pions, travelling at near light speed travel 21m (in 70 nanoseconds) before decaying into charged muons and neutrinos.  The muons may travel a further 1.85km (over 6.2 microseconds at 0.995c) before decaying into electrons and positrons.  Along with the produced charged pions, a neutral pion rapidly decays (within 60nm) into 200MeV gamma rays. The reactor chamber is made as transparent as possible, to allow these gamma rays to escape to space. Even so, considerable radiators are required to dissipate the heat absorbed by the magnetic coil shielding. Furthermore, critical components (such as crew) must be distanced by tens of kilometers from this reactor. By attaching tethers of this length to the back of the thruster, so that the cargo is towed instead of pushed, significant weight savings are achieved. A shadow shield of tungsten is mounted 2/3’s of the way on the tether for additional shielding. Antimatter production and storage is the main limiting factor in realizing this engine design but foreseeable scaling from current technology (or from game technology of 10μg/year) makes it feasible for interplanetary missions.
Frisbee R. How to build an antimatter rocket for interstellar missions. 2003.

Rubbia thin film fission hohlraum reactor – The concept proposed by C. Rubbia in 1998, can be seen as the inverse of a NERVA, in that the fissioning surface layer is deposited on the inside wall of the reaction chamber rather than as a core.  This very thin layer of metastable 242mAm can be kept at a reasonable temperature while half the fragments thermalize inside the pumped hydrogen propellant.   However, the other half fragments, injecting into the walls, need to be cooled and radiated away.   The primary advantage is that the propellant can become hotter than the solid reactor walls while still maintaining a simpler (and lighter) solid reactor system (compared to a NERVA).  Because the 242mAm can never become critical, an external neutron source, in the form of a proton accelerator against a tungsten target is required to start the fission.
Czysz P and Bruno C. Future Spacecraft Propulsion Systems. 2006
 
Ultracold neutron reactor – Neutrons are normally unstable particles, with a half life of 12 minutes.  When polarized and ultra-cooled (using vibrators or turbines) to form a dineutron or tetraneutron phase, they are believed to be stable and storable in total internal reflection bottles, lined with diamond-like carbon as the neutron reflector. Ultracold neutrons (UCN) have a huge quantum mechanical wavelength as a consequence of their slow movement (typically 0.4 µm @ 1 m/sec), and thus can spontaneously initiate fission reactions such as n-235U or n-6Li.   If the neutron source is a nuclear reactor, the neutrons must be cooled from 2 MeV to 2 meV using a heavy water moderator, and then in a UCN turbine to 0.2 IeV. 
 
VCR light bulb fission reactor – Most fission reactors avoid meltdown, but the vapor core reactor (VCR) runs so hot (25000 K) that its core vaporizes.  At this temperature, the vast majority of the electromagnetic emissions are in the hard ultraviolet range. A “bulb” transparent to this radiation, made of internally-cooled a-silica, bottles the gaseous uranium hexafluoride, while letting the fission energy shine through.  The operating pressure is 1000 atm.  The UF6 fuel is prevented from condensing on the cooled wall by a vortex flow field created by the tangential injection of a neon “buffer” gas near the inside of the transparent wall.  In a generator mode, the UV uses photovoltaics to generate electricity. In a propulsion mode, the UV heats seeded hydrogen propellant, which exits at a specific impulse of 2 ks.

Z7. RADIATORS
ANDR/In radiator – A heat pipe with Indium as a working fluid can operate between 2000 and 3000 K, higher than any other metal.  However its lower limit of operation is also higher and ideally should operate above 2353 K (the boiling point at 1 atm). Furthermore, its low freezing point makes it particularly attractive from a cold start.  However due to its high corrosion, indium has largely been ignored as a working fluid.  At temperatures approaching 3000 K, only a carbon wall and a tungsten wick system would work.  Here the wall is made of aggregated diamond nanorods (or hyper diamond) produced by the compression of fullerene.  This material with an isothermal bulk modulus of 491 GPa is harder and less compressible than diamond and an increased wear resistance.  Carbon aerogel is used as insulation in the adiabatic section. Due to the short life of the heat pipe (~2200 hrs or 3 months), multiple redundant pipes are on-hand while fouled pipes are automatically replaced and repaired by monitoring robots. Specific area is 275 kg/m2 including spare pipes and repair system.
 
Bubble membrane radiator - This concept sprays hot coolant inside a spinning bubble-shaped membrane. The cooled liquid is returned by centrifugal force. The membrane is space-manufactured from carbon nanotubes, woven cermet fabrics, or other advanced materials. A two-phase working fluid (hot liquid or gas) is centrifugally pumped to the bubble membrane, where it is sprayed on the interior surface.  The fluid wets the inner surface of the sphere and is driven in the form of a liquid film by centrifugal force to the equatorial periphery of the sphere; liquid metal pumps located there return the liquid out of the sphere through rotated shaft seals to its source. As the liquid flows along the inner surface of the envelope it loses heat by thermal radiation from the outer surface of the balloon. The specific area is 7 kg/m2, which radiates from one side at 950 K.
 
Buckytube filament radiator - Waste heat may be rejected by moving thousands of loops of thin (1 mm) flexible “Buckytubes” (carbon nanotubes), which radiate their thermal load prior to return to the heat exchanger. Cables constructed of Arm-chair type nanotubes are the strongest cables known, with design tensile strengths about 70% of the theoretical 100 GPa value. The moving filaments are heated by direct contact around a molybdenum drum filled with the heated working fluid, and then extended into space a distance of 70m by rotational inertia.  Their speed is varied according to the temperature radiated (from 273 K to 1300 K). The loops are redundantly braided to prevent single point failures from micrometeoroids. Each element is heat treated at 3300 K to increase the thermal conductivity through graphitization to about 2500 W/mK. About 12,000 filaments are needed per therm, at 77 g/filament, which exit the spacecraft at 100 m/s and return a few seconds later.  Heat-transfer and Weight Analysis
Of a Moving-Belt Radiator System for Waste Rejection in Space, Richard J. Flaherty, Lewis Research Center, Cleveland, Ohio, 1964
 
Curie point radiator  - A ferromagnetic material heated above its Curie point loses its magnetism.  If molten droplets of such a substance are slung into space, they radiate heat and solidify. Once below their Curie temperature, they regain their magnetic properties and can be shepherded by a magnetic field into a collector and returned to the heat exchanger.  The mass for a 120 MW system operating at 1200 K includes a 13 tonne magnetic heat exchanger and a rotating dust recovery electromagnet on a 25m boom, plus 7 tonnes of dust spread in a spiraling disk 27 meters in diameter (35 kg/m2). The usual medium is iron dust, which has a Curie point of 1043 K and is easily scavenged by magnetic beneficiation from regolith.
 
Electrostatic membrane radiator – This heat-rejection concept, also called a liquid-sheet radiator, encloses radiating liquid within a transparent envelope. It consists of a spinning membrane disk inflated by low gas pressure, with electrostatically-driven coolant circulating on its interior surfaces. The liquid coolant is only 300 μm thick and has an optical emissivity of 0.85 at a temperature of 1000 K. An electric field is used to lower the pressure under the film of coolant, so that leakage through a puncture in the membrane wall is avoided.  The membrane has a specific area of 4.3 kg/m2.
 
ETHER charged dust radiator –  To avoid the evaporation losses suffered by radiators that use liquid droplets in space, dust radiators use solid dust particles instead. If the particles are electrostatically charged, they are confined by the field lines established by a charged generator and its collector surfaces. If the spacecraft is charged opposite to the charge on the radiating particles, the dust executes an elliptical orbit.  The radiating particles must be charged to 10-14 coulombs to inhibit neutralization from the solar wind.  The dust radiates at 1200 K with a specific area of 71 kg/m2.

Flux-pinned superthermal radiator -Variable configuration radiators take advantage of the surprising physics of high-temperature flux-pinning superconductors. These materials resist being moved within magnetic fields, allowing stable formations of elements. No power or active feedback control if necessary. The radiating elements fly in a flux-pinned formation, not physically touching, but connected by superthermal ribbon. Superthermal compounds hypothetically conduct heat as effortlessly as superconducting materials conduct electricity. The radiating surfaces are graphite foams, which have both high emissivity (0.9) and a high thermal conductivity (1970 W/m°K) if the heat conducts in a direction parallel to the crystal layers. Operating at 928 K, the superthermal radiator has a specific area of 17 kg/m2 and 76 kWth/m2. Dr Mason Peck, 2005
 
Graphene crystal X-ray mirror radiator – An unrealized dream in fusion energy is a lightweight x-ray mirror able to reflect bremsstrahlung radiation back into the fusion core. One hypothetical scheme is a graphene on Ir(111) nano-film under active feedback. An x-ray standing wave (XSW) is created in the interface region of a graphene crystal using Bragg reflection. The XSW is periodic with Iridium (111) lattice planes spaced 0.3-nm below the graphene cell. The maxima of the XSW are shifted inward by half the lattice plane distance when scanning through the Bragg reflection according to the beam energy. Atoms exposed to the maximum of the standing wave field are strongly photo-excited. Monitoring the photoelectron signal with a pair of structured lasers allows for active feedback control. Although the mirror itself is not a radiator, the melting temperature of graphene is 4900K, the highest in nature, allowing high temperature radiator cooling. Furthermore, the x-rays reflected dramatically reduce the amount of waste heat overall. The mirror operates at a maximum specific energy of 2 MW/m2, and a specific mass of 320 kg/m2. Shu Nie et al, Sandia National Laboratories, 2011.

Hula-hoop radiator – By imparting heat to twin washer-shaped disks by direct conduction, the Hula-Hoop radiator avoids the diseconomies of scale that plague fluid radiators. Furthermore, they are robust against micrometeoroid strikes and hostile attack. Each hoop is 100m in diameter and is made of braided cermets coated with graphite, and lubricated in the heat exchanger with tungsten disulfide (WS2). Radiating at 1300 K, the hula-hoop has a specific area of 33 kg/m2.  This design is an Eklund original.
 
Li heatsink fountain radiator – In order to do away with radiator walls, the fountain design sprays coolant forward from an accelerating spacecraft, where it is cooled and recovered as it falls back into a collection trough.  The coolant is a liquid metal such as lithium, doped with nanoparticles of boron. This might produce a coolant with properties halfway between these materials, with the low density and high specific heat of lithium, and the high boiling point of boron. Doping with carbon may be necessary to improve the low emissivity of liquid metals. Lithium undergoes two phase changes as it is heated from 50 K to 1800 K.
 
Magnetocaloric refrigerator radiator – A magnetic refrigerator takes advantage of the magnetocaloric effect, the ability of some materials to heat when magnetized and cool when removed from the magnetic field.  Often used for cryogenic coolers, the external magnetic field is coupled with the magnetic sub-lattice heat source decreasing the total entropy in the solid and causing it to radiate heat.  Although the element Gadolinium is usually used, numerous new complex alloy crystals have exhibited a 2-3 magnitude larger magnetocaloric effect at temperatures in excess of 350 K (coined the Giant Magnetocaloric Effect) in fields between 1-15 Tesla.  This allows for radiation heat losses better than from a passive ideal black body (i.e. effective emissivities above 1).  Such alloys can be used as a conduit to pump heat from the rocket’s skin and ultra-thin foil radiators.  Specific area is 0.9 kg/m2 excluding power source at 450 K, pumping 2.3 × above an ideal black body.
Pecharsky V. K., Gschneidner K. A. Advanced Materials for Magnetic Cooling. 2007.
Jeppesen S. Magnetocaloric Materials. 2008.
 
Marangoni flow radiator – In zero-g, a surface tension gradient can create a heat pump with no moving parts, or drive micro-refining processes. This phenomena, called Marangoni flow, moves fluid from an area of high surface tension to one of low surface tension.  Bubbles operating at 1300 K have a specific area of 24 kg/m2. The Third Industrial Revolution, G. Harry Stine, New York, 1979.
 
Microtube array radiator – Nanofacturing techniques can fabricate large, parallel arrays of very small diameter tubing for high performance radiators. The radiating surface comprises a heavily-oxidized, metal alloy with a 100 nm film of corrosion resistant, refractory platinum alloy deposited on it. The working fluid is hydrogen, which has low pumping losses and the highest specific heat of all materials. This fluid at 0.1 to 1 MPa is circulated through the microtubes, and most of the radiation occurs from their walls, which are only 0.2 mm thick. This allows a specific area of 34 kg/m2, including hydrogen. The rejection temperature for titanium alloy tubes is from 200 K up to 1000 K, if a high temperature barrier to hydrogen diffusion is used. High speed leak detection capability and isolating valves under independent microprocessor control provide puncture survivability.  "The Microtube-Strip Heat Exchanger--Space Power Applns. for Ultra-High Conductance Gas-Gas Exchangers", F. David Doty, Gregory Hosford and Jonathan B. Spitzmesser, 1990.
 
Mo/Li heat pipe radiator - A heat pipe quickly transfers heat from one point to another. Inside the sealed pipe, at the hot interface a two-phase working fluid turns to vapor and the gas naturally flows and condenses on the cold interface. The liquid is moved by capillary action through a wick back to the hot interface to evaporate again and repeat the cycle. For high temperature applications, the working fluid is often lithium, the soft silver-white element that is the lightest known metal. Molybdenum heat pipes containing lithium can operate at the white-hot temperatures of 1700 K, and transfer heat energy at four times that of the surface of the sun. The specific area is 150 kg/m2. David Poston, Institute for Space and Nuclear Power Studies at the University of New Mexico, 2000. El Genk M., Tournier JM. Uses of Liquid-Metal and Water Heat Pipes in Space Reactor Power Systems. 2011.
 
Nuclear fuel spin polarizer radiator – The fusion of helium-3 and deuterium suffers from side reactions such as D-D, increasing neutron production. Spin-polarizing 3He-D and other aneutronic thermonuclear fuels not only suppresses the side reactions, it also increases the reactivity. Polarization is achieved by quantum tunneling across an MgO insulator into the single atom-thick sheet of carbon atoms arrayed in a honeycomb pattern known as graphene. Extremely strong and flexible, graphene exhibits good electrical conduction, heat resistance, and among the best spin-transport characteristics of any room temperature material known. After the quantum tunneling, the fuel is further treated with radiofrequency irradiations, symmetry species conversion catalysts, molecular species spatial arrangements, and anneal programs, before being stored and manipulated at liquid helium temperatures in solid, liquid and high density gaseous phases. Rapid depolarization of the plasma is prevented by providing a first-wall coating formed of a low-Z, non-metallic material having a depolarization rate greater than 1/sec. Roland Kawakami, Kawakami Laboratories, UC Riverside, 2012.
 
Pulsating heat pipe radiator –A newer concept in heat pipe technology, a pulsating heat pipe, is a meandering capillary closed loop system with no wick structure that operates as a non-equilibrium heat transfer device.  When working fluid is first added to the evacuated pipe, a system of naturally forming liquid plugs and vapor bubbles develops.  Thermal energy generates a pulsating motion, driven primarily by surface tension, oscillating between the plugs and bubbles resulting in a very high heat transfer coefficient for a given volume (a water-copper PHP has a transfer coefficient 10 times that of diamond).  PHP perform better in microgravity environments than on Earth. Varying diameter between parallel channels induces flow circulation and further increases transfer capacity. This technology has yet to be optimized and fully modeled. Here, a potassium working fluid embedded with diamond nano-particles (to increase thermal conductivity) flows through 0.7m diameter carbon-steel tubing.  Specific area is 35 kg/m2 and 212 kWth/m2.
 
Qu tube radiator – Heat pipes, particularly over long distances, are limited in the rate of heat transfer by flow rate and transition times between liquid and vapor states.  This limitation is circumvented in the Qu tube, first patented by Dr. YuZhi Qu in China in 1989. This hermetically sealed 0.9mm thick copper pipe contains three thin layers (8-12 μm) of a complex mixture of chemicals including over a dozen different metals from sodium and beryllium to aluminium and β-titanium and their various oxides.  Starting from the inner wall, there is an anti-corrosion layer, an active heat-conductive layer, and a black powder layer. The active layer works by accelerating molecular oscillations and friction associated with the third heat-generating layer. This tube is a solid state (i.e. no liquids) thermal superconductor which can operate over a wide temperature range without a gradient (over 10,000× that of silver and hundreds of times better than conventional heat pipes).  As the thermal superconductance is dependent on an even distribution of the black powder layer and a strong hermetic seal, the system is vulnerable to micrometeorite punctures.  Shorter piping and redundant systems must rely on lower conductance carbon-carbon heat brushes for interconnection. As a result, the Qu Tube is particularly sensitive to diseconomies of scale. Specific area is 7.2 kg/m2 at 630 – 650K, oscillating at 280 MHz. The patent claims it can operate up to 2000 K, but tests have only been conducted in the 300-400 K range. I elected for a reasonable 650 K which is the upper limit of oscillations quoted in the patent. Qu Y. United States Patent. Superconducting Heat Transfer Medium. 2005. AND Blackmon J. B., Entrekin S. F. Preliminary Results of an Experimental Investigation of the Qu Superconducting Heat Pipe. 2005. AND
 
Salt-cooled reflux tube radiator –In contrast to a heat pipe, that uses capillary action to return the working fluid, a reflux tube uses centrifugal acceleration. This design is more survivable than heat pipes, especially when overwrapped with a high-temperature carbon-carbon composite fabric.  Unlike metals, the strength of these composites increases up to temperatures of ~2300K. However, they degrade when subjected to high radiation levels.  The working fluid is molten fluoride salts, the only coolant (other than noble gases) compatible with carbon-based materials. Radiating at 1100 K, this radiator has a specific area of 75 kg/m2. Charles W. Forsberg, Oak Ridge National Laboratory, Proceedings of the Space Nuclear Conference 2005, San Diego, California, June 5-9, 2005
 
SS/NaK pumped loop radiator –A Rankine evaporation-condensation cycle can exchange heat using a liquid metal as a coolant, which is vaporized as it passes through a heat exchanger connected to the radiator. A liquid metal near the liquid/vapor transition is able to radiate heat at a nearly constant temperature. The usual medium is sodium or sodium-potassium, which has a saturation temperature of nearly 1200 K at 1.05 atm.  The heat pipes are oxidized aluminum or stainless steel tubes operating at up to 970 K with an emissivity of 0.9. A tube wall thickness of half a millimeter is determined by meteoroid-puncture considerations, and each pipe is an independent element so that a single puncture does not cause overall system failure.  Molecular beam cameras on long struts scan for meteoroid leaks, which are plugged with pop rivets installed by a tube crawler. Radiating at 970 K from both sides, this radiator has a specific area of 61 kg/m2, including fluid and heat exchanger. 
 
Steel/Pb-Bi pumped loop radiator – The working fluid of this design is a Lead-Bismuth Eutectic (LBE). While more corrosive than NaK (particularly above 970 K), it is more stable and able to tolerate greater temperature fluctuations.  Adding Bismuth to Lead lowers the melting point and keeps the fluid at a constant density as it is heated at the cost of increased corrosion. As an added utility, LBE is an excellent gamma shield (but transparent to particle radiation).  LBE has been extensively studied at temperatures below 870 K but next generation stainless steels are needed at temperatures exceeding 925 K where corrosion greatly limits operating life.   Radiating from both sides, this radiator has a specific area of 56 kg/m2, including LBE fluid, heat exchanger and ODS stainless steel tubing. A spiral or helical heat exchanger configuration allows for a compact, high efficiency design whereby by maximizing contact between the hot and cold surfaces. This geometry minimizes fouling, which plagues heat-pipes, both because of the low pressure drop and because a localized increased velocity forms on a fouled surface resulting in a “self-cleaning” mechanism.
 
Thermochemical heatsink radiator – While latent and sensible heat storage have been well established, long term energy storage in the form of thermochemical bonds has yet to reach commercial development.  Issues including catalysts and volumetric changes still need to be solved.  Here solid MgSO4∙7H2O sensibly absorbs heat until reaching 395 K whereupon the water dissociates (absorbing 1.1 MJ/Kg) and is boiled off and recycled. The solid MgSO4 can continue to absorb heat energy until reaching 1470 K where it dissociates to MgO and SO3 gas(absorbing 18 MJ/Kg).  The remaining magnesium oxide is allowed to absorb heat in its graphite lined tungsten container until reaching 3100 K.  Using this system, 40 tonnes of the thermochemical material can absorb a peak 240 MWth load for up to 150 minutes.  Abedin A. H., Rosen M. A. A critical review of thermochemical energy storage systems. 2011. And Barnes F. Levine J. Large Energy Storage Systems Handbook. 2011.
 
Ti/K heat pipe radiator – A Rankine evaporation-condensation cycle heat pipe uses metal vapor as the coolant, which is liquefied as it passes through a heat exchanger connected to the radiator. A liquid metal near the liquid/vapor transition is able to radiate heat at a nearly constant temperature. The pipe is made from SiC reinforced titanium or superalloy operating at up to 1100 K, and the working fluid is potassium. The pipe is covered with a lightweight thermally-conductive carbon foam, which protects the pipe from space debris and transfers heat to the radiating fins. The total specific area is 100 kg/m2.
 
Tin droplet radiator – Atomization increases the surface area with which a fluid can lose heat. A hot working fluid sprayed into space as fine streams of sub-millimeter drops readily loses heat by radiation.  The cooled droplets are recaptured and recycled back into the heat exchanger.  If tin is used as a working fluid, the kilos per power radiated is minimized, allowing heat rejection comparable to the game value of 120 MWth per therm. The low emissivity of liquid tin (0.043) is increased by mixing in carbon black, which distributes itself on the surface of the droplet. Evaporation losses are avoided by enclosing the radiator in a 1 µm plastic film, which is able to transmit radiation in the 2 to 20 µm (IR) range.  Such a film would continue to perform its function even if repeatedly punctured by micrometeoroids.  The illustration shows a triangular liquid droplet geometry.  The collector, located at the convergence point of the droplet sheet, employs centrifugal force to capture the droplets. Droplets are 50μm diameter, spaced 250 μm apart with an emissivity of 0.2 (after carbon is added). The sheet is 150m long and 62m wide at the generator end and 2.6m wide at the collector end giving an effective radiating area of 411m2.   The droplets exit at 1030 K and transit for 9 seconds resulting in an average effective temperature of 400 K.  The specific area is 9.9 kg/m2, and specific mass of 0.034 kW/kg (doubled the value used in the sources due to scaling).  The low emissivity limits the effectiveness of the radiator in regions of high solar flux.
National Academy of Sciences. Proceedings of a Symposium Advanced Compact Reactor Systems. 1982. Pages 384-386.  And Chung W. Droplet Heat Radiator. http://www.5596.org/cgi-bin/dropletradiator.php. Liquid Droplet Radiator Development Status, K. Alan White, Lewis Research Center, Cleveland, Ohio, 1987 
 
X-ray hard dielectric mirror radiator – Fusion for propulsion emphasizes the efficient rejection of fusion products as propellant rather than maximizing energy production or the “Q”. The aneutronic fuels preferred for propulsion unfortunately emit significant bremsstrahlung radiation (x-rays produced by “braked” electrons). For example, a magnetic-confined reactor burning 3He-D fuel emits more than 16% of its energy as 100-keV x-rays. For H-B fuel, it is 63% at 300-keV. Such x-rays are readily absorbed and converted into heat in less than a millimeter of steel, increasing the ship’s heat load. A dielectric mirror reactor avoids this by running the plasma hotter than normal and confining it by fields rather than solid walls, allowing the radiation to pass harmlessly into space. The high-temperature superconducting magnets of the reactor and nozzle are X-ray hardened by a dielectric mirrors composed of layers of tellurium, tungsten carbide, and polymers. The reflectivity of multilayer thin films can be extremely high if the films are constructed of pairs of quarter-wave thick layers of low absorbance dielectrics. Prototypes made at MIT reflect 99% of light falling on them, at a maximum specific energy of 3 MW/m2 and a specific mass of 1600 kg/m2. The remaining 1% of the heat is removed by Li coolant and rejected by radiators or open-cycle cooling. A Carnot-limited portion is recovered in a heat engine (e.g. hot Li turbine) to run the igniters. Ken Rachocki, LANL, 2006.
Z8. FREIGHTERS
Antiproton sail and harvester – Antiproton sail and harvester freighter – Antiprotons formed from the collision of cosmic rays and the solar wind accumulate in the planetary radiation belts. A positively-charged harvester sail scoops and directs them into a series of nested spherical decelerators, where they finally are trapped or reflected to the sail for propulsion. The antiprotons explode on contact with the sail material, a mixture of graphite, carbon-carbon fiber, and tiny amounts of uranium. Depositing the antiprotons deep into the uranium (to improve the number of fissions) increases the thrust at the expense of specific impulse. Rim-superconductors reduce translational losses of the reactants. The high specific energy of the two fission products (approx Pd-111 at 1-MeV/amu) unfortunately means half the energy must be rejected as waste heat. Unlike a fission-fragment sail, this design can tack, stop, vary the acceleration and specific impulse, and does not need a gossamer sail. The sail parameters are 2 MW/kg and 1 kg/m2. A 75m diameter sail generates 9 GWth at a thrust of 270 newtons and an exit velocity of 10%c. The starship version rides on an antimatter mass beam. Stephen Howe, 2013.
 
Archimedes Palmer-lens freighter –An advanced version of the solar-heated rocket replaces the inflatable lens with a swarm of reflecting photon sails held in formation by two Palmer lasers (see Palmer LSP aerosol lens generator). These sails are spun and deployed from regolith by photon sail-propelled Neumann devices. The swarm has a diameter of 275m. In place of rhenium foam, the solar absorber uses space-built reticulate vitreous carbon foam, allowing temperatures up to 3778 K and hydrogen specific impulses up to 1.2 ks. The carbon foam structure is protected against carbon/hydrogen reactions by means of a refractory carbide coating applied by zero-g chemical vapor infiltration (CVI). Solar power = 69 MW. 10 kN thrust, efficiency = 89%. A.J. Palmer, Hughes Research Laboratories, 1980
 
D nanotube dirt launcher freighter – Carbon nanotubes have a tensile strength of 8300 kg/mm2 (81 kN/mm2) and a density of 1800 kg/m3. A cable car system shaped like the letter “D” can operate of 6.8 km/sec and store 23 MW of energy for every kilogram of cable. The design shown has a 2 km cable, 2mm in diameter and 12 kg in mass. Its radius is 402 meters. It is driven directly by 500 MWth nuclear turbines, without using electricity. Alexander Bolonkin
Fission fragment sail –Imagine radioactive dust embedded in an absorber layer film. The impulses of the fission fragments impart a feeble thrust to this “sail”. The thrust is enhanced with an axial magnetic field too weak to affect the motions of the dust but strong enough to direct the ionized fragments into a beam at 22% efficiency. Unlike a photon sail, which can be steered by tilting in the sunlight, a fission fragment sail receives its energy equally over the entire sail surface, and thus can't be steered. The illustrated design thrusts and steers using A-LIFT (Aneutronic Laser Induced Fusion Thruster) steering patches. The thrust is initiated by a picosecond laser pulse which has been chirp-pulse amplified and directed to the first layer of a steering patch. Pulsed at 75 MHz, the laser beam delivers 20,000 TW per square millimeter at wavelengths between 1 and 10 μm. This explodes the first layer of the patch, a 5 μm thick sheet of metal foil. The teravolt per meter electric field ejects protons, which initiate H-B fusion in the second layer, a film of CH2-boron-11 composite. The fusion products are a hundred thousand 8.7 MeV alpha particles, directed by magnets for steering thrust. A 1 km diameter sail has an initial mass of 60 tonnes, but it loses a tonne per year at a 1.9 kw/m2 fission rate. With half a million second of specific impulse but only 140 newtons of thrust, this sail does not perform well in the moderate g-fields around planets. If starting in Earth-Luna L3, it takes a month for the spiral-out from Earth, only 4 years on the long leg to Jupiter, and over a year for a spiral-in to Callisto. ”Observation of neutronless fusion reactions in picosecond laser plasma”, V.P.Krainov, Lisitsa, Roussetski, Ignatyev, Andrainov, 2005.
 
Fission GCR freighter –The open-cycle gas core reactor (GCR) contains a critical fissile core in the form of a gaseous plasma. This radiatively heats seeded hydrogen propellant at triple the specific impulse of solid core versions. A 1.5 GWth reactor attains 74 kN thrust at a specific impulse of 3 ks. The 150 MWth of waste heat can be removed by doubling the mass flow rate of open-cycle cooling (to 5 kg/sec), or by radiators. The fuel is 242mAm, formed from Amercium-141, which not only has the highest known thermal fission cross section, but also has a low capture cross section, high number of neutrons per thermal fission, and long half life. This permits smaller engines than with uranium fuel, with a 1.5m diameter core, moderator thickness of 0.5m, 500 atm pressure, 65,000K core temperature, fuel-to-cavity volume ratio of 25%. To prevent fuel loss, the americium fuel core is magnetically-confined with an slightly asymmetric magnetic mirror. Gas core reactors are difficult to start, so a small pulse of antiprotons are injected to produce the 1022 neutrons needed for startup. “Nuclear Thermal Propulsion,” R. G. Ragsdale, NASA/Lewis Research Center, 1990. Kammash and Jan,1992.
 
Fission-heated steam freighter – Fission-heated steam freighter – A nuclear-heated steam rocket engine (NSR) uses a nuclear reactor to convert water propellant to superheated steam. A Calhoun pump raises the pressure of the water to 23 MPa, which acts as the moderator/reflector. Although it requires a massive pump (power ≈ pressure X area X velocity), operation above the critical pressure avoids a propellant phase change and allows operation at a higher power density and Isp. A de Laval nozzle attached directly to the reactor converts the steam expansion pressure into thrust. A mixed mean outlet temperature of 1100 K obtains an Isp of 198 seconds. The reactor is a 10 tonne particle bed yielding 1 GWth at 3 MW/liter. The water tank is Kevlar composite with 0.044mm thick walls. Alternatively, the propellant can be carried as water ice. At the maximum thrust of 380 kN, the water flow is 170 kg/sec. The energy required to melt ice into H2O (.48 MJ/kg) and H2O into steam (2.2 MJ/kg) means that almost half the thermal power is lost to water phase change. “Nuclear-heated steam rocket using Lunar Ice”. Anthony Zuppero et. al, 1997.
 
HIIPER beam-rider freighter –The Helicon Injected Inertial Plasma Electrostatic Rocket (HIIPER) employs one of the highest density plasma sources (Helicon) for plasma production and one of the most erosion-resistant accelerators (Inertial Electrostatic Confinement) for plasma acceleration. This decoupling of the ionization from the acceleration stage allows HIIPER (like VASIMR) to exhibit improved variable specific impulses and thrust to power ratios. The propellant can be a variety of gases, e.g. nitrogen, argon, or xenon. Electricity is provided by a 500 MW remote laser beam, intercepted by a gossamer laser sail of 312m diameter. Operating at 600K with an emissivity of 0.06 and an absorbance of 0.135, 250 MWe is generated. Assuming 0.100 g/m2, the sail mass is 7.5 tonnes. The pencil-thin multi-kV plasma exhaust generates just 530 newtons of thrust, about the weight of 2 airport suitcases. But the exit velocity, unsurpassed for an electric engine, is 0.45% the speed of light. Akshata Krishnamurthy, HIIPER Space Propulsion Lab, 2012.
 
Inflatable solar-heated freighter –A rocket running on solar power needs a large lightweight collecting surface with a sun-tracking system. An inflated plastic balloon of parabolic architecture focusing laser or sunlight into a “blackbody” cavity is the preferred structure. A secondary concentrator made of single-crystal zirconia, sapphire, or yttrium-aluminum-garnet (YAG) acts as a window to the chamber. The long cylindrical construction of the cavity allows most of the radiation that enters the engine to be used for heating the working fluid, i.e. the propellant. The latter enters the engine, flows through a foam-filled annulus around the receiver cavity, is heated to a high temperature, and expands through the nozzle. The high temperature metal rhenium is used throughout, which limits the temperature to 2800K and the specific impulse to about 1 ks with hydrogen or 190 seconds with water. The vessel of rhenium tubes is held by a carbon shell which is further encased in a reradiation shield to prevent heat loss. NASA Ames, 2012.
 
KESTS hoop dirt launcher freighter –Kinetic Energy Supported Transportation Structures (KESTS) are a type of electric rocket able to propel a small “rubble-pile” asteroid by anchoring two quasi-elliptical accelerator-hoops on its equator. The hoops are supported by internally-generated centrifugal forces instead of by strength of materials (unlike tether space elevators). They are analogous to the stator of a synchronous electric motor, supported by electrodynamic coupling to mass stream armatures whirling at 10 km/sec. Two counter-rotating mass streams are used to avoid forces due to angular momentum. Humans and cargo travel in vehicles magnetically linked to one of the mass streams, arriving at an upper space station in synchronous orbit with the asteroid’s rotation. A chain of dirt buckets at much higher acceleration releases regolith at the hoop apex at 10 km/sec. Two symmetrical hoops on opposite sides of the asteroid provide 56 kN total thrust at 500 MWe. .JED Cline, 2013.
 
Magnetic mirror beam-rider freighter –This beam-rider employs a magnetic mirror with two coils to reflect an incoming mass beam. The larger stern coil is 100m across and shielded weighs 120 tonnes not including radiators. The mass beam originates from the inner solar system, perhaps from a solar-powered accelerator in one of the Sol-Venus Lagrange points. The accelerator can be linear (lineac) or circular (cyclotron). It aims the mass beam using flywheels with superconducting electric motors. The beam is focused by a string of field lenses and lasers. The lasers are tuned just below a significant absorption frequency of the beam particles, so that those that stay on the beam won’t absorb a photon, while the Doppler shift of those that drift toward a laser brings them into resonance with the laser, pushing them back into the line. The beam, composed of atoms, molecules, clusters, or pellets, must be electrically neutral while flying through space (to prevent spreading), yet must be charged to be accelerated by the accelerator and decelerated by the spacecraft. Stern lasers in the spacecraft ionize the stream so that it can be decelerated and reflected by the mirror. Unreflected portions of the beam pass through the toroidal geometry, forming a bow "guard plume" that helps shield the high-speed vehicle against space dust. A cell of water guards the spacecraft against wayward beam particles. A 1.2 GW mass beam, moving at 1700 km/sec (0.56%c) and reflected at 81% efficiency, provides 1.2 kN thrust. G. David Nordley, gdnordley@aol.com
 
Poodle freighter – A product of late 1960s nuclear boldness and naivety, this thruster heats its working fluid through the natural decay of radioactive isotopes.  A very simple design, with no moving parts, the reactor uses the expensive and short lived 210Po to generate heat and thrust which limits the length of the mission.  However, since decay cannot be “turned off,” heat must either be used as thrust or radiated. On a personal note, it is really fun reading declassified 1960s era documents!
Nezgoda E. Radioisotope Propulsion Technology Program (Poodle) – Final Report. 1967. And Whiton J. Status Report – Radioisotopic propulsion systems. 1965.
 
Rotary dirt launcher freighter –The rotary pellet launcher (RPL) is a tapered tube rapidly rotating so as to accelerate small (10g) pellets of compressed regolith. Near the tip the imposed acceleration is some 100,000 gees. The pellet presses on the side of the tube with a force of approximately 8.9 kN, and exits at 4 km/sec. The rate is 230 pellets/sec. The material is Kevlar, at 60% of yield strength, lined with TiNi alloy for wear resistance. The liner is subject to considerable abrasion and wear and is replaceable. A double turret has a mass 17 tonnes. Two such turrets are needed, counter-rotating, for the spacecraft or asteroid to maintain zero net angular momentum. “Space Settlements: A Design Study”, NASA Ames, 1975.
 
Z-pinch 3He-D target fusion freighter –A Target Fusion engine has two stages: a plasma injector that produces extremely hot plasma and moves it to the second stage, a liner implosion system that magnetically compresses the hot plasma to fusion conditions. Compression is via wire-array Z-pinch and the Ulam ablation implosion effect (i.e. the lorentz force from high current in a lithium metal liner supplied by movable helical-flux generators). The imploded liner becomes part of the propellant. Target Fusion therefore combines the heating at low density approach of magnetically-confined fusion with the rapid heating of inertial-confined fusion. Because the plasma has to stay in the magnetic confinement only for fractions of a second, the requirements for the confinement are correspondingly low and less costly. Because the target is already a superheated plasma, the requirements for the pressure levels are not as stringent as in inertial fusion. G.A. Wurden http://www.lanl.gov/physics. 2013.
 
Z-pinch D-T/6Li fusion freighter –A Z-pinch fusion engine runs very large currents (Megampere scale) through plasma over short timescales (10-6 sec). The magnetic field resulting from the large current then compresses the plasma to fusion conditions. A Z-pinch rocket engine running on both D-T and n-6Li uses an annular nozzle with D-T fuel injected through the innermost nozzle and 6Li introduced through a cylindrical outer nozzle like a “shower curtain.” The 6Li propellant injection is focused in a conical manner, so that the D-T fuel and 6Li mixture meet at a specific point that acts as a cathode. 6Li serves as both a current return path to complete the circuit and as a neutron absorber. The n-6Li reaction produces additional Tritium fuel and energetic byproducts that boost the energy output. At the optimal mixture of D-T fuel and 6Li propellant, the Z-Pinch engine produces 38 kN thrust at an Isp of 19.4 ks. Neutrons and gammas that escape the 6Li liner are captured by FLiBe coolant and the waste heat is passed on to He and NaK radiators operating at 1250K. During each 1 GJ fusion pulse, the current induced in the thrust coils is used to recharge capacitor banks for the next shot, with a Q = 3. Because a large amount of energy (333 MJ) must be discharged to the D-T bolus within brief time (100 ns), Marx capacitors are used, at a frequency of 10 Hz. " Fusion Propulsion Z-Pinch Engine Concept, J. Miernik et al, Marshall Space Flight Center, 2012
 
Z9. GW THRUSTERS
Amat-catalyzed fusion-fission GW thruster –This spacecraft burns fuel pellets of Deuterium, Tritium, and Uranium-238 (nine parts D-T for every one part 238U). These are injected into the reaction chamber, compressed with ion beams, then irradiated with a 2 nsec burst of antiprotons. Although this antimatter does not “catalyze” the reaction, it does initiate uranium fission which in turn generates D-T fusion. The high energy radiation is thermalized (to 1 keV) by a 200 g WLS lead coating around the target. Even so, only a third of the reaction energy and target fragments are intercepted by the thrust shell (an 8m diameter ablative nozzle made of silicon carbide). About 20% of the energy is lost as high-energy neutrons, and a 1.2m neutron shield made of LiH is needed to protect the fuel rings. A heat engine in this shield powers the 10 MW initiator. Each 40 tonne tank of SiC propellant ejected uses 7 ng of antiprotons. At 1 Hz and an ejected mass per shot of approx 2 kg, 302 GWth is generated. At an overall efficiency of 6%, thrust is 275 kN at an exit velocity of 0.04%c. Lewis, Meyer, Smith, and Howe of Penn State University, 2000.
 
Amat-initiated H-B magnetic inertial TW thruster –Conventional inertial confinement fusion compresses a fuel pellet to many times solid state densities while simultaneously delivering energy to the core to initiate the burn. Magnetic inertial fusion circumvents this by creating fusion plasma by wall ablation inside a pellet. The ablation is initiated by a high energy beam entering through a hole in the pellet wall. The thermoelectric effect of the impact generates a strong magnetic field (12,500 T) that thermally insulates the hot dense plasma from the metal wall of the pellet. This ignites the core and allows it to burn a long time, combining the advantages of magnetic and inertial confinement. The design uses a 9 nsec pulse of a hundred billion antiprotons as the high-energy beam and hydrogen and boron in a ratio of 5 to 1 as the fuel. The H-11B fusion fuel has the advantage of producing only charged particles in the form of three alphas, and no neutrons. Each lead-shelled pellet is 50 mm in diameter and masses half a kilogram. The igniter particle cascade has 140 GJ of kinetic energy plus 30 joules of annihilation energy. At a Q = fusion energy/igniter energy = 2, 280 GJ of fusion energy is generated per shot. One shot every 2 seconds generates 83 kN thrust and 0.4%c exit velocity. “Antimatter Driven P-B11 Fusion Propulsion System”, Kammash, Martin, Godfroy, 2003.
 
Colliding FRC 3He-D fusion TW thruster –A Field-Reversed Configuration (FRC) is an ellipsoid plasma cell with an azimuthal current reversing the direction of the externally applied magnetic field. The resultant field provides for toroidal plasma confinement without requiring a toroidal vacuum vessel or coil set. In a colliding plasma mirror, two FRCs generated by theta pinches at opposite ends of a long magnetic mirror, are accelerated magnetically to a million miles per hour and slammed into each other, The fusion products escape out both ends of the machine, one end is MHD-tapped for electricity, and the other expanded in a magnetic nozzle for thrust. Fuel can be 3He-D or H-B. A 220 GWth reactor generates 5 kN thrust at Q = 2 and an exit velocity of 11%c. Helion Energy, 2010.
 
Crossfire H-B focus fusion TW thruster –The six arms of the Crossfire Reactor fire positive 600 keV ions into a negatively-charged central magnetic cusp. The ions, confined radially by magnetic fields and longitudinally by electric fields, are reflected back and forth by an electrostatic lens at the distal end of each ion gun. Pulses in the magnet currents oscillate the magnetic flux, transferring energy radially to the plasma. This pinch effect increases the fusion rate. The magnet walls are coated with a bremsstrahlung mirror to reflect the x-rays back into the plasma. The preferred fuel is hydrogen-boron. This reaction produces only charged reactants, which overcome the electric field and exit longitudinally, to be directed by magnetic and electric fields for thrust. At a power of 28 GWth, the thrust is 2.5 kN at an exit velocity of 5%c. Moacir Ferreira Jr.
 
Daedalus 3He-D inertial fusion TW thruster –Project Daedalus is the benchmark design for an unmanned interstellar probe. The card statistics are for the second stage only. (The full 2-stage starship has a dry mass of 100 mass points, and needs 1350 tanks of 3He-D fusion fuel pellets to reach Barnard’s Star.) Injected at 250/sec, these pellets are ignited by electron beams with driver energies of 2.7 GJ and 400 MJ for the first and second stages respectively. (A recent study refutes that an e-beam can ignite fusion pellets due to magnetic erosion and charge repulsion.) The high gain of the system (Q = 35), allows MHD coils in the nozzle to generate the energy needed for the next cycle. The engine bell, made of molybdenum TZM alloy and glowing orange hot (1500K), reject the waste heat. The computed burn-up fraction for the fusion fuels is 0.175 and 0.133 for the first and second stages, producing exhaust velocities of 3.5%c and 3%c, and thrusts of 7540 kN and 660 kN respectively. “Project Daedalus – The Final Report on the BIS Starship Study”, A. Bond et al., 1978.
 
Dense plasma H-B focus fusion GW thruster - This vehicle employs clean “aneutronic” dense plasma focus (DPF) fusion power using hydrogen and boron-11 fuel. The total bremsstrahlung power, found by multiplying by the pinch volume (~10-6 m3) and the repetition rate (10 Hz) is 1 GWth, at a particle number density (electron and ions) of 2 x 1022 /cm3 and an electron temperature of 1.3 MeV. At Q = 3, ignition is provided by a 33 MJ capacitor bank with a specific energy of 15.0 MJ/tonne and a specific volume of 5.0 MJ/m3. A second capacitor bank is included as a back-up. Engine parameters are anode radius = 14cm, cathode radius = 38cm, thrust = 1.7 kN , Isp = 64 ks (0.2%c), overall efficiency = 81%. Open-cycle cooling of the reaction chamber and magnetic nozzle with 2.75 kg/sec of liquid hydrogen boosts the thrust to 55 kN but lowers the Isp to 2 ks. Sean D. Knecht et al.
 
Dusty plasma TW thruster––A dusty plasma reactor suspends slightly-critical dust grains in an electric field, and uses the high-speed fission fragments directly for thrust. Upper and lower paraboloid LiH moderators reflect enough neutrons to keep the reacting dust critical. Twin cooled carbon-carbon heat shields reflect the dust infrared energy away from the moderators. For uranium dust grains, 81% of the energy released is the kinetic energy of the fission fragments, with the remaining 19% released in the form of beta, gamma, and neutrons, for a total of 207 MeV per fission. The escaping fission fragments are cooled by collision with the Li propellant, for a mean exit velocity of 3.4%c. Superconducting field coils and quadrupole current loops act as a magnetic mirror to direct the fragments and propellant for thrust. A 14 GWth dusty plasma generates 1.1 kN thrust with minimal open-cycle cooling, or 550 kN with enough Li coolant to absorb most of the neutrons. The lithium can be supplied by a Li mass beam. The nanometer-sized dust is radiatively cooled, with a thermal efficiency of 40%. NASA NAIC Spring Symposium, 2012
 
Levitated dipole 6Li-H fusion GW thruster –Most magnetic-confinement fusion vehicles use either irrational flux surfaces (i.e. tokamak, stellarator) or open magnetic field lines (i.e. magnetic mirror). Closed field line systems such as dipoles have received much less attention; however, they possess several uniquely useful properties including the confinement of high beta plasmas with low turbulent transport, and high-energy low-particle confinement in a steady state configuration created by a small number of non-interlocking coils. Levitating the dipole magnet offers steady state operation with high stability and efficient ash removal. Moreover it eliminates end losses, supporting the ignition of advanced fusion fuels with low fusion cross sections such as 6Li -H or helium-catalyzed D-D. The latter cycle suppresses the production of energetic (14.1 MeV) neutrons by removing the thermal tritium ash and replacing it with the helium-3 decay product. The chief drawback of the dipole approach is the need for a levitated superconducting ring internal to the plasma. This “floating ring” unfortunately intercepts 24% of the fusion photons and neutrons, and yet must be kept cool enough for superconduction. The ring is shielded by a millimeter thick layer of tungsten followed by a C-C fiber composite shield that constitutes a third of the ring mass. The outer layer, radiating 1 MW/m2 at 2700K, returns 400 MW back to the plasma. The C-C layer attenuates 90% of the neutron flux (60 MW for a D-D reaction). An internal shell, thermally isolated from the tungsten, provides a temperature difference that drives an internal 10 MW refrigerator to keep the internal superconductor windings cold. These windings, operating at a coil current density of 330 M A/m2, generate a 30 T peak field. The beta is 3.1 at a confinement time of 5 sec. For a 1.2 GWth fusion reactor, thrust = 1.4 kN, and exit velocity = 0.37%c, assuming an efficiency of 63%. “Helium Catalyzed D-D Fusion in a Levitated Dipole in a Z-Pinch”, J. Kesner, D.T. Garnier, A. Hansen, M. Mauel, L. Bromberg, 2003.
 
Mini-mag orion Z-pinch fission GW thruster –The Mini-MagOrion design adds two important aspects to the family of Orion concepts: first, the use of Z-Pinch magnetic compression of the fissile targets, allowing much smaller explosions (280 GJ yield vs. 20,000 GJ), and second, replacing the pusher plate with a magnetic nozzle. Only a small fraction of Cm-245 fission fuel is actually carried with the spacecraft, the remainder of the propellant (macro-particles of Cm-245 with a D-T core) is beamed to the spacecraft. The hybrid fission-fusion system has a Q of 3889 at 24% efficiency. Use of 280 GJ bombs at 1 Hz yields 1130 kN thrust with an Isp of 12 ks. "Use of Mini-Mag Orion and superconducting coils for near term interstellar transportation", Roger Lenarda, Dana Andrews, 2007.
 
Salt-water Zubrin GW thruster – The illustration shows the vision of Robert Zubrin: a rocket riding on a continuous controlled nuclear explosion just aft of the nozzle/reaction chamber.  The propellant is water, containing dissolved salts of fissile uranium or plutonium. These fuel-salts are stored in a tank made from capillary tubes of boron carbonate, a strong structural material that strongly absorbs thermal neutrons, preventing the fission chain reaction that would otherwise occur. To start the engine, the salt-water is pumped from the fuel tank into an absorber-free cylindrical nozzle.  The salt-water velocity is adjusted as it exits the tank so that the thermal neutron flux peaks sharply, outside the rocket.  At critical mass (around 50 kg of salt water), the continuous nuclear explosion produces 427 GWth, obtaining a thrust of 8600 kN and a specific impulse of 8 ksec at a thermal efficiency of 99.8% (with open-cycle cooling).  Overall efficiency is 80%. "Nuclear Salt Water Rockets: High Thrust at 10,000 sec ISP," Robert Zubrin, J. British Interplanetary Soc. 44, 1991.
 
Solem Medusa tugged Orion TW thruster –Johndale Solem’s “Medusa” is an Orion-type rocket driven by thermonuclear detonation waves behind a lightweight spinnaker canopy. The design uses 15 kiloton yield deuterium bombs with a fireball radius of 200m. A millimeter-thick spinnaker made of reinforced-Kevlar sized to this fireball is 25 tonnes. The traditional design has been augmented with a 400m magnetic mirror to ensure the fuel is completely vaporized before it hits the spinnaker. A second enhancement is proton beam fuel ignition, eliminating the need for fission materials and reducing the density of the expelled reactants (thus improving specific impulse). The 100 MJ beam is formed when a plasma bridge from a hohlraum target contacts a levitated superconducting ring charged to 1 GeV. Gigavolt multimegamperes of current strikes the end of a compressed fuel rod of deuterium encased in the hohlraum. If the current needed for ignition is below the Alfvén limit for ions, this beam is “stiff” and well in excess of the critical current to entrap the D-D fusion reaction products, a condition for detonation. The large currents generated are directed by magnetic field coils to the stern of the spacecraft, where thermionic emitters channel the electrons into space, using a process called inductive charging. This charges the ring for the next shot. A 12 kg hohlraum bomb (containing solid hydrogen propellant) is exploded every 10 seconds for maximum thrust (1700 kN), using a servo winch to keep the thrust constant. The collimation factor C0 = 0.14 with an exit velocity of 0.47%c. “Deuterium Microbomb Rocket Propulsion”, F. Winterberg, 2008.
 
Spheromak 3He-D magnetic fusion GW thruster –The Spheromak toroidal magnetic confinement is open geometry, allowing direct thrust and energy conversion of fusion products (unlike a Tokamak). The central part of the plasma focus, sustained by inductive helicity injection, cyclically produces pre-fusion plasma. A laminar dynamo is used to suppress turbulence. The residual magnetic field of the plasma focus discharge further compresses the Spheromak plasma to ignition conditions and pushes it out of the reactor into the magnetic nozzle. For a 3He-D or D-D based thruster, the low temperature plasma flowing in the region surrounding the reacting core (the so-called scrape-off region) has temperatures in the range of 100eV, corresponding to specific impulse values of 50 ks (0.16%c). Open-cycle cooling varies the thrust between 14 and 56 kN. “Studies of Conceptual Spheromak Fusion Reactors”, M. Katsurai and M. Yamada, 1982.
 
VISTA D-T inertial fusion GW thruster –The deuterium-tritium (D-T) reaction is messy. One-half its energy is high-energy neutrons, one-fourth is x-rays, and one-fourth is charged plasma debris. The conical shape of the VISTA spacecraft allows it to avoid most of the neutrons and x-rays, while redirecting the charged component with a 12 Tesla “warm” magnet. Consequently, VISTA uses only 9% of the fusion output for propulsion, and an equal amount for power generation. The gain (assuming advanced fast igniter lasers) is 600. Throttling is accomplished by varying the target firing rate between 0 and 30 Hz. Variable exhaust velocity is also available by varying the amount of hydrogen propellant enclosed in each hohlraum target pellet. Both the inside and outside of the cone provides radiator area rejecting 760 MWth. The magnet shields at 140 tons constitute most of the engine mass. “VISTA - A Vehicle for Interplanetary Space Transport Applications Powered By Inertial Confinement Fusion,” C.D. Orth, G. Klein, J. Sercel, N. Hoffman, K. Murray, F. Chang-Diaz, 2003.
 
Zubrin-GDM TW thruster –A hybrid fusion-fission rocket uses D-T fusion as a neutron source to initiate fission in a blanket of uranium salt water surrounding the long narrow fusion core.  The fusion geometry is a gas dynamic mirror (GDM), which generates electricity on one end and thrust on the other.  The Q for fusion is 2.9, using neutral beams.  The Q for Zubrin-style continuous fission is about 10, which greatly boosts the thrust. The thermal efficiency is 76%.

Z10. BERNALS
Unpromoted Bernal –The unpromoted Bernal is a dumbbell-shaped colony.  The two spheres are 67m in diameter, separated by a 334m connection.  It rotates at 1.9 rpm for 0.95 gees of artificial gravity. Assuming ½ atmospheric pressure, the structural mass is 400 tonnes.  Normally staffed by a crew of 100, it can hold up to 2000 souls maximum.  Upon reaching its destination, it is filled with 200 tonnes of air, and surrounded by a 140,000 tonnes of dirt and water shielding honeycombed with nanofibers, all made from local ISRU materials.
 
Reasons for locating a colony in a lagrange point rather than dirtside:
1.     Rawstuffs and water can  be electromagnetically catapulted from the surface for processing and assembly in an environment tailored for humans.
2.     Low thrust freighter traffic possible between the lagrange and back home in LEO.
3.     Humans can control dirtside robonauts without a troublesome time lag.
4.     Production of the final product can be performed at anything from zero to 0.7 gee, allowing unique materials and processes not possible dirtside or back at Earth.
 
Promoted Bernal –The promoted design metamorphoses from the dumbbell into a 250m Bernal Sphere with a maximum capacity of 10,000 souls. The structure is 18,120 tonnes, with 10,000 tonnes of atmosphere. The shielding is 92,800 tonnes. It rotates at 3 rpm for 0.7 gees. Bernal design uses mirrors to reflect sunlight used for growing crops. For nuclear-powered Bernals, sulfur lamps replace the mirrors. Besides sun power, need hundreds of megawatts for their faction-specific productive output: making antimatter, producing PV cells and beaming energy, electromagnetic shielding for the cycler, or electromagnetic boosting to orbit.  Promoted Bernals additionally need 50 MWe to power their dirt engines/product launcher.  The Business Bernal, a pharming facility, is exceptional in that it produces less electricity and uses a nuclear NERVA engine instead of an electric dirt engine.  
Z11. FUTURES written by Erich Schneider
Grand Scale Astronomy

Two of the futures are scientific endeavors on a grand scale. Planet Hunt (Fission GCR freighter fleet) involves using the "EM sunlens" to look for habitable planets around other stars. The sunlens is a result of changes in the structure of spacetime caused by matter as outlined by Einstein's theory of general relativity. Just like a planet or a star will bend the trajectory of a spacecraft passing close to it, it will also bend the trajectory of a beam of light passing close to it. This means that the Sun can act like the lens of a camera, focusing light from a distant point behind it at a different point. The future represents setting up the infrastructure for a telescope 550 AU from the sun to closely examine the planets around a nearby star, to see if they are capable of supporting life, and possibly to make maps of them.

The SETI future (experimental future for the Blue Goo Symbionts colonist) represents using "very long baseline interferometry" to listen for faint radio signals from other civilizations. This is a technique currently used by radiotelescope arrays on Earth, where multiple telescopes separated by a long distance emulate a telescope with a much higher resolution. In the case of the SETI future, the telescope array spans the orbit of Jupiter, with instruments spread throughout the leading and trailing Trojan asteroids.

Mega-Engineering

Ten of the futures represent titanic engineering projects.

Antimatter Creation (Antiproton Sail and Harvester freighter fleet) represents setting up an antimatter factory on a remote asteroid. Failure of the Epic Hazard represents destruction of the asteroid due to a containment failure.

The Beanstalk builders (Semi-circle nanotube dirt launcher freighter fleet) complete structures on a megascale to lower the cost of getting out of the gravity wells of the solar system - such as the Earth and Phobos space elevators (or simply an aerostat reached by a zeppelin assisted ascent). One of the most viable locations for a megastructure is the Pluto-Charon system where a material merely as strong as Kevlar (but resistant to the extreme cold) should be sufficient to build an ‘ice bridge’ between the two bodies. http://space.stackexchange.com/questions/5193/space-elevator-between-doubly-tidally-locked-bodies

Climate Mirror Terraform (Archimedes Palmer-Lens freighter fleet) involves setting up gigantic mirrors above a body with an atmosphere to heat up that atmosphere, making it more liveable and stimulating useful chemical reactions. (When applied to Venus, it represents a structure like an enormous windowshade between Venus and the Sun, with the intent of cooling the planet down as part of a larger terraforming project.)

Mini-Black Hole creation (Amat-Initiated H-B Magnetic Inertial TW thruster) should be self-explanatory. Like Antimatter Creation, this would involve creating a particle accelerator using energies far beyond what we have today. Failure of the epic hazard means the black hole has eaten the factory site!

Protium Fusion (Crossfire H-B Focus Fusion TW thruster) represents an attempt to achieve nuclear fusion the same way stars do it, fusing the protons from ordinary hydrogen to form ordinary helium (with the emission of neutrinos changing protons into neutrons). This requires temperatures much greater that what we can presently achieve, but does not require special isotopes like deuterium or tritium. Again, epic hazard failure represents a large-scale industrial accident destroying the site.

A Fusion Candle (Zubrin-GDM TW thruster) turns a gas giant into a colony ark by building a giant chimney with a fusion reactor at either end and intake in the middle - the surface end keeps the whole structure aloft and the spaceward end is a thruster using the gas giant’s atmosphere as wet mass. http://www.schlockmercenary.com/2003-08-03

A Time Vault (Cyrolibrarians future) strives to allow human technology and culture to outlive the death of the Sun - surviving the distant future when it expands into a red giant. Their strategies include replacing the hydrogen ice at the site of the vault with heavier deuterium to limit sublimation losses into space and inscribing records both on nanostructured glass tablets using femtosecond lasers and encoded into the DNA of artificial life seeded into synthesized ecologies hidden under the surface.

The Mass Beam future (Dusty Plasma TW thruster) is the next-generation version of the power beaming used by the ESA orbital laser or Push Factories. Solar power stations collect sunlight to power accelerators that fire tiny pellets of material at a distant starship. The starship vaporizes the pellets with a laser, and the resulting plasma pushes the starship via a magnetic field "mirror". The advantage of the mass beam over a laser beam is that much more of the energy used to accelerate the beam elements is transferred to the starship as momentum.

The final two mega-engineering futures use similar means for very different ends. New Venus (Wet-Nano Ecologists colonist) involves diverting an icy comet so that it will smash into Venus, adding more water to the atmosphere and possibly altering the planet's rotation to make it easier to terraform. (Venus currently has a day well over 100 Earth days in length - this is an alternative to using a climate shade as in the Climate Mirror future.) Footfall (Eugenic Pilgrims colonist) does the same thing, except smashing the comet into Earth, causing mass extinction and leaving the power-crazed madmen who orchestrated it in charge of humanity's off-world future.

Ad astra per aspera

Five of the futures correspond to various proposals for interstellar travel. Here they are, ordered by size of starship, smallest to largest.

A Star Wisp (Magnetic Mirror Beam-Rider freighter) is a tiny space probe massing no more than a few kilograms. The entire probe is a mesh of carbon wires spaced 3 millimeters apart. Computer circuitry, sensors, and radio transmitters and receivers are built into the wires themselves. The mesh is then pushed by a microwave beam with a 3mm wavelength. Its low mass allows it to be accelerated to a significant fraction of the speed of light. The beam is also used to transmit energy to the probe when it nears its destination, so it can operate its sensors and transmitters. The probe is not designed to decelerate as it passes its target star.

The famous Daedalus (Daedalus 3He-D Inertial Fusion TW thruster) design was the result of a project by the British Interplanetary Society in the mid-1970s. 50,000 tons of deuterium and helium-3 fuel plus a fusion rocket would be used in a two-stage rocket design to accelerate a 500 ton scientific payload to 12% of the speed of light over the course of about four years. This original design had Barnard's Star as its target, but its modular design allows it to be adapted to other destinations.

The liquid with the lowest known density at room temperature is a solution of lithium in anhydrous ammonia. Lithium can be combined with hydrogen nuclei (bare protons) in a fusion reaction to produce energy plus two helium atoms (3He and 4He if the more common 6Li isotope is used, two 4He atoms if the rarer 7Li isotope is used). It is thus theoretically possible to freeze large quantities of lithium dissolved in ammonia and build a starship hull out of the resulting ice, the Lithiated Ammonia Starship (Solem Medusa Tugged Orion TW thruster). The hull is then used as fuel - the ammonia (NH3) is separated into nitrogen and hydrogen, the hydrogen and lithium would be used to generate energy, and the resulting helium and leftover nitrogen would be used for reaction mass.

A similar "hull of frozen fuel" concept is used in the Enzmann Starship (Colliding FRC 3HE-D Fusion TW Thruster). In this case the hull is a multi-million ton ball of frozen deuterium with an attached fusion engine and crew/cargo module. The resulting starship would be longer than most Earth skyscrapers.

Finally, the Beehive Ark proposal (KESTS Hoop Dirt Launcher freighter fleet) involves hollowing out a small asteroid, attaching an engine to it, and using it as a generation ship to carry a large number of humans to a target star. The asteroid could be continually excavated during the journey to provide reaction mass.

The human adventure continues...

Eight of the futures are social in nature - they represent directions humanity could take in its settlement of the solar system and beyond.

The Spacefaring future (Z-pinch 3He-D Target Fusion freighter fleet) is one where large numbers of humans live and work in space. While they might still be associated with entities based on Earth, the NEO Mines future (Gypsy Alchemist colonists) envisages much closer terrestrial ties as resources near Earth are preferentially exploited. The Revolutionary future (New Attica Secessionists colonist) sees off-world development dominated by a group of humans which has cut its ties to Earth in a violent political revolution. Alternately, some kind of charismatic religious leader might become the focus of humanity's space settlement, in the Supreme Cult Leader future (Josephson Implants colonist).

It is also possible that a superior, or at least new form of, intelligence could come about as a result of research in space, away from the constraints of Earth. The Uplift concept (Creeper Neogen colonist), popularized by author David Brin, involves using a combination of genetic engineering and implanted technology to bring about human-style consciousness and intelligence in candidate animal species (nonhuman primates, dolphins, dogs, and octopuses are often considered candidates). When the intelligence is machine in nature, we have the Von Neumann future (Neumann Turings colonist) where self-replicating machines evolve intelligence, or the Artificial Consciousness (Renaissance Man colonist) where a synthetic mind is created at great risk of uncontrolled singularity. Or, researchers might use technology to network multiple human minds together to create a (possibly) superior group mind, resulting in the Pan-Sapiens future (Group Mind Immortalists). The game implies the latter future would come about through humans being forcibly incorporated into the group mind. Resistance is futile!



[1] Factory-assisted lift-offs use a fast monorail car as the launch vehicle. The rail provides traction as the car reaches orbital velocity and releases its payload into space. This follows the aerospace principle that if you have a choice to use a rocket or something else, always use something else.

[2] The Hostile Recruit operation represents persuading an employee of a rival firm to work for you. In space, where manpower is at a premium, the space workers will be able to name their own price. It is a common viewpoint that an employer has some sort of advantage over an employee. Nothing could be further from the truth. Both parties are individuals who are transacting entirely voluntarily, either for personal gain or whatever life goals each holds. Both can walk away if they no longer see an advantage. But, the employer has far more to lose, because he is invested in the enterprise. Usually heavily invested, up to his eyeballs in debt, using all his life savings, allowing that space infrastructure is expensive. If the transaction fails, the employee goes somewhere else, while the employer loses his life's work (and he has but one lifetime).
 
[3] It takes 9.5 km/sec to get to LEO, and another 3.4 km/sec to get to any of the Earth-Luna Lagrange points. The solar system escape velocity is 42 km/sec, but since you are already going 30 km/sec (the Earth’s orbital speed), this leaves 12 km/sec to travel to the stars. Cheerfully neglecting the Oberth effect here, to give the general idea behind Heinlein’s famous statement: “Once you are in low Earth orbit you’re halfway to anywhere”. The first half of your trip to anywhere is in a steep gravity field, where you need high thrust engines, launch loops, or space elevators to get your Bernals built. The second half will be with high specific impulse interplanetary engines.
 
[4] Although your colonists can be working anywhere in space, it is assumed that their families and supplies reside back home in the Bernal. Without such human factors, space cannot be colonized.
 
[5] The reactors represented in the game usually generate a few hundred megawatts of thermal power, enough to power a MW rocket. For GW and TW thrusters, however, these reactors are used not as power sources but as initiators for the primary nuclear power chain (often fusion). The ratio of nuclear power to initiator power is called the Q factor.
 
[6] Industrializing Sedna establishes a telescope at the EM Sunlens spot, 550 AU from the Earth. This planet-hunting telescope uses Sol itself as a gravitational lens, It enjoys an optical gain of 113-dB, enough to zoom-in on a landing site as far out as Epsilon Indi.
 
[7] Theories on what D worlds are range from primordial bits of the galaxy to aggregates of ices clustered around CUDOS (Compact UltraDense Objects).
 
[8] A swarm of robotic mirrors at a world’s L1 point (i.e. the Lagrange between that world and Sol) changes its albedo, and thus its climate.
 
[9] Star wisps are tiny robotic interstellar probes riding on microwave energy beamed from a push factory. A giant collimator in the outer solar system, here represented by a TNO factory, focuses the beam.
 
[10] Protium fusion is the fusion of hydrogen, such as that believed to be occurring in Sol. In a game of celestial billiards, two asteroids can be bashed together at relativistic speeds, in hopes of creating protium fusion and possibly a mini-black hole.
 
[11] The ill-advised 1979 U.N. Moon Treaty prohibits ownership of lunar resources because “the use of the moon shall be the province of all mankind.”
 
[12] A push factory on Io obtains its energy from Jupiter (using an electrodynamic tether) rather than from Sol.
 
[13] “The only known method of long-term, high-power dissipation in space is thermal radiation. ...The dramatic sixth-power decrease of mass with radiator temperature is a strong motivation to run at very high temperatures. The crew and cargo spaces...will have to be refrigerated, but the rest of the ship will run hot. At the low acceleration values we will be using, the ship can be of extremely flimsy construction-for highest performance, we want a ship made of ‘incandescent tissue paper’.” John Trenholme, 2003.
\end{document}
